\chapter{Propositions as types, proofs as programs: the Curry-Howard correspondence}
\label{chap:curry-howard}

The main interest of ours in this thesis is practical computation - so, doing computations
on a standard computer. Despite of that, a lot of space in this thesis will be devoted to 
studying logic. Why is this so?


\section{Reasoning, computation and the discrete arrow of time}
When we think about *performing a computation*, we tie it with some means of
changing some state over time

means in general, and what it
means to "physically" execute an algorithm, it is about changing some state over time.
On a Turing machine, the notion is apparent, as *states* are embedded in the definition of a turing machine.
On a physical PC, the processor is ticked by a discrete clock - nothing* can happen in between the two
clock ticks. When a tick occurs, some data might be moved between two memory places etc.
When we think about programs written in a functional programming language, while the discrete steps 
are more obscured, we still have them in the form of term normalization. The discreteness
and one-directionality of time while performing computation is also inherent in
performing concurrent or quantum computation. For now, we don't know another way of performing computation.

There is a deep link between computer science and logic, which comes up when we study this
bare nature of computation as changing a state over time. In logic, we also study some set
of provable statements with rules for transforming it in the way we believe is true.
The structure of the proof is governed by the deduction system we choose to use.
For the sake of this thesis, we only consider the following deduction systems:
- Hilbert-style, where proofs are sequences of steps
- Gentzen-style , where proofs are trees
- where proofs are dags

in the field of linear logic, *proof nets* are studied which are representations of proofs
in a nonstandard, and also non-sequential way, as a proof net is by nature non-acyclic.

In reality, perhaps one day we discover time travel and then are justified in reasoning about computation
where we can peek into the future, or the past[^1]. For now, we leave these considereations apart
and only consider finite, discrete-sequential computations.

[^1]: a noticable hyper-computational (super-Turing) model currently studied is
computation in Malament--Hogarth spacetime --- the kind of physical reality that might exist
on the edge of a black hole

\section{The Curry-Howard correspondence}
The correspondence between systems of formal logic and computational calculi can be made strict.
It was first observed by Haskell Curry in 1934, and was extended by William Howard in 1969, who
discovered that proofs of intuitionistic natural deduction syntactically are analogous to programs
of simply-typed lambda calculus.

