\chapter{Uniformity}\label{chap:uniformity}
In this chapter, we focus on the descriptions of uniformity conditions used for
families of circuits.
There is a very thorough overview of uniformity conditions for
complexity classes below \complexityi{NC}{1} in~\cite{MIXBARRINGTON1990274}.

\section{\complexity{FO}-uniformity}\label{sec:uniformity-fo}
The definition of first-order uniformity is rather technical and is based on the notion of
first-order queries, introduced in~\cite[Definition~5.16]{Immerman1999-IMMDC}.
One of the results that we discussed which uses this notion is~\autoref{thm:fo-eq-ac0}.

% immerman:
% Definition 5.16 (Uniform) Let C be a sequence of circuits as in Equation (5.14).
% Let l' E {l'eo l'thc} be the vocabulary of circuits or threshold circuits. Let / :
% STRUC[l's] ~ STRUC[l'] be a query such that for all n E N, /(0") = Cn. That
% is, on input a string of n zero's the query produces circuit n. If / E FO, then C is
% 80 5. Parallelism
% afirst-order uniform sequence of circuits. Similarly, if I E L, then Cis logspace
% uniform. If I E P, then C is polynomial-time uniform, and so on. 


\section{\compUeAst-uniformity}\label{sec:uniformity-ueast}
The notions of \(U_E\) and \(U_{E^\ast}\)-uniformity (sometimes also called $E^\ast$-uniformity) were studied in early work on circuit uniformity.
These notions are defined in terms of the \emph{direct connection language}
and the \emph{extended connection language} of a circuit family; see~\cites[Definitions~2.24 and~2.43]{10.5555/520668}.

Since then, $U_E$- and $U_{E^\ast}$-uniformity have largely been displaced by \complexity{DLOGTIME}-uniformity.
Interesting arguments about why \complexity{DLOGTIME} uniformity is the most reasonable to consider
are presented in a breakthrough paper proving that binary integer division is in 
(\complexity{DLOGTIME}-uniform) \TC{0}~\cite{HESSE2002695}.

\section{\complexity{DLOGTIME}-uniformity}\label{sec:uniform-nc1}
For very low complexity classes, the most commonly used notion of
uniformity is \complexity{DLOGTIME}-uniformity, based on random-access
Turing machines. Similarly as in~\autoref{sec:uniformity-ueast},
this notion is also based on direct and extended connection language of a circuit family.
A~circuit family is \complexity{DLOGTIME}-uniform when we can decide its
direct connection language on a random-access Turing machine in logarithmic time.
For the details, we refer to:~\cite[Section~6]{MIXBARRINGTON1990274}.

\complexity{DLOGTIME}-uniformity has some very elegant properties.
The strength of such Turing machines is well-studied.
It's mentioned that \(\complexity{AC}^0_0 \subseteq \complexity{DLOGTIME} \subseteq \complexity{AC}^0_2\) in~\cite[Page~141]{10.5555/114872},
where $\complexity{AC}^0_k$ denotes $\complexity{AC}^0$ circuits of depth $k$.
It has also been shown that the class \complexity{DLOGTIME}-uniform \complexityi{NC}{1} is equal to \complexity{ALOGTIME} (alternating
logarithmic time, which we don't introduce here)
and also equal to \complexityi{NC}{1}-uniform \complexityi{NC}{1}:~\cite[Lemma~6.2]{MIXBARRINGTON1990274}.

% Similarly to using this class of Turing machines to decide the connection languages,
% functions can also be studied. This is the basis of the notion of \complexity{DLOGTIME}-reductions.
% A very detailed example of a concrete \complexity{DLOGTIME}-reduction is presented in~\cite{694595},
% showing that the tree isomorphism problem for string-represented trees is $\complexityi{NC}{1}$-complete.


% barrington:
%We define a log-time Turing machine to have a read-only input tape of length n,
% a constant number of read-write work tapes of total length O(log n), and a read-
% write input address tape of length log n. On a given time step the machine has
% access to the bit of the input tape denoted by the contents of the address tape (or
% to the fact that there is no such bit, if the address tape holds too large a number).
% We will assume (without loss of generality) that the machine always takes the same
% amount of time on inputs of a given length (this is because some of the work tape
% can always used as a clock). The following lemma summarizes some useful
% capabilities of such a machine.

% Important: tm can decide formula langugae: {<c, i, y>: |y|=n and ith char of nth formula is c}
% very important:

\begin{remark}[Bibliography]\label{remark:bibliography-david-s-johnson}
We refer to a source that is difficult to access (e.g.\ the above~\cite[Page~141]{10.5555/114872}): this is Chapter 2 ``A Catalog of Complexity Classes'' by
David S. Johnson~\cite{10.5555/114872.114874},
which appeared in January 1991 in ``Handbook of theoretical computer science (vol. A): algorithms and complexity'',
edited by Jan van Leeuwen~\cite{10.5555/114872}.
\end{remark}


% \section{p-projections}
% In~\cite{10.1145/3149.3158}, the notion of p-projections are introduced. In~\cite{10.1007/BFb0028550},
% a complete problem for \AC{0} of depth $k$ is discussed: complete for uniform~\AC{0} under 


% REMOVE THIS? THIS IS FOR P-PROJECTIONS FOR AC0k
% \begin{definition}[Projection of Boolean functions]\label{def:boolean-proj}
% Let $f:\{0,1\}^n\!\to\!\{0,1\}$ and $g:\{0,1\}^m\!\to\!\{0,1\}$.
% We say that $f$ is a \emph{projection} of $g$ if there is a mapping
% $\varsigma : \{y_1, \dots, y_m\} \rightarrow \{0, 1, x_1, \dots, x_n, \neg x_1, \dots, \neg x_n\}$
% such that
% \[
% f(x_1, \dots, x_n) = g\big(\varsigma(y_1),\dots,\varsigma(y_m)\big).
% \]
% \end{definition}
% \begin{definition}[p\mbox{-}projection between families]\label{def:boolean-p-proj}
% Let $\mathcal{P}=(P_i)_{i\in\mathbb{N}}$ and $\mathcal{Q}=(Q_j)_{j\in\mathbb{N}}$ be families
% of Boolean functions (each $P_i,Q_j:\{0,1\}^{*}\!\to\!\{0,1\}$, arity arbitrary).
% We say that $\mathcal{P}$ is a \emph{p\mbox{-}projection} of $\mathcal{Q}$, if 
% there exists a polynomial $t:\mathbb{N}\to\mathbb{N}$ such that
% for every $i\in\mathbb{N}$ there is some $j\le t(i)$ with $P_i$ a projection of $Q_j$.
% \end{definition}

% \begin{remark}[Why the polynomial bound]
% Unrestricted projection lets a simple $P_i$ be realized only by projecting
% some $Q_j$ at an arbitrarily large index $j$, which makes the comparison vacuous.
% The polynomial bound $j\le t(i)$ enforces an \emph{efficient} correspondence of indices,
% yielding a robust, reduction-like notion. The relation $\preceq_p$ is a preorder;
% modulo $\equiv_p$ it induces a partial order on equivalence classes of families.
% \end{remark}

% \section{Bibliographical remark}
% The notions of~\autoref{def:boolean-proj} and~\autoref{def:boolean-p-proj} are from~.
% \begin{definition}[\texorpdfstring{\(\complexity{AC}^0_k\)}{AC0 depth k}]\label{def:complexity-ac0-k}
%     This is \complexityi{AC}{0} of depth $k$.
%     Complete problem:\cite{10.1007/BFb0028550}
% \end{definition}

% \begin{proposition}
%     for \(k \geqslant 3\), \(\texttt{MAZE}_k\) is complete for non-uniform $\Pi_k$ under p-projections
%     and complete for uniform $\Pi_k$ under \complexity{DLOGTIME}-uniform projections~\cite{10.1007/BFb0028550}.
% \end{proposition}