\section{Classes of binary circuits}\label{sec:defs-preliminaries-circuits}
We omit the standard definition of a Boolean circuit (a~kind of finite
directed acyclic graph of $n$ inputs and $m$ outputs).
We also omit the definitions of circuit size, depth, and value;
these are discussed in detail e.g.\ in~\cite{10.5555/520668}.

\begin{definition}[\NCipoly, \ACipoly, \TCipoly]\label{def:ac-nc-tc-poly}

Fix \(i \ge 0\).  
A language \(L \subseteq \{0,1\}^\ast\) belongs to one of the following
circuit classes if there exists a family of circuits
\(\{C_n\}_{n\in\mathbb{N}}\) such that \(C_{\len{x}}(x)=1\) iff \(x\in L\) and:

\begin{enumerate}
    \item\label{itm:circ-size} every circuit \(C_n\) has polynomially many gates w.r.t.\ \(n\);
    \item\label{itm:circ-depth} every circuit \(C_n\) has depth \(\bigO((\log n)^i)\);
    \item\label{itm:circ-nc} \(L \in \NCpoly{i}\) (Nick's class)  
          if each \(C_n\) uses only fan-in~2 \(\wedge\)-, fan-in~2 \(\vee\)-gates
          and fan-in~1 \(\neg\)-gates;

  \item\label{itm:circ-ac} \(L \in \ACpoly{i}\)  (Alternating circuits)
        if each \(C_n\) uses  
        unbounded fan-in \(\wedge\)-, unbounded fan-in \(\vee\)-gates and
        fan-in~1 \(\neg\)-gates;

  \item \label{itm:circ-tc}\(L \in \TCpoly{i}\)  (Threshold circuits)
        if each \(C_n\) uses unbounded fan-in \(\wedge\)-, unbounded fan-in \(\vee\)-gates,
        fan-in~1 \(\neg\)-gates and unbounded fan-in \emph{majority} gates
        (i.e.\ a gate that outputs 1 iff at least half of its inputs are one).
\end{enumerate}

Note that the condition \(C_n(x)=1\) iff \(x\in L\) requires the circuits
to have precisely one output node. We lift this requirement in~\autoref{def:faci-poly}.
\end{definition}


\begin{definition}[\FNCipoly, \FACipoly, \FTCipoly]\label{def:faci-poly}
A function \(F : \{0, 1\}^\ast \rightarrow \{0, 1\}^\ast\) belongs to
\FNCipoly (resp. \FACipoly, \FTCipoly) iff there exists a family of circuits
with multiple output nodes
\(
  \langle C_n \rangle_{n \in \mathbb{N}}
\)
such that whenever the input bits of \(C_n\) encode \(X \in \{0, 1\}^n\), the output bits
encode \(F(X)\); \(C_n\)
satisfies the size and depth
conditions~\ref{itm:circ-size},~\ref{itm:circ-depth} of~\autoref{def:ac-nc-tc-poly};
and additionally \(C_n\) satisfies the class condition~\ref{itm:circ-nc}
(resp.~\ref{itm:circ-ac},~\ref{itm:circ-tc}) of~\ref{def:ac-nc-tc-poly}.
\end{definition}

\begin{definition}[\compL-uniform circuit families]\label{def:logspace-uniformity}
      We say that a family of circuits \(\langle C_n \rangle_{n \in \mathbb{N}}\)
      is \compL-uniform if there exists a Turing machine operating in space \(\bigO(\log n)\),
      computing the function \(1^n \to C_n\) for some representation of \(C_n\).
\end{definition}

% Ostroznie z definiowaniem FAC tutaj. Moze uwzglednic definicje uniform fac z bournez_et_al:LIPIcs.MFCS.2019.23 ?
\begin{definition}[\NCi, \ACi, \TCi, \FNCi, \FACi, \FTCi]\label{def:aci-faci}
      A function \(F: \{0, 1\}^\ast \rightarrow \{0, 1\}\) belongs to
      \NCi (resp. \ACi, \TCi) iff there exists a \compL-uniform family of circuits satisfying the
      conditions from~\autoref{def:ac-nc-tc-poly}.
      A function \(F : \{0, 1\}^\ast \rightarrow \{0, 1\}^\ast\) belongs to
\FNCi (resp. \FACi, \FTCi) iff there exists a \compL-uniform family of circuits
satisfying the conditions from~\autoref{def:faci-poly}.
\end{definition}

% One-way permutations in $\complexityi{NC}{0}$~\cite{10.1016/0020-01908790053-6}.
% All sets complete under $\complexityi{AC}{0}$ reductions are already complete under $\complexityi{NC}{0}$ reductions~\cite{10.1145/258533.258671,AGRAWAL1998127}.
% Addition and subtraction of binary numbers lies in $\complexityi{AC}{0}$~\cite{27676}.

% \complexity{DLOGTIME}-uniform $\complexityi{AC}{0}$:~\cite{hella2023regularrepresentationsuniformtc0}.
% Review the definitions of $\complexityi{AC}{0}$, $\complexityi{TC}{0}$, FATC0, and FTC0 provided in~\cite{612309}.