\chapter{Implicit Computational Complexity}\label{chap:icc}
Implicit computational complexity (ICC) studies how to guarantee resource bounds
without appealing to external machine models.
Instead of analysing running time or space after the fact, ICC designs
languages and recursion schemes whose syntactic constraints ensure that every
definable function belongs to a chosen complexity class.
The aim is a principled foundation for programming languages that ``build in''
complexity guarantees by construction.

Our goals in this chapter are twofold.
For one, we want to introduce
the most promising ICC approaches we have found towards obtaining a language for
\complexity{FL} or for \complexity{FP}. Indeed, the characterizations we will study
resulted in the creation of two actual programming languages. They are purely of academic
interest, but there are close to none alternative languages for complexity classes
that reached the stage of being actually implemented.
In the rest of this chapter we explain why the approach we studied most ---
of using techniques from ICC --- did not lead to a practical programming language for our purposes,
even though it initially seems very appealing.
However, more importantly, we want to share \emph{negative} evidence of
usefullness of the characterizations studied in ICC towards our ultimate goal
of not just creating a programming language, but using it to \emph{certify}
the complexity of known algorithms.

We will begin by sharing our experience with ICC,
then proceed to discuss the characterizations we have studied, most importantly:
in~\autoref{def:bc-eps} we examine a programming language based on the first
function algebra (untyped) for \complexity{FL} due to Neergaard; in~\autoref{sec:intml} we study
Dal Lago and Schöpp's programming language \texttt{IntML} with the type system 
guaranteeing the capturing of \complexity{FL} and \complexity{FNL} (nondeterministic version of \complexity{FL},
which we don't introduce formally).

\begin{remark}
  Recall that, given a description of a Turing machine, the problem of deciding whether it halts
  is undecidable. Hence, deciding whether an arbitrary computer program belongs to the complexity
  class \compP{} is also undecidable. This is only a problem, however, if we take \emph{arbitrary}
  programs as input. If we limit the scope to programs written in a special, limited programming
  language, we can easily design the language so that it does not admit constructs such as
  \texttt{while}-loops or general recursion, and therefore every program in it necessarily terminates.
  Moreover, membership in the syntax of such a language can be easy to decide. In this way, the
  undecidability problem is not resolved, but ``shifted'' to the difficulty of programming: given a
  \texttt{C} program implementing an algorithm, it becomes undecidable whether there exists a
  corresponding program in our restricted language.
\end{remark}


That was the approach that took us the most time to study and did not result in obtaining
``practical programming languages''. We want to place ourselves in the program of ``we need a better way of
communicating negative results (failed research)'', and signal why, despite the apparent
usefulness of the results for our purpose, our goal failed. The reason is the difference between
\emph{intenstional} and \emph{extensional} expressive power of characterizations of complexity classes.
With type systems capturing low complexity classes, it becomes very apparent that these function
classes are only \emph{extensionally} equivalent to their corresponding complexity classes.
For an example of how easily we can lose \emph{intensional} expressive power in a language with
linear types, please look at~\autoref{lst:haskell-linear-example}.
They shift the problem of undecidability of deciding if a given Turing machine works in some complexity,
to the problem if the algorithm implemented by it is transferrable to the particular programming language.

\subsection{Extensional and intensional equality of sets of functions}\label{subsec:intensional}

The field appear very fragmented and the approaches don't scale:~\cite{DalLagoMartini2006MuTutorial}.

\begin{remark}
    We don't introduce tiered recursion, ramified recursion and Leivant's works in general here:~\cite{LeivantRamyaa11}.
    See also~\cite{3ffa7833-e2d2-3419-abb5-7f266190ba48} for discussion on tiering as recursion technique.
\end{remark}



Accessible introductions to ICC include the three-part presentation~\cite{martini2006implicit1,martini2006implicit2,martini2006implicit3},
the talk~\cite{ronchi2019logic},
and a short overview~\cite{DalLago2012}.


\begin{remark}
    We focused on the characterizations of \complexity{FL} and \complexity{FP}.
\end{remark}

% \subsection{Related works on specifically the \complexity{FL} class}
% Early function algebras for \complexity{FL} appeared in~\cite{10.1145/1008293.1008295} and~\cite{lind1974logspace},
% but these were explicit characterizations.

% In~\cite{10.1007/978-3-662-46678-0_27}, an interesting approach using coinduction is utilized to capture \complexity{FL}.

% In~\cite{hofmann2006logspace} a good overview of languages for \complexity{FL} is presented,
% and in~\cite{schoepp2006spaceefficiency}, the history of \complexity{FL} characterizations is traced.

Description of logspace, ptime (decisive):~\cite{Jones99}; logspace, linspace:~\cite{Kristiansen05}.~\cite{Bonfante06}
For a broad literature survey, see~\cite{bloch1994function}.




\subsection{Characterizations not easily adjustable for a programming language}
Before the seminal works that founded the field of Implicit Complexity, many characterizations
of complexity classes had been known already. All of them suffered at least one of the two problems:
either it only characterized a class of relations in a given complexity (as opposed to functions),
or the characterization wasn't purely syntactic. We will refer to the latter of being ``explicit''
instead of ``implicit''.

\subsection{Characterizations of classes of relations}
Characterizations of classes of relations, such as \complexity{P} (as opposed to \complexity{FP}),
are not of interest to us because they don't generalize at all to a programming language
allowing to write functions with output. Nevertheless, we investigated the concepts used
there and describe some of them briefly in this subsection.

Polynomial-time relations have been characterized without explicit
size bounds in~\cite{doi:10.1137/0216051}.
In~\cite{COMPTON1990241}, uniform \(\complexityi{NC}{1}\) was characterized,
and in~\cite{ALLEN19911} uniform \(\complexity{NC}\), though their definitions
still concealed polynomial bounds and targeted relations instead of functions.

In more modern works, decisive complexity classes have been successfully characterized in~\cite{JONES1999151} by a fragment of Lisp in~\(\complexity{L}\) and
\(\complexity{P}\). The same concept has been extended to account for
for nondeterminism in~\cite{10.1007/11784180_8}.
The authors of~\cite{kristiansenvoda2005} investigated both imperative
and functional programming languages whose fragments yield hierarchies containing \emph{decisional} \complexity{L},
\complexity{LINSPACE}, \complexity{P}, and \complexity{PSPACE}.
Related contributions include~\cite{kristiansen2005neat} and~\cite{Oitavem+2010+355+362}.



The modern study of ICC begins with two breakthroughs:~\cite{151625}~and~\cite{10.1007/BF01201998}
gave the first implicit characterisations of polynomial-time computable functions.

However, the idea of Bellantoni and Cook seemed to best align with being the foundation
of a practical programming language. Hence, we decided to solely focus on it and its successors.

Since then (since BC, leivant), numerous classes have been captured implicitly; see, for
example,~\cite{NIGGL201047}~and~\cite{10.1016/j.ic.2015.12.009}
for overviews of \(\complexity{FP}\) and \(\complexity{FNC}\) characterisations.

See~\cite{10.1007/s00153-022-00828-4} for implicit characterizations of counting classes such as $\complexity{\mathbin{\#}P}$ (not introduced here).

\section{Recursion-theoretic approach}\label{sec:recursion-theory}
In this section we focus on techniques from recursion theory that were successfully
utilized in Implicit Computational Complexity.

\subsection{Origins of recursion theory}
While not the primary focus of this work, the field of recursion theory developed concepts
that later became foundational for ICC\@. An important formal system studied there is \emph{primitive recursion}.


\begin{definition}[Primitive recursive functions]\label{def:primitive-recursive}
      \(\complexity{PR}\) is the smallest class of functions containing~\ref{itm:pra-const}--\ref{itm:pra-proj} and closed under~\ref{itm:pra-comp} and~\ref{itm:pra-rec}:
\begin{enumerate}
\item\label{itm:pra-const}\textbf{(constants)} for every \(n\in\mathbb{N}\) and \(k\ge 0\), the \(k\)-ary constant function
      \(c_{n}^{(k)}(\vec x)=n\);
\item\label{itm:pra-succ} \textbf{(successor)} \(S(x)=x+1\);
\item\label{itm:pra-proj} \textbf{(projections)} for \(k\ge 1\) and \(1\le i\le k\),
      \(\pi_i^{(k)}(x_1,\dots,x_k)=x_i\);
\item\label{itm:pra-comp} \textbf{(composition)} if \(h:\mathbb{N}^m\to\mathbb{N}\) and
      \(g_1,\dots,g_m:\mathbb{N}^k\to\mathbb{N}\) are in \(\complexity{PR}\), then
      \(f(\vec x)=h\big(g_1(\vec x),\dots,g_m(\vec x)\big)\) is in \(\complexity{PR}\);
\item\label{itm:pra-rec} \textbf{(primitive recursion)} if \(g:\mathbb{N}^k\to\mathbb{N}\) and
      \(h:\mathbb{N}^{k+2}\to\mathbb{N}\) are in \(\complexity{PR}\), then the unique
      \(f:\mathbb{N}^{k+1}\to\mathbb{N}\) is in \(\complexity{PR}\) with:
      \[
      f(0,\vec x)=g(\vec x),\qquad
      f(S(y),\vec x)=h\big(y,\,f(y,\vec x),\,\vec x\big).
      \]
\end{enumerate}
\end{definition}

\begin{example}[Addition]
Define \(\mathrm{Add}:\mathbb{N}^2\to\mathbb{N}\) by primitive recursion:
\[
\mathrm{Add}(0,x)=x, \qquad
\mathrm{Add}(S(y),x)=S\big(\mathrm{Add}(y,x)\big).
\]
\end{example}

\begin{definition}[LOOP language]
Let \(\mathrm{Var}=\{x_0,x_1,x_2,\dots\}\).
LOOP programs are generated by the grammar
\[
\begin{aligned}
P ::=~& x_i := 0
\;\mid\; x_i := x_i + 1
\;\mid\; P \,;\, P
\;\mid\; \texttt{LOOP}~x_i~\texttt{DO}~P~\texttt{END},
\end{aligned}
\]
where \(x_i\in\mathrm{Var}\).

\noindent
We assume standard semantics, with a remark that \(\texttt{LOOP}~x_i~\texttt{DO}~P~\texttt{END}\) repeats \(P\) exactly as many times as the value stored in \(x_i\) \emph{at loop entry} (changes to \(x_i\) inside \(P\) do not change the iteration count).
\end{definition}

\begin{theorem}[{\cite{10.1145/800196.806014}}~\complexity{LOOP} captures precisely \complexity{PR}]\label{thm:loop-captures-pr}
      % TODO: make statement precise, perhaps based on https://www.cs.cornell.edu/courses/cs6110/2012sp/MeyerAndRitchie-67.pdf
      The functions computable by \complexity{LOOP} programs are precisely the primitive recursive functions
\end{theorem}

This simple connection actually satisfies our criteria of a ``programming language capturing a complexity class'', as
the \(\complexity{LOOP}\) language captures exactly \(\complexity{PR}\)\footnote{As recognized at \url{https://complexityzoo.net/Complexity_Zoo:P}.}, the class of primitive recursive functions.
Moreover, we can even stratify the primitive recursive functions into a hierarchy like in~\cite{Grzegorczyk1953}.

\begin{remark}[Bibliography]
      Historically, the origins of primitive recursion can be traced back to~\cite{Grassmann1861} and~\cite{Dedekind1888},
      but the class was probably first considered as the primary object of study in~\cite{Skolem1923-vanHeijenoort}.
      For the details of the historical origins, consult~\cite{Adams2011}.
\end{remark}


\subsection{Explicit characterizations}\label{subsec:explicit}
Some characterizations discussed in the literature contain so-called \emph{explicit}
conditions --- e.g.\ they explicitly require a function to not grow faster than polynomially.
Such conditions cannot be checked syntactically and must be enforced
by an additional proof outside of the algebra. We call these characterizations
explicit, as opposed to implicit. Despite not being convenient to use for our purpose, these concepts
have been very important for the field and we will discuss one example in this subsection.
For more examples of explicit characterizations, please see~\cite{4568079} for an algebra for polynomial-time
functions;~\cite{COMPTON1990241} for uniform \complexityi{NC}{1};~\cite{ALLEN19911} for uniform \complexity{NC}.

For a good overview of implicit versus explicit characterizations, refer to~\cite{aubert:hal-01111737}.

A famous example of an explicit characterization is Cobham's algebra
for polynomial-time functions, using the recursion scheme defined below.
We will use the notation \(S_0(y)=2y\) and \(S_1(y)=2y+1\) for appending a binary digit to \(y\).
We will denote by $\len{x}$ the length of binary representation of $x$; in particular, $\len{0} = 1, \len{1}=1, \len{2}=2$.

\begin{definition}[{\cite[Definition~VIII.2.14~(Page~174)]{Odifreddi1999CRT2}}]\label{def:bounded-binary-primitive-recursion}
A function \(f\) is defined from functions \(g\), \(h_0\), \(h_1\), and \(s\) by \emph{bounded primitive recursion on binary notation}
if, for every \(\vec x\) and \(y\in\mathbb{N}\),
\begin{align}
  f(\vec x, 0)      &= g(\vec x);\\
  f(\vec x, S_0(y)) &=
    \begin{cases}
      0 & \text{if } y = 0,\\
      h_0\big(\vec x, y, f(\vec x, y)\big) & \text{otherwise;}
    \end{cases}\\
  f(\vec x, S_1(y)) &= h_1\big(\vec x, y, f(\vec x, y)\big);\\
  f(\vec x, y)      &\le s(\vec x, y)\label{eq:explicit-cobham}.
\end{align}
\end{definition}

\begin{remark}\label{remark:cobham-recursion}
  The recursion parameter $y$ is written in binary, so the definition of $f(\vec x,y)$
  unfolds only $\mathcal{O}(\len{y})$ many steps.
  Moreover, the
  side condition $f(\vec x,y)\le s(\vec x,y)$ implies that every value of $f(\vec x,y)$
  is at most $s(\vec x,y)$. Hence, if the defining functions
  $g,h_0,h_1$ and the bounding function $s$ are polynomially bounded (in the lengths
  of their arguments in binary), then $f$ will also be polynomially bounded.
  
  In this definition, the
  bound~\eqref{eq:explicit-cobham} is \emph{explicit}: when one writes a function in this
  style, there is no obvious way to check mechanically that $f(\vec x,y)\le s(\vec x,y)$
  holds, other than by supplying a separate mathematical proof.
\end{remark}


\begin{definition}[Cobham's algebra for \complexity{FP}]\label{def:cobham}
The class \complexity{Cob} is the smallest class of functions containing~\ref{itm:cobham-zero}--\ref{itm:cobham-smash} and closed
under composition and bounded primitive recursion on binary notation:
\begin{enumerate}
\item\label{itm:cobham-zero} for every $k \ge 0$, the $k$-ary constant function $0(\vec{x}) = 0$;
\item the binary successor functions \(S_0(x)=2x\) and \(S_1(x)=2x+1\);
\item for \(k\ge 1\) and \(1\le i\le k\), projections
                  \(\pi_i^{(k)}(x_1,\dots,x_k)=x_i\);

\item\label{itm:cobham-smash} the weak exponential \((x,y)\mapsto x^{\len{y}}\) denoted $x\mathbin{\#}y$; sometimes called \emph{the smash function}.
\end{enumerate}
\end{definition}

\begin{remark}
This algebra was originally defined for decimal digits.
Note that all of our initial functions are polynomially bounded in the lengths
of their arguments. For the smash function we have, for $x,y \ge 1$,
\[
  x < 2^{\len{x}},
  \qquad
  x^{\len{y}} < 2^{\len{x}\cdot\len{y}},
  \qquad
  \len{x^{\len{y}}} \le \len{x}\cdot\len{y} + 1,
\]
hence
\[
  \len{x \mathbin{\#} y}
  = \len{x^{\len{y}}}
  \le \len{x}\cdot\len{y} + 1.
\]
Composition preserves polynomial boundedness, and bounded recursion on binary
notation also preserves it (recall~\autoref{remark:cobham-recursion}),
so every function in $\complexity{Cob}$ is polynomially bounded.
\end{remark}

\begin{theorem}[{\cite[Proposition~VIII.2.15~(Page~175)]{Odifreddi1999CRT2}}]\label{thm:cobham-is-fp}
      The class \complexity{Cob} contains precisely the functions from \complexity{FP}.
\end{theorem}

\begin{remark}
      Cobham's characterization underlies the arithmetical theory \complexity{PV},
      which is also designed to capture
      polynomial time reasoning in the style discussed in~\autoref{chap:bounded-arithmetic}.
      We will not introduce \complexity{PV} in this work, but we will mention it once again in~\autoref{sec:bounded-arith-imppv}
      to discuss another concept for a programming language. Please see~\cite[End~of~section~6]{10.1007/BF01201998} for
      a brief discussion.
\end{remark}

\begin{remark}[Bibliography]
This algebra from~\autoref{def:cobham} was originally published in~\cite{Cobham1964-COBTIC} and never proved by the author to
actually capture \complexity{FP}.
In~\cite[Remark~15.3.22]{Tourlakis2022Computability} there is a discussion of several proofs of this
theorem from the literature, each based on different, and in general non-equivalent, definitions.
The proof we referenced in the header of~\autoref{thm:cobham-is-fp} is due to Odifreddi.
\end{remark}



\subsection{Characterization of \complexity{FP} with safe-recursion}

Bellantoni and Cook introduced a function algebra \(\complexity{BC}\) whose key
innovation is the separation of arguments into \emph{normal} inputs (controlling
recursion depth) and \emph{safe} inputs (being passed around without
influencing that depth).
We write \(f(\vec{x};\vec{a})\), with normal inputs \(\vec{x}\) to the left of
the semicolon and safe inputs \(\vec{a}\) to the right.

\begin{definition}[Bellantoni and Cook's algebra for \complexity{FP}]\label{def:bellantoni-cook}
We will use the notation $a0$ for the binary representation of $2*a$. Similarily, we will use $a1$ for $2*a + 1$,
and the more general $ai$ when $i$ is known from the context to be equal to either $0$ or $1$.

      The class \(\complexity{BC}\) is the smallest class of functions on non-negative
integers that contains~\ref{itm:bc-constant}--\ref{itm:bc-cond} and is closed under~\ref{itm:bc-rec} and~\ref{itm:bc-comp}:
\begin{enumerate}
  \item\label{itm:bc-constant}\textbf{(constant)} \(0(;)=0\);
  \item\textbf{(projection)} for \(m,n\ge 0\) and \(1\le j\le m+n\),
        \[
          \pi_j(x_1,\dots,x_m;\,a_1,\dots,a_n)=
          \begin{cases}
            x_j     & \text{if } j\le m,\\
            a_{j-m} & \text{otherwise.}
          \end{cases};
        \]
  \item \textbf{(successors)} \(s_i(;a)=2a+i\) for \(i\in\{0,1\}\);
  \item \textbf{(predecessor)} \(p(;0)=0\) and \(p(;ai)=a\);
  \item\label{itm:bc-cond}\textbf{(conditional)}\[
        C(;a,b,c)=
        \begin{cases}
          b & \text{if } a\bmod 2 = 0,\\
          c & \text{otherwise};
        \end{cases}
        \]
  \item\label{itm:bc-rec}\textbf{(predicative recursion on notation (\emph{safe} recursion))}\
        if \(g,h_0,h_1\in\complexity{BC}\), then also \(f \in \complexity{BC}\) with
        \begin{align*}
        f(0,\vec{x};\vec{a}) &= g(\vec{x};\vec{a}),\\
        f(yi,\vec{x};\vec{a}) &= h_i\bigl(y,\vec{x};\vec{a},f(y,\vec{x};\vec{a})\bigr)\qquad \text{for } i \in \{0, 1\}
        \end{align*}
  \item\label{itm:bc-comp}\textbf{(safe composition)}\
        if \(h,\vec{r},\vec{t}\in\complexity{BC}\) with each component of
        \(\vec{r}\) taking only normal arguments and each component of
        \(\vec{t}\) taking both normal and safe arguments, then also \(f \in \complexity{BC}\) with
        \[
          f(\vec{x};\vec{a}) =
          h\bigl(\vec{r}(\vec{x};\,);\ \vec{t}(\vec{x};\vec{a})\bigr).
        \]
\end{enumerate}
\end{definition}

\begin{theorem}[{\cite[Theorem~3.3,4.2]{10.1007/BF01201998}}]\label{thm:icc-belantoni-cook-fp}
      Let $f(\vec{x})$ be a function in \complexity{FP}. Then $f(\vec{x};)\in\complexity{BC}$.
      Let $f(\vec{x};\vec{y})$ be a function in \complexity{BC}. Then $f(\vec{x}, \vec{y}) \in \complexity{FP}$.
\end{theorem}


\begin{remark}[Intuition]
  The key idea of safe recursion is that safe arguments can never flow back into normal positions.
  In safe composition, the normal arguments of $h$ are obtained only from
  functions that themselves take only normal inputs. In the recursion scheme, the
  recursive value $f(y,\vec{x};\vec{a})$ is passed to $h_i$ as a safe argument.
  As a result, the
  depth of recursion can depend only on the normal inputs, and not on values computed
  during the recursion.
\end{remark}

\begin{remark}\label{remark:icc-bc-algebra-bitstrings}
The theorem is formulated in terms of computation on non-negative integers, but the proof transfers to
the case of general binary strings (e.g.\ being able to start with a zero)~\cite{10.1007/BF01201998}.
\end{remark}

\begin{remark}[Formalization of polynomial time functions]\label{remark:heraud-nowak}
Interestingly, in~\cite{10.1007/978-3-642-22863-6_11} the authors claim formalizing a~proof that
the Bellantoni-Cook algebra formulated on binary strings (recall~\autoref{remark:icc-bc-algebra-bitstrings})
captures precisely \complexity{FP}.


This sparked our interest, as it could suggest that the authors have formalized the notion of an \complexity{FP}
function in a proof assistant. In particular, one could hope for a compiler translating \complexity{BC} programs
into Turing machines running in \complexity{FP}. This is not the case, however, as
they have only formalized the proof that the class \complexity{BC} is the same as the class
\complexity{Cob}.\footnote{Link to the source code: \url{https://github.com/davidnowak/bellantonicook}.}
\end{remark}


\begin{remark}
In~\cite{10.1007/BF01201998} it is also shown how to readily use their safe recursive algebra
to characterize functions from \complexity{FL} with ``small output'', but this characterization
relied on using unary representation of natural numbers on input, which is more of a 
hack than a true characterization of this class.
\end{remark}


\subsection{Characterization of \complexity{FL} with affine safe recursion}
Møller-Neergaard refined the safe recursion discipline to capture \complexity{FL}
by strengthening the treatment of safe data: each safe value may be \emph{used} at most once.
An analogy from quantum computing is that after we ``measure'' a value by, for example, testing
one of its bits in a conditional, it disappears (goes out of scope). As in the
Bellantoni-Cook setting, arguments are split into normal and safe ones, and recursion is
permitted only on the normal arguments. In addition, the composition and recursion schemes are
designed so that safe arguments are never duplicated, and recursion has a course-of-value
flavour: successive recursive calls are allowed to ``jump back'' along the recursion chain and
do not need to visit every intermediate value. These restrictions together yield a function
algebra that characterizes \complexity{FL}. As the precise definition is quite technical and
will not be used later, we refer the interested reader to~\autoref{def:bc-eps}
and~\autoref{thm:neergaard-theorem} in~\autoref{chap:appendix-neergaard}.

This algebra was very important for us, as its description in~\cite{10.1007/978-3-540-30477-7_21}
explicitly names it as a programming language. It seemed very promising indeed to be implemented
on a computer as a simple, useful programming language.

However, in practice we found it very difficult to program in this algebra.
Neergaard's system can define every function in \complexity{FL}, but not every
\emph{algorithmic technique} commonly used to implement \complexity{FL} Turing machines.
In other words, it matches \complexity{FL} extensionally, but its intensional
expressive power (cf.~\autoref{subsec:intensional}) is quite limited. For instance,
even a function that always returns $0$ requires a non-trivial use of the recursion
scheme, simply to introduce normal arguments.

It seemed unlikely that we could add support for structures such as pairs and lists to this programming language.
In an unpublished technical report by the author, accessible at~\cite{Neergaard2004BCeps},
it was discussed that the type of \emph{pairs} of numbers seems not to be implementable in this
language. The property of the whole data structure \emph{disappearing}
after we check one bit of it seems to render implementing typical data structures very difficult.


\paragraph{Neergaard's original code from 2004}
The author's original code of the interpreter referenced in the paper is not publicly accessible.
We have managed to obtain the original Moscow ML code from 2004 from the author on a permissive license,
port it to a modern version of SML/NJ and release it with the author's
permission.\footnote{The code is available at:~\url{https://github.com/ruplet/neergaard-logspace-characterization}.}

\begin{remark}[Bibliography]
      The technical report~\cite{Neergaard2004BCeps} has never been published and seems to be
      a preliminary version of the author's publication from the same year. Despite that,
      it contains crucial insight on the limits of this characterization.
      There are papers closely related to Neergaard's publication and necessary
      to also be studied while exploring Neergaard's work. Please also
      see~\cite{NeergaardMairson2003HowLight}. There is also~\cite{MurawskiOng2000SafeRecursion},
      but it seems inaccessible online. Probably the contents of that work is similar to~\cite{MURAWSKI2004197}
\end{remark}

\begin{remark}[Origins of this thesis]
For insight into the timeline of this thesis, it's worth to note that we have first discovered
this paper ourselves around February 2023.
\end{remark}
\section{Linear types}\label{sec:linear-types}
The type systems of most popular programming languages are fairly weak:
they do not provide direct support
for verifying high-level correctness or complexity properties.
Functional programming languages with strong type systems exist, allowing us to
verify quite useful correctness properties --- one example is the language Haskell.
Going yet
stronger, Lean and Rocq are at the same time programming languages and proof assistants, i.e.\
their type systems allow us to prove even very abstract mathematical properties of the functions they define.
However, none of the mainstream languages allows us to enforce useful computational complexity properties
of programs.

One of the properties that is notoriously difficult to enforce
is preventing data from being copied\footnote{We can delete the copy constructor in C++, but this is of course bypassable.}
or discarded without being used.\footnote{A linter can detect unused variables, but this does not enforce a guarantee.}
We will call variables that are \emph{non-copyable} \emph{affine}, and reserve the term \emph{linear} for variables
that must be used \emph{exactly once}, i.e.\ they may be neither duplicated nor discarded.
The affine requirement can be enforced by making the variable go out of scope just after it was used.
Affine variables are important in logspace computations, where copying large enough data
is simply not implementable. Linear variables, on the other hand, are reminiscent of
the requirement that every class construction must be matched by a corresponding destruction operation
on every execution path.
We will see an example of affinity enforced by linear types in Haskell in~\autoref{lst:haskell-linear-example}.

These ideas have been studied for decades in proof theory.
There, a proof system may or may not allow the so-called \emph{structural} rules:
weakening, contraction and exchange.
Proof systems that lack one of these rules are studied in the field of linear logic,
introduced in~\cite{GIRARD19871} with a clear intent to be used in computer science.
The semantics of linear logic has a natural interpretation in terms of \emph{resources}:
a proposition may be used exactly once, or at most once, rather than freely duplicated and discarded.
The concepts from linear logic have been carried over almost directly to \emph{linear type systems},
which, in (typically functional) programming languages, can control the ability to clone and discard data.

As it turns out, this level of control is also enough to limit the computational complexity of definable functions.
The class \complexity{FL} has been captured by a variant of affine logic in~\cite{4276584};
it was also studied in~\cite{10.1007/11874683_40},~\cite{DaLagoSchopp10} and~\cite{mazza:LIPIcs.CSL.2015.24}.
The class \complexity{FP} was characterized in~\cite{Leivant93}.
Most importantly for us, the programming language \texttt{IntML}
was introduced in~\cite{DALLAGO2016150}, which we will discuss in~\autoref{sec:intml}.

\begin{remark}
The formal bridge between linear logics and linear type systems is the
Curry-Howard correspondence,
also known as ``propositions as types'' or ``proofs as programs''.
The connections between logic, type systems and complexity theory are already well explored in the literature,
e.g.\ \cite{BenedettiPhd}.
We will not repeat these definitions here.
A good introduction to the Curry-Howard correspondence is the book by S\o{}rensen and Urzyczyn~\cite{10.5555/1197021}.
\end{remark}

\begin{remark}
Controlling the computational complexity of programs through the complexity of their specification
was already discussed in~\autoref{chap:descriptive-complexity}, where the control went through model theory.
Here, the restrictions are entirely in the world of proof theory, which is more directly connected to
computation than model theory.
\end{remark}

\begin{remark}[Support for linear types in mainstream programming languages]
Some mainstream programming languages offer some support for linear  types.
Haskell (GHC~$\ge 9.0.1$) supports a limited version of linear types.
In Rust, affine reasoning can be expressed through the ownership and borrowing mechanism.
Going less mainstream, Idris~2, F$^\star$ and Q$^\star$ (a quantum programming language)
also support different variants of linear types.
\end{remark}

\begin{rawlisting}
\begin{Verbatim}[commandchars=\\\{\}]
\PY{c+cm}{\PYZob{}\PYZhy{}}\PY{c+cm}{\PYZsh{} LANGUAGE LinearTypes \PYZsh{}}\PY{c+cm}{\PYZhy{}\PYZcb{}}\PY{+w}{ }\PY{c+c1}{\PYZhy{}\PYZhy{} compilation: `ghc Linear.hs`, ghc \PYZgt{}= 9.0.1}
\PY{k+kr}{module}\PY{+w}{ }\PY{n+nn}{Linear}\PY{+w}{ }\PY{k+kr}{where}
\PY{k+kr}{import}\PY{+w}{ }\PY{n+nn}{Prelude}

\PY{c+c1}{\PYZhy{}\PYZhy{} Define own bitstring type for ints, as operations from Prelude}
\PY{c+c1}{\PYZhy{}\PYZhy{} on Int are *not* linear and will not typecheck.}
\PY{k+kr}{data}\PY{+w}{ }\PY{k+kt}{Bit}\PY{+w}{  }\PY{o+ow}{=}\PY{+w}{ }\PY{k+kt}{Zero}\PY{+w}{ }\PY{o}{|}\PY{+w}{ }\PY{k+kt}{One}
\PY{k+kr}{data}\PY{+w}{ }\PY{k+kt}{Bits}\PY{+w}{ }\PY{o+ow}{=}\PY{+w}{ }\PY{k+kt}{Nil}\PY{+w}{ }\PY{o}{|}\PY{+w}{ }\PY{k+kt}{B0}\PY{+w}{ }\PY{k+kt}{Bits}\PY{+w}{ }\PY{o}{|}\PY{+w}{ }\PY{k+kt}{B1}\PY{+w}{ }\PY{k+kt}{Bits}\PY{+w}{  }\PY{c+c1}{\PYZhy{}\PYZhy{} Prepend 0 or 1 as the LSB.}
\PY{n+nf}{const\PYZus{}5}\PY{+w}{ }\PY{o+ow}{::}\PY{+w}{ }\PY{k+kt}{Bits}\PY{+w}{ }\PY{o+ow}{=}\PY{+w}{ }\PY{k+kt}{B1}\PY{+w}{ }\PY{p}{(}\PY{k+kt}{B0}\PY{+w}{ }\PY{p}{(}\PY{k+kt}{B1}\PY{+w}{ }\PY{k+kt}{Nil}\PY{p}{)}\PY{p}{)}
\PY{n+nf}{const\PYZus{}6}\PY{+w}{ }\PY{o+ow}{::}\PY{+w}{ }\PY{k+kt}{Bits}\PY{+w}{ }\PY{o+ow}{=}\PY{+w}{ }\PY{k+kt}{B0}\PY{+w}{ }\PY{p}{(}\PY{k+kt}{B1}\PY{+w}{ }\PY{p}{(}\PY{k+kt}{B1}\PY{+w}{ }\PY{k+kt}{Nil}\PY{p}{)}\PY{p}{)}

\PY{n+nf}{burn}\PY{+w}{ }\PY{o+ow}{::}\PY{+w}{ }\PY{k+kt}{Bits}\PY{+w}{ }\PY{o}{\PYZpc{}}\PY{l+m+mi}{1}\PY{o+ow}{\PYZhy{}\PYZgt{}}\PY{+w}{ }\PY{n+nb}{()}
\PY{n+nf}{burn}\PY{+w}{ }\PY{k+kt}{Nil}\PY{+w}{      }\PY{o+ow}{=}\PY{+w}{ }\PY{n+nb}{()}
\PY{n+nf}{burn}\PY{+w}{ }\PY{p}{(}\PY{k+kt}{B0}\PY{+w}{ }\PY{n}{xs}\PY{p}{)}\PY{+w}{  }\PY{o+ow}{=}\PY{+w}{ }\PY{n}{burn}\PY{+w}{ }\PY{n}{xs}
\PY{n+nf}{burn}\PY{+w}{ }\PY{p}{(}\PY{k+kt}{B1}\PY{+w}{ }\PY{n}{xs}\PY{p}{)}\PY{+w}{  }\PY{o+ow}{=}\PY{+w}{ }\PY{n}{burn}\PY{+w}{ }\PY{n}{xs}

\PY{n+nf}{evenBits}\PY{+w}{ }\PY{o+ow}{::}\PY{+w}{ }\PY{k+kt}{Bits}\PY{+w}{ }\PY{o}{\PYZpc{}}\PY{l+m+mi}{1}\PY{o+ow}{\PYZhy{}\PYZgt{}}\PY{+w}{ }\PY{k+kt}{Prelude}\PY{o}{.}\PY{k+kt}{Bool}
\PY{n+nf}{evenBits}\PY{+w}{ }\PY{k+kt}{Nil}\PY{+w}{      }\PY{o+ow}{=}\PY{+w}{ }\PY{k+kt}{Prelude}\PY{o}{.}\PY{k+kt}{True}
\PY{n+nf}{evenBits}\PY{+w}{ }\PY{p}{(}\PY{k+kt}{B0}\PY{+w}{ }\PY{n}{xs}\PY{p}{)}\PY{+w}{  }\PY{o+ow}{=}\PY{+w}{ }\PY{k+kr}{case}\PY{+w}{ }\PY{n}{burn}\PY{+w}{ }\PY{n}{xs}\PY{+w}{ }\PY{k+kr}{of}\PY{+w}{ }\PY{n+nb}{()}\PY{+w}{ }\PY{o+ow}{\PYZhy{}\PYZgt{}}\PY{+w}{ }\PY{k+kt}{Prelude}\PY{o}{.}\PY{k+kt}{True}
\PY{n+nf}{evenBits}\PY{+w}{ }\PY{p}{(}\PY{k+kt}{B1}\PY{+w}{ }\PY{n}{xs}\PY{p}{)}\PY{+w}{  }\PY{o+ow}{=}\PY{+w}{ }\PY{k+kr}{case}\PY{+w}{ }\PY{n}{burn}\PY{+w}{ }\PY{n}{xs}\PY{+w}{ }\PY{k+kr}{of}\PY{+w}{ }\PY{n+nb}{()}\PY{+w}{ }\PY{o+ow}{\PYZhy{}\PYZgt{}}\PY{+w}{ }\PY{k+kt}{Prelude}\PY{o}{.}\PY{k+kt}{False}

\PY{n+nf}{half}\PY{+w}{ }\PY{o+ow}{::}\PY{+w}{ }\PY{k+kt}{Bits}\PY{+w}{ }\PY{o}{\PYZpc{}}\PY{l+m+mi}{1}\PY{o+ow}{\PYZhy{}\PYZgt{}}\PY{+w}{ }\PY{k+kt}{Bits}
\PY{n+nf}{half}\PY{+w}{ }\PY{k+kt}{Nil}\PY{+w}{      }\PY{o+ow}{=}\PY{+w}{ }\PY{k+kt}{Nil}
\PY{n+nf}{half}\PY{+w}{ }\PY{p}{(}\PY{k+kt}{B0}\PY{+w}{ }\PY{n}{xs}\PY{p}{)}\PY{+w}{  }\PY{o+ow}{=}\PY{+w}{ }\PY{n}{xs}
\PY{n+nf}{half}\PY{+w}{ }\PY{p}{(}\PY{k+kt}{B1}\PY{+w}{ }\PY{n}{xs}\PY{p}{)}\PY{+w}{  }\PY{o+ow}{=}\PY{+w}{ }\PY{n}{xs}

\PY{n+nf}{plus2x1}\PY{+w}{ }\PY{o+ow}{::}\PY{+w}{ }\PY{k+kt}{Bits}\PY{+w}{ }\PY{o}{\PYZpc{}}\PY{l+m+mi}{1}\PY{o+ow}{\PYZhy{}\PYZgt{}}\PY{+w}{ }\PY{k+kt}{Bits}
\PY{n+nf}{plus2x1}\PY{+w}{ }\PY{n}{x}\PY{+w}{ }\PY{o+ow}{=}\PY{+w}{ }\PY{k+kt}{B1}\PY{+w}{ }\PY{n}{x}

\PY{n+nf}{branchConst}\PY{+w}{ }\PY{o+ow}{::}\PY{+w}{ }\PY{k+kt}{Bits}\PY{+w}{ }\PY{o}{\PYZpc{}}\PY{l+m+mi}{1}\PY{o+ow}{\PYZhy{}\PYZgt{}}\PY{+w}{ }\PY{k+kt}{Bits}
\PY{n+nf}{branchConst}\PY{+w}{ }\PY{n}{x}\PY{+w}{ }\PY{o+ow}{=}
\PY{+w}{  }\PY{k+kr}{if}\PY{+w}{ }\PY{n}{evenBits}\PY{+w}{ }\PY{n}{x}
\PY{+w}{    }\PY{k+kr}{then}\PY{+w}{ }\PY{n}{half}\PY{+w}{ }\PY{n}{const\PYZus{}5}
\PY{+w}{    }\PY{k+kr}{else}\PY{+w}{ }\PY{n}{plus2x1}\PY{+w}{ }\PY{n}{const\PYZus{}6}

\PY{c+c1}{\PYZhy{}\PYZhy{} collatzBad2 :: Bits \PYZpc{}1\PYZhy{}\PYZgt{} Bits}
\PY{c+c1}{\PYZhy{}\PYZhy{} collatzBad2 x =}
\PY{c+c1}{\PYZhy{}\PYZhy{}   if evenBits x}
\PY{c+c1}{\PYZhy{}\PYZhy{}     then half x       \PYZhy{}\PYZhy{} ERROR: x already consumed by \PYZsq{}evenBits x\PYZsq{}}
\PY{c+c1}{\PYZhy{}\PYZhy{}     else plus2x1 x    \PYZhy{}\PYZhy{} ERROR: x already consumed by \PYZsq{}evenBits x\PYZsq{}}
\end{Verbatim}

\caption{Example of linear types in Haskell}\label{lst:haskell-linear-example}
\end{rawlisting}
\begin{remark}\label{remark:haskell-linear}
  In~\autoref{lst:haskell-linear-example}, the type system prevents us from using a linear argument
  more than once. The function \texttt{collatzBad2} fails to type-check precisely because the
  argument \texttt{x} would be consumed by \texttt{evenBits x} and then used again in the branches.
This effect makes most of the standard algorithms not transferrable to this formalism.  
\end{remark}

\subsection{IntML}\label{sec:intml}
In 2013, Dal Lago and Sch\"opp introduced \texttt{IntML}, a functional language with a linear type
system that characterizes \complexity{FL}~\cite{DALLAGO2016150}.
An implementation of \texttt{IntML} is available on
GitHub.\footnote{\url{https://github.com/uelis/IntML}. Following private communication with the authors, 
a permissive license was added to the repository, as it was not included originally.}
To the best of our knowledge, it remains the only language within the linear-logic branch of ICC that has both
a working implementation and some potential for (academic) practical use.

From the point of view of this thesis, however, linear-logic-based approaches --- including \texttt{IntML} ---
run into the \emph{same} issue of intensional vs.\ extensional expressive power discussed in~\autoref{subsec:intensional}.
These systems characterize classes such as \complexity{FL} and \complexity{FP} \emph{extensionally}.
The characterizations capture the right functions, but not the usual \emph{algorithmic techniques}
used to implement them. \texttt{IntML} looks like a very good starting point for
a practical programming language. However, it would still be very hard to use it
to certify the complexity of standard algorithms.

In this thesis, we will not pursue this line further as a practical basis for certifying
the complexity.
