\section{Decisional Turing machine complexity classes}\label{sec:preliminaries-turing}

In this section we introduce standard complexity classes such as 
\(\complexity{L}\), \(\complexity{P}\) and \(\complexity{NP}\). It is important to note that these 
definitions only talk about the complexity of solving \emph{decisional} problems,
i.e.\ computing functions \(f : \{0,1\}^\ast \to \{0,1\}\).

\begin{definition}[Computing a function~{\cites[Definition~1.3]{10.5555/1540612}[Definition~1.4]{DRAFT10.5555/1540612}}]
Let~\(f : \{0,1\}^{\ast} \to \{0,1\}^{\ast}\) and \(T : \mathbb{N} \to \mathbb{N}\) be functions, and let
\(M\) be a Turing machine.  
\begin{enumerate}
  \item We say that \(M\) \emph{computes} \(f\) if, for every input
        \(x \in \{0,1\}^{\ast}\), when \(M\) is started in its initial configuration
        on input \(x\), it eventually halts with the string \(f(x)\) written on its
        output tape.

  \item We say that \(M\) computes \(f\) in time \(T(n)\) if, for every
        input \(x\), the computation of \(M\) on input \(x\) halts within
        at most \(T(|x|)\) steps.
\end{enumerate}
\end{definition}

\begin{notation}
    We say that a machine \emph{decides} a language \(L \subseteq \{0, 1\}^\ast\) iff it computes
    the function \(f_L: \{0, 1\}^\ast \rightarrow \{0, 1\}\), where \(f_L(x) = 1 \iff x \in L\).
\end{notation}

\begin{definition}[Time complexity~{\cites[Definition~1.19]{DRAFT10.5555/1540612}[Definition~1.12]{10.5555/1540612}}]\label{def:turing-dtime}
\hfill\\
Let~\(T~:~\mathbb{N} \to \mathbb{N}\) be a function.
We define \(\mathsf{DTIME}(T(n))\) to be the class of all Boolean
functions that are computable by some deterministic
Turing machine running in at most \(c \cdot T(n)\) steps on every input of
length \(n\), for some constant \(c > 0\).
\end{definition}

\begin{definition}[Space complexity~{\cites[Definition~4.1]{DRAFT10.5555/1540612}[Definition~4.1]{10.5555/1540612}}]\label{def:turing-dspace}
\hfill\\
Let~\(S~:~\mathbb{N} \to \mathbb{N}\) be a function and let
\(L \subseteq \{0,1\}^{\ast}\) be a language.  
% \begin{enumerate}
%   \item 
  We say that \(L \in \complexity{SPACE}(S(n))\) iff there exist a constant
        \(c > 0\) and a deterministic Turing machine \(M\) deciding \(L\) such that,
        on every input of length \(n\), the machine \(M\) visits at most
        \(c \cdot S(n)\) distinct cells on its read--write work tapes
        (the input tape is read-only and does not count toward the space
        bound).

%   \item Likewise, \(L \in \complexity{NSPACE}(S(n))\) if there exist a constant
%         \(c > 0\) and a nondeterministic Turing machine \(M\) deciding \(L\) such
%         that, for every input of length \(n\) and for every computation branch,
%         \(M\) uses no more than \(c \cdot S(n)\) nonblank work-tape cells.
% \end{enumerate}
\end{definition}


\begin{definition}[Logarithmic space~{\cites[Definition~4.5]{10.5555/1540612}[Definition~4.5]{DRAFT10.5555/1540612}}]\label{def:complexity-l}
\[
    \complexity{L} = \complexity{DSPACE}(\log n).
\]
\end{definition}


\begin{definition}[Polynomial time~{\cites[Definition~1.13]{10.5555/1540612}[Definition~1.20]{DRAFT10.5555/1540612}}]\label{def:complexity-p}
\[
    \complexity{P} = \bigcup_{c \ge 1} \complexity{DTIME}(n^c).
\]
\end{definition}



\begin{definition}[The class \texorpdfstring{\complexity{NP}}{NP}{~\cites[Definition~2.1]{10.5555/1540612}[Definition~2.1]{DRAFT10.5555/1540612}}]\label{def:complexity-np}
\hfill\\
A~language~\(L \subseteq \{0,1\}^{\ast}\) belongs to \(\complexity{NP}\) if there exist
\begin{enumerate}
  \item a polynomial \(p : \mathbb{N} \to \mathbb{N}\) (bounding the length of a certificate), and
  \item a deterministic polynomial-time Turing machine \(M\)
        (called a \emph{verifier} for \(L\)),
\end{enumerate}
such that for every input string \(x \in \{0,1\}^{\ast}\),
\[
  x \in L
  \;\Longleftrightarrow\;
  \exists\,u \in \{0,1\}^{\,p(|x|)} \ \text{with}\ M(x,u) = 1.
\]

Whenever \(x \in L\) and a string \(u \in \{0,1\}^{p(|x|)}\) satisfies \(M(x,u)=1\),
the string \(u\) is called a \emph{certificate} (or \emph{witness}) for \(x\)
with respect to the language \(L\) and the verifier~\(M\).
\end{definition}