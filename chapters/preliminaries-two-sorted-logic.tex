\section{Two-sorted first-order logic}\label{sec:defs-two-sorted}
% Sec. 2.2: Akapit 1 sugeruje, jakbyśmy nic nie mówili o logice 2-sortowej, a zaraz po nim miało nastąpić coś
% innego. Przeformułuj, aby był jaśniejszy (w stylu "pomijamy rzeczy podobne, a wypiszemy tylko różnice").

% Ponadto rozdziały 2.1 i 2.2 mają zupełnie inny charakter. W 2.1 jest logika nad dowolną sygnaturą, a w 2.2
% konkretnie liczby i napisy. Pasuje coś powiedzieć na początku 2.2 - że choć logikę 2-sortową można ogólnie
% definiować dla dowolnych sortów, to my piszemy tylko o tych konkretnych, liczby i napisy.

% I też właśnie trzeba konkretnie napisać, że sorty to liczby i napisy, przed Def. 2.2.1 - bo teraz w środku
% tej definicji nagle się pojawia coś o napisach i liczbach i nie wiadomo o co chodzi.



Two-sorted first-order logic extends the single-sorted setting in a routine way, so we only record
the additions that we use later.  A systematic presentation can be found
in~\cites[Section~4B]{CookNguyenDraft}[Section~IV.2]{Cook_Nguyen_2010}.
In principle one could work with arbitrary pairs of sorts, but in this thesis we instantiate the
framework to the familiar number sort (ranging over $\mathbb{N}$) and string sort (ranging over
finite binary strings).  The goal of this section is therefore to emphasise what changes when we
move from the definitions of \autoref{sec:defs-single-sorted} to this concrete two-sorted
setting.

\begin{definition}[Two-sorted first-order vocabularies]\label{def:two-sorted-vocabulary}
A \emph{two-sorted first-order vocabulary} (often abbreviated simply as a
two-sorted language) \(\mathcal{L}\) consists of collections of function and predicate
symbols, much like an ordinary single-sorted vocabulary, but now the
symbols may accept arguments of either of the two sorts.  Moreover, the
function symbols come in two varieties:
\begin{enumerate}
  \item \emph{number-valued} function symbols, whose outputs lie in the
        number sort; and
  \item \emph{string-valued} function symbols, whose outputs lie in the
        string sort.
\end{enumerate}

For any pair \(n,m \in \mathbb{N}\), the vocabulary contains:
\begin{enumerate}
  \item a set of \((n,m)\)-ary number-function symbols,
  \item a set of \((n,m)\)-ary string-function symbols, and
  \item a set of \((n,m)\)-ary predicate symbols.
\end{enumerate}
A \((0,0)\)-ary function symbol is simply a constant symbol, which may be
either a constant of the number sort or a constant of the string sort.

We use \(f,g,h,\dots\) as metavariables for number-valued function symbols,
\(F,G,H,\dots\) for string-function symbols, and \(P,Q,R,\dots\) for predicate
symbols.
\end{definition}

% TODO: TUTAJ DOPISAC REMARK ZE TO BEDZIE JAKO AKSJOMAT!
% Okolice Def. 2.2.2: Implicite zakładamy związek między |X| oraz t\in X, więc nie dowolna struktura tylko
% coś w stylu że nie zachodzi t\in X dla liczb >= |X|.

\begin{definition}[The language \(\mathcal{L}^{2}_{A}\)]\label{def:L2A}
As an example, consider the following two-sorted extension of the
arithmetical language \(\mathcal{L}_{A}\)~(\autoref{def:LA}):
\[
  \mathcal{L}^{2}_{A} \;=\; [\,0,\ 1,\ +,\ \cdot,\ |\cdot| \;;\ =_{1},\ =_{2},\ \le,\ \in\,].
\]

Here the symbols \(0,1,+,\cdot,=_{1},\le\) are
symbols of \(\mathcal{L}_{A}\) (with \(=_{1}\) corresponding to the usual equality of
numbers).  
The symbol \(|X|\) is a number-valued function symbol giving the
length of a string \(X\).  
The binary predicate \(\in\) relates a number and a string and is used to
express membership: intuitively, \(i \in X\) means that the \(i\)-th bit of
the string \(X\) is \(1\).  
The symbol \(=_{2}\) denotes equality between objects of the second sort.

For convenience, when \(t\) is a number term, we abbreviate
\[
  X(t) \;\coloneqq\; t \in X.
\]
Thus \(X(i)\) plays the role of the \(i\)-th bit of the binary string \(X\).

In \(\mathcal{L}^{2}_{A}\), the symbols \(+\) and \(\cdot\) each have arity \((2,0)\); the
length function \(|\cdot|\) has arity \((0,1)\); and the predicate
\(\in\) has arity \((1,1)\).
\end{definition}



\begin{notation}[Bounded formulas]\label{def:bounded-formulas}
Let \(\mathcal{L}\) be a two-sorted vocabulary.  
If \(x\) is a number variable and \(X\) a string variable that do not occur
in the \(\mathcal{L}\)-number term \(t\), we use the following abbreviations:
\begin{align*}
  \exists x \le t\ldotp \varphi
    &\;\;\text{stands for}\;\;
      \exists x\ldotp (x \le t \;\wedge\; \varphi), \\[2pt]
  \forall x \le t\ldotp \varphi
    &\;\;\text{stands for}\;\;
      \forall x\ldotp (x \le t \;\to\; \varphi), \\[2pt]
  \exists X \le t\ldotp \varphi
    &\;\;\text{stands for}\;\;
      \exists X\ldotp (\,|X| \le t \;\wedge\; \varphi), \\[2pt]
  \forall X \le t\ldotp \varphi
    &\;\;\text{stands for}\;\;
      \forall X\ldotp (\,|X| \le t \;\to\; \varphi).
\end{align*}

A quantifier appearing in one of these forms is called \emph{bounded},
and a \emph{bounded formula} is a formula in which every quantifier is
bounded.
\end{notation}


\paragraph{Notation.}
The expression
\(\exists \vec{x} \le \vec{t}\ldotp \varphi\)
abbreviates a block of bounded number quantifiers
\(\exists x_1 \le t_1\ldotp \cdots \exists x_k \le t_k\ldotp \varphi\)
for some \(k\), where no variable \(x_i\) occurs in any term \(t_j\)
(even when \(i < j\)).  
The same convention applies to \(\forall \vec{x} \le \vec{t}\), \(\exists \vec{X} \le \vec{t}\), and
\(\forall \vec{X} \le \vec{t}\).


\begin{definition}[The \texorpdfstring{$\Sigma^1_{1}(\mathcal{L})$, $\Sigma^B_{i}(\mathcal{L})$,
  and $\Pi^B_{i}(\mathcal{L})$}{Σ¹₁(L), Σᵢᴮ(L), Πᵢᴮ(L)} formulas]\label{def:SigmaB-PiB-hierarchy}
Let \(\mathcal{L} \supseteq \mathcal{L}^{2}_{A}\) be a two-sorted vocabulary.  
% Druga potencjalna modyfikacja to dopuszczenie kwantyfikatorów po liczbach między tymi po napisach.
% Obecnie w ogóle nie może być kwantyfikatora po napisie wewnątrz takiego po liczbie. To istotne?
% Warto to napisać wprost jeszcze raz pod spodem.
\begin{enumerate}
  \item The class \(\Sigma^{B}_{0}(\mathcal{L}) = \Pi^{B}_{0}(\mathcal{L})\) consists of all
        \(\mathcal{L}\)-formulas whose only quantifiers are \emph{bounded number quantifiers}
        (string variables may occur free).

  \item For \(i \ge 0\), the class \(\Sigma^{B}_{i+1}(\mathcal{L})\) (resp.\ \(\Pi^{B}_{i+1}(\mathcal{L})\))
        consists of formulas of the form
        \[
          \exists \vec{X} \le \vec{t}\ldotp \varphi(\vec{X})
          \quad\text{(resp.\ }\forall \vec{X} \le \vec{t}\ldotp \varphi(\vec{X})\text{)},
        \]
        where:
        \begin{enumerate}
            \item \(\vec{X}\) is a vector of string variables,
            % TODO: Def. 2.2.3 (ii) - "not involving variables from X" -
            % nasuwa się pytanie co by było, gdybyśmy dopuścili? Nic by się nie zmieniło.
            % Więc warto gdzieś tam dopisać, że to tylko założenie dla porządku, które nic nie zmienia.
            \item \(\vec{t}\) is a vector of \(\mathcal{L}^{2}_{A}\)-terms not involving variables from \(\vec{X}\),
            \item \(\varphi\) is a \(\Pi^{B}_{i}(\mathcal{L})\) formula  
                  (resp.\ a \(\Sigma^{B}_{i}(\mathcal{L})\) formula).
        \end{enumerate}

  \item A \(\Sigma^{1}_{1}(\mathcal{L})\) formula is a formula of the form
        \(
          \exists \vec{X}\ldotp \varphi,
        \)
        where \(\vec{X}\) is a vector of zero or more string variables and
        \(\varphi\) is a \(\Sigma^{B}_{0}(\mathcal{L})\) formula.
\end{enumerate}

We usually write \(\Sigma^{B}_{i}\) for \(\Sigma^{B}_{i}(\mathcal{L}^{2}_{A})\) and
\(\Pi^{B}_{i}\) for \(\Pi^{B}_{i}(\mathcal{L}^{2}_{A})\).
\end{definition}


\begin{remark}
The formalism described above coincides with the usual weak monadic second-order logic (WMSO) on
words.  We phrase it as a two-sorted first-order system only to match the presentation
in~\cite{CookNguyenDraft,Cook_Nguyen_2010}, but in practice we freely identify it with WMSO and use
that terminology when convenient, following the convention in~\cite{COOK2003193}.
% Ostatnie zdanie rozdziału, o MSO: w MSO na słowach skończonych też kwantyfikujemy po zbiorach skończonych.
% Więc oprócz tego, że zbiory są skończone warto napisać, że "pomimo iż liczby są z nieskończonego zbioru
% liczb naturalnych" (w sumie też nigdzie nie jest napisane, że to liczby naturalne, a nie całkowite,
% wymierne, czy jeszcze jakieś inne).
% W tym kontekście używa się często nazwy WMSO (weak MSO) - czy ona nie byłaby odpowiednia?
\end{remark}
