\chapter{Circuits}\label{chap:circuits}
In this chapter, we focus on the descriptions of uniformity conditions used for
families of circuits.




We need \complexity{FO}-uniformity for~\autoref{sec:fo-eq-ac0}.

\(U_E\) and \(U_{E^\ast}\)-uniformity is defined in terms of \emph{diret connection language}
and \emph{extended connection language} in~\cites[Definition~2.24,~Definition~2.43~]{10.5555/520668}.
There is a very thorough overview of uniformity conditions below NC1 in ~\cite{MIXBARRINGTON1990274}.





% barrington:
%We define a log-time Turing machine to have a read-only input tape of length n,
% a constant number of read-write work tapes of total length O(log n), and a read-
% write input address tape of length log n. On a given time step the machine has
% access to the bit of the input tape denoted by the contents of the address tape (or
% to the fact that there is no such bit, if the address tape holds too large a number).
% We will assume (without loss of generality) that the machine always takes the same
% amount of time on inputs of a given length (this is because some of the work tape
% can always used as a clock). The following lemma summarizes some useful
% capabilities of such a machine.
dlogtime uniformity:\cite[Section~6]{MIXBARRINGTON1990274}:
deterministic log-time tm can decide formula langugae: {<c, i, y>: |y|=n and ith char of nth formula is c}
very important:


The class dlogtime-uniform nc1 is alogtime is nc1-uniform nc1:\cite[Lemma~6.2]{MIXBARRINGTON1990274}.


\section{uniform \complexityi{NC}{1} is \complexity{ALOGTIME}}\label{sec:uniform-nc1}
For low uniformities: MIXBARRINGTON1990274.
In the below, \(R\) is said to be \(\complexity{NC}^1\)-reducible to \(S\) (written \(R \leqslant S\))
iff there is a \(\compUeAst\)-uniform family of circuits with oracle nodes for \(S\).
In the below, \(\complexity{FL}^*\) consists of all problems \(R\) such that \(R \leqslant S\)
for some \(S\) in \(FL\)~\cite[Page~8]{COOK19852}.


Bibliographical remark: we refer to Chapter 2 ``A Catalog of Complexity Classes'' by David S. Johnson~\cite{10.5555/114872.114874},
which appeared in January 1991 in ``Handbook of theoretical computer science (vol. A): algorithms and complexity'',
edited by Jan van Leeuwen~\cite{10.5555/114872}.

For probably the first published recognition of the widespread inconsistency of decisional vs functional
complexity classes in the literature, with examples of inconsistent places see~\cite[Page~131]{10.5555/114872}.

For an overview of the different uniformity conditions in a state-of-the-art paper, with
arguments about why \complexity{DLOGTIME} uniformity is the most reasonable to consider, please see\cite{HESSE2002695}.
% KEEP IT logspace-uniform here, but remark that FO-uniformity will be required later when we define FO-queries
% immerman:
% Definition 5.16 (Uniform) Let C be a sequence of circuits as in Equation (5.14).
% Let l' E {l'eo l'thc} be the vocabulary of circuits or threshold circuits. Let / :
% STRUC[l's] ~ STRUC[l'] be a query such that for all n E N, /(0") = Cn. That
% is, on input a string of n zero's the query produces circuit n. If / E FO, then C is
% 80 5. Parallelism
% afirst-order uniform sequence of circuits. Similarly, if I E L, then Cis logspace
% uniform. If I E P, then C is polynomial-time uniform, and so on. 

The below is from:~\cite{RUZZO1981365}.
\begin{definition}[\complexity{DLOGTIME}-uniformity]

\end{definition}

\subsection{\complexity{DLOGTIME}-uniformity}\label{subsec:dlogtime-uniformity}


\begin{enumerate}
\item TODO:\@ Work out the example of a \complexity{DLOGTIME} reduction showing that tree isomorphism for string-represented trees is $\complexityi{NC}{1}$-complete~\cite{694595}.
\end{enumerate}



uniform FAC0 does NOT admit a nice circuit characterization. FAC0/poly
is standard (see LogicalFoundations Definition V.2.3) circuits with output.
But the notion of uniformity doesn't generalize to circuits with outputs!
Our only hope is in ``polynomial number of AC0-decidable outputs''!


% TODO: Clarify what notion of uniformity should be used for $\complexityi{NC}{0}$, given that \complexity{DLOGTIME} Turing machines can do things that cannot be done by
% any $\complexityi{NC}{0}$ circuit.
% \subsection{\complexityi{NC}{1}}
% \begin{enumerate}
% \item TODO: Present the argument that division is in \complexity{DLOGTIME}-uniform $\complexityi{NC}{1}$~\cite{ITA_2001__35_3_259_0}.
% \item TODO: use \complexity{DLOGTIME} reductions for $\complexityi{NC}{1}$.
% \item TODO: Clarify how $U_{E^*}$ reductions relate to $\complexityi{NC}{1}$ and why $U_{E^*}$-uniform $\complexityi{NC}{1}$ equals \complexity{ALOGTIME}~\cite{RUZZO1981365}.
% \end{enumerate}
% TODO: Confirm that this class is contained in $\complexityi{AC}{i}$.
