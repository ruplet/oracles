% Prehistory of Implicit Computational Complexity (pre-1992)
% This is a skimmable catalog organized by "why this is not yet fully ICC".
% Uses \cite{...} keys defined in the separate .bib file.

\section{Prehistory (Pre-1992): What Came Before Fully Implicit Characterisations}


subsection:
BLOOP, Hostadter


\subsection*{A. Subrecursive hierarchies \& loop programs (explicit growth controls)}
\textbf{Why not full ICC?} These lines of work stratify total functions by growth-rate
(e.g., Grzegorczyk hierarchy) or by syntactic loop depth; they do not align cleanly with
natural complexity classes such as \(\mathsf{P}\) or \(\mathsf{L}\), and typically rely on overt
bounds or syntactic depth measures rather than a machine-free, class-exact discipline.

\noindent\textbf{Representative papers:}
\begin{itemize}
  \item Grzegorczyk, \emph{Some Classes of Recursive Functions} \cite{Grzegorczyk1953}.
  \item Ritchie, \emph{Classes of Predictably Computable Functions} \cite{Ritchie1963}.
  \item Meyer--Ritchie, \emph{The Complexity of Loop Programs} \cite{MeyerRitchie1967}.
\end{itemize}

\subsection*{B. Machine-based resource measures (time/space) and axioms for complexity}
\textbf{Why not full ICC?} These works found the field and define machine measures, but
the characterisations hinge on external (explicit) resource bounds, not on intrinsic
program/logic discipline that \emph{guarantees} a class without naming time/space.

\noindent\textbf{Representative papers:}
\begin{itemize}
  \item Hartmanis--Stearns, \emph{On the Computational Complexity of Algorithms} \cite{HartmanisStearns1965}.
  \item Blum, \emph{A Machine-Independent Theory of the Complexity of Recursive Functions} \cite{Blum1967}.
\end{itemize}

\subsection*{C. Function algebras with \emph{explicit} polynomial bounds}
\textbf{Why not full ICC?} Cobham’s and follow-ups capture \(\mathsf{FP}\) but do so by \emph{building in}
polynomial-bounded schemata or growth predicates; the bound is explicit, not implicit.

\noindent\textbf{Representative papers:}
\begin{itemize}
  \item Cobham, \emph{The Intrinsic Computational Difficulty of Functions} \cite{Cobham1964}.
  \item Gurevich, \emph{Algebras of Feasible Functions} \cite{Gurevich1983}.
\end{itemize}

\subsection*{D. Descriptive complexity (relations, often with order/fixed-point)}
\textbf{Why not full ICC?} These give \emph{logical} characterisations of \emph{relations} (decision problems);
they typically require ordered structures and capture classes at the \emph{relational} level.
They do not, by themselves, give a function algebra/typing discipline that defines
\emph{functions} without resource mentions.

\noindent\textbf{Representative papers:}
\begin{itemize}
  \item Fagin, \emph{Generalized First-Order Spectra and Polynomial-Time Recognizable Sets} \cite{Fagin1974}.
  \item Vardi, \emph{The Complexity of Relational Query Languages} \cite{Vardi1982}.
  \item Immerman, \emph{Relational Queries Computable in Polynomial Time} \cite{Immerman1986}.
  \item Immerman, \emph{Languages That Capture Complexity Classes} \cite{Immerman1987}.
\end{itemize}

\subsection*{E. Early bounded arithmetic \& PV (polytime ``built in'')}
\textbf{Why not full ICC?} PV and Buss’s bounded arithmetics are foundational for feasible
reasoning, but they \emph{bake in} polytime via function symbols/axioms and proof theory,
rather than giving a syntax/typing or recursion discipline that enforces \(\mathsf{P}\) without
naming time.

\noindent\textbf{Representative papers/books:}
\begin{itemize}
  \item Cook, \emph{Feasibly Constructive Proofs and the Propositional Calculus (Preliminary Version)} \cite{Cook1975}.
  \item Buss, \emph{Bounded Arithmetic} \cite{Buss1986}.
\end{itemize}

\subsection*{F. Early logspace function characterisations (explicit/logical constraints)}
\textbf{Why not full ICC?} Early \(\mathsf{L}\)-oriented work characterises low space using explicit resource
bounds or logics over relations; again, not yet an implicit \emph{functional} discipline.

\noindent\textbf{Representative papers:}
\begin{itemize}
  \item Lind--Meyer, \emph{A Characterization of Log-space Computable Functions} \cite{LindMeyer1973}.
  \item Lind, \emph{Computing in Logarithmic Space} (tech memo) \cite{Lind1974}.
\end{itemize}

\subsection*{G. Circuit-depth algebras/logics (toward low-level uniformity)}
\textbf{Why not full ICC?} Elegant and close to implicit viewpoints, but generally target relations
(or need BIT/order/uniformity assumptions) and still stop short of a pure function-algebra
or type-theoretic ICC account for \(\mathsf{NC}\)/\(\mathsf{NC}^1\).

\noindent\textbf{Representative papers:}
\begin{itemize}
  \item Compton--Laflamme, \emph{An Algebra and a Logic for \(\mathsf{NC}^{1}\)} \cite{ComptonLaflamme1990}.
  \item Allen, \emph{Arithmetizing Uniform \(\mathsf{NC}\)} \cite{Allen1991}.
\end{itemize}

% End of prehistory catalog
