\section{Linear types}\label{sec:linear-types}
The type systems of most popular programming languages are fairly weak:
they do not provide direct support
for verifying high-level correctness or complexity properties.
Functional programming languages with strong type systems exist, allowing us to
verify quite useful correctness properties --- one example is the language Haskell.
Going yet
stronger, Lean and Rocq are at the same time programming languages and proof assistants, i.e.\
their type systems allow us to prove even very abstract mathematical properties of the functions they define.
However, none of the mainstream languages allows us to enforce useful computational complexity properties
of programs.

One of the properties that is notoriously difficult to enforce
is preventing data from being copied\footnote{We can delete the copy constructor in C++, but this is of course bypassable.}
or discarded without being used.\footnote{A linter can detect unused variables, but this does not enforce a guarantee.}
We will call variables that are \emph{non-copyable} \emph{affine}, and reserve the term \emph{linear} for variables
that must be used \emph{exactly once}, i.e.\ they may be neither duplicated nor discarded.
The affine requirement can be enforced by making the variable go out of scope just after it was used.
Affine variables are important in logspace computations, where copying large enough data
is simply not implementable. Linear variables, on the other hand, are reminiscent of
the requirement that every class construction must be matched by a corresponding destruction operation
on every execution path.
We will see an example of affinity enforced by linear types in Haskell in~\autoref{lst:haskell-linear-example}.

These ideas have been studied for decades in proof theory.
There, a proof system may or may not allow the so-called \emph{structural} rules:
weakening, contraction and exchange.
Proof systems that lack one of these rules are studied in the field of linear logic,
introduced in~\cite{GIRARD19871} with a clear intent to be used in computer science.
The semantics of linear logic has a natural interpretation in terms of \emph{resources}:
a proposition may be used exactly once, or at most once, rather than freely duplicated and discarded.
The concepts from linear logic have been carried over almost directly to \emph{linear type systems},
which, in (typically functional) programming languages, can control the ability to clone and discard data.

As it turns out, this level of control is also enough to limit the computational complexity of definable functions.
The class \complexity{FL} has been captured by a variant of affine logic in~\cite{4276584};
it was also studied in~\cite{10.1007/11874683_40},~\cite{DaLagoSchopp10} and~\cite{mazza:LIPIcs.CSL.2015.24}.
The class \complexity{FP} was characterized in~\cite{Leivant93}.
Most importantly for us, the programming language \texttt{IntML}
was introduced in~\cite{DALLAGO2016150}, which we will discuss in~\autoref{sec:intml}.

\begin{remark}
The formal bridge between linear logics and linear type systems is the
Curry-Howard correspondence,
also known as ``propositions as types'' or ``proofs as programs''.
The connections between logic, type systems and complexity theory are already well explored in the literature,
e.g.\ \cite{BenedettiPhd}.
We will not repeat these definitions here.
A good introduction to the Curry-Howard correspondence is the book by S\o{}rensen and Urzyczyn~\cite{10.5555/1197021}.
\end{remark}

\begin{remark}
Controlling the computational complexity of programs through the complexity of their specification
was already discussed in~\autoref{chap:descriptive-complexity}, where the control went through model theory.
Here, the restrictions are entirely in the world of proof theory, which is more directly connected to
computation than model theory.
\end{remark}

\begin{remark}[Support for linear types in mainstream programming languages]
Some mainstream programming languages offer some support for linear  types.
Haskell (GHC~$\ge 9.0.1$) supports a limited version of linear types.
In Rust, affine reasoning can be expressed through the ownership and borrowing mechanism.
Going less mainstream, Idris~2, F$^\star$ and Q$^\star$ (a quantum programming language)
also support different variants of linear types.
\end{remark}

\begin{rawlisting}
\begin{Verbatim}[commandchars=\\\{\}]
\PY{c+cm}{\PYZob{}\PYZhy{}}\PY{c+cm}{\PYZsh{} LANGUAGE LinearTypes \PYZsh{}}\PY{c+cm}{\PYZhy{}\PYZcb{}}\PY{+w}{ }\PY{c+c1}{\PYZhy{}\PYZhy{} compilation: `ghc Linear.hs`, ghc \PYZgt{}= 9.0.1}
\PY{k+kr}{module}\PY{+w}{ }\PY{n+nn}{Linear}\PY{+w}{ }\PY{k+kr}{where}
\PY{k+kr}{import}\PY{+w}{ }\PY{n+nn}{Prelude}

\PY{c+c1}{\PYZhy{}\PYZhy{} Define own bitstring type for ints, as operations from Prelude}
\PY{c+c1}{\PYZhy{}\PYZhy{} on Int are *not* linear and will not typecheck.}
\PY{k+kr}{data}\PY{+w}{ }\PY{k+kt}{Bit}\PY{+w}{  }\PY{o+ow}{=}\PY{+w}{ }\PY{k+kt}{Zero}\PY{+w}{ }\PY{o}{|}\PY{+w}{ }\PY{k+kt}{One}
\PY{k+kr}{data}\PY{+w}{ }\PY{k+kt}{Bits}\PY{+w}{ }\PY{o+ow}{=}\PY{+w}{ }\PY{k+kt}{Nil}\PY{+w}{ }\PY{o}{|}\PY{+w}{ }\PY{k+kt}{B0}\PY{+w}{ }\PY{k+kt}{Bits}\PY{+w}{ }\PY{o}{|}\PY{+w}{ }\PY{k+kt}{B1}\PY{+w}{ }\PY{k+kt}{Bits}\PY{+w}{  }\PY{c+c1}{\PYZhy{}\PYZhy{} Prepend 0 or 1 as the LSB.}
\PY{n+nf}{const\PYZus{}5}\PY{+w}{ }\PY{o+ow}{::}\PY{+w}{ }\PY{k+kt}{Bits}\PY{+w}{ }\PY{o+ow}{=}\PY{+w}{ }\PY{k+kt}{B1}\PY{+w}{ }\PY{p}{(}\PY{k+kt}{B0}\PY{+w}{ }\PY{p}{(}\PY{k+kt}{B1}\PY{+w}{ }\PY{k+kt}{Nil}\PY{p}{)}\PY{p}{)}
\PY{n+nf}{const\PYZus{}6}\PY{+w}{ }\PY{o+ow}{::}\PY{+w}{ }\PY{k+kt}{Bits}\PY{+w}{ }\PY{o+ow}{=}\PY{+w}{ }\PY{k+kt}{B0}\PY{+w}{ }\PY{p}{(}\PY{k+kt}{B1}\PY{+w}{ }\PY{p}{(}\PY{k+kt}{B1}\PY{+w}{ }\PY{k+kt}{Nil}\PY{p}{)}\PY{p}{)}

\PY{n+nf}{burn}\PY{+w}{ }\PY{o+ow}{::}\PY{+w}{ }\PY{k+kt}{Bits}\PY{+w}{ }\PY{o}{\PYZpc{}}\PY{l+m+mi}{1}\PY{o+ow}{\PYZhy{}\PYZgt{}}\PY{+w}{ }\PY{n+nb}{()}
\PY{n+nf}{burn}\PY{+w}{ }\PY{k+kt}{Nil}\PY{+w}{      }\PY{o+ow}{=}\PY{+w}{ }\PY{n+nb}{()}
\PY{n+nf}{burn}\PY{+w}{ }\PY{p}{(}\PY{k+kt}{B0}\PY{+w}{ }\PY{n}{xs}\PY{p}{)}\PY{+w}{  }\PY{o+ow}{=}\PY{+w}{ }\PY{n}{burn}\PY{+w}{ }\PY{n}{xs}
\PY{n+nf}{burn}\PY{+w}{ }\PY{p}{(}\PY{k+kt}{B1}\PY{+w}{ }\PY{n}{xs}\PY{p}{)}\PY{+w}{  }\PY{o+ow}{=}\PY{+w}{ }\PY{n}{burn}\PY{+w}{ }\PY{n}{xs}

\PY{n+nf}{evenBits}\PY{+w}{ }\PY{o+ow}{::}\PY{+w}{ }\PY{k+kt}{Bits}\PY{+w}{ }\PY{o}{\PYZpc{}}\PY{l+m+mi}{1}\PY{o+ow}{\PYZhy{}\PYZgt{}}\PY{+w}{ }\PY{k+kt}{Prelude}\PY{o}{.}\PY{k+kt}{Bool}
\PY{n+nf}{evenBits}\PY{+w}{ }\PY{k+kt}{Nil}\PY{+w}{      }\PY{o+ow}{=}\PY{+w}{ }\PY{k+kt}{Prelude}\PY{o}{.}\PY{k+kt}{True}
\PY{n+nf}{evenBits}\PY{+w}{ }\PY{p}{(}\PY{k+kt}{B0}\PY{+w}{ }\PY{n}{xs}\PY{p}{)}\PY{+w}{  }\PY{o+ow}{=}\PY{+w}{ }\PY{k+kr}{case}\PY{+w}{ }\PY{n}{burn}\PY{+w}{ }\PY{n}{xs}\PY{+w}{ }\PY{k+kr}{of}\PY{+w}{ }\PY{n+nb}{()}\PY{+w}{ }\PY{o+ow}{\PYZhy{}\PYZgt{}}\PY{+w}{ }\PY{k+kt}{Prelude}\PY{o}{.}\PY{k+kt}{True}
\PY{n+nf}{evenBits}\PY{+w}{ }\PY{p}{(}\PY{k+kt}{B1}\PY{+w}{ }\PY{n}{xs}\PY{p}{)}\PY{+w}{  }\PY{o+ow}{=}\PY{+w}{ }\PY{k+kr}{case}\PY{+w}{ }\PY{n}{burn}\PY{+w}{ }\PY{n}{xs}\PY{+w}{ }\PY{k+kr}{of}\PY{+w}{ }\PY{n+nb}{()}\PY{+w}{ }\PY{o+ow}{\PYZhy{}\PYZgt{}}\PY{+w}{ }\PY{k+kt}{Prelude}\PY{o}{.}\PY{k+kt}{False}

\PY{n+nf}{half}\PY{+w}{ }\PY{o+ow}{::}\PY{+w}{ }\PY{k+kt}{Bits}\PY{+w}{ }\PY{o}{\PYZpc{}}\PY{l+m+mi}{1}\PY{o+ow}{\PYZhy{}\PYZgt{}}\PY{+w}{ }\PY{k+kt}{Bits}
\PY{n+nf}{half}\PY{+w}{ }\PY{k+kt}{Nil}\PY{+w}{      }\PY{o+ow}{=}\PY{+w}{ }\PY{k+kt}{Nil}
\PY{n+nf}{half}\PY{+w}{ }\PY{p}{(}\PY{k+kt}{B0}\PY{+w}{ }\PY{n}{xs}\PY{p}{)}\PY{+w}{  }\PY{o+ow}{=}\PY{+w}{ }\PY{n}{xs}
\PY{n+nf}{half}\PY{+w}{ }\PY{p}{(}\PY{k+kt}{B1}\PY{+w}{ }\PY{n}{xs}\PY{p}{)}\PY{+w}{  }\PY{o+ow}{=}\PY{+w}{ }\PY{n}{xs}

\PY{n+nf}{plus2x1}\PY{+w}{ }\PY{o+ow}{::}\PY{+w}{ }\PY{k+kt}{Bits}\PY{+w}{ }\PY{o}{\PYZpc{}}\PY{l+m+mi}{1}\PY{o+ow}{\PYZhy{}\PYZgt{}}\PY{+w}{ }\PY{k+kt}{Bits}
\PY{n+nf}{plus2x1}\PY{+w}{ }\PY{n}{x}\PY{+w}{ }\PY{o+ow}{=}\PY{+w}{ }\PY{k+kt}{B1}\PY{+w}{ }\PY{n}{x}

\PY{n+nf}{branchConst}\PY{+w}{ }\PY{o+ow}{::}\PY{+w}{ }\PY{k+kt}{Bits}\PY{+w}{ }\PY{o}{\PYZpc{}}\PY{l+m+mi}{1}\PY{o+ow}{\PYZhy{}\PYZgt{}}\PY{+w}{ }\PY{k+kt}{Bits}
\PY{n+nf}{branchConst}\PY{+w}{ }\PY{n}{x}\PY{+w}{ }\PY{o+ow}{=}
\PY{+w}{  }\PY{k+kr}{if}\PY{+w}{ }\PY{n}{evenBits}\PY{+w}{ }\PY{n}{x}
\PY{+w}{    }\PY{k+kr}{then}\PY{+w}{ }\PY{n}{half}\PY{+w}{ }\PY{n}{const\PYZus{}5}
\PY{+w}{    }\PY{k+kr}{else}\PY{+w}{ }\PY{n}{plus2x1}\PY{+w}{ }\PY{n}{const\PYZus{}6}

\PY{c+c1}{\PYZhy{}\PYZhy{} collatzBad2 :: Bits \PYZpc{}1\PYZhy{}\PYZgt{} Bits}
\PY{c+c1}{\PYZhy{}\PYZhy{} collatzBad2 x =}
\PY{c+c1}{\PYZhy{}\PYZhy{}   if evenBits x}
\PY{c+c1}{\PYZhy{}\PYZhy{}     then half x       \PYZhy{}\PYZhy{} ERROR: x already consumed by \PYZsq{}evenBits x\PYZsq{}}
\PY{c+c1}{\PYZhy{}\PYZhy{}     else plus2x1 x    \PYZhy{}\PYZhy{} ERROR: x already consumed by \PYZsq{}evenBits x\PYZsq{}}
\end{Verbatim}

\caption{Example of linear types in Haskell}\label{lst:haskell-linear-example}
\end{rawlisting}
\begin{remark}\label{remark:haskell-linear}
  In~\autoref{lst:haskell-linear-example}, the type system prevents us from using a linear argument
  more than once. The function \texttt{collatzBad2} fails to type-check precisely because the
  argument \texttt{x} would be consumed by \texttt{evenBits x} and then used again in the branches.
This effect makes most of the standard algorithms not transferrable to this formalism.  
\end{remark}

\subsection{IntML}\label{sec:intml}
In 2013, Dal Lago and Sch\"opp introduced \texttt{IntML}, a functional language with a linear type
system that characterizes \complexity{FL}~\cite{DALLAGO2016150}.
An implementation of \texttt{IntML} is available on
GitHub.\footnote{\url{https://github.com/uelis/IntML}. Following private communication with the authors, 
a permissive license was added to the repository, as it was not included originally.}
To the best of our knowledge, it remains the only language within the linear-logic branch of ICC that has both
a working implementation and some potential for (academic) practical use.

From the point of view of this thesis, however, linear-logic-based approaches --- including \texttt{IntML} ---
run into the \emph{same} issue of intensional vs.\ extensional expressive power discussed in~\autoref{subsec:intensional}.
These systems characterize classes such as \complexity{FL} and \complexity{FP} \emph{extensionally}.
The characterizations capture the right functions, but not the usual \emph{algorithmic techniques}
used to implement them. \texttt{IntML} looks like a very good starting point for
a practical programming language. However, it would still be very hard to use it
to certify the complexity of standard algorithms.

In this thesis, we will not pursue this line further as a practical basis for certifying
the complexity.
