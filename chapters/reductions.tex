\chapter{Reductions}
\label{chap:reductions}
The definitions below are from~\cite{10.5555/520668}.


\section{Uniformity}
Dlogtime-uniformity. First-order reductions. Logspace-uniformity. $U_{E^*}$. Direct / extended connection language.
Below it should not matter which of these types we use? But I don't have citation for that.
When we don't say poly, it means class is uniform. Otherwise: Immerman showed that the class FO is the same as a uniform version
of AC 0 . Originally AC0 was defined in its nonuniform version, which
we shall refer to as AC0/poly. A language in AC0/poly is specified by
a polynomial size bounded depth family $\langle C_n \rangle$ of Boolean circuits, where
each circuit Cn has n input bits, and is allowed to have ¬-gates, as well as
unbounded fan-in $\land$-gates and $\lor$-gates. In the uniform version, the circuit
C n must be specified in a uniform way; for example one could require that
$\langle C_n \rangle$ is in FO. See also Appendix A.5.


\section{$\text{NC}^i$}
$\text{AC}^i$ is the class of languages accepted by uniform circuit
families of polynomial size and depth $O(\log^i n)$, consisting of unbounded fan-in
AND, and OR gates, along with NOT gates.

Contained in $\text{AC}^i$.

\section{$\text{AC}^i$}
$\text{AC}^i$ is the class of languages accepted by uniform circuit
families of polynomial size and depth $O(\log^i n)$, consisting of unbounded fan-in
AND, and OR gates, along with NOT gates.

\section{$\text{AC}^i[m]$}
$\text{AC}^i[m]$ is defined as $\text{AC}^i$, but in addition unbounded fan-in $\text{MOD}_m$ gates
are allowed, which output 1 iff the number of input wires carrying a value of 1 is a
multiple of $m$.

\section{$\text{ACC}^i$}
$\text{ACC}^i = \bigcup_m \text{AC}^i[m]$. $\text{ACC}^0$ is contained in $\text{TC}^0$.

\section{$\text{TC}^i$}
$\text{TC}^i$ is the class of languages accepted by uniform circuit families 
of polynomial size and depth $O(\log^i n)$, consisting of unbounded fan-in MAJORITY
gates, along with NOT gates.

Contained in $\text{NC}^{i + 1}$.

\section{$\text{AC}^0$-reduction}
\label{sec:ac0red}
Definition IX.1.1 CN10. We say that a string function F
(resp. a number function f) is $\text{AC}^0$ -reducible to L if there is a sequence
of string functions $F_1, \dots, F_n (n \geqslant 0)$ such that
$F_i$ is $\Sigma^B_0$-definable from $L \cup \{F_1, \dots , F_{i-1}\}$, for $i = 1, \dots, n$
and F (resp. f) is $\Sigma^B_0$ -definable from $L \cup \{F_1, \dots , F_{i-1}\}$. A relation R is
$\text{AC}^0$-reducible to L if there is a sequence $F_1, \dots, F_n$ as above, and R is
represented by a $\Sigma^B_0(L \cup \{F_1, \dots, F_n\})$ formula.

\section{LOGSPACE reduction}

\section{P reduction}

