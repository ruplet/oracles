% ============================================
% Descriptive Complexity Section (LaTeX)
% TODO: literature review https://plato.stanford.edu/archives/win2017/entries/computational-complexity/#DesCom
% ============================================

\section{Descriptive Complexity: An Overview}

TODO: glue the below 2 paragraphs
Descriptive complexity is a branch of complexity theory that uses sentences in finite model theory
to describe computational queries.

In descriptive complexity theory, an object (e.g., a set of graphs) in
a complexity class is specified as the set of all finite models of a given
formula. In this work, we only consider the case where the object is a language 
\(L \subseteq \{0,1\}^*\) and the model is a finite binary string.

The central problem of descriptive complexity is to characterize a
complexity class by the \emph{power of logic required to define its problems}.
This is in contrast to \emph{Implicit Complexity Theory}, discussed later,
which seeks the weakest system sufficient to \emph{implement algorithms}
of the class; and in contrast to \emph{Bounded Arithmetic}, which studies
the weakest \emph{theory} required to define a function \emph{and prove its correctness}.
This perspective has led to elegant logical characterizations of many traditional
complexity classes.

As an example, Grädel's Theorem in descriptive complexity states that
\(\mathsf{NL}\) is the class of finite models of the second-order Krom formulas
\cite{GRADEL199235}.
Here, a \emph{Krom formula} is a formula in conjunctive normal form (CNF)
where each clause contains at most
two literals. The satisfiability problem for Krom formulas, Krom-SAT, is
complete for \(\mathsf{coNL}\) (or equivalently \(\mathsf{NL}\), by the Immerman-Szelepcsényi
Theorem).

Similarly, Grädel~\cite{GRADEL199235} shows that second-order Horn formulas
express precisely the properties decidable in polynomial time.
A \emph{Horn clause} is a clause with at most one positive literal and any number
of negative literals, and a Horn formula is a conjunction of such clauses.
By a result independently proved by Immerman~\cite{IMMERMAN198686}
and Vardi~\cite{10.1145/800070.802186},
\(\mathsf{PTIME}\) is also characterized by first-order logic with a least fixed-point operator
(\(\mathrm{FO[LFP]}\)).
Similar results exist for most commonly studied complexity classes:
\(\mathsf{NP}\) (Fagin's theorem), \(\mathsf{coNP}\), \(\mathsf{PH}\), \(\mathsf{PSPACE}\), and \(\mathsf{EXPTIME}\).

For an accessible introduction, see the classical and self-contained reference
by Immerman~\cite{Immerman1999-IMMDC}.

% -------------------------------------------------
\subsection{Ordered and Unordered Structures}

Most of the results discussed in this chapter assume that the input is
given as an \emph{ordered structure}, i.e., for two positions \(x\) and \(y\)
of the input word, we can test whether \(y\) is to the right of \(x\).
In fact, many theorems also assume the presence of a \emph{successor relation},
allowing us to test whether \(y\) is the \emph{immediate} right neighbor of \(x\).

In practice, computers operate on linearly ordered memory
(via addresses), so the assumption of ordered structures is both natural
and realistic.
However, this assumption is \emph{crucial}:
without an explicit ordering, \(\mathrm{FO[LFP]}\) does \textbf{not} capture
\(\mathsf{PTIME}\)~\cite{Cai1992}.
The existence of a logic that characterizes \(\mathsf{PTIME}\) on
\emph{unordered structures} remains a major open problem in computer science as of 2025.

Fagin's theorem says that existential second-order logic characterizes
\(\mathsf{NP}\).
It is worth noting that this theorem does \emph{not} assume order on the input ---
intuitively, \(\mathsf{NP}\) is strong enough to ``guess'' the order relation.

% -------------------------------------------------
\subsection{Descriptive Complexity and Type Systems}

After investigating systems within descriptive complexity, we explored
whether their logical frameworks could be combined with type-theoretic
approaches to design programming languages that capture complexity classes.
While this attempt to bridge descriptive complexity and type systems
was conceptually motivated, the two fields remain quite distant in practice,
and we were not successful in fully integrating them.
Nonetheless, this line of inquiry led to a broader investigation of
type systems that enforce resource bounds --- particularly those inspired
by linear logic and implicit computational complexity.
The results of this exploration are presented in the next chapter.

% -------------------------------------------------
\subsection{\texorpdfstring{$\mathrm{FO} = \mathrm{AC}^0$}{FO = AC0} and the Definition of \texorpdfstring{$\mathrm{FAC}^0$}{FAC0}}

Due to their declarative nature, results from descriptive complexity do not
directly help in designing programming languages.
However, one result from this field will be important for us in
Chapter~V (on bounded arithmetic). There, we will study a logical system
characterizing \(\mathrm{FAC}^0\): a class of \emph{reductions} (functions) that is very weak,
yet strong enough to define nontrivial computations.
The definition of \(\mathrm{FAC}^0\) will be based on a logical characterization of
\(\mathrm{AC}^0\).

A foundational result in descriptive complexity theory is that
\[
\mathrm{FO} = \mathrm{AC}^0,
\]
i.e., relations definable by a first-order sentence are precisely those
decidable by \(\mathrm{AC}^0\) circuits
(Corollary~5.32 in~\cite{Immerman1999-IMMDC}).%
\footnote{%
Under reasonable assumptions, namely:
FO formulas have access to an order relation on the input structure and to
basic arithmetic predicates \(+\) and \(\times\);
\(\mathrm{AC}^0\) is FO-uniform; and
all structures considered are ordered and have at least two elements (so that
\(0\) and \(1\) are distinct).
See Provisos~1.14 and~1.15 of~\cite{Immerman1999-IMMDC} for details.%
}
\footnote{%
Part of Corollary~5.32 relevant for our purposes refers to Theorem~5.22,
which states that \(\mathrm{FO}[t(n)] = \mathrm{AC}[t(n)]\)
for all polynomially bounded and FO-constructible functions \(t(n)\).
Here, \(\mathrm{FO}[t(n)]\) denotes formulas with a quantifier block iterated
\(t(n)\) times (Definition~4.24), and
\(\mathrm{AC}[t(n)]\) denotes FO-uniform circuits of depth \(O(t(n))\)
(Definition~5.17).
For our purposes, \(t(n) = 1\), circuits have depth \(O(1)\),
and \(\mathrm{AC}[t(n)] = \mathrm{AC}^0\).%
}

TODO: semantic notion of definability, i.e. a function is definable means
its graph is expressible by a formula in some logic.

This characterization will later be used (Definition~V.2.3 in~\cite{Cook_Nguyen_2010})
to define functions in \(\mathrm{FAC}^0\) as those of polynomially bounded output length
whose \emph{bit-graphs} lie in \(\mathrm{AC}^0\).
That is how we will transition to studying functional complexity classes.

% % -------------------------------------------------
% \section*{References}

% \bibliographystyle{plain}
% \bibliography{references}


% <!-- IMPORTANT: what is bit? what is <=, +, -.
% section 1.2 of immerman
% bit(i, 0) holds iff i is odd.
% PLUS(i, j, k) meaning i + j = k
% 2. TIMES(i, j, k) meaning i x j = k
% 3. BIT(i, j) meaning bit j in the binary representation of i is  -->

% <!-- this PLUS, TIMES seems to be like operations on unary numbers. This is unary numbers because we can only do it up until `n` (because we only have forall x, PLUS(x), PLUS(0), PLUS(1), PLUS(max),...) -->