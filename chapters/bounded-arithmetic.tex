% Important for literature review: https://plato.stanford.edu/archives/win2017/entries/computational-complexity/#BouAri

\chapter{Bounded arithmetic}
\label{chap:bounded-arithmetic}

 Find IDelta0 and that it can't express exp function. Look for
 Find Buss' PV theories and Cook's $V^i$ hierachy.
 ważne: Cook and Urquhart's theory IPV

\section{The zoo of arithmetical theories}
Historically, the single-sorted equational PV theory~\cite{10.1145/800116.803756} was
the first designed to capture polynomial time reasoning.
Its language contains one symbol for every function in FP --- functions are mapped to symbols
using Cobham's original characterization of FP with explicit bounds, as mentioned in \ref{sec:cobham-characterization}.

We also have Buss' $S^1_2$ theory. Apparently, to characterize below-polytime classes, a split to two sorts (strings and indices)
is necessary. This is the approach of Cook and Nguyen 2010, who introduce $V^i$ theories, which we focus on here.

% Paper:
% Suppose IS12 proves Exists y . A(y, c).
% Then S12 proves istnieje dowod
% https://mathweb.ucsd.edu/~sbuss/ResearchWeb/intuitionisticBA/intuitionisticBA_OCR.pdf
% Tutaj strona 160 (58 pdfa) wykazuje conjecture 3 z powyzszego!:
% https://doi.org/10.1016/0168-0072(93)90044-E
% no i Godel encoding jest wazny, bo jego dowod 1 twierdzenia o niezupelnosci
% to tak naprawde interpreter!


\section{IMP(PV) programming language}
In MIT reading group notes released in July 2025~\cite{Li2025FeasibleMathematics}, the authors show how the equational theory PV
can be used to simulate set theory, an imperative programming language, and Hoare semantics for it.

\begin{itemize}
  \item \textbf{Programmable data structures with internal correctness proofs.} \\
  Lists and maps are defined within PV or PV-PL together with proofs of their intended properties. Meta-theorems support recursion and induction on lists, analogous to the existing recursion and induction rules over strings.

  \item \textbf{An imperative layer with verifiable functionality.} \\
  A small language, \textbf{IMP(PV)}, is introduced with a Hoare-style proof system. Programs written in IMP(PV) can be compiled into PV functions, and Hoare proofs of functionality can be translated back into proofs in PV.

  \item \textbf{Machine-level simulation.} \\
  Turing machines are simulated by IMP(PV) programs and hence by PV functions, establishing the corresponding results about the expressiveness of the system.

  \item \textbf{Feasible set operations and reasoning.} \\
  A finite set theory is developed within PV-PL, allowing operations such as union and intersection on sets encoded by enumeration. Proofs in first-order logic over these finite sets (with quantifiers of the form $\forall x \in S$ and $\exists x \in S$) can be translated into PV proofs.

  \item \textbf{Proof methodology.} \\
  All arguments are written as ``pseudo-proofs'' in structured natural language, which can be directly translated into proof trees in PV-PL and then into formal PV proofs via the translation theorem.
\end{itemize}

\noindent
For precise formal definitions and proofs, see Chapter~4 of the book.

\paragraph{What the language is.}
\textbf{IMP(PV)} is a lightweight imperative layer on top of PV: expressions are PV terms; commands add assignment, sequencing, conditionals, and a bounded \texttt{for}-loop that iterates by repeatedly applying \(\mathrm{TR}(\cdot)\) to a loop variable until it equals \(\varepsilon\).
All variables are global.

\medskip
% --- Core syntax (split across lines to avoid overfull boxes)
\noindent\textbf{Core syntax (only what we need).}
\[
\begin{array}{l}
\text{Expressions: } e \text{ is any PV term.} \\[0.3em]
\text{Commands: } \text{Cmd} \;::=\; \texttt{skip}
\;\mid\; \texttt{let } x := e
\;\mid\; \text{Cmd};\ \text{Cmd} \\[0.25em]
\hphantom{\text{Commands: } \text{Cmd} \;::=\;} 
\;\mid\; \texttt{if } e \ \texttt{then}\ c_1\ \texttt{else}\ c_0 \\[0.25em]
\hphantom{\text{Commands: } \text{Cmd} \;::=\;}
\;\mid\; \texttt{for } x := e;\ x \neq \varepsilon;\ x := \mathrm{TR}(x)\ \texttt{do } c\,.
\end{array}
\]
Here \( \mathrm{TR} \) is the one-step right-trim on strings/lists, and all variables are global.

\medskip
\noindent\textbf{Resource discipline.}
To keep programs within polynomial time we use:
\begin{enumerate}
  \item \emph{Read-only loop index:} the index \(x\) of a \texttt{for}-loop is not reassigned inside the loop and is not reused as an inner-loop index.
  \item \emph{Length cap:} a program is paired with a bound \(\omega\); values exceeding \(|\omega|\) bits are truncated to \(\mathrm{Suf}(y,\omega)\).
\end{enumerate}

\medskip
\noindent\textbf{Hoare semantics in PV-PL.}
A Hoare triple \(\{\,\varphi_1\,\}\ c\ \{\,\varphi_2\,\}\) uses PV-PL formulas \(\varphi_1,\varphi_2\). The rules we rely on are:

\begin{itemize}
  \item \emph{Consequence:} if \( \mathrm{PV\mbox{-}PL} \vdash \varphi_1 \Rightarrow \varphi'_1 \) and \( \mathrm{PV\mbox{-}PL} \vdash \varphi'_2 \Rightarrow \varphi_2 \), then
  \[
    \dfrac{\{\varphi'_1\}\ c\ \{\varphi'_2\}}{\{\varphi_1\}\ c\ \{\varphi_2\}}.
  \]
  \item \emph{Skip:} \(\{\varphi\}\ \texttt{skip}\ \{\varphi\}\).
  \item \emph{Assignment:} \(\{\ \varphi[x/e]\ \}\ \texttt{let }x := e\ \{\ \varphi\ \}\).
  \item \emph{Sequencing:} from \(\{\varphi_1\}\ c_1\ \{\psi\}\) and \(\{\psi\}\ c_2\ \{\varphi_2\}\) infer \(\{\varphi_1\}\ c_1;c_2\ \{\varphi_2\}\).
  \item \emph{Conditional:}
  \[
  \dfrac{\{\varphi_1 \land \mathrm{LastBit}(e)=1\}\ c_1\ \{\varphi_2\}
        \qquad
        \{\varphi_1 \land \mathrm{LastBit}(e)\neq 1\}\ c_0\ \{\varphi_2\}}
        {\{\varphi_1\}\ \texttt{if }e\ \texttt{then }c_1\ \texttt{else }c_0\ \{\varphi_2\}}.
  \]
  \item \emph{For-loop (invariant \(\varphi\)):} for \(i\in\{0,1\}\),
  \[
  \dfrac{\{\ \varphi[x/s_i(x)]\ \}\ c\ \{\ \varphi\ \}}
        {\{\ \varphi[x/e]\ \}\ \texttt{for }x := e;\ x\neq\varepsilon;\ x := \mathrm{TR}(x)\ \texttt{do } c\ \{\ \varphi[x/\varepsilon]\ \}}.
  \]
\end{itemize}

\medskip
\noindent\textbf{Example (length-by-iteration).}
\[
\begin{array}{l}
\texttt{LenITR} := \texttt{let } z := \ell;\ \texttt{let } y := \varepsilon; \\[0.2em]
\qquad \texttt{for } x := \ell;\ x \neq \varepsilon;\ x := \mathrm{TR}(x)\ \texttt{do } \\
\qquad\qquad \texttt{if } \mathrm{IsEps}(z)\ \texttt{ then }\ \texttt{skip}\ \texttt{ else }\ \texttt{let } y := s_0(y);\ \texttt{let } z := \mathrm{Tail}(z).
\end{array}
\]
Correctness is expressed as the Hoare triple
\[
\{\ \mathrm{IsList}(\ell)=1\ \}\ \texttt{LenITR}\ \{\ y = \mathrm{Len}(\ell)\ \},
\]
proved by a loop invariant that relates \(z\), \(y\), and the iterator equalities (details as in the chapter).


\medskip
\noindent\textbf{Compilation back to PV (how commands become functions).}
Every length-restricted program \((P,\omega)\) compiles to a PV function \([P]_{PV}\).
The translation uses a tuple \(\pi\) to store all variables, interprets expressions as \(e[x_i/\pi_i]\) followed by \(\mathrm{Suf}(\cdot,\omega)\), maps \texttt{skip} to the identity on \(\pi\), \texttt{let} to tuple update, sequencing to composition, \texttt{if} to a PV conditional, and the \texttt{for}-loop to a PV recursion that mimics one loop step per recursive call.
The details of each clause and the proof obligations they satisfy are provided in the chapter.

\medskip
\noindent\textbf{Two preservation properties we will cite (not re-prove here).}
\begin{itemize}
  \item \emph{Length preservation:} for any \((P,\omega)\), the translation \([P]_{PV}\) enforces \(x_i=\mathrm{Suf}(x_i,\omega)\) at the end for every program variable (proved by structural induction on \(P\)).
  \item \emph{Proof preservation:} a derivation of \(\{\,\varphi_1\,\}(P,\omega)\{\,\varphi_2\,\}\) yields a PV-PL derivation of the corresponding assertion obtained by substituting variables with tuple components and interpreting \(P\) as \([P]_{PV}\).
\end{itemize}
Both statements, and the admissibility of the Hoare rules under this translation, are established in the chapter.

\medskip
\noindent\textbf{Why bounded arithmetic must be in place first.}
Every step above---the loop rule, the compilation of \texttt{for}, the use of \(\mathrm{TR}\), \(\mathrm{Suf}\), tuple updates, and the truncation invariant---relies on equalities and inductions that are proved \emph{inside} PV/PV-PL.
Thus, implementing the language with its Hoare system requires a solid formalization of this arithmetic (lists/strings, iterators, truncation, and the associated induction principles), so that programs and their proofs are checked against PV-PL rather than an external meta-theory.
For all formal statements and proofs, we refer to Chapter~4.


\section{}

\section{Type-theoretical treatment}
It seems possible to reason about bounded arithmetic in type-theoretical framework.
The below is inspired by~\cite{Li2025FeasibleMathematics}.

Feasible set theory. The universal and existential quantification over feasible sets
can be viewed as a “typed” version of Buss’s sharply bounded quantifiers [Bus86],
as the numbers of elements in sets encoded by lists are bounded by their encoding
lengths.


% \section{Reverse mathematics}
% Bounded reverse mathematics: https://www.cs.toronto.edu/~sacook/banff_survey.pdf
% Examine reverse mathematics. This resulted from study of how large of sets can mathematics (ZFC) create.
% \label{sec:aaa}
 
% Find a paper which requires formalization of 'provability in $V^i'$ to construct Hoare semantics

% proposition 5.32 a string function is sigma0b-bit-definable iff it is in FAC0
% follows from def 5.32, corollary 5.17
% def 5.32:
% for Phi: set of formulas (e.g. sigma0b)
% string function F(x, Y) is Phi-bit-definable if is formula phi in Phi and number term t(x, Y) s.t.
% F(x, Y)(i) iff i < t(x, Y) and phi(i, x, Y)
% the RHS of this is a bit-defining axiom of F.

% corollary 5.17: string function is in FAC0 iff is p-bounded
% and its bit graph is represented by a sigma0b formula
% the same holds for a  number function, with graph replacing bit-graph
% proof: follows from sigma0b representation theorem 4.17

% theorem 4.17: relation R(x, X) is in AC0 iff it is represented by some Sigma0b formula phi(x,X)
% proof: like theorem 3.58

% there are functions whose graphs are in AC0 (representable by sigma0b formulas),
% but which do not belong to FAC0 (section: proof of witnessing theorem for v0)

% theorem 3.58.
% one side: compline LTH turing machines to formulas. we're not going to do that.
% second side: Delta0n subset LTH.
% bounded quantifiers correspond to exists, forall states in ATM.
% only interesting case is that `R(x, y, z) iff x*y=z` is in LTH.
% use corollary 3.60 which shows L subseteq LTH and that multiplication is in L.

% some intuitionistic logic in lean:
% https://github.com/DafinaTrufas/Intuitionistic-Logic-Lean

% Bounded arithmetics in Lean:
% https://github.com/FormalizedFormalLogic/Foundation

% hierarchia arytmetyk
% https://en.wikipedia.org/wiki/Ordinal_analysis

% Weak systems of arithmetic:
% https://golem.ph.utexas.edu/category/2011/10/weak_systems_of_arithmetic.html

% IΔ₀ and linear time hierarchy!
% Elementary Function Arithmetic = EFA
% Primitive Recursive Arithmetic = PRA
% RCA0*
% IDelta0 + Exp

% Primitive recursion:
% https://ftp.cs.ru.nl/CSI/CompMath.Found/week9.pdf

% A constructive proof of the Gödel-Rosser incompleteness theorem has been completed using the Coq proof assistant
% http://r6.ca/Goedel/goedel1.html

% Redukcje w Coq różne:  jest tutaj Ackermann function, PRA
% https://github.com/rocq-community/hydra-battles

% Definicje e.g. arytmetyki Q Robinsona:
% file:///home/maryjane/Downloads/1235421930-2.pdf

% Coding of sets and sequences, strona 31 pdfa (295):
% file:///home/maryjane/Downloads/1235421934.pdf
% Książka:
% https://projecteuclid.org/eBooks/perspectives-in-logic/Metamathematics-of-First-Order-Arithmetic/toc/pl/1235421926

% Funkcja jest dowodliwa w PA wtedy i tylko wtedy gdy rośnie wystarczająco wolno!
% https://math.stackexchange.com/questions/4859305/are-there-examples-of-statements-not-provable-in-pa-that-do-not-require-fast-gro?rq=1
% Czyli funkcje, które są niedowodliwe, a rosną wolno (np. są 0/1), nie mogą być wyrażalne!

% https://mathoverflow.net/questions/382179/what-can-i-delta-0-prove

% Euler's phi function in IDelta0:
% https://link.springer.com/article/10.1007/BF01375521



