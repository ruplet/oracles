\chapter{Bounded arithmetic}
\label{chap:bounded-arithmetic}



\section{Reverse mathematics}
\label{sec:aaa}


Examine reverse mathematics. This resulted from study of how large of sets can mathematics (ZFC) create. Find bounded arithmetic. Find IDelta0 and that it can't express exp function. Find Buss' PV theories and Cook's V^i hierachy. Look for
a way to implement / formalize results about V^i. Find a way to do it. Find a paper which requires formalization of 'provability in V^i' to construct Hoare semantics
of an imperative programming language. This is presented at AITP.

All of my work on the formaliation of bounded arithmetic has been moved to
https://github.com/ruplet/formalization-of-bounded-arithmetic



proposition 5.32 a string function is sigma0b-bit-definable iff it is in FAC0
follows from def 5.32, corollary 5.17
def 5.32:
for Phi: set of formulas (e.g. sigma0b)
string function F(x, Y) is Phi-bit-definable if is formula phi in Phi and number term t(x, Y) s.t.
F(x, Y)(i) iff i < t(x, Y) and phi(i, x, Y)
the RHS of this is a bit-defining axiom of F.

corollary 5.17: string function is in FAC0 iff is p-bounded
and its bit graph is represented by a sigma0b formula
the same holds for a  number function, with graph replacing bit-graph
proof: follows from sigma0b representation theorem 4.17

theorem 4.17: relation R(x, X) is in AC0 iff it is represented by some Sigma0b formula phi(x,X)
proof: like theorem 3.58

there are functions whose graphs are in AC0 (representable by sigma0b formulas),
but which do not belong to FAC0 (section: proof of witnessing theorem for v0)


theorem 3.58.
one side: compline LTH turing machines to formulas. we're not going to do that.
second side: Delta0n subset LTH.
bounded quantifiers correspond to exists, forall states in ATM.
only interesting case is that `R(x, y, z) iff x*y=z` is in LTH.
use corollary 3.60 which shows L subseteq LTH and that multiplication is in L.

some intuitionistic logic in lean:
https://github.com/DafinaTrufas/Intuitionistic-Logic-Lean

Bounded arithmetics in Lean:
https://github.com/FormalizedFormalLogic/Foundation

ważne: palalansouki


file:///home/maryjane/Downloads/1235421934.pdf
strona 52 pdf, strona 316 ksiazki
chapter V Bounded arithmetic
tu jest o kodowaniu CFG

Paper:
Suppose IS12 proves Exists y . A(y, c).
Then S12 proves istnieje dowod
https://mathweb.ucsd.edu/~sbuss/ResearchWeb/intuitionisticBA/intuitionisticBA_OCR.pdf
Tutaj strona 160 (58 pdfa) wykazuje conjecture 3 z powyzszego!:
https://doi.org/10.1016/0168-0072(93)90044-E
no i Godel encoding jest wazny, bo jego dowod 1 twierdzenia o niezupelnosci
to tak naprawde interpreter!

ważne: Cook and Urquhart's theory IPV

przykłady rozszerzeń arytmetyki:
https://en.wikipedia.org/wiki/Conservative_extension

hierarchia arytmetyk
https://en.wikipedia.org/wiki/Ordinal_analysis

Weak systems of arithmetic:
https://golem.ph.utexas.edu/category/2011/10/weak_systems_of_arithmetic.html

IΔ₀ and linear time hierarchy!
Elementary Function Arithmetic = EFA
Primitive Recursive Arithmetic = PRA
RCA0*
IDelta0 + Exp

Primitive recursion:
https://ftp.cs.ru.nl/CSI/CompMath.Found/week9.pdf

A constructive proof of the Gödel-Rosser incompleteness theorem has been completed using the Coq proof assistant
http://r6.ca/Goedel/goedel1.html

Redukcje w Coq różne:  jest tutaj Ackermann function, PRA
https://github.com/rocq-community/hydra-battles

Definicje e.g. arytmetyki Q Robinsona:
file:///home/maryjane/Downloads/1235421930-2.pdf

Coding of sets and sequences, strona 31 pdfa (295):
file:///home/maryjane/Downloads/1235421934.pdf
Książka:
https://projecteuclid.org/eBooks/perspectives-in-logic/Metamathematics-of-First-Order-Arithmetic/toc/pl/1235421926

Funkcja jest dowodliwa w PA wtedy i tylko wtedy gdy rośnie wystarczająco wolno!
https://math.stackexchange.com/questions/4859305/are-there-examples-of-statements-not-provable-in-pa-that-do-not-require-fast-gro?rq=1
Czyli funkcje, które są niedowodliwe, a rosną wolno (np. są 0/1), nie mogą być wyrażalne!

https://mathoverflow.net/questions/382179/what-can-i-delta-0-prove

Euler's phi function in IDelta0:
https://link.springer.com/article/10.1007/BF01375521

czym jest transfinite induction w ERA, PRA, PA:
https://mathoverflow.net/questions/123713/era-pra-pa-transfinite-induction-and-equivalences?rq=1


24. Cook-Nguyen (history):
The universal theory VPV is based on the single-sorted theory PV [39].
which historically was the first theory designed to capture polynomial time
reasoning. 

