\chapter{Preliminaries}\label{chap:preliminaries}
In this chapter we introduce the standard definitions for logic and complexity theory, necessary to
fix the language we will use in the rest of the work. 
We restrict attention to the logical frameworks needed later (single-sorted first-order logic and
its two-sorted counterpart) and omit the analogous presentations for propositional calculus or
full second-order systems. The semantics is entirely classical
(as in: not intuitionistic). 

In~\autoref{sec:defs-preliminaries-circuits} we first introduce the so-called \emph{non-uniform}
circuit families (e.g.\ \FACipoly), then define a weak notion of uniformity and the appropriate
\emph{uniform} circuit families (e.g.\ \FACi).
The notion of uniformity used in this thesis (cf.~\autoref{def:logspace-uniformity})
and the results using it are usually not interesting. However, in this work
we will not focus too much on that problem. We leave out the complicated
divagations about the appropriate (i.e.\ much weaker) notions of uniformity to~\autoref{chap:uniformity}.

\begin{remark}[Bibliography]
The definitions in~\autoref{sec:defs-single-sorted} are in the style of~\cites[Section~2B]{CookNguyenDraft}[Section~II.2]{Cook_Nguyen_2010}.
The definitions in~\autoref{sec:defs-two-sorted} are in the style of~\cites[Section~4B]{CookNguyenDraft}[Section~IV.2]{Cook_Nguyen_2010}.
We will mostly need them for~\autoref{chap:bounded-arithmetic}, but also for~\autoref{chap:descriptive-complexity} --- 
the results discussed in the latter are mostly from~\cite{Immerman1999-IMMDC},
where different style of definitions is used.
However,~\cite{Immerman1999-IMMDC} doesn't introduce two-sorted logic.
The single-sorted definitions differences are mostly negligible and the consistent treatment of
single- and two-sorted logic, is important for~\autoref{chap:bounded-arithmetic}.

Much effort has been put into
considering different versions of the definitions in~\autoref{sec:preliminaries-turing}
to keep them consistent
with the definitions of functional complexity classes studied later in~\autoref{sec:functional-complexity-classes},
e.g.\ the class \complexity{FNP} defined in~\autoref{def:complexity-fnp}.
These are much less standard, and often have no clear consensus in the literature.

The definitions of decisional circuit classes are from~\cite{10.5555/520668}.
\end{remark}


\section{Single-sorted first-order logic}\label{sec:defs-single-sorted}

\begin{definition}[First-order vocabulary and syntax]
A \emph{first-order vocabulary} (or \emph{language}) $\mathcal{L}$ consists of:
\begin{enumerate}
  \item For each $n \ge 0$, a (possibly empty) set of $n$-ary \emph{function symbols}.
        We use $f,g,h,\dots$ as meta-variables for function symbols.
        A $0$-ary function symbol is called a \emph{constant symbol};
  \item For each $n \ge 0$, a set of $n$-ary \emph{predicate symbols}, which is
        nonempty for at least one $n$.
        We use $P,Q,R,\dots$ as meta-variables for predicate symbols.
\end{enumerate}

In addition, the following logical symbols are available to build first-order
terms and formulas:
\begin{enumerate}
  \item An infinite set of \emph{variables}. We use $x,y,z,\dots$ and sometimes
        $a,b,c,\dots$ as meta-variables for variables;
  \item The connectives $\lnot$, $\land$, $\lor$ (not, and, or) and
        logical constants $\bot$, $\top$ (False, True);\todo{consistency: semicolon in enumerate}
  \item The quantifiers $\forall$, $\exists$ (for all, there exists);
  \item Parentheses $(\, ,\,)$.
\end{enumerate}
\end{definition}

% Expressive power of different possibilities
% \item $\text{FO}[+, *] = \text{FO}[\mathrm{BIT}]: Section~1.2.1 Immerman $.
% \item $\text{FO}[+]$ is less expressive than $\text{FO}[<, *] = \text{FO}[<, /] = \text{FO}[<, \mathrm{COPRIME}]$~\cite{10.1002/malq.200310041}.


\begin{definition}[$\mathcal{L}$-terms]
Let $\mathcal{L}$ be a first-order vocabulary.
The set of \emph{$\mathcal{L}$-terms} is defined inductively as follows:
\begin{enumerate}
  \item Every variable is an $\mathcal{L}$-term;
  \item If $f$ is an $n$-ary function symbol of $\mathcal{L}$ and
        $t_1,\dots,t_n$ are $\mathcal{L}$-terms, then
        \[
          f(t_1,\dots,t_n)
        \]
        is an $\mathcal{L}$-term.
\end{enumerate}
\end{definition}

\begin{definition}[$\mathcal{L}$-formulas]
Let $\mathcal{L}$ be a first-order vocabulary.
The set of \emph{first-order formulas in $\mathcal{L}$} (or
\emph{$\mathcal{L}$-formulas}) is defined
inductively as follows:
\begin{enumerate}
  \item The logical constants $\bot$ and $\top$ are atomic formulas;
  \item If $P$ is an $n$-ary predicate symbol in $\mathcal{L}$ and
        $t_1,\dots,t_n$ are $\mathcal{L}$-terms, then
        \[
          P(t_1,\dots,t_n)
        \]
        is an \emph{atomic} $\mathcal{L}$-formula;
  \item If $A$ and $B$ are $\mathcal{L}$-formulas, then
        $\lnot A$, $(A \land B)$, and $(A \lor B)$ are $\mathcal{L}$-formulas;
  \item If $A$ is an $\mathcal{L}$-formula and $x$ is a variable, then
        $\forall x\,A$ and $\exists x\,A$ are $\mathcal{L}$-formulas.
\end{enumerate}
For example,
\[
  (\lnot \forall x\,P x \,\lor\, \exists x\,\lnot P x)
  \quad\text{and}\quad
  (\forall x\,\lnot P x y \,\land\, \lnot \forall z\,P f y z)
\]
are $\mathcal{L}$-formulas (for suitable choices of $P$ and $f$ in $\mathcal{L}$).
\end{definition}

\begin{definition}[The language of arithmetic]\label{def:LA}
The \emph{language of arithmetic} is
\[
  L_{A} = [\,0,\,1,\,+,\,\cdot \;;\; =,\,\le\,],
\]
where \(0\) and \(1\) are constant symbols, \(+\) and \(\cdot\) are binary
function symbols, and \(=\) and \(\le\) are binary predicate symbols.
We will write these symbols in infix form.
\end{definition}

\begin{definition}[Free and bound variables]\label{def:free-bound}
Let $A$ be a formula and $x$ a variable.
An occurrence of $x$ in $A$ is \emph{bound} if it lies within a subformula
of $A$ of the form $\forall x\,B$ or $\exists x\,B$.
Any other occurrence of $x$ in $A$ is called \emph{free}.
\end{definition}

\begin{definition}[Closed terms, closed formulas, sentences]\label{def:closed-sentence}
A formula is \emph{closed} if it contains no free occurrence of any variable.
A term is \emph{closed} if it contains no variables at all.
A closed formula is also called a \emph{sentence}.
\end{definition}

\begin{definition}[$\mathcal{L}$-structure]\label{def:L-structure}
Let $\mathcal{L}$ be a first-order vocabulary.
An \emph{$\mathcal{L}$-structure} $\mathcal{M}$ consists of:
\begin{enumerate}
  \item A nonempty set $M$, called the \emph{universe}
        (Variables are intended to range over $M$);
  \item For each $n$-ary function symbol $f$ in $\mathcal{L}$, an associated function
        \(
          f^{\mathcal{M}} : M^n \to M
        \);
  \item For each $n$-ary predicate symbol $P$ in $\mathcal{L}$, an associated relation
        \(
          P^{\mathcal{M}} \subseteq M^n
        \).
\end{enumerate}
\end{definition}

\begin{remark}
    Note that to ``syntactical'' relations, we assign ``real'' relations defined on
    the underlying elements of the structure. We will want to treat some of these
    relations specially, e.g.\ to make sure that the ``\(=\)'' relation is
    interpreted as the actual equality, or that a designated ``\(\text{PLUS}(x, y, z)\)''
    relation holds only if the underlying objects are actual natural numbers,
    for which we have $x + y = z$. \todo{add autoref to where we talk about that}
% Thus the predicate symbol $=$ receives special treatment: it must always be
% interpreted as actual equality on the universe. We can also consider
% logics where we don't take the $=$ symbol as granted. For our purposes, however,
% the things we will be able to say about the class of models of a given formula
% will be more interesting if we already assume that whenever $p = q$ holds
% in the structure, then the underlying objects of the universe are also equal
% (in the meta-mathematical sense).
\end{remark}


\begin{definition}[Object Assignment]
Let $\mathcal{M}$ be a structure with universe $M$.  
An \emph{object assignment} \(\sigma\) for $\mathcal{M}$ is a mapping
from variables to the universe $M$.
\end{definition}

\begin{notation}
Let $x$ be a variable and $m \in M$.  
We write $\sigma(m/x)$ for the
assignment that is the same as $\sigma$ except that it maps $x$ to $m$.
\end{notation}

\begin{definition}[Basic Semantic Definition]
Let $\mathcal{L}$ be a first-order vocabulary, let $\mathcal{M}$ be an $\mathcal{L}$-structure
with universe $M$, and let $\sigma$ be an object assignment for $\mathcal{M}$.

\paragraph{Interpretation of terms.}
Each $\mathcal{L}$-term $t$ is assigned an element $t^{\mathcal{M}}[\sigma] \in M$,
defined by structural induction on $t$:
\begin{enumerate}
  \item For each variable $x$,
  \(
    x^{\mathcal{M}}[\sigma] = \sigma(x).
  \)
  \item 
  \(
    (f t_1 \dots t_n)^{\mathcal{M}}[\sigma]
      = f^{\mathcal{M}}\bigl(t_1^{\mathcal{M}}[\sigma],\dots,t_n^{\mathcal{M}}[\sigma]\bigr).
  \)
\end{enumerate}

\paragraph{Satisfaction of formulas.}
For an $\mathcal{L}$-formula $A$, the relation
\[
  \mathcal{M} \models A[\sigma]
\]
(read: ``$\mathcal{M}$ satisfies $A$ under $\sigma$'') is defined by structural
induction on $A$:
\begin{enumerate}
  \item $\mathcal{M} \models \top$ and $\mathcal{M} \not\models \bot$.
  \item For an atomic formula $P t_1 \dots t_n$ (with $P$ an $n$-ary
        predicate symbol),
  \[
    \mathcal{M} \models (P t_1 \dots t_n)[\sigma]
    \;\;\text{iff}\;\;
    \bigl\langle t_1^{\mathcal{M}}[\sigma],\dots,t_n^{\mathcal{M}}[\sigma]\bigr\rangle
    \in P^{\mathcal{M}}.
  \]
  \item If $\mathcal{L}$ contains $=$, then for terms $s,t$,
  \[
    \mathcal{M} \models (s = t)[\sigma]
    \;\;\text{iff}\;\;
    s^{\mathcal{M}}[\sigma] = t^{\mathcal{M}}[\sigma].
  \]
  \item $\mathcal{M} \models \neg A[\sigma]$ iff $\mathcal{M} \not\models A[\sigma]$.
  \item $\mathcal{M} \models (A \lor B)[\sigma]$ iff
        $\mathcal{M} \models A[\sigma]$ or $\mathcal{M} \models B[\sigma]$.
  \item $\mathcal{M} \models (A \land B)[\sigma]$ iff
        $\mathcal{M} \models A[\sigma]$ and $\mathcal{M} \models B[\sigma]$.
  \item $\mathcal{M} \models (\forall x\,A)[\sigma]$ iff
        $\mathcal{M} \models A[\sigma(m/x)]$ for all $m \in M$.
  \item $\mathcal{M} \models (\exists x\,A)[\sigma]$ iff
        $\mathcal{M} \models A[\sigma(m/x)]$ for some $m \in M$.
\end{enumerate}

\noindent
If $t$ is a closed term, then $t^{\mathcal{M}}[\sigma]$
is independent of $\sigma$, and we simply write $t^{\mathcal{M}}$.
Similarly, if $A$ is a sentence, we often write
$\mathcal{M} \models A$ instead of $\mathcal{M} \models A[\sigma]$, since
the choice of $\sigma$ does not matter.
\end{definition}

\section{Two-sorted first-order logic}\label{sec:defs-two-sorted}
% Sec. 2.2: Akapit 1 sugeruje, jakbyśmy nic nie mówili o logice 2-sortowej, a zaraz po nim miało nastąpić coś
% innego. Przeformułuj, aby był jaśniejszy (w stylu "pomijamy rzeczy podobne, a wypiszemy tylko różnice").

% Ponadto rozdziały 2.1 i 2.2 mają zupełnie inny charakter. W 2.1 jest logika nad dowolną sygnaturą, a w 2.2
% konkretnie liczby i napisy. Pasuje coś powiedzieć na początku 2.2 - że choć logikę 2-sortową można ogólnie
% definiować dla dowolnych sortów, to my piszemy tylko o tych konkretnych, liczby i napisy.

% I też właśnie trzeba konkretnie napisać, że sorty to liczby i napisy, przed Def. 2.2.1 - bo teraz w środku
% tej definicji nagle się pojawia coś o napisach i liczbach i nie wiadomo o co chodzi.



Two-sorted first-order logic extends the single-sorted setting in a routine way, so we only record
the additions that we use later.  A systematic presentation can be found
in~\cites[Section~4B]{CookNguyenDraft}[Section~IV.2]{Cook_Nguyen_2010}.
In principle one could work with arbitrary pairs of sorts, but in this thesis we instantiate the
framework to the familiar number sort (ranging over $\mathbb{N}$) and string sort (ranging over
finite binary strings).  The goal of this section is therefore to emphasise what changes when we
move from the definitions of \autoref{sec:defs-single-sorted} to this concrete two-sorted
setting.

\begin{definition}[Two-sorted first-order vocabularies]\label{def:two-sorted-vocabulary}
A \emph{two-sorted first-order vocabulary} (often abbreviated simply as a
two-sorted language) \(\mathcal{L}\) consists of collections of function and predicate
symbols, much like an ordinary single-sorted vocabulary, but now the
symbols may accept arguments of either of the two sorts.  Moreover, the
function symbols come in two varieties:
\begin{enumerate}
  \item \emph{number-valued} function symbols, whose outputs lie in the
        number sort; and
  \item \emph{string-valued} function symbols, whose outputs lie in the
        string sort.
\end{enumerate}

For any pair \(n,m \in \mathbb{N}\), the vocabulary contains:
\begin{enumerate}
  \item a set of \((n,m)\)-ary number-function symbols,
  \item a set of \((n,m)\)-ary string-function symbols, and
  \item a set of \((n,m)\)-ary predicate symbols.
\end{enumerate}
A \((0,0)\)-ary function symbol is simply a constant symbol, which may be
either a constant of the number sort or a constant of the string sort.

We use \(f,g,h,\dots\) as metavariables for number-valued function symbols,
\(F,G,H,\dots\) for string-function symbols, and \(P,Q,R,\dots\) for predicate
symbols.
\end{definition}

% TODO: TUTAJ DOPISAC REMARK ZE TO BEDZIE JAKO AKSJOMAT!
% Okolice Def. 2.2.2: Implicite zakładamy związek między |X| oraz t\in X, więc nie dowolna struktura tylko
% coś w stylu że nie zachodzi t\in X dla liczb >= |X|.

\begin{definition}[The language \(\mathcal{L}^{2}_{A}\)]\label{def:L2A}
As an example, consider the following two-sorted extension of the
arithmetical language \(\mathcal{L}_{A}\)~(\autoref{def:LA}):
\[
  \mathcal{L}^{2}_{A} \;=\; [\,0,\ 1,\ +,\ \cdot,\ |\cdot| \;;\ =_{1},\ =_{2},\ \le,\ \in\,].
\]

Here the symbols \(0,1,+,\cdot,=_{1},\le\) are
symbols of \(\mathcal{L}_{A}\) (with \(=_{1}\) corresponding to the usual equality of
numbers).  
The symbol \(|X|\) is a number-valued function symbol giving the
length of a string \(X\).  
The binary predicate \(\in\) relates a number and a string and is used to
express membership: intuitively, \(i \in X\) means that the \(i\)-th bit of
the string \(X\) is \(1\).  
The symbol \(=_{2}\) denotes equality between objects of the second sort.

For convenience, when \(t\) is a number term, we abbreviate
\[
  X(t) \;\coloneqq\; t \in X.
\]
Thus \(X(i)\) plays the role of the \(i\)-th bit of the binary string \(X\).

In \(\mathcal{L}^{2}_{A}\), the symbols \(+\) and \(\cdot\) each have arity \((2,0)\); the
length function \(|\cdot|\) has arity \((0,1)\); and the predicate
\(\in\) has arity \((1,1)\).
\end{definition}



\begin{notation}[Bounded formulas]\label{def:bounded-formulas}
Let \(\mathcal{L}\) be a two-sorted vocabulary.  
If \(x\) is a number variable and \(X\) a string variable that do not occur
in the \(\mathcal{L}\)-number term \(t\), we use the following abbreviations:
\begin{align*}
  \exists x \le t\ldotp \varphi
    &\;\;\text{stands for}\;\;
      \exists x\ldotp (x \le t \;\wedge\; \varphi), \\[2pt]
  \forall x \le t\ldotp \varphi
    &\;\;\text{stands for}\;\;
      \forall x\ldotp (x \le t \;\to\; \varphi), \\[2pt]
  \exists X \le t\ldotp \varphi
    &\;\;\text{stands for}\;\;
      \exists X\ldotp (\,|X| \le t \;\wedge\; \varphi), \\[2pt]
  \forall X \le t\ldotp \varphi
    &\;\;\text{stands for}\;\;
      \forall X\ldotp (\,|X| \le t \;\to\; \varphi).
\end{align*}

A quantifier appearing in one of these forms is called \emph{bounded},
and a \emph{bounded formula} is a formula in which every quantifier is
bounded.
\end{notation}


\paragraph{Notation.}
The expression
\(\exists \vec{x} \le \vec{t}\ldotp \varphi\)
abbreviates a block of bounded number quantifiers
\(\exists x_1 \le t_1\ldotp \cdots \exists x_k \le t_k\ldotp \varphi\)
for some \(k\), where no variable \(x_i\) occurs in any term \(t_j\)
(even when \(i < j\)).  
The same convention applies to \(\forall \vec{x} \le \vec{t}\), \(\exists \vec{X} \le \vec{t}\), and
\(\forall \vec{X} \le \vec{t}\).


\begin{definition}[The \texorpdfstring{$\Sigma^1_{1}(\mathcal{L})$, $\Sigma^B_{i}(\mathcal{L})$,
  and $\Pi^B_{i}(\mathcal{L})$}{Σ¹₁(L), Σᵢᴮ(L), Πᵢᴮ(L)} formulas]\label{def:SigmaB-PiB-hierarchy}
Let \(\mathcal{L} \supseteq \mathcal{L}^{2}_{A}\) be a two-sorted vocabulary.  
% Druga potencjalna modyfikacja to dopuszczenie kwantyfikatorów po liczbach między tymi po napisach.
% Obecnie w ogóle nie może być kwantyfikatora po napisie wewnątrz takiego po liczbie. To istotne?
% Warto to napisać wprost jeszcze raz pod spodem.
\begin{enumerate}
  \item The class \(\Sigma^{B}_{0}(\mathcal{L}) = \Pi^{B}_{0}(\mathcal{L})\) consists of all
        \(\mathcal{L}\)-formulas whose only quantifiers are \emph{bounded number quantifiers}
        (string variables may occur free).

  \item For \(i \ge 0\), the class \(\Sigma^{B}_{i+1}(\mathcal{L})\) (resp.\ \(\Pi^{B}_{i+1}(\mathcal{L})\))
        consists of formulas of the form
        \[
          \exists \vec{X} \le \vec{t}\ldotp \varphi(\vec{X})
          \quad\text{(resp.\ }\forall \vec{X} \le \vec{t}\ldotp \varphi(\vec{X})\text{)},
        \]
        where:
        \begin{enumerate}
            \item \(\vec{X}\) is a vector of string variables,
            % TODO: Def. 2.2.3 (ii) - "not involving variables from X" -
            % nasuwa się pytanie co by było, gdybyśmy dopuścili? Nic by się nie zmieniło.
            % Więc warto gdzieś tam dopisać, że to tylko założenie dla porządku, które nic nie zmienia.
            \item \(\vec{t}\) is a vector of \(\mathcal{L}^{2}_{A}\)-terms not involving variables from \(\vec{X}\),
            \item \(\varphi\) is a \(\Pi^{B}_{i}(\mathcal{L})\) formula  
                  (resp.\ a \(\Sigma^{B}_{i}(\mathcal{L})\) formula).
        \end{enumerate}

  \item A \(\Sigma^{1}_{1}(\mathcal{L})\) formula is a formula of the form
        \(
          \exists \vec{X}\ldotp \varphi,
        \)
        where \(\vec{X}\) is a vector of zero or more string variables and
        \(\varphi\) is a \(\Sigma^{B}_{0}(\mathcal{L})\) formula.
\end{enumerate}

We usually write \(\Sigma^{B}_{i}\) for \(\Sigma^{B}_{i}(\mathcal{L}^{2}_{A})\) and
\(\Pi^{B}_{i}\) for \(\Pi^{B}_{i}(\mathcal{L}^{2}_{A})\).
\end{definition}


\begin{remark}
The formalism described above coincides with the usual weak monadic second-order logic (WMSO) on
words.  We phrase it as a two-sorted first-order system only to match the presentation
in~\cite{CookNguyenDraft,Cook_Nguyen_2010}, but in practice we freely identify it with WMSO and use
that terminology when convenient, following the convention in~\cite{COOK2003193}.
% Ostatnie zdanie rozdziału, o MSO: w MSO na słowach skończonych też kwantyfikujemy po zbiorach skończonych.
% Więc oprócz tego, że zbiory są skończone warto napisać, że "pomimo iż liczby są z nieskończonego zbioru
% liczb naturalnych" (w sumie też nigdzie nie jest napisane, że to liczby naturalne, a nie całkowite,
% wymierne, czy jeszcze jakieś inne).
% W tym kontekście używa się często nazwy WMSO (weak MSO) - czy ona nie byłaby odpowiednia?
\end{remark}

\section{Functions computable by Turing machines}\label{sec:preliminaries-turing-functions}
We introduce the notion of computing a general $\{0,1\}^{\ast} \to \{0,1\}^{\ast}$ function
early, as this is the primary interest of this thesis. Most of the literature in computational complexity
focuses solely
on computing Boolean functions which we introduce in~\autoref{sec:preliminaries-turing}.
To properly discuss this imbalance, we postpone introducing the \emph{functional}
complexity classes until~\autoref{chap:reductions}.


\begin{definition}[{\cites[Definition~1.3]{10.5555/1540612}[Definition~1.4]{DRAFT10.5555/1540612}}]
Let~\(f : \{0,1\}^{\ast} \to \{0,1\}^{\ast}\) and \(T : \mathbb{N} \to \mathbb{N}\) be functions, and let
\(M\) be a Turing machine.  We say that \(M\) \emph{computes} \(f\) if, for every input
        \(x \in \{0,1\}^{\ast}\), when \(M\) is started in its initial configuration
        on input \(x\), it eventually halts with the string \(f(x)\) written on its
        output tape.
\end{definition}


\section{Decisional Turing machine complexity classes}\label{sec:preliminaries-turing}

In this section we introduce standard complexity classes such as 
\(\complexity{L}\), \(\complexity{P}\) and \(\complexity{NP}\). It is important to note that these 
classes only contain \emph{decision} problems, i.e.\ only require
computing a function \(f : \{0,1\}^\ast \to \{0,1\}\). Complexity classes for general
functions will appear in.

\begin{notation}
    We say that a machine \emph{decides} a language \(L \subseteq \{0, 1\}^\ast\) iff it computes
    the function \(f_L: \{0, 1\}^\ast \rightarrow \{0, 1\}\), where \(f_L(x) = 1 \iff x \in L\).
\end{notation}

\begin{definition}[Time complexity~{\cites[Definition~1.19]{DRAFT10.5555/1540612}[Definition~1.12]{10.5555/1540612}}]\label{def:turing-dtime}
\hfill\\
Let~\(T~:~\mathbb{N} \to \mathbb{N}\) be a function.
We define \(\complexity{DTIME}(T(n))\) to be the class of all Boolean
functions that are computable by some deterministic
Turing machine running in at most \(c_1 \cdot T(n) + c_2\) steps on every input of
length \(n\), for some constants \(c_1, c_2 > 0\).
\end{definition}

\begin{definition}[Space complexity~{\cites[Definition~4.1]{DRAFT10.5555/1540612}[Definition~4.1]{10.5555/1540612}}]\label{def:turing-dspace}
\hfill\\
Let~\(S~:~\mathbb{N} \to \mathbb{N}\) be a function and let
\(L \subseteq \{0,1\}^{\ast}\) be a language.  
% \begin{enumerate}
%   \item 
  We say that \(L \in \complexity{SPACE}(S(n))\) iff there exist constants
        \(c_1, c_2 > 0\) and a deterministic Turing machine \(M\) deciding \(L\) such that,
        on every input of length \(n\), the machine \(M\) visits at most
        \(c_1 \cdot S(n) + c_2\) distinct cells on its read-write work tapes
        (the input tape is read-only and does not count toward the space
        bound).

%   \item Likewise, \(L \in \complexity{NSPACE}(S(n))\) if there exist a constant
%         \(c > 0\) and a nondeterministic Turing machine \(M\) deciding \(L\) such
%         that, for every input of length \(n\) and for every computation branch,
%         \(M\) uses no more than \(c \cdot S(n)\) nonblank work-tape cells.
% \end{enumerate}
\end{definition}


\begin{definition}[Logarithmic space~{\cites[Definition~4.5]{10.5555/1540612}[Definition~4.5]{DRAFT10.5555/1540612}}]\label{def:complexity-l}
\[
    \complexity{L} = \complexity{SPACE}(\log n).
\]
\end{definition}


\begin{definition}[Polynomial time~{\cites[Definition~1.13]{10.5555/1540612}[Definition~1.20]{DRAFT10.5555/1540612}}]\label{def:complexity-p}
\[
    \complexity{P} = \bigcup_{c \ge 1} \complexity{DTIME}(n^c).
\]
\end{definition}



\begin{definition}[The class \texorpdfstring{\complexity{NP}}{NP}{~\cites[Definition~2.1]{10.5555/1540612}[Definition~2.1]{DRAFT10.5555/1540612}}]\label{def:complexity-np}
\hfill\\
A~language~\(L \subseteq \{0,1\}^{\ast}\) belongs to \(\complexity{NP}\) if there exist
\begin{enumerate}
  \item a polynomial \(p : \mathbb{N} \to \mathbb{N}\) (bounding the length of a certificate); and
  \item a deterministic polynomial-time Turing machine \(M\)
        (called a \emph{verifier} for \(L\)),
\end{enumerate}
such that for every input string \(x \in \{0,1\}^{\ast}\),
\[
  x \in L
  \;\Longleftrightarrow\;
  \exists\,u \in \{0,1\}^{\,p(\len{x})} \ \text{with}\ M(x,u) = 1.
\]

Whenever \(x \in L\) and a string \(u \in \{0,1\}^{p(\len{x})}\) satisfies \(M(x,u)=1\),
the string \(u\) is called a \emph{certificate} (or \emph{witness}) for \(x\)
with respect to the language \(L\) and the verifier~\(M\).
\end{definition}

\section{Classes of binary circuits}\label{sec:defs-preliminaries-circuits}


\begin{definition}
\todo[inline]{Define what is a circuit perhaps?}
\end{definition}

\begin{definition}[\NCipoly, \ACipoly, \TCipoly]\label{def:ac-nc-tc-poly}

Fix \(i \ge 0\).  
A language \(L \subseteq \{0,1\}^\ast\) belongs to one of the following
circuit classes if there exists a family of circuits
\(\{C_n\}_{n\in\mathbb{N}}\) such that \(C_n(x)=1\) iff \(x\in L\) and:

\begin{enumerate}
    \item\label{itm:circ-size} every circuit \(C_n\) has polynomially many gates w.r.t. \(n\);
    \item\label{itm:circ-depth} every circuit \(C_n\) has depth \(\bigO((\log n)^i)\);
    \item\label{itm:circ-nc} \(L \in \NCpoly{i}\) (Nick's class)  
          if each \(C_n\) uses only fan-in~2 \(\wedge\)-, fan-in~2 \(\vee\)-gates
          and fan-in~1 \(\neg\)-gates;

  \item\label{itm:circ-ac} \(L \in \ACpoly{i}\)  (Alternating circuits)
        if each \(C_n\) uses  
        unbounded fan-in \(\wedge\)-, unbounded fan-in \(\vee\)-gates and
        fan-in~1 \(\neg\)-gates;

  \item \label{itm:circ-tc}\(L \in \TCpoly{i}\)  (Threshold circuits)
        if each \(C_n\) uses unbounded fan-in \(\wedge\)-, unbounded fan-in \(\vee\)-gates,
        fan-in~1 \(\neg\)-gates and unbounded fan-in \emph{majority} gates
        (i.e.\ a gate that outputs 1 iff at least half of its inputs are one).
\end{enumerate}

Note that the condition \(C_n(x)=1\) iff \(x\in L\) requires the circuits
to have precisely one output node. We lift this requirement in~\autoref{def:aci-faci}.
% The full hierarchies are:
% \[
%   \complexity{AC}_{/poly}
%     = \bigcup_{i\ge 0} \ACpoly{i}, 
%   \qquad
%   \complexity{NC}_{/poly}
%     = \bigcup_{i\ge 0} \NCpoly{i}, 
%   \qquad
%   \complexity{TC}_{/poly}
%     = \bigcup_{i\ge 0} \TCpoly{i}.
% \]
\end{definition}


\begin{definition}[\FNCipoly, \FACipoly, \FTCipoly]\label{def:faci-poly}
A function \(F : \{0, 1\}^\ast \rightarrow \{0, 1\}^\ast\) belongs to
\FNCipoly (resp. \FACipoly, \FTCipoly) iff there exists a family of circuits
with multiple output nodes
\(
  \langle C_n \rangle_{n \in \mathbb{N}}
\)
such that whenever the input bits of \(C_n\) encode \(X \in \{0, 1\}^n\), the output bits
encode \(F(X)\); \(C_n\)
satisfies the size and depth
conditions~\ref{itm:circ-size},~\ref{itm:circ-depth} of~\autoref{def:ac-nc-tc-poly};
and additionally \(C_n\) satisfies the class condition~\ref{itm:circ-nc}
(resp.~\ref{itm:circ-ac},~\ref{itm:circ-tc}) of~\ref{def:ac-nc-tc-poly}.
\end{definition}

\begin{definition}[\compL-uniform circuit families]\label{def:logspace-uniformity}
      We say that a family of circuits \(\langle C_n \rangle_{n \in \mathbb{N}}\)
      is \compL-uniform if there exists a Turing machine operating in space \(\bigO(\log n)\),
      computing the function \(1^n \to C_n\) for some representation of \(C_n\).
\end{definition}

\begin{definition}[\NCi, \ACi, \TCi, \FNCi, \FACi, \FTCi]\label{def:aci-faci}
      A function \(F : \{0, 1\}^\ast \rightarrow \{0, 1\}^\ast\) belongs to
\FNCi (resp. \FACi, \FTCi) iff there exists a \compL-uniform family of circuits
satisfying the conditions from~\autoref{def:faci-poly}.
\end{definition}


% \subsection{Cicuit complexity classes}
% The complexity classes in this subsection are \emph{circuit complexity} classes,
% which means that the computation is done by \emph{boolean circuits}, which we will
% now introduce. Note that this is a different computational model to Turing machines, finite
% automata or lambda calculi, and thus comes with its own notion of complexity. The definitions
% in this subsection are based on~\cite[Definition 5.17]{Immerman1999-IMMDC}.



% \begin{definition}[Words accepted by a circuit]
% We say that a circuit \emph{accepts} a given binary word iff the value of its root is 1 with values of leaves
% set according to the input.
% \end{definition}

% \begin{definition}[Language decided by circuit family]
% We say that a family $\langle C_n \rangle_{n \in \mathbb{N}}$
% of boolean circuits decides a language $L$ iff for every $w \in \{0, 1\}^*$, $C_{|w|}$ accepts
% $w$ iff $w \in L$.
% \end{definition}

% \begin{definition}[\complexity{NC} circuits (Nick's class)]
% A boolean circuit is \emph{\complexity{NC}} iff the gates are only binary AND and OR gates.
% \end{definition}

% \begin{definition}[\complexity{AC} circuits (Alternating circuits)]
% A boolean circuit is \emph{\complexity{AC}} iff the gates are only unlimited fan-in AND and OR gates.
% It is a theorem that we can also allow unary NOT gates at leaves only.
% \end{definition}

% \begin{definition}[\complexity{TC} circuits (Threshold circuits)]
% A boolean circuit is \emph{\complexity{AC}} iff the gates are only unlimited fan-in Threshold(?) gates.
% It is a theorem that we can also allow unlimited fan-in AND and OR gates(?).
% \end{definition}

% \begin{definition}[\complexity{NC[t(n)]_{/poly}}, \complexity{AC[t(n)]_{/poly}}, \complexity{TC[t(n)]_{/poly}}]
% We define $\complexity{NC[t(n)]_{/poly}}$ to be the class of $\complexity{NC}$ circuits that:
% \begin{enumerate}
%    \item have polynomially-many internal nodes w.r.t. $n$, the number of leaves.
%    \item have depth $\bigO(t(n))$.
% \end{enumerate}
% Let $\complexityi{NC_{/poly}}{i} = \complexity{NC[(\log n)^i]_{/poly}}$. We define
% $\complexity{AC[t(n)]_{/poly}}, \complexity{TC[t(n)]_{/poly}}, \complexityi{AC_{/poly}}{i}, \complexityi{TC_{/poly}}{i}$ analogously.

% In particular, $\complexityi{NC_{/poly}}{0}$, $\complexityi{AC_{/poly}}{0}$, $\complexityi{TC_{/poly}}{0}$ are families of circuits with
% uniformly constant depth.

% For now we have not talked about how do we generate the consecutive circuits $C_n$.
% Later in~\autoref{sec:uniformity} we will restrict the family of (some standard) descriptions of the circuits
% to have to be \emph{computable efficiently} and then we will be able to drop the $_{/poly}$ suffix in our complexity classes.
% \end{definition}


% \section{\texorpdfstring{$\complexityi{NC}{i}_{/poly}$}{NC\string^ipoly}}

% \subsection{\complexityi{NC}{0}}
% For example, each output of an \complexityi{NC}{0} computable function can depend on only finitely many
% inputs. Thus, \complexityi{NC}{0} can't even compute an AND of all its inputs (in contrast, the unbounded
% fan-in AND is an \complexityi{AC}{0} function).
% \paragraph{TODO: Notes on $\complexityi{NC}{0}$.}
% TODO: Provide an introductory overview of the key references on $\complexityi{NC}{0}$ listed below.
% \begin{enumerate}
% \item TODO: Summarize the construction of one-way permutations in $\complexityi{NC}{0}$~\cite{10.1016/0020-01908790053-6}.
% \item TODO: Explain why all sets complete under $\complexityi{AC}{0}$ reductions are already complete under $\complexityi{NC}{0}$ reductions~\cite{10.1145/258533.258671,AGRAWAL1998127}.
% \item TODO: Describe Immerman's page~81 discussion of addition in $\complexityi{NC}{0}$ and MAJORITY in $\complexityi{NC}{1}$~\cite{Immerman1999-IMMDC}.
% \item TODO: Detail why addition and subtraction of binary numbers lies in $\complexityi{AC}{0}$~\cite{27676}.
% \end{enumerate}


% Given $x, y, z$: binary representations of natural numbers,
% it is decidable in uniform \complexityi{AC}{0} if $x + y = z$, but it is not decidable 
% in \complexityi{AC}{0} if $x * y = z$.

% \subsection{\complexityi{AC}{0}}
% \begin{enumerate}
% \item TODO: Summarize why addition is in $\complexityi{AC}{0}$~\cite{BussLectureNotes}.
% \item TODO: Explain the equivalence $\text{FO}[+, *] = \complexity{DLOGTIME}$-uniform $\complexityi{AC}{0}$ (see \url{https://complexityzoo.net/Complexity_Zoo:A#ac0}).
% \item TODO: Discuss the characterization of \complexity{DLOGTIME}-uniform $\complexityi{AC}{0}$ presented in~\cite{hella2023regularrepresentationsuniformtc0}.
% \item TODO: Review the definitions of $\complexityi{AC}{0}$, $\complexityi{TC}{0}$, FATC0, and FTC0 provided in~\cite{612309}.
% \end{enumerate}



% TODO: Check whether $\complexityi{TC}{i}$ is contained in $\complexityi{NC}{i + 1}$.
% \subsection{\complexityi{TC}{0}}
% \begin{enumerate}
% \item TODO: Confirm that multiplication is in $\complexityi{TC}{0}$~\cite{BussLectureNotes}, and locate the supporting details in~\cite{doi:10.1137/0213028}.
% \end{enumerate}

