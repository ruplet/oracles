\chapter{Definition of Neergaard's safe affine recursion}\label{chap:appendix-neergaard}

We continue to write functions as $f(\vec{x};\vec{y})$, with normal inputs
$\vec{x}$ and safe inputs $\vec{y}$. For $\delta \in \mathbb{N}$, put
$\texttt{shift}(y,\delta) = \lfloor y / 2^\delta \rfloor$, i.e.\ the number
obtained from $y$ by dropping its $\delta$ least significant bits.
Let $\mathbb{N}_2$ denote the natural numbers in binary notation.
  For $m,n \ge 0$, we write $\mathbb{N}_2^{m,n}$ for the set of functions
  with $m$ normal and $n$ safe arguments. All the arguments and the return value are in $\mathbb{N}_2$.

\begin{definition}[{\cite{10.1007/978-3-540-30477-7_21}}~Neergaard's $\complexity{BC}_\varepsilon^{-}$ algebra]\label{def:bc-eps}
  The class $\complexity{BC}_\varepsilon^{-}$ is the smallest class of functions over
  $\mathbb{N}_2$ that contains the initial functions~\ref{itm:bc-eps-zero}--\ref{itm:bc-eps-cond}
  and is closed under
  \ref{itm:neergaard-composition} and~\ref{itm:neergaard-recursion} below:
  \begin{enumerate}
    \item\label{itm:bc-eps-zero}\textbf{(zero)}
      \(
        0(;) \;=\; \varepsilon;
      \)
      \item\label{itm:bc-eps-pred}\textbf{(predecessor)}
      \(
        p(;\varepsilon) \;=\; \varepsilon,\quad
        p(; yb) \;=\; y \quad\text{for } b \in \{0,1\};
      \)
      \item\label{itm:bc-eps-proj}\textbf{(projections)}
      for $m,n \ge 0$ and $1 \le j \le m+n$,
      \(
        \pi^{m,n}_j(x_1,\dots,x_m ; x_{m+1},\dots,x_{m+n}) \;=\; x_j;
      \)
      \item\label{itm:bc-eps-succ}\textbf{(successors)}
      \(
        s_0(; y) \;=\; y0,\quad
        s_1(; y) \;=\; y1;
      \)
      \item\label{itm:bc-eps-cond}\textbf{(conditional)}
      \[
        c(; y_1,y_2,y_3) \;=\;
        \begin{cases}
          y_2 & \text{if $y_1$ ends in $1$ (i.e.\ $y_1 = z1$ for some $z$)},\\
          y_3 & \text{otherwise;}
        \end{cases}
      \]
    \item\label{itm:neergaard-composition}
      \textbf{(safe affine composition)}\
      let $f\in \complexity{BC}_\varepsilon^{-}$ have arity $(M,N)$, let
      $g_1,\dots,g_M \in \complexity{BC}_\varepsilon^{-}$ with
      $g_i$~of~arity~$(m,0)$, and let
      $h_1,\dots,h_N \in \complexity{BC}_\varepsilon^{-}$ with
      $h_i$ of arity $(m, n_i)$.
      Put $n := n_1 + \dots + n_N$.
      The \emph{safe affine composition}
      $f \circ \langle g_1,\dots,g_M ; h_1,\dots,h_N \rangle$
      is the function $F$ of arity $(m, n)$ defined by
      \[
        F(\vec{x};\vec{y}) =
        f\Bigl(g_1(\vec{x};\,),\dots,g_M(\vec{x};\,)\ ;\ 
              h_1(\vec{x};\vec{y}_1),\dots,h_N(\vec{x};\vec{y}_N)\Bigr),
      \]
      where each $\vec{y}_j$ is a (possibly empty) subtuple of the safe inputs
      $\vec{y}$ and every safe variable occurs in \emph{at most} one of the
      tuples $\vec{y}_1,\dots,\vec{y}_N$. In particular, no safe input may be
      duplicated, but some safe inputs may be unused.

    \item\label{itm:neergaard-recursion}
      \textbf{(safe affine course-of-value recursion)}\
      let $g,h_0,h_1,d_0,d_1 \in \complexity{BC}_\varepsilon^{-}$ be such that
      \[
        g : \mathbb{N}_2^{m,n} \to \mathbb{N}_2,\quad
        h_0,h_1 : \mathbb{N}_2^{m+1,1},\quad
        d_0,d_1 : \mathbb{N}_2^{m+1,0}.
      \]
      \emph{Safe affine course-of-value recursion} of them is the function
      $f : \mathbb{N}_2^{m+1,n}$ given by
      \[
      \begin{aligned}
        f(0,\vec{x};\vec{y}) &= g(\vec{x};\vec{y}),\\
        f(tb,\vec{x};\vec{y}) &=
          h_b \Bigl(t,\vec{x};\,f(\texttt{shift}(t,\delta_b),\vec{x};\vec{y})\Bigr)
      \end{aligned}
      \]
      for $b \in \{0,1\}$, where
      $\delta_b = \len{d_b(t,\vec{x};\,)}$ depends only on the normal
      arguments $(y,\vec{x})$. Here the recursive value
      $f(\texttt{shift}(t,\delta_b),\vec{x};\vec{y})$ is supplied to $h_b$ as
      its unique safe argument and can therefore be used at most once.
  \end{enumerate}
\end{definition}

\begin{theorem}[{\cite[Theorem~1]{10.1007/978-3-540-30477-7_21}}]\label{thm:neergaard-theorem}
    For any function definable in $\complexity{BC}_\varepsilon^{-}$ there is a Turing machine evaluating
    the function in \complexity{FL}. The Turing machine can be constructed from the function expression
    in logarithmic space in the size of the $\complexity{BC}_\varepsilon^{-}$-expression.
\end{theorem}



\begin{remark}[Intuition]
The recursive result is passed back in a safe
  position and can be used only once.
  Once a safe value is
  ``measured'' by the conditional operator,
    it must be recomputed; we \emph{cannot} duplicate it.
Dropping the affinity constraint collapses back to the original
\complexity{BC} algebra for \(\complexity{FP}\).
\end{remark}


\begin{remark}[Bibliography]
      In~\cite{murawski2000can}, with further refinements in~\cite{MURAWSKI2004197},
      \(\complexity{BC}^{-}\) (note the lack of $\varepsilon$ subscript) was introduced, an algebra that was contained in \complexity{FL},
      but was not known (and unlikely) to be \complexity{FL}-complete.
      In~\cite{10.1007/978-3-540-30477-7_21} this was improved to the result that
      \(\complexity{BC}_\varepsilon^{-} = \complexity{FL}\),
      with a short discussion that using course-of-value affine recursion instead of predicative affine recursion
      seems to be the reason why \(\complexity{BC}_\varepsilon^{-}\) is \complexity{FL}-complete,
      whereas \(\complexity{BC}^{-}\) is probably not.
\end{remark}

Writing programs in \(\complexity{BC}_\varepsilon^{-}\) is not reminiscent of any mainstream programming language.
The types being linear make most of the standard programming techniques not usable.
As part of our work, we have implemented an interpreter for the algebra in Haskell and tried to reproduce
some of the propositions from~\cite{10.1007/978-3-540-30477-7_21}. We present one listing, just to give
a hint of how the corresponding function look like; this is a Haskell representation of a function from the
algebra, not a programming language directly for the algebra. We present the full source code as
attachment to this thesis.\todo{How to add attachment to thesis?}

  % \begin{rawlisting}
  % \begin{Verbatim}[commandchars=\\\{\},numbers=left,firstnumber=1,stepnumber=1]
\PY{n+nf}{constZero}\PY{+w}{ }\PY{o+ow}{::}\PY{+w}{ }\PY{k+kt}{Int}\PY{+w}{ }\PY{o+ow}{\PYZhy{}\PYZgt{}}\PY{+w}{ }\PY{k+kt}{Int}\PY{+w}{ }\PY{o+ow}{\PYZhy{}\PYZgt{}}\PY{+w}{ }\PY{k+kt}{Func}
\PY{n+nf}{constZero}\PY{+w}{ }\PY{l+m+mi}{0}\PY{+w}{ }\PY{l+m+mi}{0}\PY{+w}{ }\PY{o+ow}{=}\PY{+w}{ }\PY{k+kt}{ZeroFunc}
\PY{n+nf}{constZero}\PY{+w}{ }\PY{l+m+mi}{0}\PY{+w}{ }\PY{n}{nSafe}\PY{+w}{ }\PY{o+ow}{=}
\PY{+w}{  }\PY{k+kt}{Composition}\PY{+w}{ }\PY{l+m+mi}{0}\PY{+w}{ }\PY{p}{(}\PY{n}{nSafe}\PY{+w}{ }\PY{o}{+}\PY{+w}{ }\PY{l+m+mi}{1}\PY{p}{)}\PY{+w}{ }\PY{l+m+mi}{0}\PY{+w}{ }\PY{p}{[}\PY{n}{nSafe}\PY{p}{,}\PY{+w}{ }\PY{l+m+mi}{0}\PY{p}{]}\PY{+w}{ }\PY{p}{(}\PY{k+kt}{Proj}\PY{+w}{ }\PY{l+m+mi}{0}\PY{+w}{ }\PY{l+m+mi}{2}\PY{+w}{ }\PY{l+m+mi}{2}\PY{p}{)}\PY{+w}{ }\PY{k+kt}{[]}\PY{+w}{ }\PY{p}{[}\PY{k+kt}{Proj}\PY{+w}{ }\PY{l+m+mi}{0}\PY{+w}{ }\PY{n}{nSafe}\PY{+w}{ }\PY{l+m+mi}{1}\PY{p}{,}\PY{+w}{ }\PY{k+kt}{ZeroFunc}\PY{p}{]}
\PY{n+nf}{constZero}\PY{+w}{ }\PY{n}{nNormal}\PY{+w}{ }\PY{n}{nSafe}\PY{+w}{ }\PY{o+ow}{=}
\PY{+w}{  }\PY{k+kr}{let}\PY{+w}{ }\PY{n}{g}\PY{+w}{ }\PY{o+ow}{=}\PY{+w}{ }\PY{n}{constZero}\PY{+w}{ }\PY{p}{(}\PY{n}{nNormal}\PY{+w}{ }\PY{o}{\PYZhy{}}\PY{+w}{ }\PY{l+m+mi}{1}\PY{p}{)}\PY{+w}{ }\PY{n}{nSafe}\PY{+w}{ }\PY{k+kr}{in}
\PY{+w}{  }\PY{k+kr}{let}\PY{+w}{ }\PY{n}{h}\PY{+w}{ }\PY{o+ow}{=}\PY{+w}{ }\PY{k+kt}{Proj}\PY{+w}{ }\PY{n}{nNormal}\PY{+w}{ }\PY{l+m+mi}{1}\PY{+w}{ }\PY{p}{(}\PY{n}{nNormal}\PY{+w}{ }\PY{o}{+}\PY{+w}{ }\PY{l+m+mi}{1}\PY{p}{)}\PY{+w}{ }\PY{k+kr}{in}
\PY{+w}{  }\PY{k+kr}{let}\PY{+w}{ }\PY{n}{d}\PY{+w}{ }\PY{o+ow}{=}\PY{+w}{ }\PY{k+kt}{Proj}\PY{+w}{ }\PY{n}{nNormal}\PY{+w}{ }\PY{l+m+mi}{0}\PY{+w}{ }\PY{l+m+mi}{1}\PY{+w}{ }\PY{k+kr}{in}
\PY{+w}{  }\PY{k+kt}{Recursion}\PY{+w}{ }\PY{p}{(}\PY{n}{nNormal}\PY{+w}{ }\PY{o}{\PYZhy{}}\PY{+w}{ }\PY{l+m+mi}{1}\PY{p}{)}\PY{+w}{ }\PY{n}{nSafe}\PY{+w}{ }\PY{n}{g}\PY{+w}{ }\PY{n}{h}\PY{+w}{ }\PY{n}{h}\PY{+w}{ }\PY{n}{d}\PY{+w}{ }\PY{n}{d}
\end{Verbatim}

  % \caption{Constant zero function of arity $(M, N)$ }\label{lst:neergaard-const-zero}
  % \end{rawlisting}
  % \begin{remark}
  %   Notice that the function depicted in~\autoref{lst:neergaard-const-zero} only returns 0,
  %   but nevertheless we need to use recursion to introduce normal arguments.
  % \end{remark}

\begin{rawlisting}
\begin{Verbatim}[commandchars=\\\{\},numbers=left,firstnumber=1,stepnumber=1]
\PY{c+c1}{\PYZhy{}\PYZhy{} Proposition 1. Let m and n by numbers in binary. Right shift shiftR(m : n) of}
\PY{c+c1}{\PYZhy{}\PYZhy{} m by |n| and selection of bit |n| from m are definable in BCeps\PYZhy{}.}

\PY{c+c1}{\PYZhy{}\PYZhy{} shiftR(n : m) = m \PYZgt{}\PYZgt{} |n|}
\PY{n+nf}{shiftR}\PY{+w}{ }\PY{o+ow}{::}\PY{+w}{ }\PY{k+kt}{Func}
\PY{n+nf}{shiftR}\PY{+w}{ }\PY{o+ow}{=}
\PY{+w}{  }\PY{k+kr}{let}\PY{+w}{ }\PY{n}{g}\PY{+w}{ }\PY{o+ow}{=}\PY{+w}{ }\PY{k+kt}{Proj}\PY{+w}{ }\PY{l+m+mi}{0}\PY{+w}{ }\PY{l+m+mi}{1}\PY{+w}{ }\PY{l+m+mi}{1}\PY{+w}{ }\PY{k+kr}{in}
\PY{+w}{  }\PY{k+kr}{let}\PY{+w}{ }\PY{n}{d}\PY{+w}{ }\PY{o+ow}{=}\PY{+w}{ }\PY{n}{oneNormalToZero}\PY{+w}{ }\PY{k+kr}{in}
\PY{+w}{  }\PY{c+c1}{\PYZhy{}\PYZhy{} h(timer : recursive) = tail(recursive)}
\PY{+w}{  }\PY{k+kr}{let}\PY{+w}{ }\PY{n}{h}\PY{+w}{ }\PY{o+ow}{=}\PY{+w}{ }\PY{k+kt}{Composition}\PY{+w}{ }\PY{l+m+mi}{0}\PY{+w}{ }\PY{l+m+mi}{1}\PY{+w}{ }\PY{l+m+mi}{1}\PY{+w}{ }\PY{p}{[}\PY{l+m+mi}{1}\PY{p}{]}\PY{+w}{ }\PY{k+kt}{Tail}\PY{+w}{ }\PY{k+kt}{[]}\PY{+w}{ }\PY{p}{[}\PY{k+kt}{Proj}\PY{+w}{ }\PY{l+m+mi}{1}\PY{+w}{ }\PY{l+m+mi}{1}\PY{+w}{ }\PY{l+m+mi}{2}\PY{p}{]}\PY{+w}{ }\PY{k+kr}{in}
\PY{+w}{  }\PY{k+kt}{Recursion}\PY{+w}{ }\PY{l+m+mi}{0}\PY{+w}{ }\PY{l+m+mi}{1}\PY{+w}{ }\PY{n}{g}\PY{+w}{ }\PY{n}{h}\PY{+w}{ }\PY{n}{h}\PY{+w}{ }\PY{n}{d}\PY{+w}{ }\PY{n}{d}
\end{Verbatim}

\caption{Function \texttt{shiftR} from the original paper.}\label{lst:neergaard-shiftr}
\end{rawlisting}
\begin{remark}
  The example in~\autoref{lst:neergaard-shiftr} shows the implementation of a function that
  shifts bit to the right (dropping least-significant bits). The implementation is standard,
  but it's worth to note that in the original paper, there is an error --- the function \texttt{shiftR} originally
  has flipped safe and normal arguments, making it impossible to implement. We outline our proof by contradiction for this fact.
  Consider a function $\mathtt{cond'(; x, y, z)}$ returning $y$ if $x$ is empty, $z$ otherwise.
  We can prove directly that this function is not implementable in \(\complexity{BC}^{-}\) due to
  the algebra not being able to differentiate between an empty string and a string beginning with a sufficiently
  long prefix of zeros.
  Assume the original \texttt{shiftR} is implementable. Then we can implement $\mathtt{cond'}$ using
  \texttt{shiftR}, which is a contradiction.
\end{remark}