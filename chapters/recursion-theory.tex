% TODO: why do we focus on L and P: because *reductions* chapter!

\chapter{Recursion-Theoretic Implicit Complexity}
\label{chap:recursion-theory}

\section{Recursion theory}
While not the primary focus of this work, the field of recursion theory developed concepts
that later became foundational for ICC\@. An important formal system studied there is \emph{primitive recursion}.

\begin{definition}[Primitive recursive functions]
\(\mathsf{PR}\) is the smallest class of functions containing \(\text{(i)}\)--\(\text{(iii)}\) and closed under \(\text{(iv)}\), \(\text{(v)}\).
\begin{enumerate}[label=(\roman*)]
\item \textbf{(Constants)} For every \(n\in\mathbb{N}\) and \(k\ge 0\), the \(k\)-ary constant function
      \(c_{n}^{(k)}(\vec x)=n\).
\item \textbf{(Successor)} \(S(x)=x+1\).
\item \textbf{(Projections)} For \(k\ge 1\) and \(1\le i\le k\),
      \(\pi_i^{(k)}(x_1,\dots,x_k)=x_i\).
\item[(iv)] \textbf{(Composition)} If \(h:\mathbb{N}^m\to\mathbb{N}\) and
      \(g_1,\dots,g_m:\mathbb{N}^k\to\mathbb{N}\) are in \(\mathsf{PR}\), then
      \(f(\vec x)=h\big(g_1(\vec x),\dots,g_m(\vec x)\big)\) is in \(\mathsf{PR}\).
\item[(v)] \textbf{(Primitive recursion)} If \(g:\mathbb{N}^k\to\mathbb{N}\) and
      \(h:\mathbb{N}^{k+2}\to\mathbb{N}\) are in \(\mathsf{PR}\), then the unique
      \(f:\mathbb{N}^{k+1}\to\mathbb{N}\) defined by
      \[
      f(0,\vec x)=g(\vec x),\qquad
      f(S(y),\vec x)=h\big(y,\,f(y,\vec x),\,\vec x\big).
      \]

\end{enumerate}
\end{definition}

\begin{example}[Addition]
Define \(\mathrm{Add}:\mathbb{N}^2\to\mathbb{N}\) by primitive recursion:
\[
\mathrm{Add}(0,y)=y, \qquad
\mathrm{Add}(S(x),y)=S\big(\mathrm{Add}(x,y)\big).
\]
\end{example}


\begin{definition}[LOOP language]
Let \(\mathrm{Var}=\{x_0,x_1,x_2,\dots\}\).
LOOP programs are generated by the grammar
\[
\begin{aligned}
P ::=~& x_i := 0
\;\mid\; x_i := x_i + 1
\;\mid\; P \,;\, P
\;\mid\; \texttt{LOOP}~x_i~\texttt{DO}~P~\texttt{END},
\end{aligned}
\]
where \(x_i\in\mathrm{Var}\).

\noindent
We assume standard semantics, with a remark that \(\texttt{LOOP}~x_i~\texttt{DO}~P~\texttt{END}\) repeats \(P\) exactly as many times as the value stored in \(x_i\) \emph{at loop entry} (changes to \(x_i\) inside \(P\) do not change the iteration count).
\end{definition}

Interestingly, in~\cite{10.1145/800196.806014} it has been shown that the functions definable by LOOP programs
are precisely the primitive recursive functions.
This simple example actually satisfies our criteria of a `programming language capturing a complexity class', as
the LOOP language captures exactly the primitive recursive functions\footnote{As recognized in \url{https://complexityzoo.net/Complexity_Zoo:P}.}.
Moreover, we can even stratify the primitive recursive functions into a hierarchy like in~\cite{Grzegorczyk1953}.


Historically, the origins of primitive recursion can be traced back to~\cite{Grassmann1861} and~\cite{Dedekind1888},
but the class was probably first considered as the primary object of study in~\cite{Skolem1923-vanHeijenoort}.
For the details of the historical origins, consult~\cite{Adams2011}.
