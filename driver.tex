% Template source: https://www.mimuw.edu.pl/pl/informator-dla-studentow/prace-i-egzaminy-dyplomowe-odbior-dyplomu/

% Niniejszy plik stanowi przykład formatowania pracy magisterskiej na
% Wydziale MIM UW.  Szkielet użytych poleceń można wykorzystywać do
% woli, np. formatujac wlasna prace.
%
% Zawartosc merytoryczna stanowi oryginalnosiagniecie
% naukowosciowe Marcina Wolinskiego.  Wszelkie prawa zastrzeżone.
%
% Copyright (c) 2001 by Marcin Woliński <M.Wolinski@gust.org.pl>
% Poprawki spowodowane zmianami przepisów - Marcin Szczuka, 1.10.2004
% Poprawki spowodowane zmianami przepisow i ujednolicenie
% - Seweryn Karłowicz, 05.05.2006
% Dodanie wielu autorów i tłumaczenia na angielski - Kuba Pochrybniak, 29.11.2016
% drobne poprawki i komentarze - Paweł Goldstein, 24.04.2025

% dodaj opcję [licencjacka] dla pracy licencjackiej
% dodaj opcję [en] dla wersji angielskiej (mogą być obie: [licencjacka,en])
\documentclass[en]  {pracamgr}
%include standalone forest, tikz figures
\usepackage{standalone}

% TODO: CSL doesn't work with Biblatex. Use luatex, citeproc for it: https://github.com/zepinglee/citeproc-lua
% Select citation style (CSL sources live under thesis-tex/).
% https://github.com/citation-style-language/styles/blob/2e3e9964846c9a1ad709e6974307b85f70a37f06/apa.csl
% \PassOptionsToPackage{style=apa}{biblatex} % default APA, see thesis-tex/apa.csl
% https://github.com/citation-style-language/styles/blob/2e3e9964846c9a1ad709e6974307b85f70a37f06/ieee.csl
% \PassOptionsToPackage{style=ieee}{biblatex} % switch to IEEE, see thesis-tex/ieee.csl
% https://github.com/citation-style-language/styles/blob/2e3e9964846c9a1ad709e6974307b85f70a37f06/chicago-author-date-17th-edition.csl
% \PassOptionsToPackage{style=chicago-author-date-17th-edition}{biblatex}

% \usepackage[pl]{babel}
\usepackage[
  backend=biber,
  style=alphabetic,
  bibencoding=utf8 % this is default, but make it explicit
]{biblatex}
% recommended while using biblatex with babel
\usepackage{csquotes}
% make references clickable, hide boxes around links
\usepackage[hidelinks, pdfpagelabels]{hyperref}

% Ensure correct handling and display of accented and non-ASCII characters:
% - fontenc[T1]: use 8-bit fonts with full Latin alphabet (proper hyphenation, PDF copy/paste)
% - inputenc[utf8]: tell LaTeX that the source files are saved in UTF-8 encoding
% - lmodern: load Latin Modern fonts (Unicode-aware version of Computer Modern)
\usepackage[T1]{fontenc} 
\usepackage[utf8]{inputenc}
\usepackage{lmodern}

\usepackage{amsmath,amssymb,amsthm}
\usepackage{mathtools}
\usepackage{microtype}
\usepackage{proof}
\usepackage{enumitem}
\usepackage{xspace}

% Package todonotes Warning: The length marginparwidth is less than 2cm and will 
% most likely cause issues with the appearance of inserted todonotes. The issue c
% an be solved by adding a line like \setlength {\marginparwidth }{2cm} prior to 
% loading the todonotes package. on input line 228.
\setlength{\marginparwidth}{2cm}
% TODOs in draft
% size=\scriptsize
\usepackage[colorinlistoftodos,textsize=scriptsize]{todonotes}
\usepackage{xcolor}
\setuptodonotes{
  color=orange!20,     % lighter background
  linecolor=orange!50, % border
  bordercolor=orange!50,
  textcolor=black,
  caption={TODO},      % text used in list of todos
  prepend={TODO:} % prefix inside note
}

% theorem-like environments
\theoremstyle{plain}
\newtheorem{theorem}{Theorem}[chapter]
% \newtheorem{lemma}[theorem]{Lemma}
\newtheorem{corollary}[theorem]{Corollary}

\theoremstyle{definition}
\newtheorem{example}[theorem]{Example}
\theoremstyle{definition}
\newtheorem{definition}{Definition}[section]
\theoremstyle{plain}
\newtheorem{proposition}[definition]{Proposition}
\theoremstyle{remark}
\newtheorem{remark}[theorem]{Remark}

% Workaround for "duplicate destination name{page.i}" on front matter:
% disable page anchors just for the title pages, then re-enable.
\usepackage{etoolbox}
\makeatletter
\pretocmd{\maketitle}{\hypersetup{pageanchor=false}}{}{}
\apptocmd{\tableofcontents}{\clearpage\hypersetup{pageanchor=true}}{}{}
\makeatother

\addbibresource{chapters/foundations.bib}
\addbibresource{chapters/reductions.bib}
\addbibresource{chapters/linear-logic.bib}
\addbibresource{chapters/bounded-arithmetic.bib}
\addbibresource{chapters/descriptive-complexity.bib}
\addbibresource{chapters/recursion-theory.bib}


\newcommand{\llbracket}{\mathopen{[\![}}
\newcommand{\rrbracket}{\mathclose{]\!]}}

% \addbibresource{references.bib}

% https://tex.stackexchange.com/a/404990
% https://mirror.aarnet.edu.au/pub/CTAN/macros/latex/contrib/hyperref/doc/hyperref-doc.html#x1-140004
\addto\extrasenglish{%
  \def\chapterautorefname{Chapter}%
}
\addto\extrasenglish{%
  \def\chaptername{Chapter}%
}

\addto\extrasenglish{%
  \def\sectionautorefname{Section}%
}
\addto\extrasenglish{%
  \def\sectionname{Section}%
}
\usepackage{subcaption}
\usepackage{listings}
\lstset{
  basicstyle=\ttfamily,
  columns=fullflexible,
  frame=single,
  breaklines=true,
  literate={ą}{{\k a}}1
  	  {Ą}{{\k A}}1
           {ż}{{\. z}}1
           {Ż}{{\. Z}}1
           {ź}{{\' z}}1
           {Ź}{{\' Z}}1
           {ć}{{\' c}}1
           {Ć}{{\' C}}1
           {ę}{{\k e}}1
           {Ę}{{\k E}}1
           {ó}{{\' o}}1
           {Ó}{{\' O}}1
           {ń}{{\' n}}1
           {Ń}{{\' N}}1
           {ś}{{\' s}}1
           {Ś}{{\' S}}1
           {ł}{{\l}}1
           {Ł}{{\L}}1
}
\usepackage{url}
\usepackage{xr}
\usepackage{tabularx}
\usepackage{booktabs}
\usepackage{graphicx}
\usepackage{forest}
\usepackage{tikz-qtree}
% https://tex.stackexchange.com/a/18927


\usepackage{pdflscape}

\usepackage[titletoc]{appendix}
\usepackage{amsthm}


\autor{Paweł Balawender}{429141}

\title{Practical programming languages capturing complexity classes}
\titlepl{Praktyczne języki programowania wyrażające klasy złożoności}
% \tytulang{An implementation of a difference blabalizer based on the theory of $\sigma$ -- $\rho$ phetors}

%kierunek: 
% - matematyka, informacyka, ...
% - Mathematics, Computer Science, ...
\kierunek{Computer Science}

% informatyka - nie okreslamy zakresu (opcja zakomentowana)
% matematyka - zakres moze pozostac nieokreslony,
% a jesli ma byc okreslony dla pracy mgr,
% to przyjmuje jedna z wartosci:
% {metod matematycznych w finansach}
% {metod matematycznych w ubezpieczeniach}
% {matematyki stosowanej}
% {nauczania matematyki}
% Dla pracy licencjackiej mamy natomiast
% mozliwosc wpisania takiej wartosci zakresu:
% {Jednoczesnych Studiow Ekonomiczno--Matematycznych}

% \zakres{Tu wpisac, jesli trzeba, jedna z opcji podanych wyzej}

% Praca wykonana pod kierunkiem:
% (podać tytuł/stopień imię i nazwisko opiekuna
% Instytut
% ew. Wydział ew. Uczelnia (jeżeli nie MIM UW))
\opiekun{dr hab. Paweł Parys, prof. UW\\
  Institute of Informatics, University of Warsaw\\
  }

% miesiąc i~rok:
\date{July~2025}

%Podać dziedzinę wg klasyfikacji Socrates-Erasmus:
\dziedzina{ 
%11.0 Matematyka, Informatyka:\\ 
%11.1 Matematyka\\ 
%11.2 Statystyka\\ 
11.3 Computer Science\\ 
%11.4 Sztuczna inteligencja\\ 
%11.5 Nauki aktuarialne\\
%11.9 Inne nauki matematyczne i informatyczne
}

% Klasyfikacja tematyczna wedlug AMS (matematyka) lub ACM (informatyka)
% ACM: https://www.acm.org/publications/class-2012
\klasyfikacja{F. Theory of computation\\
  F.3. Logics and meanings of programs\\
  F.3.3. Studies of program constructs}

% Słowa kluczowe:
\keywords{TODO keywords at the end}

% Tu jest dobre miejsce na Twoje własne makra i~środowiska:
% \newtheorem{defi}{Definicja}[section]
\newcommand{\bcem}{\ensuremath{\text{BC}_{\varepsilon}^-}}

% https://tex.stackexchange.com/a/354690
\DeclareRobustCommand{\bigO}{%
  \text{\usefont{OMS}{cmsy}{m}{n}O}%
}

% We will not use `complexity' CTAN package, because it is wacky
% https://tex.stackexchange.com/q/391604
\newcommand{\complexity}[1]{\ensuremath{\mathsf{#1}}\xspace}
\newcommand{\complexityi}[2]{\ensuremath{\mathsf{#1}^{#2}}\xspace}


% koniec definicji

\includeonly{chapters/foundations}

\begin{document}
\hypersetup{pageanchor=false}
\maketitle

\begin{abstract}
    In this work, I study which features make sense to add to a programming language
    from the computational complexity perspective. Specifically, 
    I focus on how these features can be used to capture various complexity classes, 
    such as LOGSPACE, PTIME, and PSPACE\@. By analyzing the expressiveness and limitations 
    of these features, I aim to provide insights into the design of programming languages
     that are both practical and theoretically sound.
\end{abstract}

\hypersetup{pageanchor=true}
\tableofcontents
%\listoffigures
%\listoftables

\chapter{Models of computation, programming paradigms and complexity measures}\label{chap:foundations}
One of the first decisions a programming language designer has to make is choosing
the programming paradigm convenient for writing the programs of interest.
In the practice of programming, imperative languages
have no competition when a user needs to reason about the computational complexity of
programs.
The structure of imperative programs closely mirrors how the computation
is executed on modern CPUs and GPUs.
In turn, it is inherently unintuitive to reason about the
complexity of programs written e.g.\ in Haskell or Prolog.

Yet if we want to understand what classes of functions can be characterized syntactically,
we have to temporarily step away from the imperative mindset. As argued in~\cite{10.1007/978-3-642-27660-6_3},
the very notion of an ``algorithm'' is still evolving, so we shouldn't limit our considerations
to a single paradigm. This chapter tests whether
alternative models of computation could be better suited
to form the basis of languages that capture popular complexity classes.

We can foreshadow that the answer is (perhaps surprisingly) positive: even though the complexity classes are defined on Turing machines,
the characterizations studied in literature are rarely imperative. 

\section{Finite automata and transducers}
Finite-state transducers compute string-to-string functions and have simple descriptions.
In~\cite{bojańczyk2018polyregularfunctions},
four characterizations are given for the class of \emph{polyregular} functions --- a~class of string-to-string
functions computed by a particular kind of transducer. The definitions described there readily
constitute the basis of a programming language. Another programming
language for transducers is studied in~\cite{DBLP:conf/fsmnlp/Schmid05}. A particular
class of string-to-string functions defined using logic, \emph{MSO transductions},
is characterized to be precisely the class of functions computable by two-way deterministic finite transducers (2DFT)
in~\cite{engelfriet1999msodefinablestringtransductions}. All of these results
provide  basis for new programming languages and are therefore more than relevant to our work.

Because transducer classes remain poorly understood, it is usually unclear whether an arbitrary problem 
belongs to the class recognized by a given flavor of transducer.
Writing programs in such a programming language would therefore often be equivalent
to doing new research on transducers, so for now we focus
on more well-established classes of functions. A~good overview of existing research on
transducers can be found in~\cite{muscholl_et_al:LIPIcs.STACS.2019.2}.

The existing research on finite-state automata has not been directly useful for our work ---
perhaps because the expressiveness of this model is inherently limited to Boolean-valued functions.

\section{Turing machines}
This model of computation underpins the imperative programming style.
There is a variety of flavors of Turing machines, and the details of a specific definition will most
of the time not affect our considerations in this work. When not explicitly stating otherwise,
when describing a computational process we will implicitly assume the realization of it on some kind
of a Turing machine.

\begin{definition}[Time complexity]\label{def:turing-dtime}
Let \(T : \mathbb{N} \to \mathbb{N}\) be some function.  A language \(L\) belongs to \(\complexity{DTIME}(T(n))\) iff there exists a deterministic Turing machine that satisfies:
\begin{enumerate}[label= (\roman*), ref= (\roman*)]
    \item\label{itm:dtime} for some constant \(c > 0\) and every input \(w \in \{0,1\}^{n}\) it terminates within \(c \cdot T(n)\) steps;
    \item for every \(w \in L\) the machine outputs \(1\);
    \item for every \(w \notin L\) the machine outputs \(0\).
\end{enumerate}
We will say that a Turing machine belongs to \(\complexity{DTIME}(T(n))\) iff it satisfies~\ref{itm:dtime}.
\end{definition}

\begin{definition}[Space complexity]\label{def:turing-dspace}
Let \(T : \mathbb{N} \to \mathbb{N}\) be some function.  A language \(L\) belongs to \(\complexity{DSPACE}(T(n))\) iff there exists a deterministic Turing machine that satisfies:
\begin{enumerate}[label= (\roman*), ref= (\roman*)]
    \item\label{itm:dspace} for some constant \(c > 0\) and every input \(w \in \{0,1\}^{n}\) it halts while using at most \(c \cdot T(n)\) work-tape cells;
    \item for every \(w \in L\) the machine outputs \(1\);
    \item for every \(w \notin L\) the machine outputs \(0\).
\end{enumerate}
We will say that a Turing machine belongs to \(\complexity{DSPACE}(T(n))\) iff it satisfies~\ref{itm:dspace}.
\end{definition}

The most popular complexity classes are directly based on the above notions.
Yet surprisingly few elegant, high-level
imperative languages are known to characterize well-studied complexity classes beyond the trivial ones.
This scarcity is precisely why much of the work surveyed in later chapters relies on non-imperative paradigms.

\subsection{Random-access Turing machines}
Random-access Turing machines are Turing machines with a special ``pointer'' tape,
of length logarithmic to the size of input, and a special state such that when the binary
number on the pointer tape is \(n\), \(n\)-th digit of the input is written to the work tape.

Reasoning about computation in complexity classes such as \(\complexity{L}\) or \(\complexity{P}\) is the same
for traditional Turing machines and random-access Turing machines. The choice of model starts to matter for
notions of complexity such as \(\complexity{DTIME}(n)\) or \(\complexity{DTIME}(n^2)\) (\emph{fine-grained} complexity classes).
Due to insufficient existing research on implicit characterizations of
fine-grained complexity classes, we will not consider them besides a brief discussion
in~\autoref{subsec:fine-grained-reductions}. Computation on transducers is a promising way to obtain robust characterizations of these
classes --- e.g.~\emph{polyregular} functions are computable in linear time by a kind of a transducer~\cite{bojańczyk2018polyregularfunctions}.
This is not a \emph{characterization}, however, because it is unlikely that every \(\complexity{DTIME}(n)\) function is definable in this language.
The notion of \(\complexity{DLOGTIME} = \complexity{DTIME}(\log n)\)-uniformity explored in~\autoref{subsec:dlogtime-uniformity} is also
only defined for random-access Turing machines.

\section{Circuits}
Circuits as a model of computation are the theoretical foundation of parallel programming.
The theoretical implications turn out to not be widely applicable in practice, however.
In this work, we will mostly use circuits to reason about very weak complexity.

As we will see in~\autoref{sec:uniformity}, we will almost always want to 
define a circuit family by a function (of a low complexity) \(n \rightarrow C_n\), computing the description
of the circuit for a given input size \(n\). If we obtain a language for
such functions, we will be able to use it to \todo{compute with complexity bound,
i.e.\ that if the resulting circuit is correct, it will be logspace-uniform ac0}compute circuit families. We will study
a logical characterization of a variant of \(\complexity{AC}^0\) in~\autoref{sec:vac0}.

A very rich \todo{there is nothing new for me
in this paper; i have also a very good literature overview, but scattered around this PDF}
overview of different characterizations of circuit complexity classes is
in~\cite{antonelli2025characterizingsmallcircuitclasses}.

\section{Discrete differential equations}
An original point of view on computation is to describe functions
as solutions to discrete differential equations. For example, in~\cite{bournez_et_al:LIPIcs.MFCS.2019.23},
\(\complexity{FP}\)~(\autoref{def:fp}) and \(\complexity{FNP}\)~(\autoref{def:fnp}) are characterized.
Characterizations of various circuit classes from
\(\complexity{FAC}^0\) to \(\complexity{FAC}^1\)~(\autoref{def:faci}) 
are described in~\cite{antonelli2025characterizingsmallcircuitclasses}.

\section{Logic programming and descriptive complexity}
Logic provides very deep complexity-theoretic connections,
primarily through descriptive complexity theory, which we explore in more detail in~\autoref{chap:descriptive-complexity}.

\section{Untyped recursion}\label{sec:untyped-lambda}
Another classical paradigm is that of general recursive functions, or equivalently the untyped
lambda calculus. Because these systems are Turing complete, the interesting question
is how to constrain recursion so that the resulting language captures a specific class.
We treat these ideas in~\autoref{chap:recursion-theory}.

\section{Typed lambda calculus}
Typed lambda calculus underpins functional programming. In this section we focus on typed variants,
unlike in~\autoref{sec:untyped-lambda}.

Lambda calculus does
not line up cleanly with traditional Turing machine complexity measures.  For instance,
for some representation of strings \(\{0, 1\}^\ast\),~\cite[Theorem~3.4]{10.5555/788018.788832} identifies
the functions \(\{0, 1\}^\ast \rightarrow \{0, 1\}\) definable in simply typed
lambda calculus (STLC), with regular languages. But with a different encoding of inputs,~\cite{HILLEBRAND1996117} relates STLC to the whole \(\complexity{ELEMENTARY}\) class. 
Moreover,~\cite{zakrzewski2007definablefunctionssimplytyped} states that
with a different ``standard'' encoding, STLC instead characterizes extended polynomials, and further shows
that if we slightly modify the encoding, the class is yet different.
For more discussion, see also~\cite{27863}.
Consequently, it is not obvious how to reason about complexity theory in the language of lambda calculus.

Nevertheless, typed lambda calculi have been utilized
very successfully to syntactically characterize complexity classes.
Recall that one of the reasons linear logic is studied is the potential
to reason about resource creation and utilization.
Concepts from linear logic have been implemented in the theory of type systems
to transfer the resource interpretation.
This will be discussed further in~\autoref{chap:linear-types}.

\section{Set theory as inspiration for model of computation}
An interesting connection appears when we think of traditional notions of ``complexity'' of sets in set theory
from the point of view of computational complexity. 
If we treat taking complements, intersections, countable unions as operations in a programming language,
perhaps we could design a programming language for constructing sets. By itself such a language would probably
not be the most interesting one. However, thinking about mathematical reasoning
in terms of a computational process is a very powerful technique. It has been deeply explored
under the name of Curry-Howard or proofs-as-programs correspondence.\footnote{A good introduction to the
immensely deep topic of proofs-as-programs is~\cite{10.5555/1197021}.} For the computational content
of set theory in particular, an interesting brief discussion is presented in~\cite{Tao2010ComputationalPerspective}.

For our purposes, an interesting connection appears when we look at descriptive set theory.
The most basic classes of sets distinguished there are open and closed sets. Slightly higher,
an \(F_\sigma\) set is a countable union of closed sets; a \(G_\delta\) set is a countable intersection
of open sets. A~few levels higher up the \emph{Boldface hierarchy}, which in a way quantify 
the complexity of sets, Borel sets are considered:

\begin{definition}[Borel sets]
Let \(X\) be a topological space. The class of Borel sets of \(X\), \(\mathcal{B}(X)\) is the smallest class of sets
containing every open set of \(X\) and closed under~\ref{itm:borel-comp} and~\ref{itm:borel-union}:
\begin{enumerate}[label= (\roman*), ref= (\roman*)]
    \item\label{itm:borel-comp} if \(A\) is Borel, then its complement \(X \setminus A\) is Borel;
    \item\label{itm:borel-union} if \(A_n\) is Borel for each \(n \in \mathbb{N}\), then the countable union \(\bigcup_{n \in \mathbb{N}} A_n\) is Borel.
\end{enumerate}
\end{definition}

As this is a standard least fixed point definition, it is strikingly similar to
definitions of classes of recursive functions that we will consider later, e.g.~\autoref{def:primitive-recursive}.
We can also take a computational point of view on theorems about the determinacy of Gale-Stewart games (which we shall not introduce here).
It is widely known that Gale-Stewart games are determined when the underlying set is open or closed.
Allowing the underlying set to be more and more complicated, we quickly 
reach the limits of provability in Zermelo-Fraenkel set theory: determinacy for Borel sets is a
difficult theorem, and determinacy for analytic and projective sets is
independent of ZF, yet provable assuming as axiom the existence of an appropriately large cardinal.

A good question to ask is if by carefully curating the axioms, we could
obtain a mathematical theory such that theorems corresponding
to computation in our desired complexity class are provable, and
the theorems that wouldn't be ``implementable'' are not.

We will circle back
to this intuition while considering the \(\texttt{PIGEON}\) computational problem in~\ref{subsubsec:tfnp} and
the unprovability of the related pigeonhole principle in the weak theories studied in~\ref{subsec:vac0-php}.
Especially the last fact about unprovability is interesting for us in this section,
as it resembles e.g.\ independence of continuum hypothesis from ZFC, but in a strictly computational setting
of not being able to perform enough computation in a low complexity class.
% This is also a hint that going this route, a collection of powerful tools used in logic to prove
% independence, will become available to us as tools for proving a problem not being solvable
% in a given complexity class.

While this line of thought is very far from ``practical programming languages'', these considerations
inspired\footnote{This line of study grew out of presenting the topic
at the JAiO master's seminar at the University of Warsaw ---
slides are available online at~\cite{Balawender2025BorelDeterminacySeminar}.}
the approach that we study in~\autoref{chap:bounded-arithmetic}.


% \section{Models of hypercomputation}
% Notes (for future work):
% \begin{itemize}
% \item Sequential Time. An algorithm determines a sequence of computational
% states for each valid input. Specifically, the time is discrete. what if time is not a successor structure? i.e. we can travel back in Time
% \item Malament-Hogarth spacetime.
% \end{itemize}

% todo: coin the term Oracle Oriented Programming
\chapter{Reductions}\label{chap:reductions}
Plenty of theorems of the form ``problem $P$ is \emph{complete} for class $C$ under
reductions in class $R$'' have been described in the literature. In this chapter
we analyze when are they useful to capture complexity classes syntactically,
and when they don't suffice.

\begin{definition}[Circuit Value Problem (\problem{CVP})]
  Given a representation of a Boolean circuit on input, compute its output.
  \todo[inline]{more formality}
\end{definition}

\begin{theorem}[%
  {\cite{10.1145/990518.990519}}%
]\label{thm:cvp-pcomplete}
  The circuit value problem is complete for \compP{} under \compL{} reductions.
  \begin{remark}
    We define \compL-reductions formally in~\autoref{def:logspace-reductions}.
  \end{remark}
\end{theorem}

Knowing~\autoref{thm:cvp-pcomplete}, we could design a language for \compP{} in such a way:
first, design a language for \compL{} which is (perhaps) simpler; then, as the compiler provider,
include a standard library function \(P\), which
the \compL-functions from the language could call like an oracle to get solution for \(P\).
In this chapter we try to understand
if such a language would truly have the full power of polynomial-time functions.
The core of this chapter is~\autoref{thm:fl-is-l-red}, stating that this holds at least for the class
of logspace-computable functions and a particular notion of weak reductions. But the heart
of this intuition is not developed until~\autoref{thm:fc-is-fac0-closure}, where we get
such characterizations for most of the complexity classes that we consider in this thesis.

\section{Decisional and functional complexity classes}\label{sec:decisional-and-functional-complexity-classes}
We can't answer the question of expressive power of \(\problem{CVP} + \compL \text{-reductions}\)
by comparing it with any decisional class.
Focusing solely on the complexity of Boolean functions
as we did for now e.g.\ in~\autoref{sec:preliminaries-turing} is 
not sufficient to reason about general functions with output.
In this chapter we will introduce the still standard, but less talked about, 
\emph{functional} complexity classes,
that study general functions \(f : \{0,1\}^\ast \to \{0,1\}^\ast\). In complexity theory these are
usually implicitly used to define \emph{reductions}. The results
about these classes  transfer much better to our interests than results about the decisional complexity
classes. As we discuss  in~\autoref{chap:recursion-theory}, programming-language-like characterizations
of decisional complexity classes are far more abundant and predate the characterizations of
functional complexity classes. It is the latter, however, that is viable for our purposes of
designing a programming language.

A thorough overview of complexity classes of functions is described in~\cite{SELMAN1994357}.
A more thorough discussion of decision vs search is in~\cite{10.5555/889581}.

\paragraph{Exponential-length output}
The difference between \compP{} and \compFP{} is obvious when we look at the below example:
\begin{example}
    Running time of any Turing machine computing the function \(x \rightarrow 2^x\) for input and output in binary,
    is exponential. At the same time, given an input \(x, y\), checking if \(y = 2^x\) is easily in polynomial time.
\end{example}
This problem is, however, usually artificially mitigated by requiring the length \(|f(x)|\)
of the function's output to be polynomially bounded as in~\autoref{def:logspace-reductions}.


\paragraph{Self-reducibility}
A typical, and ubiquitous in the literature way of defining function problems is to require
an external proof that \(|f(x)|\) is polynomially bounded,
then repeatedly decide if \(i\)-th bit of \(f(x)\) is 0 or 1. For example, let's look at the below

\begin{definition}[{\cites[Definition~4.16]{10.5555/1540612}[Definition~4.14]{DRAFT10.5555/1540612}}~\compL-reductions]\label{def:logspace-reductions}
A function \(f : \{0,1\}^{\ast} \to \{0,1\}^{\ast}\) is said to be
\emph{logspace computable} if
\begin{enumerate}
  \item \(f\) is \emph{polynomially bounded}, meaning that there exists a
        constant \(c > 0\) such that
        \[
          |f(x)| \le |x|^{\,c}
          \qquad\text{for all } x \in \{0,1\}^{\ast},
        \]
        and
  \item the following two languages lie in \(\mathsf{L}\):
        \[
          L_{f}
            = \{\,\langle x,i\rangle \mid f(x)_{i} = 1\,\},
          \qquad
          L'_{f}
            = \{\,\langle x,i\rangle \mid i \le |f(x)|\,\}.
        \]
\end{enumerate}

In other words, a deterministic \(\bigO(\log |x|)\)-space machine can, given
\((x,i)\), determine whether \(i\) is within the length of \(f(x)\) and, if so,
whether the \(i\)-th bit of \(f(x)\) is \(1\).

A language \(B\) is \emph{logspace reducible} to a language \(C\), if there exists
a function
\(f : \{0,1\}^{\ast} \to \{0,1\}^{\ast}\) that is logspace
computable and such that \(x \in B\) iff \(f(x) \in C\) for every
\(x \in \{0,1\}^{\ast}\).
\end{definition}


Famously, we can also apply this trick to solve the problem \problem{FSAT}, the corresponding functional
problem to the \problem{SAT} of finding a specific
satisfying assignment with polynomially many calls to the decision procedure \problem{SAT}. 
In general, many functional problems are solvable in polynomial time with polynomially
many calls to their corresponding decisional problems. We say that
problems with that property are \emph{self-reducible}.

However, some search problems are unlikely to be self-reducible.
A good example is the problem of integer factorization, which
is still, as of November 2025, conjectured to not be in \compP even despite
the recent breakthrough in which \problem{PRIME} was proved to be in \compP.
A particularly important class of such problems is considered in~\ref{subsubsec:tfnp}.
But first, let's define the classes \complexity{FP} and \complexity{FNP}.

\section{Functional complexity classes}\label{sec:functional-complexity-classes}

\todo{perhaps define poly-time computable functoins in preliminaries, and here just link to that}
\begin{definition}[{\cites[Section~17.2]{10.5555/1540612}[Section~9.1]{DRAFT10.5555/1540612}}~The class \complexity{FP} (Version 1)]\label{def:fp-ver1}
  \(\mathsf{FP}\)~consists of all functions
\[
  f : \{0,1\}^{\ast} \to \{0,1\}^{\ast}
\]
that are computable by a deterministic polynomial-time Turing machine.
In contrast to decision problems (which output a single bit), functions
in \(\mathsf{FP}\) may produce outputs of arbitrary polynomial length.

\begin{remark}
  This definition is used e.g.\ in~\cite{COOK19852} besides the work cited above.
  It is rather equivalent to~\autoref{def:fp} below, also ubiquitous in the literature.
\end{remark}
\end{definition}

\todo[inline]{I have a definition of also poly-time reductions e.g. for NP in tex, but seems like i'm not using it anywhere}
\begin{definition}[{\cite[Section 28.10]{Rich2007Automata}}~\complexity{FP} (Version 2)]\label{def:fp}

A binary relation \(P(x, y)\) is in \(\complexity{FP}\) iff there exists a polynomial-time Turing machine
that, given an arbitrary input \(x\):
\begin{enumerate}[label=(\roman*)]
    \item outputs some \(y\) such that \(P(x, y)\) if any exists;
    \item signals that no such \(y\) exists otherwise.
\end{enumerate}

\begin{remark}
  This version can make it more obvious to compare \(\complexity{FP}\) with \(\complexity{FNP}\) (defined later) ---
  but only assuming that \(\complexity{FNP}\) is defined using nondeterministic Turing machines,
  which is not true in our case; we will use the \emph{verifier}-style definition.
\end{remark}


% transducers of polynomial growth are studied in~\cite{10.1145/3531130.3533326}, where also
% are given pebble, functinal, impearitve and logical models of computation.
\end{definition}


\begin{remark}[\complexity{P} vs \complexity{FP}]
These two classes are often identified due to similar properties.
The notion of completeness for both of them, despite being differently defined, practically is practically the same
to the robustness of \(\complexity{P}\) computations under being repeated for every bit of the output.
Indeed, even in Stephen Cook's 1982 ACM Turing Award lecture~\cite[Section 6]{10.1145/358141.358144},
it is not clearly distinguished between
\(\complexity{P}\)-completeness and \(\complexity{FP}\)-completeness: the 3 proofs cited in this lecture
as proofs of \(\complexity{FP}\)-completeness of some functions \(f(x)\) only themselves prove the
\(\complexity{P}\)-completeness of problems of the form ``decide if \(i\)-th bit of the result \(f(x)\) is zero''.

The two classes, however, are not the same.
In~\cite[Theorem 4.1]{KRENTEL1988490}, it is proved that
if 
\(\complexity{FP}^{\complexity{SAT}}[\bigO(\log{n})] = \complexity{FP}^{\complexity{SAT}}[n^{\bigO(1)}]\)
then also \(\complexity{P}=\complexity{NP}\).
In turn, as noted in~\cite[discussion after Theorem 8]{doi:10.1142/9789812794499_0029}, the corresponding result for
\(\complexity{P}^\complexity{NP}\) versus \(\complexity{P}^\complexity{NP}[\bigO(\log{n})]\) is not known,
and indeed fails relative to some oracles.

For a good discussion specifically on \complexity{FP}-completeness,
which is relatively hard to find, there is an argument that finding the lexicographically
first maximal clique in an undirected graph
is \complexityi{NC}{i}-complete for \complexity{FP} in~\cite[Proposition~6.1]{COOK19852}.
\end{remark}

% POLYTIME-REDUCIBILITY
% We will say that a function is \emph{polynomial-time computable} if it is
% computable by a Turing machine running in polynomial time, not using the
% decider trick for \emph{logspace computability} in~\autoref{def:logspace-reductions}.
% \begin{definition}[\compP-reductions~{\cites[Definition~2.7]{10.5555/1540612}[Definition~2.7]{DRAFT10.5555/1540612}}]\label{def:p-reductions}
% Let \(A,B \subseteq \{0,1\}^{\ast}\) be languages.  
% We say that \(A\) is \emph{polynomial-time Karp (many-one) reducible} to \(B\),
% if there exists a polynomial-time
% computable function
% \[
%   f : \{0,1\}^{\ast} \to \{0,1\}^{\ast}
% \]
% such that for every input \(x \in \{0,1\}^{\ast}\),\todo{consistency: \(\iff\) vs \(\Longleftrightarrow\)}
% \[
%   x \in A
%   \;\Longleftrightarrow\;
%   f(x) \in B.
% \]
% In this case, the function \(f\) is called a \emph{polynomial-time reduction}
% from \(A\) to \(B\).
% \end{definition}









\subsection{\complexity{FNP}}
The definition of \complexity{FNP} is tricky to get right.
A very good discussion of the awkwardness of the definitions is present in~\cite{37813}.
For extensive discussion on the different definitions, see~\cite{37812},~\cite{71617}.
In Papadimitriou's book, it's defined in yet another way, as a class function problems for \complexity{NP},
not in terms of a specific computational model.

\begin{definition}[The class \texorpdfstring{\complexity{FNP}}{FNP}]\label{def:complexity-fnp}
A binary relation \(P(x, y)\) is in \(\complexity{FNP}\) if there exist:
\begin{enumerate}
  \item a polynomial \(p : \mathbb{N} \to \mathbb{N}\) such that if for a given \(x\) exists a solution \(y\) such
  that \(P(x, y)\), then there also exist a ``short'' solution \(y'\) such that
  \(P(x, y')\) and \(|y'| \leqslant p(|x|)\);
  \item a deterministic polynomial-time Turing machine \(M\)
        (a \emph{verifier}), 
        such that for every input pair \((x, y)\),
        \[
            P(x, y)
            \;\Longleftrightarrow\;
            M(x, y) = 1.
            \]
\end{enumerate}

\begin{remark}
  This definition is in style of~\cite[28.10~and~Theorem~28.9]{Rich2007Automata}, where also the
  other, nondeterministic Turing machines-based definition is listed.

  The other definition might come off as more intuitive:
  that a relation $P$ is in \complexity{FNP} iff there is a nondeterministic polynomial-time
  algorithm that, given an arbitrary input $x$,
  can find some $y$ such that $P(x, y)$ or signal that it doesn't exist~\cite{bournez_et_al:LIPIcs.MFCS.2019.23}.
  However, as such nondeterministic Turing machines don't seem to be physically realisable, we don't want to
  introduce that computational model in this work.
\end{remark}

\end{definition}




\subsection{\complexity{NP} vs \complexity{FNP} and the total search problems}\label{subsubsec:tfnp}
\begin{definition}[\complexity{TFNP}]\label{def:tfnp}
A binary relation \(P(x, y)\) is in \(\complexity{TFNP}\) (total \(\complexity{FNP}\)) iff it is
in \(\complexity{FNP}\) and for every \(x\) exists at least one \(y\) such that \(P(x, y)\).
\end{definition}

An interesting example of a problem in \(\complexity{TFNP}\) is \problem{PIGEON} defined below,
for which we mathematically know that the answer exists, but finding it is not trivial.

\begin{definition}[\problem{PIGEON}]\label{def:pigeon}
Given a binary string encoding a Boolean circuit \(C:\{0,1\}^n\!\to\!\{0,1\}^n\), return either
an input \(x\) such that \(C(x) = 0^n\), or two distinct inputs \(x \neq y\) such that \(C(x) = C(y)\).
\end{definition}

\begin{remark}\label{remark:link-pigeonhole}
    This class will be of our interest in~\autoref{subsec:vac0-php}, where we will discuss mathematical theories
    so weak that the pigeonhole principle is not their theorem. The intuition behind it is that
    the computational content of these theories is not strong enough to perform an exhaustive
    linear search of the whole domain.
\end{remark}

The class \complexity{PPP}, a subclass of \complexity{TFNP} problems for which the solution is
guaranteed to exist by the pigeonhole principle, is conjectured to not be equal to \complexity{FP}.
If \(\complexity{PPP} = \complexity{FP}\), then one-way permutations do not exist~\cite[Proposition~3]{PAPADIMITRIOU1994498},
which would have tremendous implications for cryptography.

The class \complexity{TFNP} is discussed in yet more detail in~\cite[Section 1.1]{10.5555/1104410}.







% CIRCUIT-REDUCIBILITY
% uniform FAC0 does NOT admit a nice circuit characterization. FAC0/poly
% is standard (see LogicalFoundations Definition V.2.3) circuits with output.
% But the notion of uniformity doesn't generalize to circuits with outputs!
% Our only hope is in ``polynomial number of AC0-decidable outputs''!

% FAC0 vs AC0-reduction: Definition IX.1.1: AC0-reductions are Turing reductions
% and are circuits with oracle gates for some problem L!
% Section IX.2.1: if C is relations ac0-reducible to F, then FC is FAC0 closure of F.


% \section{\texorpdfstring{$\complexityi{AC}{0}$-reduction}{AC\string^0-reduction}}
% \label{sec:ac0red}
% Definition IX.1.1 CN10. We say that a string function F
% (resp. \  a number function f) is $\complexityi{AC}{0}$-reducible to $L$ if there is a sequence
% of string functions $F_1, \dots, F_n (n \geqslant 0)$ such that
% $F_i$ is $\Sigma^B_0$-definable from $L \cup \{F_1, \dots , F_{i-1}\}$, for $i = 1, \dots, n$
% and F (resp. \ f) is $\Sigma^B_0$-definable from $L \cup \{F_1, \dots , F_{i-1}\}$. A relation R is
% $\complexityi{AC}{0}$-reducible to $L$ if there is a sequence $F_1, \dots, F_n$ as above, and R is
% represented by a $\Sigma^B_0(L \cup \{F_1, \dots, F_n\})$ formula.

% In Chapter~2 of~\cite{edbd4873718c414f90d22dadf0dba2b1} there is an extensive discussion about
% the different subtleties of defining $\complexityi{AC}{0}$ functions and numerous different characterizations
% of Dlogtime-uniform $\complexityi{AC}{0}$-computable functions.

\begin{definition}
  \todo[inline]{I really don't want to introduce circuit reductions... skipping it.}
\end{definition}


\subsection{Language for FL}

\begin{theorem}[{\cite[Proposition~4.1]{COOK19852}}]\label{thm:fl-is-l-red}
  We obtain precisely the class \complexity{FL} from the closure of \compL under \complexityi{NC}{1} circuit reductions,
  symbolically: \(\complexity{FL} = \complexity{L}^\ast\)
  
  \begin{remark}
    Originally, this theorem is proved with \complexityi{NC}{1}-reducibility meaning
    reducibility by \compUeAst-uniform \complexityi{NC}{1} circuits. We don't introduce these
    notions in this work (except for a brief discussion of \compUeAst-uniformity in~\autoref{sec:uniformity-ueast})
  \end{remark}
\end{theorem}

An overview of problems complete for \compL is present in~\cite{COOK1987385}.

\subsection{Language for \complexity{FP}}
We can derive an analogous result to~\autoref{thm:fl-is-l-red} for the class \complexity{FP}.
We are discussing it here, postponing the considerations to~\autoref{thm:fc-is-fac0-closure}.

% # Circuit Value Problem
% - For a given single-tape, polynomial-time Turing machine `M` and input `x`, in [@Kozen2006],
% there is an explicit construction of a boolean circuit over (0, 1, `and`, `or`, `not`)
% (with fan-in 2 for `and`, `or` and 1 for `not`), with one output node, such that its value
% is 1 if and only if machine `M` accepts input `x`. The construction is in LOGSPACE.
% So CVP is P-complete w.r.t. LOGSPACE-reductions.
% - This is a good example of a LOGSPACE-reduction, being a good benchmark for the LF programming
% language and for the circuit description language
% - The problem is that we can't generate tests for it; we have no database of Turing machines descriptions
The notion of \compP-completeness is defined formally e.g.\ in~\cites[Definition~6.25]{DRAFT10.5555/1540612}[Definition~6.28]{10.5555/1540612}.
A very detailed description of one problem complete for \compP under \compL-reductions is in~\cite{Kozen2006}.

\subsection{Not-a-Language for \complexity{FNP}}
There is a relatively agreed-upon notion of reductions between \complexity{FNP} problems:

\begin{definition}[{~\cites{Goebel2011NashComplexity}{Goldberg2021SearchTotal}}~Polynomial-time reductions for \complexity{FNP}]
    Let \(\texttt{HardProblem}\), \(\texttt{NewProblem}\) be search problems in \(\complexity{FNP}\).
    We say that \(\texttt{HardProblem}\) (many-one) reduces to \(\texttt{NewProblem}\) if there exist
    \(f, g\) in \complexity{FP} such that:
    \[\texttt{NewProblem}(f(x), y) \implies \texttt{HardProblem}(x, g(y))\]

    For a given input \(x\) of \(\texttt{HardProblem}\), we can run \(\texttt{NewProblem}(f(x))\)
    to obtain some result \(y\), such that \(g(y)\) is the result of \(\texttt{HardProblem}(x)\).
\end{definition}

There is also plenty of \complexity{NP}-complete problems described in the literature.
However, it is very unclear if we will get \complexity{FNP} this way. The class 
\(\complexity{FP}^{\complexity{NP}}\) is well-studied and nothing suggests it to be equal to \complexity{FNP}.


\subsection{Semantic and syntactic complexity classes}
Some of the complexity classes remain notoriously difficult to be characterized implicitly,
by e.g.\ showing a complete problem and reductions for it.
However, as it turns out, not all complexity classes studied have known
complete problems.
The classes for which a complete problem exists are called ``syntactic'' complexity classes,
as opposed to ``semantic'', e.g.~in~\cite{DBLP:conf/innovations/GoldbergP18}, the authors define a new complexity class
\(\complexity{PTFNP \subseteq \complexity{TFNP}}\), for which
they prove the existence of a complete problem, and then call this class ``syntactic''.

An interesting discussion of this problem, centered around the class \complexity{inv-P}
we discussed in~\autoref{remark:unordered-structures}, is present in~\cite{dawar2012syntactic}.
For the class \complexity{BPP}, some discussion is in~\cite{35236}.
Despite that, a ``less implicit'' characterization of BPP was studied in~\cite{lago2012higherordercharacterizationprobabilisticpolynomial}.

Interestingly, PP has been characterized implicitly by Ugo Dal Lago:~\cite{dallago_et_al:LIPIcs.MFCS.2021.35}.



\begin{remark}[Bibliography]
For probably the first published recognition of the widespread inconsistency of decisional vs functional
complexity classes in the literature, with examples of inconsistent places see~\cite[Page~131]{10.5555/114872}
(and our bibliographical~\autoref{remark:bibliography-david-s-johnson}).

\(\complexity{TFNP}\) was first introduced in~\cite{MEGIDDO1991317}.
\end{remark}







\section{Oracle-oriented programming}
If you use reductions + a single oracle for a difficult problem, you can get
a powerful programming language. This would be a new, nice paradigm that I aimed to
realize.

Due to possible quadratic blowup of output size (even for ac0 circuits i think? but maybe not if fo-uniform?),
it is unsatisfactory for us to have a single complete problem solving e.g. sat in worst-time \(2^n\).
Because then it gets \(2^{n^2}\) etc., which is bad. Transductions (or fine-grained time complexity classes
such as dlintime) are a potential nice class for this.

\subsection{Oracle Turing machines and the technique of forcing}\label{subsec:oracle-forcing}
\todo[inline]{optional. Baker-Gill-Solovay proof uses forcing. }
% Some discussion in~\cite{14091}.



% ### Proving unprovability: Kripke semantics
% - even though searching for a countermodel in Kripke semantics is completely infeasible computationally, we have a good tool for the job!
% - i tested and it works, find countermodels and proofs of intuitionistic formulas. code: https://github.com/ferram/jtabwb_provers/tree/master

% ### Proving unprovability: forcing
% - Extending Type Theory with Forcing (INRIA, 2012)
% > Implementation of forcing in Coq as a program transformation and show a proof of the negation of CH  
% > https://hal.science/hal-00685150/document

% - A beginner’s guide to forcing
% > https://arxiv.org/pdf/0712.1320

% - Forcing for dummies blogpost
% > https://timothychow.net/mathstuff/forcingdum.txt

% - Baker-Gill-Solovay theorem proof
% > Forcing as a method to prove that something can or cannot be done using an oracle  
% > https://math.stackexchange.com/questions/2616541/simple-applications-of-forcing-in-recursion-theory  
% > https://en.wikipedia.org/wiki/Forcing_(computability)

% forcing zeby badac jezyki programowania: to jak oracles w computational complexity!
% https://cstheory.stackexchange.com/a/14093
% see here for oracle A such that NEXP^A = P^{NP^{A}}
% what it means in logic when you have P^A,B vs P^A^B?
% https://link.springer.com/article/10.1007/s00037-001-8190-2




\subsection{Fine-grained reductions}\label{subsec:fine-grained-reductions}
% We will not realistically capture $\text{TIME}(\bigO(n))$ or anything of this kind,
% as the field of fine-grained complexity is relatively modern and little or none interesting
% characterizations of these classes have been found as of writing this work.
% \begin{enumerate}
% \item TODO: Review Neil D. Jones's ``Constant Time Factors Do Matter'' for its discussion of NLIN-complete problems (\url{https://dl.acm.org/doi/pdf/10.1145/167088.167244}).
% \item TODO: Summarize the insights from Gurevich and Shelah's ``Nearly Linear Time'' concerning the definition of $\complexity{DTIME}(n)$ and nearly-linear-time-complete problems under QL reductions (\url{https://link.springer.com/content/pdf/10.1007/3-540-51237-3_10.pdf}).
% \end{enumerate}
\todo[inline]{just mention QL (quasilinear-time functions), NLT (robust complexity class for 
    \( \complexity{DTIME} (n (\log{n})^{\bigO(1)}) \) 
    on RAM machines~\cite{10.1007/3-540-51237-3_10})
}




% \item TODO: Example programming language characterizing \complexity{L}: finite number of variables each bounded by $n$.
% \item TODO: Explore the alternative characterization using a finite number of input pointers, relating it to multi-head two-way automata~\cite{423885},~\cite{10.1007/BF00289513}.



% \subsection{Related works on specifically the \complexity{FL} class}
% Early function algebras for \complexity{FL} appeared in~\cite{10.1145/1008293.1008295} and~\cite{lind1974logspace},
% but these were explicit characterizations.
% In~\cite{10.1007/BF01201998} it was shown how to readily use their concept
% to characterize functions from \complexity{FL} with ``small output'', but this characterization
% relied on using unary representation of natural numbers on input, which is more of a 
% hack than a true characterization of this class.
% In~\cite{murawski2000can}, with further refinements in~\cite{MURAWSKI2004197},
% \(BC^{-}\) was introduced, an algebra that was contained in \complexity{FL},
% but was not known (and unlikely) to be \complexity{FL}-complete.
% In~\cite{Neergaard04} this was improved to the result that \(BC_\varepsilon^{-} = \complexity{FL}\),
% with a short discussion that using course-of-value affine recursion instead of predicative affine recursion
% seem to be the reason behind \(BC_\varepsilon^{-}\) being FL-complete, and \(BC^{-}\) being probably not.

% In~\cite{4276584}, Stratified Bounded Affine Logic is introduced to capture \complexity{FL} computation.
% In~\cite{10.1007/978-3-662-46678-0_27}, an interesting approach using coinduction is utilized to capture \complexity{FL}.

% In~\cite{hofmann2006logspace} a good overview of languages for \complexity{FL} is presented,
% and in~\cite{schoepp2006spaceefficiency}, the history of \complexity{FL} characterizations is traced.

\chapter{Recursion theory}
\label{chap:recursion-theory}
Here I introduce the recursion-theoretical approach to Implicit Computational Complexity.

\section{Cobham's characterization of FP}
\label{sec:cobham-characterization}
Here I copy-paste how Cobham in 1965 defined FP functions as a function algebra.

\chapter{Linear types}\label{chap:linear-types}

% \subsection{How can STLC be HOL, if System F is weaker than second order logic?}
% % system f is second-order propositional, which is different to second-order
% % https://mathoverflow.net/questions/344446/proof-theory-and-subsystems-of-second-order-arithmetic-in-particular-the-revers
% \subsection{Can STLC compute less functions (after type erasue) than System F?}

% \subsection{Curry-Howard correspondence}

\section{Implicit Computational Complexity: linear logic approach}\label{sec:icc-linear}
\url{https://github.com/uelis/IntML}
\todo[inline]{Here, just show that Ugo dal Lago created IntML.
Say roughly what linear types are, don't get into details.}

% This chapter introduces the minimal fragment of linear logic and its term assignment needed later. The purpose is to contrast ordinary (intuitionistic) derivations with linear derivations where duplication and discarding of assumptions are controlled. Only the connectives $\multimap$ and ${!}$ are used, together with the falsity constant $\bot$ for completeness. No multiplicative products or additive connectives are introduced.

% \section{Structural rules}
% \label{sec:linear-structural}

% In intuitionistic propositional reasoning and in the simply typed $\lambda$-calculus, contexts admit \emph{contraction} (assumptions may be duplicated) and \emph{weakening} (assumptions may be ignored). Linear logic removes these two principles from the default context and admits them only through an explicit modality ${!}$.

% \begin{enumerate}
%   \item \textit{No weakening in the linear context:} every linear assumption must be used.
%   \item \textit{No contraction in the linear context:} linear assumptions cannot be duplicated.
%   \item \textit{Exchange:} reordering assumptions is permitted.
% \end{enumerate}

% \section{Propositional linear logic}
% \label{sec:lin-logic}

% \paragraph{Atomic formulas.}
% Fix a countable set $\mathcal{P}=\{P,Q,R,\dots\}$ of propositional variables. The set of atomic formulas is $\mathcal{A}_0 := \mathcal{P} \cup \{\bot\}$, where $\bot$ is falsity.

% \paragraph{Formulas.}
% \[
% A ::= A_0 \mid (A \multimap A) \mid {!}A \qquad (A_0 \in \mathcal{A}_0).
% \]

% \paragraph{Contexts and judgments.}
% Judgments have the form $\Gamma \,;\, \Delta \vdash A$, where $\Gamma$ (unrestricted context) contains formulas of the form ${!}B$, and $\Delta$ (linear context) contains linear formulas. In rules below, $\Delta_1 \uplus \Delta_2$ denotes a disjoint split of the multiset $\Delta$.

% \paragraph{Rules.}
% Identity:
% \[
% \infer[\mathrm{id}]
%       {\Gamma \,;\, A \vdash A}{}
% \qquad
% \infer[\mathrm{id}!]
%       {\Gamma,{!}A \,;\, \cdot \vdash A}{}
% \]

% Linear implication:
% \[
% \infer[\multimap\mathrm{I}]
%       {\Gamma \,;\, \Delta \vdash A \multimap B}
%       {\Gamma \,;\, \Delta, A \vdash B}
% \qquad
% \infer[\multimap\mathrm{E}]
%       {\Gamma \,;\, \Delta_1 \uplus \Delta_2 \vdash B}
%       {\Gamma \,;\, \Delta_1 \vdash A \multimap B
%        \quad
%        \Gamma \,;\, \Delta_2 \vdash A}
% \]

% Exponentials (${!}$):
% \[
% \begin{gathered}
% \infer[\mathrm{dereliction}]
%       {\Gamma \,;\, \Delta, {!}A \vdash A}{} \\[4pt]
% \infer[\mathrm{weak}!]
%       {\Gamma,{!}A \,;\, \Delta \vdash C}
%       {\Gamma \,;\, \Delta \vdash C} \\[4pt]
% \infer[\mathrm{contr}!]
%       {\Gamma,{!}A \,;\, \Delta \vdash C}
%       {\Gamma,{!}A,{!}A \,;\, \Delta \vdash C}
% \end{gathered}
% \]
% \[
% \infer[\mathrm{promotion}]
%       {\Gamma \,;\, \Delta \vdash {!}A}
%       {\Gamma \,;\, \cdot \vdash A}
% \]

% The unrestricted context $\Gamma$ admits weakening and contraction via the explicit rules above. The linear context $\Delta$ admits neither.

% \section{Linear \texorpdfstring{$\lambda$}{lambda}-calculus (term assignment)}
% \label{sec:lin-lambda}

% \paragraph{Atomic types.}
% Fix a countable set $\mathcal{T}_0=\{\alpha,\beta,\gamma,\dots\}$ of base types and include $\bot$. Atomic types are $\mathcal{U}_0:=\mathcal{T}_0 \cup \{\bot\}$.

% \paragraph{Types and terms.}
% \[
% \tau ::= U_0 \mid (\tau \multimap \tau) \mid {!}\tau
% \qquad (U_0 \in \mathcal{U}_0).
% \]
% \[
% t ::= x \mid \lambda x{:}\tau.\, t \mid (t\,u) \mid \mathsf{abort}_{\tau}(t).
% \]
% Here $\mathsf{abort}_{\tau}(t)$ is the eliminator for $\bot$.

% \paragraph{Typing.}
% Typing judgments have the form $\Gamma \,;\, \Delta \vdash t : \tau$, where $\Gamma$ records unrestricted variables and $\Delta$ records linear variables.
% \[
% \begin{gathered}
% \infer[\mathrm{var}]
%       {\Gamma \,;\, x{:}\tau \vdash x : \tau}{} \\[4pt]
% \infer[\mathrm{var}!]
%       {\Gamma, x:{!}\tau \,;\, \cdot \vdash x : \tau}{}
% \end{gathered}
% \]
% \[
% \begin{gathered}
% \infer[\multimap\mathrm{I}]
%       {\Gamma \,;\, \Delta \vdash \lambda x{:}\sigma.\, t : \sigma \multimap \rho}
%       {\Gamma \,;\, \Delta, x{:}\sigma \vdash t : \rho} \\[4pt]
% \infer[\multimap\mathrm{E}]
%       {\Gamma \,;\, \Delta_1 \uplus \Delta_2 \vdash t\,u : \rho}
%       {\Gamma \,;\, \Delta_1 \vdash t : \sigma \multimap \rho
%        \quad
%        \Gamma \,;\, \Delta_2 \vdash u : \sigma}
% \end{gathered}
% \]
% \[
% \begin{gathered}
% \infer[\mathrm{dereliction}]
%       {\Gamma \,;\, \Delta, x:{!}\sigma \vdash x : \sigma}{} \\[4pt]
% \infer[\mathrm{promotion}]
%       {\Gamma \,;\, \Delta \vdash t : {!}\sigma}
%       {\Gamma \,;\, \cdot \vdash t : \sigma} \\[4pt]
% \infer[\bot\mathrm{E}]
%       {\Gamma \,;\, \Delta \vdash \mathsf{abort}_{\tau}(t) : \tau}
%       {\Gamma \,;\, \Delta \vdash t : \bot}
% \end{gathered}
% \]

% \paragraph{Evaluation.}
% Evaluation is by $\beta$-reduction, closed under the usual congruence rules:
% \[
% (\lambda x{:}\sigma.\, t)\,u \;\to_\beta\; t[x:=u].
% \]
% No reduction rule is given for $\mathsf{abort}_{\tau}$.

% \section{Curry--Howard identification}
% \label{sec:ch-linear}
% Under the identification of formulas with types and derivations with well-typed terms: $\multimap$ corresponds to the linear function space, and the rules for ${!}$ correspond to the admissibility of duplication and discarding in the unrestricted context. The separation $\Gamma \,;\, \Delta$ enforces that each variable in $\Delta$ is used exactly once.

% \section{Example}
% \label{sec:example-linear}

% \paragraph{Statement.}
% \[
% \vdash\; (A \multimap B) \multimap A \multimap B.
% \]

% \paragraph{Derivation.}
% \[
% \infer[\multimap\mathrm{I}]{\; \vdash (A \multimap B) \multimap A \multimap B \;}{
%   \infer[\multimap\mathrm{I}]{\; A \multimap B \vdash A \multimap B \;}{
%     \infer[\multimap\mathrm{E}]{\; A \multimap B, A \vdash B \;}{
%       \infer[\mathrm{id}]{A \multimap B \vdash A \multimap B}{}
%       &
%       \infer[\mathrm{id}]{A \vdash A}{}
%     }
%   }
% }
% \]

% \paragraph{Program.}
% \[
% \lambda f{:}A \multimap B.\,\lambda a{:}A.\, f\,a \;:\; (A \multimap B) \multimap A \multimap B.
% \]

% \paragraph{Executable instance (Haskell, \texttt{LinearTypes}).}
% \begin{verbatim}
% {-# LANGUAGE LinearTypes #-}
% module Main where

% -- apply : (a %1-> b) %1-> a %1-> b
% apply :: (a %1-> b) %1-> a %1-> b
% apply f a = f a

% idL :: a %1-> a
% idL x = x

% main :: IO ()
% main = print (apply idL (42 :: Int))
% \end{verbatim}

% \section{Implicit computational complexity (orientation)}
% \label{sec:icc-orientation}
% The rules for ${!}$ make duplication and discarding explicit in derivations and in typing. 
% Restrictions on the use of ${!}$ (e.g., stratification) yield bounds on normalization and can 
% characterize complexity classes. One instance is the language \emph{IntML} by Dal Lago and Sch\"opp, 
% designed to capture \textbf{\complexity{FLOGSPACE}} via a linear typing discipline \cite{DALLAGO2016150}.
%  An implementation is available.\footnote{\url{https://github.com/uelis/IntML}.}






% \chapter{Linear types}
% \label{chap:linear-types}

% Linear logic can be thought of not as true facts we can reason upon or deduce,
% but as resources that we can obtain and utilize (sometimes, destructively).


% suppose that we have a web server. clients come to us and ask for a resource of type `b`.
% we don't have it yet, but we have a function `a -> b`, and there is a small chance that
% one client will send an erroneous request, with a valid object of type $a$ in it.
% after this moment, we will just call our function, and send the b obtained to all clients.

% gamma |- alpha, alpha->beta -> gamma|- b.

% now, what if we have a bakery and the erroneous client give us a cake instead of buying it?
% logically, the naive approach would be to 
% we will be able to satisfy one clinet, but not the whole queue. yet, traditionally in logic we
% don't have a way to 


% What if we could encode in the type system that a function can only read an argument once?
% Or that if you call a function A -> B on argument of type A, then you have access to an element
% of type B, but lose the access to the original argument A (goes out of scope)?
% Consider such a semantics of computer program:
% - (contraction) once you use a variable, it goes out of scope
% - (weakening) function is incorrect if at the end of it, some variable is still in scope (was unused)

% This allows us to reason about resource usage in programs.

% This idea is studied by the field of linear type systems.


% A popular way to think of typed programs is to
% think of an object of type $a -> b$ as a method to 
% transform objects of type a to objects of type b.
% so, if you obtain something of type $a$ (so, the type a is nonempty),
% and something of type $a -> b$, then you can also obtain something of type $b$.


% \section{Linear logic}
% % perhaps take intro from this: https://theses.hal.science/tel-01123737v1/file/2015ENSL0981.pdf

% Linear type systems are fully inspired by substructural logics.
% They consider changing contraction and weakening rule, but keep exchange rule.

% The formal bridge between the theories of linear types and the different kinds of linear logics is the Curry--Howard correspondence
% (also known as propositions as types, or proofs as programs).
% Extensive literature exists in this area, also known as the Curry--Howard correspondence. A good introduction to it is a book by S\o{}rensen and Urzyczyn: [@10.5555/1197021].

% The resource interpretation is also there.

% \section{Linear types and resources}

% In intuitionistic logic and the simply typed $\lambda$-calculus, assumptions in the context $\Gamma$ can be freely duplicated (contraction) or ignored (weakening). This corresponds to programs that can copy or discard variables arbitrarily.

% \textbf{Linear logic}, introduced by Girard, restricts these structural rules. In the corresponding $\lambda$-calculus, each variable must be used \emph{exactly once}. This leads to \textbf{linear types}, where values are treated as \emph{resources}:
% \begin{enumerate}[nosep]
%   \item A value must be consumed once (no weakening).
%   \item A value cannot be duplicated (no contraction).
%   \item The order of usage does not matter (exchange is allowed).
% \end{enumerate}

% Thus, linear types provide a fine-grained way to model computation where resources (such as memory cells, channels, or tokens) cannot be copied or discarded at will. This perspective opens the door to implicit control of computational complexity, since unrestricted duplication of resources is closely related to uncontrolled growth in computation.

% \section{Modern results in Implicit Computational Complexity}


% Bang and paragraph are parts of lambda term syntax. The theorem then is that such a term,
% ignoring types, normalizes in polynomial number of beta steps?


% In 2013, Dal Lago and Sch\"opp introduced \textbf{IntML}, a functional programming language with a linear type system that characterizes \complexity{FLOGSPACE} \cite{DALLAGO2016150}. This marked a significant milestone for Implicit Computational Complexity (ICC). An implementation of IntML is available on GitHub\footnote{\url{https://github.com/uelis/IntML}. Following my private communication with the authors, a permissive license was added to the repository, as it was not included originally.}, and to the best of my knowledge it remains the only language within the linear-logic branch of ICC that has both a working implementation and some potential for (academic) practical use.

% Despite this achievement, IntML's complex typing rules make it difficult to translate most standard imperative algorithms into the language without substantial modification. While, in principle, the language could serve as a platform for reimplementing well-known algorithms within this new paradigm, the steep learning curve creates a significant barrier to adoption. As a result, it is unlikely to become a convenient tool for algorithm designers.

% For these reasons, we decided not to pursue IntML further and not to focus on linear type systems in this work. Instead, we turn to another language discussed in Section~\autoref{chap:icc-recursion-theory}.

% \section{Support for linear types in mainstream programming languages}
% Haskell has some. Rust also. Idris 2 has some. F*, Q*. quantum programming uses that a lot!





% \chapter[Propositions as types]{Propositions as types, proofs as programs: the Curry--Howard correspondence}
% \label{chap:curry-howard}

% The aim of this thesis is to reason about computation and its cost on standard machines. To do so, we will fix a minimal logical and computational core that is sufficient to state and track the shape of computations without introducing additional connectives or control features. In this chapter we present the implicational fragment of intuitionistic logic and the simply typed lambda calculus with only function types. We use these and nothing more.

% \section{Computation and stepwise transformation}
% A computation proceeds by a sequence of discrete steps that transform state. On a Turing machine, this discreteness is explicit in the transition function; on conventional hardware it is enforced by the processor clock; in functional languages it is captured by stepwise term rewriting. Counting such steps yields a cost measure employed later in the thesis. No additional effects or control features are used in this chapter.



% \section{Implicational propositional calculus}
% \label{sec:ipc}

% \paragraph{Language.}
% Fix a countable set $\mathcal{P}=\{P,Q,R,\dots\}$ of propositional variables (atomic formulas). Formulas are generated by
% \[
% A ::= \bot \mid P \mid (A \to A) \qquad (P \in \mathcal{P}).
% \]
% Negation is an abbreviation $\neg A := A \to \bot$.

% \paragraph{Hilbert-style calculus.}
% A derivation is a finite sequence of formulas. The system consists of the following axiom schemata and the single inference rule \emph{modus ponens}:
% \[
% \begin{array}{ll}
% \text{(I1)} & A \to (B \to A), \\[2pt]
% \text{(I2)} & \bigl(A \to (B \to C)\bigr) \to \bigl((A \to B) \to (A \to C)\bigr), \\[2pt]
% \text{(EFQ)} & \bot \to A, \\[6pt]
% \multicolumn{2}{l}{\infer[\mathrm{MP}]{B}{A \to B \quad A}.}
% \end{array}
% \]

% \paragraph{Derivability.}
% Let $\Gamma$ be a finite set (or list) of formulas. A formula $C$ is \emph{derivable from $\Gamma$}, written $\Gamma \vdash C$, if there exists a finite sequence $A_1,\dots,A_n$ with $A_n=C$ and, for each $i\le n$, one of the following holds: (i) $A_i$ is an instance of (I1), (I2), or (EFQ); (ii) $A_i \in \Gamma$; (iii) there exist $j,k<i$ and a formula $D$ with $A_j = D \to A_i$ and $A_k = D$ (an application of MP). A \emph{theorem} is a formula $C$ such that $\vdash C$.





% \section{Simply typed lambda calculus}
% \label{sec:stlc}

% \paragraph{Atomic types.}
% Fix a countable set $\mathcal{T}_0=\{\alpha,\beta,\gamma,\dots\}$ of base types (atomic types).

% \paragraph{Types and terms.}
% Types are generated by
% \[
% \tau ::= \bot \mid \alpha \mid (\tau \to \tau) \qquad (\alpha \in \mathcal{T}_0).
% \]
% Terms are
% \[
% t ::= x \mid \lambda x{:}\tau.\, t \mid (t\,u) \mid \mathsf{abort}_{\tau}(t).
% \]
% Here $\mathsf{abort}_{\tau}(t)$ is the eliminator for the empty type $\bot$.

% \paragraph{Typing.}
% Typing judgments have the form $\Gamma \vdash t : \tau$, where $\Gamma$ maps variables to types. The rules are:
% \[
% \infer{\Gamma, x{:}\tau \vdash x : \tau}{}
% \qquad
% \infer{\Gamma \vdash \lambda x{:}\tau.\, t : \tau \to \sigma}{\Gamma, x{:}\tau \vdash t : \sigma}
% \qquad
% \infer{\Gamma \vdash t\,u : \sigma}{\Gamma \vdash t : \tau \to \sigma \quad \Gamma \vdash u : \tau}
% \]
% \[
% \infer{\Gamma \vdash \mathsf{abort}_{\tau}(t) : \tau}{\Gamma \vdash t : \bot}
% \]

% \paragraph{Evaluation.}
% Evaluation is by $\beta$-reduction, closed under the usual congruence rules:
% \[
% (\lambda x{:}\tau.\, t)\,u \;\to_\beta\; t[x:=u].
% \]
% There is no reduction rule for $\mathsf{abort}_{\tau}$.







% \section{Curry--Howard (implicational-only)}
% Under the correspondence:
% \[
% \begin{aligned}
% \text{propositions} &\;\leftrightarrow\; \text{types}, \\
% \text{proofs} &\;\leftrightarrow\; \text{well-typed terms}, \\
% \text{modus ponens / normalization} &\;\leftrightarrow\; \text{application / $\beta$-reduction}.
% \end{aligned}
% \]
% Concretely:
% \begin{enumerate}
%   \item A derivation of $\Gamma \vdash A \to B$ corresponds to a term $\lambda x{:}A.\,t$ with $\Gamma, x{:}A \vdash t : B$.
%   \item A derivation using $\to$-elimination corresponds to application $t\,u$.
% \end{enumerate}
% We will rely only on this fragment.

% \section{Minimal running example}
% We use the tautology
% \[
% A \to (B \to A),
% \]
% which reads: given a value of type $A$, and given a value of type $B$, we can return the original $A$.

% \subsection*{Logical proof (natural deduction)}
% TODO: IS THIS CORRECT? IT LOOOKS LIKE DEDUCTION THEOREM AND NOT IMPLICATION INTRODUCTION
% \[
% \infer[\to\mathrm{I}]{\vdash A \to (B \to A)}{
%   \infer[\to\mathrm{I}]{A \vdash B \to A}{
%     \infer[\mathrm{Ax}]{A, B \vdash A}{}
%   }
% }
% \]
% Reading this bottom-up: assume $A$; under that assumption, assume $B$; by the assumption $A$ we conclude $A$; discharge the $B$-assumption to obtain $B \to A$; discharge the $A$-assumption to obtain $A \to (B \to A)$.

% \subsection*{Program (simply typed lambda calculus) and executable instance}
% The corresponding term is the ``K'' combinator:
% \[
% \lambda a{:}A.\,\lambda b{:}B.\, a \;:\; A \to (B \to A).
% \]
% To run it on a computer, instantiate the atomic types with concrete ones. For example, take $A = \mathsf{int}$ and $B = \mathsf{bool}$ in OCaml:

% \begin{verbatim}
% (* k : int -> bool -> int *)
% let k (a : int) (_b : bool) = a

% let () =
%   let r1 = k 42 true in
%   let r2 = k 7 false in
%   Printf.printf "%d %d\n" r1 r2
% \end{verbatim}

% This program is a direct executable instance of the proof. The application
% \[
% (\lambda a.\,\lambda b.\,a)\ 42\ \mathsf{true} \;\to_\beta\; (\lambda b.\,42)\ \mathsf{true} \;\to_\beta\; 42.
% \]
% No other features are used.

% \section{Summary}
% We fixed the implicational fragment of intuitionistic logic and the simply typed lambda calculus with only function types. Curry--Howard in this fragment identifies proofs with typed lambda terms and proof normalization with $\beta$-reduction. The single example $A \to (B \to A)$ illustrates both sides and compiles to an ordinary program without introducing any additional connectives or control constructs.

\chapter{Bounded arithmetic}\label{chap:bounded-arithmetic}

In mathematics we typically assume some (pretty strong) foundational axioms we rely on
to prove theorems. If we choose set theory as the foundation (as we usually do), a debatable
concept is whether we should use the axiom of choice or not. More popularly in computer science,
we often want to be explicit about using Kőnig's
lemma\footnote{note that Kőnig's lemma is a form of countable choice from finite sets.}
and Ramsey's theorem.
If we think of the concept
we introduced in~\autoref{chap:reductions} (i.e.\ writing a program in a language in which calls
of the computation-heavy oracle are explicit), we would often ask ourselves the same question ---
can we write the program without relying on that function, i.e.\ write the program in a lower complexity?
It turns out the similarity is not a coincidence,
and to explore this connection further we need to study the theories of \emph{bounded arithmetic}.

We will start by looking at interesting theorems about one such theory, just after
introducing the necessary definitions.

\section{Single-sorted logic and \IDeltaZero}

\begin{definition}[{\cite[Definition~III.1.1]{Cook_Nguyen_2010}}]
A \emph{theory} over a vocabulary $\mathcal{L}$ is a set $\mathcal{T}$ of $\mathcal{L}$-formulas that is closed
under logical consequence and under universal closure.

Note that we have not defined ``logical consequence'', which refers to a particular \emph{proof system}
(also named \emph{proof calculus}). We will not define proof systems for logic in this work.
For those interested, we will just mention that
all the results discussed in this chapter assume standard Gentzen-style proof calculus for classical logic,
$\mathrm{LK}$~\cite[Section~II.2.3]{Cook_Nguyen_2010} for single-sorted logic
and $\mathrm{LK}^2$~\cite[Section~IV.4]{Cook_Nguyen_2010} for two-sorted logic.
\end{definition}

\begin{definition}[{\cite[Definition~II.2.3]{Cook_Nguyen_2010}}]
The vocabulary of arithmetic is
\[
\mathcal{L}_A = \langle 0,1,+,\cdot\ ;\ =,\le \rangle .
\]
Here $0,1$ are constant symbols; $+$ and $\cdot$ are binary function symbols; and
$=$ and $\le$ are binary predicate symbols. We will implicitly assume the
standard interpretation of these symbols as the appropriate functions on natural numbers
whenever talking about the semantics of $\mathcal{L}_A$-formulas.
\end{definition}

\begin{definition}[{\cite[Figure~1]{Cook_Nguyen_2010}} Axioms 1-BASIC of Peano arithmetic]
    \[
\begin{array}{@{}l l@{}}
\mathrm{B1.}\; x + 1 \neq 0
&
\mathrm{B5.}\; x \cdot 0 = 0
\\[2pt]
\mathrm{B2.}\; x + 1 = y + 1 \rightarrow x = y
&
\mathrm{B6.}\; x \cdot (y + 1) = (x \cdot y) + x
\\[2pt]
\mathrm{B3.}\; x + 0 = x
&
\mathrm{B7.}\; (x \le y \land y \le x) \rightarrow x = y
\\[2pt]
\mathrm{B4.}\; x + (y + 1) = (x + y) + 1
&
\mathrm{B8.}\; x \le x + y
\\[6pt]
\multicolumn{2}{@{}l@{}}{\text{C.}\; 0 + 1 = 1}
\end{array}
\]
\end{definition}

\begin{definition}[{\cite[Definition~III.1.4]{Cook_Nguyen_2010}} Induction Scheme]
Let $\Phi$ be a set of formulas. The \emph{$\Phi$-IND axioms} are all formulas of the form
\begin{equation}\label{eq:Phi-IND}
\bigl(\varphi(0)\ \land\ \forall x\,(\varphi(x)\rightarrow \varphi(x+1))\bigr)
\ \rightarrow\ \forall z\,\varphi(z),
\end{equation}
where $\varphi$ ranges over formulas in $\Phi$.  Note that $\varphi(x)$ may have free
variables other than~$x$.
\end{definition}

\begin{definition}[{\cite[Definition~III.1.5]{Cook_Nguyen_2010}} Peano Arithmetic]
The theory $\arithPA$ has as axioms $\mathrm{B1},\ldots,\mathrm{B8}$, together with the $\Phi$-IND axioms,
where $\Phi$ is the set of all $\mathcal{L}_A$-formulas.

Peano Arithmetic is a powerful theory capable of formalizing the major theorems of
number theory. We define subsystems of $\arithPA$ by restricting the induction axioms
to certain sets of formulas. 
\end{definition}

\begin{definition}[{\cite[Definition~III.1.7]{Cook_Nguyen_2010}} \IOPEN, \IDeltaZero, \ISigmaOne]
Let $\mathrm{OPEN}$ be the set of \emph{open} (i.e.\ quantifier-free) formulas, let $\Delta_{0}$
be the set of \emph{bounded} formulas, and let $\Sigma_{1}$ be the set of formulas
of the form $\exists \vec{x}\,\varphi$, where $\varphi$ is bounded and $\vec{x}$ is a
(possibly empty) tuple of variables.  

The theories $\IOPEN$, $\IDeltaZero$, and $\ISigmaOne$ are the subsystems of $\arithPA$
obtained by restricting the induction scheme so that $\Phi$ is $\mathrm{OPEN}$, $\Delta_{0}$,
and $\Sigma_{1}$, respectively.
\end{definition}

\begin{lemma}[{\cite[Example~III.1.8]{Cook_Nguyen_2010}}]
The following formulas (and their universal closures) are theorems of $\IOPEN$:
\[
\begin{array}{@{}l l@{}}
\mathrm{O1.}\; (x+y)+z = x+(y+z)
& \text{(Associativity of $+$)} \\[2pt]
\mathrm{O2.}\; x+y = y+x
& \text{(Commutativity of $+$)} \\[2pt]
\mathrm{O3.}\; x\cdot (y+z) = (x\cdot y)+(x\cdot z)
& \text{(Distributive law)} \\[2pt]
\mathrm{O4.}\; (x\cdot y)\cdot z = x\cdot (y\cdot z)
& \text{(Associativity of $\cdot$)} \\[2pt]
\mathrm{O5.}\; x\cdot y = y\cdot x
& \text{(Commutativity of $\cdot$)} \\[2pt]
\mathrm{O6.}\; x+z = y+z \rightarrow x=y
& \text{(Cancellation for $+$)} \\[2pt]
\mathrm{O7.}\; 0 \le x
& \\[2pt]
\mathrm{O8.}\; x \le 0 \rightarrow x=0
& \\[2pt]
\mathrm{O9.}\; x \le x
& \\[2pt]
\mathrm{O10.}\; x \neq x+1
&
\end{array}
\]
\end{lemma}

\begin{lemma}[{\cite[Example~III.1.9]{Cook_Nguyen_2010}}]
The following formulas (and their universal closures) are theorems of $\IDeltaZero$:
\[
\begin{array}{@{}l l@{}}
\mathrm{D1.}\; x\neq 0 \rightarrow \exists y\le x\,(x=y+1)
& \text{(Predecessor)} \\[2pt]
\mathrm{D2.}\; \exists z\,\bigl(x+z=y \,\lor\, y+z=x\bigr)
& \\[2pt]
\mathrm{D3.}\; x\le y \leftrightarrow \exists z\,(x+z=y)
& \\[2pt]
\mathrm{D4.}\; (x\le y \land y\le z) \rightarrow x\le z
& \text{(Transitivity)} \\[2pt]
\mathrm{D5.}\; x\le y \lor y\le x
& \text{(Total order)} \\[2pt]
\mathrm{D6.}\; x\le y \leftrightarrow x+z \le y+z
& \\[2pt]
\mathrm{D7.}\; x\le y \rightarrow x\cdot z \le y\cdot z
& \\[2pt]
\mathrm{D8.}\; x\le y+1 \leftrightarrow (x\le y \lor x=y+1)
& \text{(Discreteness 1)} \\[2pt]
\mathrm{D9.}\; x<y \leftrightarrow x+1\le y
& \text{(Discreteness 2)} \\[2pt]
\mathrm{D10.}\; x\cdot z = y\cdot z \land z\neq 0 \rightarrow x=y
& \text{(Cancellation for $\cdot$)}
\end{array}
\]
\end{lemma}

Using the above lemmas as building blocks, we can prove quite a few nontrivial theorems.
We will now introduce the core notion of arithmetic --- what does it mean to \emph{define} a function
\emph{in a theory}.


\begin{definition}[{\cite[Definition~III.3.2]{Cook_Nguyen_2010}} Predicates and Functions definable in a Theory]
    Let $\mathcal{T}$ be a theory with vocabulary $\mathcal{L}$, and let $\Phi$ be a set of $\mathcal{L}$-formulas.

\begin{enumerate}
    \item
    A predicate symbol $P(x)\notin \mathcal{L}$ is \emph{$\Phi$-definable in $\mathcal{T}$} if there exists
    an $\mathcal{L}$-formula $\varphi(x)\in\Phi$ such that
    \begin{equation}\label{eq:defining-predicate}
    P(x)\;\leftrightarrow\;\varphi(x).
    \end{equation}

    \item
    A function symbol $f(x)\notin \mathcal{L}$ is \emph{$\Phi$-definable in $\mathcal{T}$} if there exists
    a formula $\varphi(x,y)\in\Phi$ such that
    \begin{equation}\label{eq:unique-existence}
    \mathcal{T} \vdash \forall x\,\exists! y\,\varphi(x,y),
    \end{equation}
    and moreover
    \begin{equation}\label{eq:defining-function}
    y=f(x)\;\leftrightarrow\;\varphi(x,y).
    \end{equation}
\end{enumerate}

    We call \eqref{eq:defining-predicate} a \emph{defining axiom} for $P(x)$ and
    \eqref{eq:defining-function} a \emph{defining axiom} for $f(x)$.
    A symbol is \emph{definable in $\mathcal{T}$} if it is $\Phi$-definable in $\mathcal{T}$ for some $\Phi$.
\end{definition}

\begin{definition}
    We will say that a function is \emph{provably total} in $\mathcal{T}$ iff it
    is $\Sigma_1$-definable in $\mathcal{T}$.
\end{definition}


In~\cite[Section~III.3]{Cook_Nguyen_2010} it is argued that: functions $\lfloor x/y \rfloor,\;\big \lfloor \sqrt{x} \big \rfloor,
\;\max(0, x - y),\;x\!\!\mod y$ are definable in \IDeltaZero; relation $x \mid y$ is definable in \IDeltaZero{},
and, interestingly, the relation $\;\exp(x, y)$ where $\exp(x, y)$ iff $y = 2^x$, is also definable in \IDeltaZero{}.
We don't introduce the specific logical formula defining the relation $\exp(x, y)$, as it is complicated and discussed
in~\cite[Section~III.3]{Cook_Nguyen_2010}.
For a different point of view to these problems, e.g.\ in~\cite{Jumelet1995} it is shown
that Euler's $\varphi$ function is provably total in \IDeltaZero{}.
However, the limits of expressive power of \IDeltaZero{} are low.

\begin{theorem}[{\cite[Section~III.2]{Cook_Nguyen_2010}}]
\[
\IDeltaZero \nvdash \forall x\,\exists y\,\exp(x,y).
\]
Note that $\arithPA$ easily proves $\forall  x\,\exists y\,\exp(x,y)$.
\end{theorem}

It is interesting to study the theory $\IDeltaZero + \exp$ of $\IDeltaZero$ axioms with an additional axiom
stating that the exponential function is definable. As it turns out, this theory enables us to
reason about syntactic constructs such as coding of sets and sequences or context-free
grammar parsing~\cite[Chapter~V,~Section~3]{HajekPudlak1993Metamathematics}
\footnote{note that they use the name $\mathrm{I}\Sigma_0 + \Omega_1$ instead of $\IDeltaZero + \exp$ which is the same.}.

% In~\cite{Buchholz1987ProvablyCF} is is shown that a function to not be provably total in Peano
% arithmetic requires it to be growing too fast. An intuition behind it for the sake of our
% thesis is that functions that are difficult to \emph{prove correct}, but grow slowly (in particular,
% solve decisional problems and only output a boolean value), must have graphs that are not
% easily definable by a logical sentence.

It turns out that a function is $\Sigma_1$-definable in \IDeltaZero{} iff it is in \complexity{FLTH}, functional
version of linear-time hierarchy~\cite[Theorem~III.4.8]{Cook_Nguyen_2010}; for the definition of \complexity{LTH},
refer to~\cite[Section~III.4.1]{Cook_Nguyen_2010} --- as this complexity class is far from what we call ``feasible''
in this work, we don't introduce the details here. Instead, we will now introduce a theory with a good
computational complexity characterization.

\section{Two-sorted logic and \compVZero}\label{sec:theory-v0}

\begin{definition}[Axioms of 2-BASIC]
\[
\begin{array}{@{}l l@{}}
\mathrm{B1.}\; x + 1 \neq 0
&
\mathrm{B7.}\; (x \le y \land y \le x) \rightarrow x = y
\\[2pt]
\mathrm{B2.}\; x + 1 = y + 1 \rightarrow x = y
&
\mathrm{B8.}\; x \le x + y
\\[2pt]
\mathrm{B3.}\; x + 0 = x
&
\mathrm{B9.}\; 0 \le x
\\[2pt]
\mathrm{B4.}\; x + (y + 1) = (x + y) + 1
&
\mathrm{B10.}\; x \le y \lor y \le x
\\[2pt]
\mathrm{B5.}\; x \cdot 0 = 0
&
\mathrm{B11.}\; x \le y \leftrightarrow x < y + 1
\\[2pt]
\mathrm{B6.}\; x \cdot (y + 1) = (x \cdot y) + x
&
\mathrm{B12.}\; x \neq 0 \rightarrow \exists y \le x\, (y + 1 = x)
\\[6pt]
\text{L1.}\; X(y) \rightarrow y < \len{X}
&
\text{L2.}\; y + 1 = \len{X} \rightarrow X(y)
\\[6pt]
\multicolumn{2}{@{}l@{}}{
\text{SE.}\;
\bigl(\len{X} = \len{Y} \land \forall i < \len{X}\, (X(i) \leftrightarrow Y(i))\bigr)
\rightarrow X = Y
}
\end{array}
\]

\end{definition}


\begin{definition}[Comprehension Axiom]\label{def:V.1.2}
Let $\Phi$ be a set of formulas. The \emph{comprehension axiom scheme for $\Phi$},
denoted $\Phi\text{-}\mathrm{COMP}$, consists of all formulas of the form
\begin{equation}\label{eq:Phi-COMP}
\exists X \le y\;\forall z<y\;\bigl(X(z)\leftrightarrow \varphi(z)\bigr),
\end{equation}
where $\varphi(z)\in\Phi$ and $X$ does not occur free in $\varphi(z)$.
In \eqref{eq:Phi-COMP}, the formula $\varphi(z)$ may have free variables of both
sorts in addition to~$z$.  We are mainly interested in the cases where
$\Phi$ is one of the classes $\Sigma^{B}_{i}$.
\end{definition}

% \begin{notation}\label{not:V.1.2}
% Since \eqref{eq:Phi-COMP} asserts the existence of a finite set $X$ of numbers,
% we will sometimes use standard set-theoretic notation to describe~$X$:
% \begin{equation}\label{eq:set-notation}
% X=\{\,z : z<y \land \varphi(z)\,\}.
% \end{equation}
% \end{notation}

\begin{definition}[$V^{i}$]\label{def:arith-vi}
For $i\ge 0$, the theory $V^{i}$ has vocabulary $\mathcal{L}^{2}_{A}$ and is axiomatized by
2-BASIC together with $\Sigma^{B}_{i}\text{-}\mathrm{COMP}$.

Note that there are no explicit induction axioms for $\mathrm{V}^{i}$.
\end{definition}

\begin{theorem}[{\cite[Corollary~V.1.8]{Cook_Nguyen_2010}}]
    Induction is provable in $\mathrm{V}^{i}$. Induction for $\Delta_0$ formulas
    is a theorem of \compVZero.

    Note that this implies that any theorem $\varphi$ provable in \IDeltaZero{} is also
    provable in \compVZero.
\end{theorem}

\begin{theorem}[{\cite[Theorem~V.1.9]{Cook_Nguyen_2010}}]
    For every formula $\varphi$ in the vocabulary $\mathcal{L}_A$ of single-sorted arithmetic,
    if $\compVZero \vdash \varphi$, then also $\IDeltaZero \vdash \varphi$.
    In other words, \compVZero{} is a \emph{conservative extension} of \IDeltaZero.
\end{theorem}

\begin{remark}[{\cite[Section~IV.3]{Cook_Nguyen_2010}} Two-sorted complexity classes]
    When operating in two-sorted logic, we need to redefine what does it mean for a relation to be in a complexity
    class. We will think of numerical arguments $x_i$ of a relation $R(\vec{x}, \vec{X})$ to be passed to
    the deciding Turing machine in unary representation. The string arguments $X_i$ representing finite sets
    of numbers are passed as follows. For a string argument $S$ define $S(i) = 1$ when $i \in S$, 0 otherwise.
    Then the representation $\squarequotes[4]{S}$ of $S$, when the largest member of $S$ is $n$, is defined as
    the following concatenation of bits:
    \[\squarequotes[4]{S} = S(n)S(n - 1) \dots S(1)S(0)\]
    If $S$ is empty then $\squarequotes[4]{S}$ is the empty string.
    Note that $\len{\squarequotes[4]{S}}$ is the same as our
    interpretation of $S$ inside of the theory: $\len{S} = \max(S) + 1$ or $0$ if $S$ empty.
    
    We will write $\unary{x}$ to denote unary representation of $x$, i.e.\ $1^{\len{x}}$
    and $\binary{x}$ to denote binary representation.
    The ultimate input to the Turing machine deciding if $R(\vec{x}, \vec{X})$
    for $\len{\vec{x}}= n, \len{\vec{X}} = N, \len{X_i}=N_i$ is:

    \[
    \unary{n}\;0 \quad \unary{x_1}\;0\;\unary{x_2}\;0\;\dots\;0\;\unary{x_n}\;0\quad\unary{N}
    \; 0 \quad
    \unary{N_1}\;0\;\squarequotes[5][7]{X_1}\;0\;\dots\;0\;\unary{N_N}\;0\;\squarequotes[5][7]{X_N}
    \]


    Note that a purely numerical relation $R(x)$ is in two-sorted polynomial time iff it is computed
    in time $2^{\bigO(n)}$ for $n = \len{\binary{x}}$. The notion of polynomial-time complexity for
    relations with only string arguments $R(\vec{X})$ coincides with our standard intuition.
\end{remark}

\begin{definition}[{\cite[Definition~V.2.1]{Cook_Nguyen_2010}}]
A number function $f$ or string function $F$ is
(\emph{$p$-bounded}) iff there exists a polynomial $p(x,y)$ such that, for all
inputs $x,Y$,
\[
f(x,Y)\ \le\ p\bigl(x,\lvert Y\rvert\bigr)
\qquad\text{or}\qquad
\lvert F(x,Y)\rvert\ \le\ p\bigl(x,\lvert Y\rvert\bigr),
\]
respectively.
\end{definition}


\begin{definition}[{\cite[Definition~V.2.3]{Cook_Nguyen_2010}} Two-sorted functional complexity classes]
Let $\complexity{C}$ be a two-sorted complexity class of relations.  
The corresponding \emph{function class} $\complexity{FC}$ consists of:
\begin{enumerate}
\item all $p$-bounded number functions whose graphs belong to $\complexity{C}$, and
\item all $p$-bounded string functions whose bit graphs belong to $\complexity{C}$.
\end{enumerate}

Note that the classes \complexityi{FAC}{0}, \complexity{FP}, \complexity{FL} are defined in a different way
to what we have used earlier. However, the difference will not matter in this work.
\end{definition}


We don't repeat the definitions of definability in a theory
for the two-sorted case~\cite[Definition~V.4.1]{Cook_Nguyen_2010}.
Recall the definition of $\Sigma_0^B$ formulas~(\autoref{def:SigmaB-PiB-hierarchy}).

\begin{theorem}[{\cite[Corollary~V.5.3]{Cook_Nguyen_2010}}]
    A function is in \FACZero{} iff it is $\Sigma_0^B$-definable in \compVZero.
\end{theorem}

\begin{definition}
    The theory \complexity{VC} for a complexity class $\complexity{C}$ has vocabulary $\mathcal{L}^2_A$
    and is axiomatized
    by the axioms of \compVZero and one additional axiom depending on the choice of the class $\complexity{C}$.
    The additional axiom can be thought of adding an oracle for a $C$-complete problem to \compVZero.
    We skip the (lengthy) technicalities of~\cite[Definition~IX.2.1]{Cook_Nguyen_2010}.
\end{definition}

The below theorem is the central result of our interest in this thesis.

\begin{theorem}[{\cite[Theorem~IX.2.3]{Cook_Nguyen_2010}}]
    A function is provably total in \complexity{VC} iff it is in \complexity{FC}.
\end{theorem}

By adding a single axiom to the theory of \compVZero, we can obtain arithmetical hierarchies
in which the functions that we can define and prove correct are precisely the functions
from a given complexity class $\complexity{C} = \complexityi{FTC}{0}, \complexityi{FNC}{1}, \complexity{FL}, \complexity{FP}$.

This way, we obtain theories with very nice properties. They foster certification of complexity
of an algorithm (if the proof of
correctness itself is feasible, see~\autoref{subsec:complexity-alg-proof}).
At the same time, they enable us to prove theorems about the correctness of functions defined.
In~\cite{buss2025logspaceconstructiveprooflsl}, the authors formalize the breakthrough
result \(\complexity{L}=\complexity{SL}\) of~\cite{10.1145/1391289.1391291} inside of the weak
theory of bounded arithmetic~\complexity{VL}.
The complexity of computational content of proofs of the Discrete Jordan Curve Theorem is
examined in~\cite{10.1145/2071368.2071377}.
Expander construction in \complexityi{VNC}{1} was conducted in~\cite{BUSS2020102796}.

Another elegant property of these theories is that the proof of a problem
not being solvable in a given complexity is exactly a proof of independence
of the axiom (corresponding to the problem) from the theory (corresponding to the complexity class).

\begin{theorem}[{\cites[Corollary~7.21]{CookNguyenDraft}[Corollary~VII.2.4]{Cook_Nguyen_2010}}~Independence of \problem{PHP} from \complexityi{VAC}{0}]\label{subsec:vac0-php}
    \[\complexityi{VAC}{0} \nvdash \problem{PHP}\]
\end{theorem}

\subsection{Complexity of algorithm vs complexity of proof}\label{subsec:complexity-alg-proof}
Even when an algorithm is simple, it seems to not always be trivial to ``feasibly'' prove
that it computes the correct result. In our setting, this results in knowing that
a problem can be solved in a complexity class $\complexity{C}$, but not knowing if the corresponding
function can be defined in the theory $\complexity{VC}$ (i.e.\ proved total and correct).
See \cites[Section~IX.7.3]{Cook_Nguyen_2010}[Section~9G.3]{CookNguyenDraft} for
an open problem whether the breakthrough result that binary integer division
is in \complexity{DLOGTIME}-uniform \complexityi{TC}{0}~\cite{HESSE2002695},
means that it can also be proved in the corresponding \complexityi{VTC}{0} theory.
Note that this problem apparently has been solved (affirmatively) in~\cite{Jerabek2022}.


% \begin{definition}[V.4.12] (semantic)
% A string function is said to be \(\Sigma^B_0\)\textit{-definable from a collection} \(L\) of
% two-sorted functions and relations if it is \(p\)-bounded and its bit graph is represented by
% a \(\Sigma^B_0(L)\) formula.  
% Similarly, a number function is \(\Sigma^B_0\)\textit{-definable from} \(L\) if it is \(p\)-bounded
% and its graph is represented by a \(\Sigma^B_0(L)\) formula.
% \medskip
% This \emph{semantic} notion of \(\Sigma^B_0\)-definability should not be confused with \(\Sigma^B_0\)-definability \emph{in a theory} (Definition~V.4.1), which involves provability.  
% The next result connects these two notions.
% \end{definition}

% there are functions whose graphs are in \complexityi{AC}{0} (representable by sigma0b formulas),
% but which do not belong to \complexityi{FAC}{0} (section: proof of witnessing theorem for v0)


\begin{remark}[Bibliography]
The field of bounded arithmetic was initiated by Samuel Buss in his PhD thesis:~\cite{Buss1986}, in which
the theories $\mathrm{S}^1_2$ were introduced to capture reasoning about the polynomial-time hierarchy \complexity{PH}
(not introduced in our thesis).
The first theory designed to capture polynomial time reasoning was the
equational theory $\mathrm{PV}$ (as in: polynomially-verifiable [proofs])
theory from~\cite{10.1145/800116.803756}.
The two-sorted logic language for capturing complexity classes has been introduced by Zambella in~\cite{00d3b11b-ff1c-386f-a929-6943478c4a28}.
Despite the theories being designed to reason about computation, they are theories of classical logic,
which might come off as worrying given our considerations from \todo[inline]{sec: chapter icc, section intuit. logic}.
Intuitionistic counterparts such as $\mathrm{IS}^1_2$ for $\mathrm{S}^1_2$ and $\mathrm{IPV}$ for $\mathrm{PV}$
have also been studied. However, much less is known about their expressive power.
For the relation of $\mathrm{IS}^1_2$ and $\mathrm{S}^1_2$, please see~\cite{10.1007/3-540-16486-3_91}. In
particular,~\cite[Conjecture~3]{10.1007/3-540-16486-3_91} asks: if $\mathrm{IS}^1_2 \vdash \exists y \ldotp \phi(y, c)$,
then is it true that there is a function $f$, provably correct in $\mathrm{S}^1_2$, such that $f$ \emph{computes}
the Gödel encoding of that $\mathrm{IS}^1_2$ proof? In~\cite[Corollary~8.19]{COOK1993103}, that conjecture
is answered affirmatively. The intuitionistic version $\mathrm{IPV}$ of the theory $\mathrm{PV}$ is discussed
in some detail in~\cite{COOK1993103}.

For a good introduction to \emph{bounded reverse mathematics},
with a very thorough overview of arithmetical theories corresponding to complexity classes below \compFP,
refer to~\cite{Ngu08}.
\end{remark}




\section{Programming language}
\todo[inline]{In progress}
Now, we want to formalize these arithmetical theories so that a computer can check our programs and ensure we didn't
go out of a given complexity at any point.
With such a programming language (and a formalization of arithmetic in general),
we will be able to readily transfer a huge amount of results from paper to computer.

We have two goals for formalization:
\begin{enumerate}
    \item for logicians to believe us the formalization is sound
    \item to be able to extract code with certified complexity from proofs
\end{enumerate}

There is very little work available on the formalization of arithmetic.
A formalization of consistency of Peano arithmetic in Coq was presented in~\cite{O_Connor_2005}.
A formalization of the so-called \emph{Hydra battles} related to unprovability in Peano arithmetic
was shown in~\cite{casteran:hal-03404668}. There is an impressive ongoing project of formalization
of bounded arithmetic in the model-theoretical style in the Lean
community.\footnote{\url{https://github.com/FormalizedFormalLogic/Foundation}} Their approach doesn't
align with our goal of certifying complexity, as their focus is on other arithmetical theories, which differ
significantly from what we need. Somehow related, some work on intuitionistic logic in Lean has also been done
in~\cite{Trufa__2024} even though Lean is not a natural environment for intuitionistic thinking, as it
assumes classical axioms very deeply in its standard libraries, unlike Rocq which is constructive by heart.


An idea for a programming language based on bounded arithmetic was discussed
in~\cite{Li2025FeasibleMathematics}. The language they discuss is $\mathrm{IMP}~(\complexity{PV})$,
based on the equational theory $\complexity{PV}$ which is different from (and less interesting than)
the theories we have discussed. There, the authors show how to design an imperative programming language
with Hoare logic as the verification mechanism (a.k.a. a type system). Note that for their concept
to be implementable in practice, a \emph{formalization} of $\complexity{PV}$ is necessary.

\todo[inline]{Say how i use curry-howard correspondence to extract code from proofs as introduced in chapter linear logic}


% design:
% deep embedding of proof theory vs going with model theory: rocq internship
% proof relevance/irrelevance vs "extension of a theory" / actually defining new functions in theory
% for formalizing V0: go with single-sorted interpretation, or modify Mathlib
% for formalizing anything: separate Ex and All vs set Ex phi = !All x !phi

% This was presented at AITP2025. And was the topic of
% my visit of Yannick Forster at INRIA. My source code is here:~\url{https://github.com/ruplet/formalization-of-bounded-arithmetic}.
% The presentations PDFs are also there, and reviews of my abstract from aitp.

% deep vs shallow embeddings:
% https://people.cs.nott.ac.uk/psztxa/publ/tt-in-tt.pdf
% https://research-information.bris.ac.uk/ws/portalfiles/portal/330955816/LIPIcs_ITP_2022_28.pdf
% https://drops.dagstuhl.de/storage/00lipics/lipics-vol237-itp2022/LIPIcs.ITP.2022.28/LIPIcs.ITP.2022.28.pdf
% https://cstheory.stackexchange.com/questions/1370/shallow-versus-deep-embeddings
% My work on formalizing bounded arithmetic is here: [https://github.com/ruplet/formalization-of-bounded-arithmetic](https://github.com/ruplet/formalization-of-bounded-arithmetic) - in this repo there is also my presentation from AITP, the abstract and the reviews it received.
% This is also the subject of my visit at INRIA, beginning 8th September 2025.





\begin{rawlisting}
\begin{Verbatim}[commandchars=\\\{\}]
\PY{c+c1}{\PYZhy{}\PYZhy{} D1. x ≠ 0 → ∃ y ≤ x, x = y + 1  (Predecessor)}
\PY{c+c1}{\PYZhy{}\PYZhy{} proof: induction on x}
\PY{k+kn}{theorem}\PY{+w}{ }\PY{n}{pred\PYZus{}exists}\PY{+w}{ }\PY{o}{:}
\PY{+w}{  }\PY{n+nb+bp}{∀}\PY{+w}{ }\PY{o}{\PYZob{}}\PY{n}{x}\PY{+w}{ }\PY{o}{:}\PY{+w}{ }\PY{n}{M}\PY{o}{\PYZcb{}}\PY{o}{,}\PY{+w}{ }\PY{n}{x}\PY{+w}{ }\PY{n+nb+bp}{≠}\PY{+w}{ }\PY{l+m+mi}{0}\PY{+w}{ }\PY{n+nb+bp}{→}\PY{+w}{ }\PY{n+nb+bp}{∃}\PY{+w}{ }\PY{n}{y}\PY{+w}{ }\PY{n+nb+bp}{≤}\PY{+w}{ }\PY{n}{x}\PY{o}{,}\PY{+w}{ }\PY{n}{x}\PY{+w}{ }\PY{n+nb+bp}{=}\PY{+w}{ }\PY{n}{y}\PY{+w}{ }\PY{n+nb+bp}{+}\PY{+w}{ }\PY{l+m+mi}{1}\PY{+w}{ }\PY{o}{:=}
\PY{k}{by}
\PY{+w}{  }\PY{k}{let}\PY{+w}{ }\PY{n}{ind1}\PY{+w}{ }\PY{o}{:}\PY{+w}{ }\PY{n}{peano}\PY{n+nb+bp}{.}\PY{n}{Formula}\PY{+w}{ }\PY{o}{(}\PY{n}{Vars2}\PY{+w}{ }\PY{n+nb+bp}{.}\PY{n}{y}\PY{+w}{ }\PY{n+nb+bp}{.}\PY{n}{x}\PY{o}{)}\PY{+w}{ }\PY{o}{:=}\PY{+w}{ }\PY{n}{x}\PY{+w}{ }\PY{n+nb+bp}{=}\PY{n+nb+bp}{\PYZsq{}}\PY{+w}{ }\PY{o}{(}\PY{n}{y}\PY{+w}{ }\PY{n+nb+bp}{+}\PY{+w}{ }\PY{l+m+mi}{1}\PY{o}{)}
\PY{+w}{  }\PY{k}{let}\PY{+w}{ }\PY{n}{ind2}\PY{+w}{ }\PY{o}{:}\PY{+w}{ }\PY{n}{peano}\PY{n+nb+bp}{.}\PY{n}{Formula}\PY{+w}{ }\PY{o}{(}\PY{n}{Vars1}\PY{+w}{ }\PY{n+nb+bp}{.}\PY{n}{x}\PY{o}{)}\PY{+w}{ }\PY{o}{:=}
\PY{+w}{    }\PY{o}{(}\PY{n}{Formula}\PY{n+nb+bp}{.}\PY{n}{iBdEx\PYZsq{}}\PY{+w}{ }\PY{n}{x}\PY{+w}{ }\PY{o}{(}\PY{n}{display2}\PY{+w}{ }\PY{n+nb+bp}{.}\PY{n}{y}\PY{+w}{ }\PY{n}{ind1}\PY{o}{)}\PY{n+nb+bp}{.}\PY{n}{flip}\PY{o}{)}
\PY{+w}{  }\PY{k}{let}\PY{+w}{ }\PY{n}{ind}\PY{+w}{ }\PY{o}{:=}\PY{+w}{ }\PY{n}{idelta0}\PY{n+nb+bp}{.}\PY{n}{delta0\PYZus{}induction}\PY{+w}{ }\PY{n+nb+bp}{\PYZdl{}}\PY{+w}{ }\PY{n}{display1}\PY{+w}{ }\PY{n+nb+bp}{\PYZdl{}}\PY{+w}{ }\PY{o}{(}\PY{n}{x}\PY{+w}{ }\PY{n+nb+bp}{≠}\PY{n+nb+bp}{\PYZsq{}}\PY{+w}{ }\PY{l+m+mi}{0}\PY{o}{)}\PY{+w}{ }\PY{n+nb+bp}{⟹}\PY{+w}{ }\PY{n}{ind2}

\PY{+w}{  }\PY{n}{unfold}\PY{+w}{ }\PY{n}{ind2}\PY{+w}{ }\PY{n}{ind1}\PY{+w}{ }\PY{k}{at}\PY{+w}{ }\PY{n}{ind}

\PY{+w}{  }\PY{n}{specialize}\PY{+w}{ }\PY{n}{ind}\PY{+w}{ }\PY{o}{(}\PY{k}{by}
\PY{+w}{    }\PY{n}{rw}\PY{+w}{ }\PY{o}{[}\PY{n}{IsDelta0}\PY{n+nb+bp}{.}\PY{n}{display1}\PY{o}{]}
\PY{+w}{    }\PY{c+c1}{\PYZhy{}\PYZhy{} TODO: this lemma can\PYZsq{}t be in @[delta0\PYZus{}simps],}
\PY{+w}{    }\PY{c+c1}{\PYZhy{}\PYZhy{} as it creates a goal \PYZsq{}φ.IsOpen\PYZsq{} \PYZhy{} which might be not true!}
\PY{+w}{    }\PY{n}{rw}\PY{+w}{ }\PY{o}{[}\PY{n}{IsDelta0}\PY{n+nb+bp}{.}\PY{n}{of\PYZus{}open}\PY{n+nb+bp}{.}\PY{n}{imp}\PY{o}{]}
\PY{+w}{    }\PY{n+nb+bp}{·}\PY{+w}{ }\PY{n}{constructor}
\PY{+w}{      }\PY{n+nb+bp}{·}\PY{+w}{ }\PY{n}{unfold}\PY{+w}{ }\PY{n}{Term}\PY{n+nb+bp}{.}\PY{n}{neq}
\PY{+w}{        }\PY{n}{rw}\PY{+w}{ }\PY{o}{[}\PY{n}{IsDelta0}\PY{n+nb+bp}{.}\PY{n}{of\PYZus{}open}\PY{n+nb+bp}{.}\PY{n}{not}\PY{o}{]}
\PY{+w}{        }\PY{n}{constructor}\PY{n+nb+bp}{;}\PY{+w}{ }\PY{n}{constructor}\PY{n+nb+bp}{;}\PY{+w}{ }\PY{n}{constructor}
\PY{+w}{        }\PY{n}{constructor}\PY{n+nb+bp}{;}\PY{+w}{ }\PY{n}{constructor}
\PY{+w}{      }\PY{n+nb+bp}{·}\PY{+w}{ }\PY{n}{constructor}
\PY{+w}{    }\PY{n+nb+bp}{·}\PY{+w}{ }\PY{n}{unfold}\PY{+w}{ }\PY{n}{Term}\PY{n+nb+bp}{.}\PY{n}{neq}
\PY{+w}{      }\PY{n}{rw}\PY{+w}{ }\PY{o}{[}\PY{n}{IsOpen}\PY{n+nb+bp}{.}\PY{n}{not}\PY{o}{]}
\PY{+w}{      }\PY{n}{constructor}\PY{n+nb+bp}{;}\PY{+w}{ }\PY{n}{constructor}
\PY{+w}{  }\PY{o}{)}
\PY{+w}{  }\PY{n}{simp\PYZus{}induction}\PY{+w}{ }\PY{k}{at}\PY{+w}{ }\PY{n}{ind}

\PY{+w}{  }\PY{n}{apply}\PY{+w}{ }\PY{n}{ind}\PY{+w}{ }\PY{n+nb+bp}{?}\PY{n}{base}\PY{+w}{ }\PY{n+nb+bp}{?}\PY{n}{step}\PY{+w}{ }\PY{n+nb+bp}{\PYZlt{}}\PY{n+nb+bp}{;}\PY{n+nb+bp}{\PYZgt{}}\PY{+w}{ }\PY{n}{clear}\PY{+w}{ }\PY{n}{ind}\PY{+w}{ }\PY{n}{ind1}\PY{+w}{ }\PY{n}{ind2}
\PY{+w}{  }\PY{n+nb+bp}{·}\PY{+w}{ }\PY{n}{simp}\PY{+w}{ }\PY{n}{only}\PY{+w}{ }\PY{o}{[}\PY{n}{IsEmpty}\PY{n+nb+bp}{.}\PY{n}{forall\PYZus{}iff}\PY{o}{]}
\PY{+w}{  }\PY{n+nb+bp}{·}\PY{+w}{ }\PY{n}{intro}\PY{+w}{ }\PY{n}{a}\PY{+w}{ }\PY{n}{hind}\PY{+w}{ }\PY{n}{h}
\PY{+w}{    }\PY{n}{exists}\PY{+w}{ }\PY{n}{a}
\PY{+w}{    }\PY{n}{constructor}
\PY{+w}{    }\PY{n+nb+bp}{·}\PY{+w}{ }\PY{n}{exact}\PY{+w}{ }\PY{n}{B8}\PY{+w}{ }\PY{n}{a}\PY{+w}{ }\PY{l+m+mi}{1}
\PY{+w}{    }\PY{n+nb+bp}{·}\PY{+w}{ }\PY{n}{rfl}
\end{Verbatim}

\caption{Lean example}\label{lst:lean-example}
\end{rawlisting}




\printbibliography[heading=bibintoc]
% \begin{thebibliography}{99}
% \addcontentsline{toc}{chapter}{Bibliografia}

% \bibitem[Bea65]{beaman} Juliusz Beaman, \textit{Morbidity of the Jolly
%     function}, Mathematica Absurdica, 117 (1965) 338--9.
% \end{thebibliography}

\end{document}


%%% Local Variables:
%%% mode: latex
%%% TeX-master: t
%%% coding: utf-8
%%% End:
