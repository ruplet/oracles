\documentclass[en]  {pracamgr}
%include standalone forest, tikz figures
\usepackage{standalone}
\usepackage{biblatex}
% \usepackage[backend=biber,style=authoryear]{biblatex}
% recommended while using biblatex with babel
\usepackage{csquotes}
% make references clickable, hide boxes around links
\usepackage[hidelinks, pdfpagelabels]{hyperref}
\usepackage[T1]{fontenc}
\usepackage[utf8]{inputenc}
\usepackage{lmodern}
\usepackage{amsmath,amssymb,amsthm}
\usepackage{mathtools}
\usepackage{microtype}
\usepackage{proof}
\usepackage{enumitem}
% theorem-like environments
\theoremstyle{plain}
\newtheorem{theorem}{Theorem}[chapter]
% \newtheorem{lemma}[theorem]{Lemma}
\newtheorem{corollary}[theorem]{Corollary}

\theoremstyle{definition}
\newtheorem{example}[theorem]{Example}
\theoremstyle{definition}
\newtheorem{definition}{Definition}[section]
\theoremstyle{plain}
\newtheorem{proposition}[definition]{Proposition}
\theoremstyle{remark}
\newtheorem{remark}[theorem]{Remark}

% Workaround for "duplicate destination name{page.i}" on front matter:
% disable page anchors just for the title pages, then re-enable.
\usepackage{etoolbox}
\makeatletter
\pretocmd{\maketitle}{\hypersetup{pageanchor=false}}{}{}
\apptocmd{\tableofcontents}{\clearpage\hypersetup{pageanchor=true}}{}{}
\makeatother

\addbibresource{chapters/reductions.bib}
\addbibresource{chapters/linear-logic.bib}
\addbibresource{chapters/bounded-arithmetic.bib}
\addbibresource{chapters/descriptive-complexity.bib}
\addbibresource{chapters/recursion-theory.bib}
\addbibresource{chapters/prehistory-icc.bib}


\newcommand{\llbracket}{\mathopen{[\![}}
\newcommand{\rrbracket}{\mathclose{]\!]}}

% \addbibresource{references.bib}

% https://tex.stackexchange.com/a/404990
% https://mirror.aarnet.edu.au/pub/CTAN/macros/latex/contrib/hyperref/doc/hyperref-doc.html#x1-140004
\addto\extrasenglish{%
  \def\chapterautorefname{Chapter}%
}
\addto\extrasenglish{%
  \def\chaptername{Chapter}%
}

\addto\extrasenglish{%
  \def\sectionautorefname{Section}%
}
\addto\extrasenglish{%
  \def\sectionname{Section}%
}
\usepackage{subcaption}
\usepackage{listings}
\lstset{
  basicstyle=\ttfamily,
  columns=fullflexible,
  frame=single,
  breaklines=true,
  literate={ą}{{\k a}}1
  	  {Ą}{{\k A}}1
           {ż}{{\. z}}1
           {Ż}{{\. Z}}1
           {ź}{{\' z}}1
           {Ź}{{\' Z}}1
           {ć}{{\' c}}1
           {Ć}{{\' C}}1
           {ę}{{\k e}}1
           {Ę}{{\k E}}1
           {ó}{{\' o}}1
           {Ó}{{\' O}}1
           {ń}{{\' n}}1
           {Ń}{{\' N}}1
           {ś}{{\' s}}1
           {Ś}{{\' S}}1
           {ł}{{\l}}1
           {Ł}{{\L}}1
}
\usepackage{url}
\usepackage{xr}
\usepackage{tabularx}
\usepackage{booktabs}
\usepackage{graphicx}
\usepackage{forest}
\usepackage{tikz-qtree}
% https://tex.stackexchange.com/a/18927


\usepackage{pdflscape}

\usepackage[titletoc]{appendix}
\usepackage{amsthm}


\autor{Paweł Balawender}{429141}

\title{Practical programming languages capturing complexity classes}
\titlepl{Praktyczne języki programowania wyrażające klasy złożoności}
% \tytulang{An implementation of a difference blabalizer based on the theory of $\sigma$ -- $\rho$ phetors}

%kierunek: 
% - matematyka, informacyka, ...
% - Mathematics, Computer Science, ...
\kierunek{Computer Science}

% informatyka - nie okreslamy zakresu (opcja zakomentowana)
% matematyka - zakres moze pozostac nieokreslony,
% a jesli ma byc okreslony dla pracy mgr,
% to przyjmuje jedna z wartosci:
% {metod matematycznych w finansach}
% {metod matematycznych w ubezpieczeniach}
% {matematyki stosowanej}
% {nauczania matematyki}
% Dla pracy licencjackiej mamy natomiast
% mozliwosc wpisania takiej wartosci zakresu:
% {Jednoczesnych Studiow Ekonomiczno--Matematycznych}

% \zakres{Tu wpisac, jesli trzeba, jedna z opcji podanych wyzej}

% Praca wykonana pod kierunkiem:
% (podać tytuł/stopień imię i nazwisko opiekuna
% Instytut
% ew. Wydział ew. Uczelnia (jeżeli nie MIM UW))
\opiekun{dr hab. Paweł Parys, prof. UW\\
  Institute of Informatics, University of Warsaw\\
  }

% miesiąc i~rok:
\date{July~2025}

%Podać dziedzinę wg klasyfikacji Socrates-Erasmus:
\dziedzina{ 
%11.0 Matematyka, Informatyka:\\ 
%11.1 Matematyka\\ 
%11.2 Statystyka\\ 
11.3 Computer Science\\ 
%11.4 Sztuczna inteligencja\\ 
%11.5 Nauki aktuarialne\\
%11.9 Inne nauki matematyczne i informatyczne
}

%Klasyfikacja tematyczna wedlug AMS (matematyka) lub ACM (informatyka)
\klasyfikacja{F. Theory of computation\\
  F.3. Logics and meanings of programs\\
  F.3.3. Studies of program constructs}

% Słowa kluczowe:
\keywords{Implicit Computational Complexity, Bounded arithmetic, Lean 4}

% Tu jest dobre miejsce na Twoje własne makra i~środowiska:
% \newtheorem{defi}{Definicja}[section]
\newcommand{\bcem}{\ensuremath{\text{BC}_{\varepsilon}^-}}

% koniec definicji

\begin{document}
\hypersetup{pageanchor=false}
\maketitle

\begin{abstract}
    In this work, I study which features make sense to add to a programming language
    from the computational complexity perspective. Specifically, 
    I focus on how these features can be used to capture various complexity classes, 
    such as LOGSPACE, PTIME, and PSPACE\@. By analyzing the expressiveness and limitations 
    of these features, I aim to provide insights into the design of programming languages
     that are both practical and theoretically sound.
\end{abstract}

\hypersetup{pageanchor=true}
\tableofcontents
%\listoffigures
%\listoftables

\chapter{Reductions}
\label{chap:reductions}
A lot of the definitions below are from~\cite{10.5555/520668}.
definitions of ac, nc, tc:
immerman definition 5.17




\section{Decisional and functional problems}
When we hear a sentence like ``prime factorization is in NP'', it usually
means that the \emph{decisional} version of this problem is decidable in NP.

But we actually usually want to reason about complexity of \emph{functions},
and it appears to not be equivalent. E.g. \ the graph of exponential function is 
decidable in polytime, but calculating exponential function is not.

There is surprisingly little literature on functional complexity classes.
An interesting case is TFNP, studied later in~\ref{sec:complexity-class-tfnp}.
Another interesting case are \emph{transductors}.

\section{Uniformity}
Dlogtime-uniformity. First-order reductions. Logspace-uniformity. $U_{E^*}$. Direct / extended connection language.
Below it should not matter which of these types we use? But I don't have citation for that.
When we don't say poly, it means class is uniform. Otherwise: Immerman showed that the class FO is the same as a uniform version
of AC 0 . Originally AC0 was defined in its nonuniform version, which
we shall refer to as AC0/poly. A language in AC0/poly is specified by
a polynomial size bounded depth family $\langle C_n \rangle$ of Boolean circuits, where
each circuit Cn has n input bits, and is allowed to have ¬-gates, as well as
unbounded fan-in $\land$-gates and $\lor$-gates. In the uniform version, the circuit
C n must be specified in a uniform way; for example one could require that
$\langle C_n \rangle$ is in FO\@. See also Appendix A.5.

We say that family of circuits in AC0 is uniform if the function $n \rightarrow C_n$ is
 *simple to compute*; there is a variety of notions of uniformity of AC0 circuits, which
  we will explore in (TODO: chapter reductions). For the sake of this section we can assume
   that by AC0 we will denote the class of circuit families $C_n$, for which there is a 
   LOGSPACE Turing machine $M$ which on input $0^n$ outputs a *standard representation*


\section{\texorpdfstring{$\text{NC}^i$}{NC\string^i}}
$\text{AC}^i$ is the class of languages accepted by uniform circuit
families of polynomial size and depth $O(\log^i n)$, consisting of unbounded fan-in
AND, and OR gates, along with NOT gates.

Contained in $\text{AC}^i$.

\subsection{NC0}
For example, each output of an NC0 computable function can depend on only finitely many
inputs. Thus, NC0 can't even compute an AND of all its inputs (in contrast, the unbounded
fan-in AND is an AC0 function).
# NC0
- One-way permutations in NC0: [@10.1016/0020-01908790053-6]
- All sets complete under AC0 reductions are in fact already complete under NC0 reductions: [@10.1145/258533.258671] [@AGRAWAL1998127]
- Immerman at page 81 shows how to do addition in NC0 and MAJORITY in NC1 [@Immerman1998-IMMDC]
- Addition and substraction of binary numbers is in AC0: [@27676]
- This is not trivial, as these algorithms in NC0/AC0 use Chinese Remainder Representation or Fermat's Little Theorem!



It is not clear what should be the notion of uniformity for $\text{NC}^0$,
as Dlogtime Turing machines can do things that cannot be done by
any $\text{NC}^0$ circuit.
\subsection{NC1}
- Division is in DLOGTIME-uniform NC1: [@ITA_2001__35_3_259_0]
- Reductions for NC1: Dlogtime!
- Reductions for NC1: UE* reductions; UE*-uniform NC1 = ALOGTIME [@RUZZO1981365]


\section{\texorpdfstring{$\text{AC}^i$}{AC\string^i}}
$\text{AC}^i$ is the class of languages accepted by uniform circuit
families of polynomial size and depth $O(\log^i n)$, consisting of unbounded fan-in
AND, and OR gates, along with NOT gates.

Given $x, y, z$: binary representations of natural numbers,
it is decidable in uniform AC0 if $x + y = z$, but it is not decidable 
in AC0 if $x * y = z$.

A family of circuits $\langle C_n \rangle_{n \in \mathbb{N}}$ for $n$:
number of input bits is in AC0 iff every $C_n$ is of depth $O(1)$ and polynomial 
size w.r.t $n$, contains only unlimited-fanin AND gates and OR gates, and optional 
NOT gates at the inputs. 

\subsection{AC0}
- Addition is in AC0 [@BussLectureNotes]
- FO[+, *] = DLOGTIME-uniform AC0 (https://complexityzoo.net/Complexity_Zoo:A#ac0)
- Discussion of DLOGTIME-uniform AC0: [@hella2023regularrepresentationsuniformtc0]
- Tutaj mamy definicje AC0, TC0, FATC0, FTC0 etc.: [@AGRAWAL2000395]
notions of uniformity of ac0:
0. U_{E^*}-uniformity
1. direct connection language / extended connection language is decidable by FO.
2. is decidable by DLogTime on a random-access turing machine (check if ith bit of representation of nth circuit is b)



\section{\texorpdfstring{$\text{AC}^i[m]$}{AC\string^i[m]}}
$\text{AC}^i[m]$ is defined as $\text{AC}^i$, but in addition unbounded fan-in $\text{MOD}_m$ gates
are allowed, which output 1 iff the number of input wires carrying a value of 1 is a
multiple of $m$.

\section{\texorpdfstring{$\text{ACC}^i$}{ACC\string^i}}
$\text{ACC}^i = \bigcup_m \text{AC}^i[m]$. $\text{ACC}^0$ is contained in $\text{TC}^0$.

\section{\texorpdfstring{$\text{TC}^i$}{TC\string^i}}
$\text{TC}^i$ is the class of languages accepted by uniform circuit families 
of polynomial size and depth $O(\log^i n)$, consisting of unbounded fan-in MAJORITY
gates, along with NOT gates.

Contained in $\text{NC}^{i + 1}$.
\subsection{TC0}
- Multiplication is in TC0 [@BussLectureNotes], this resource cites this: [@doi:10.1137/0213028] but i cant find relevant info there


\section{P}
Poly-time relations. To define reductions between P-complete problems,
first-order/$\text{AC}^0$ reductions are typically used.

\subsection{inv-P}
Permutation-invariant P; P on unordered structures.

\section{FP}
Poly-time functions.

\section{NP}
reductions used here are precisely the class FL of logspace reductions!

\section{FNP}
- https://cs.stackexchange.com/questions/71617/function-problems-and-fp-subseteq-fnp
- https://cstheory.stackexchange.com/questions/37812/what-exactly-are-the-classes-fp-fnp-and-tfnp
- https://complexityzoo.net/Complexity_Zoo:F#fp


\section{TFNP}
\label{sec:complexity-class-tfnp}

\section{\texorpdfstring{$\text{AC}^0$-reduction}{AC\string^0-reduction}}
\label{sec:ac0red}
Definition IX.1.1 CN10. We say that a string function F
(resp. a number function f) is $\text{AC}^0$ -reducible to L if there is a sequence
of string functions $F_1, \dots, F_n (n \geqslant 0)$ such that
$F_i$ is $\Sigma^B_0$-definable from $L \cup \{F_1, \dots , F_{i-1}\}$, for $i = 1, \dots, n$
and F (resp. f) is $\Sigma^B_0$ -definable from $L \cup \{F_1, \dots , F_{i-1}\}$. A relation R is
$\text{AC}^0$-reducible to L if there is a sequence $F_1, \dots, F_n$ as above, and R is
represented by a $\Sigma^B_0(L \cup \{F_1, \dots, F_n\})$ formula.

In (chapter 2, \cite{edbd4873718c414f90d22dadf0dba2b1}) there is an extensive discussion about
the different subtleties of defining $\text{AC}^0$ functions and numerous different characterizations
of Dlogtime-uniform $\text{AC}^0$-computable functions.


\section{\texorpdfstring{$\text{NC}^0$}{NC\string^0} reductions}
In~\cite{edbd4873718c414f90d22dadf0dba2b1}, it is shown that, surprisingly, all known NP-complete problems
are complete under $\text{NC}^0/poly$ reductions already. Another candidate for a problem that is NP-complete
under poly-time reductions but not under logspace reductions is discussed in~\cite{18631}.

\section{DLOGTIME-uniformity}
- example of DLOGTIME reduction to show tree isomorphism for string-represented trees, is NC1-complete [@694595]


\section{First-order reduction}
Definicja first-order redukcji:  
[https://link.springer.com/book/10.1007/3-540-28788-4](https://link.springer.com/book/10.1007/3-540-28788-4)  
rozdzial 12.3

# FO
- FO[+, *] = FO[BIT]
- FO[<, *] = FO[BIT] (Crane Beach Conjecture)
- FO[+] is less expressive than FO[<, *] = FO[<, /] = FO[<, COPRIME] [@10.1002/malq.200310041]
3. fo-query: definition 1.26, immerman
fo-uniformity: there is a first order query I : STRUC[ts] -> STRUC[tc] with I(0^n) = C_n; definition 5.16 immerman
fo-reduction is defined as a first-order query (definition 1.26)
also there is definition how to represent a circuit!

\section{First-order projection}
also named fops in immerman.
definition 11.7 first-order projections
\section{Quantifier-free fo-projections}
qfps

\section{LOGSPACE reduction}
- Reductions for L: e.g. first-order reductions, Immerman 1999 p.51
- USTCONN is complete for L
- a programming language for L: a finite number of variables, each <= n
- alternative: a finite number of input pointers; this is really similar to multi-head two-way automaton [@423885] [@10.1007/BF00289513]

\section{P reduction}

\section{Complete problems}
\label{sec:complete-problems}
For every class C from: AC0, L, P, there is some problem F such that F is C-complete under AC0 reductions.
So AC0 reductions are enough for us, and AC0 reductions is(?) the same class as FAC0 functions.

\subsection{Functional complexity classes and completeness}
To show FL-completeness, it suffices to show L-completeness: FL = L* = L + NC1 
reductions [@COOK19852, proposition 4.1]

\section{Which class we will focus on?}
\label{sec:classes-of-interest}
For chapters about ICC we will focus on L and P, as these are the most abundant reductions,
and other classes might have problems like in the below subsections.
Later, in the chapter about bounded arithmetic we will find that we actually want to care
about characterization of $\text{AC}^0$-reductions.

- Why P and L are important and robust complexity classes
> The smallest class containing linear time and closed under subroutines is P. The smallest class containing log space and closed under subroutines is still log space. So P and L are the smallest robust classes for time and space respectively which is why they feel right for modeling efficient computation.  
> https://cstheory.stackexchange.com/a/3448/71933

\subsection{Fine-grained complexity theory}
We will not realistically capture $\text{TIME}(O(n))$ or anything of this kind,
as the field of fine-grained complexity is relatively modern and little or none interesting
characterizations of these classes have been found as of writing this work.


- Neil D. Jones: Constant Time Factors Do Matter
> NLIN-complete problem  
> https://dl.acm.org/doi/pdf/10.1145/167088.167244

- Gurevich, Shelah: Nearly linear time
> couple problems with defining DTIME(n) (dependency on computational model)  
> nearly-linear-time-complete problem under QL reductions  
> https://link.springer.com/content/pdf/10.1007/3-540-51237-3_10.pdf

\subsection{Semantic and syntactic complexity classes}
Which classes can we realistially try to characterize? Probably not BPP, nor permutation-invariant PTIME.
https://mathoverflow.net/questions/35236/is-there-a-syntactic-characterization-for-bpp-bqp-or-qma

- On Syntactic and Semantic Complexity Classes
> Anuj Dawar  
> University of Cambridge Computer Laboratory  
> Spitalfields Day, Isaac Newton Institute, 9 January 2012  
> https://www.newton.ac.uk/files/seminar/20120109163017301-152985.pdf  
> e.g. NP = ESO (Fagin 1974), so NP is syntactical  
> major open problem:  
> Does P admit a syntactic characterisation?  
> Can the class P be “built up from below” by finitely many operations?  
> If a complexity class C has a complete problem L, it is a syntactic class.  
> because we can enumerate all AC0 reductions  
> Two Possible Worlds:
> Either
> - there is no effective syntax for inv-P
> - there is no classification possible of polynomial-time graph problems
> - there is an inexhaustible supply of efficient algorithmic techniques to be discovered
> - P neq NP
> Or,
> - there is an effective syntax for inv-P
> - there is a P-complete graph problem under FO-reductions
> - all polynomial-time graph problems can be solved by easy variations of one algorithm.


- About semantic and syntactic complexity classes
> An interesting difference is that PR functions can be explicitly enumerated, whereas functions in R cannot be (since otherwise the halting problem would be decidable). In this sense, PR is a "syntactic" class whereas R is "semantic."  
> https://complexityzoo.net/Complexity_Zoo:P#pr



% TODO: why do we focus on L and P: because *reductions* chapter!

\chapter{Recursion-Theoretic Implicit Complexity}
\label{chap:recursion-theory}

\section{Recursion theory}
While not the primary focus of this work, the field of recursion theory developed concepts
that later became foundational for ICC\@. An important formal system studied there is \emph{primitive recursion}.

\begin{definition}[Primitive recursive functions]
\(\mathsf{PR}\) is the smallest class of functions containing \(\text{(i)}\)--\(\text{(iii)}\) and closed under \(\text{(iv)}\), \(\text{(v)}\).
\begin{enumerate}[label=(\roman*)]
\item \textbf{(Constants)} For every \(n\in\mathbb{N}\) and \(k\ge 0\), the \(k\)-ary constant function
      \(c_{n}^{(k)}(\vec x)=n\).
\item \textbf{(Successor)} \(S(x)=x+1\).
\item \textbf{(Projections)} For \(k\ge 1\) and \(1\le i\le k\),
      \(\pi_i^{(k)}(x_1,\dots,x_k)=x_i\).
\item[(iv)] \textbf{(Composition)} If \(h:\mathbb{N}^m\to\mathbb{N}\) and
      \(g_1,\dots,g_m:\mathbb{N}^k\to\mathbb{N}\) are in \(\mathsf{PR}\), then
      \(f(\vec x)=h\big(g_1(\vec x),\dots,g_m(\vec x)\big)\) is in \(\mathsf{PR}\).
\item[(v)] \textbf{(Primitive recursion)} If \(g:\mathbb{N}^k\to\mathbb{N}\) and
      \(h:\mathbb{N}^{k+2}\to\mathbb{N}\) are in \(\mathsf{PR}\), then the unique
      \(f:\mathbb{N}^{k+1}\to\mathbb{N}\) is in \(\mathsf{PR}\) with:
      \[
      f(0,\vec x)=g(\vec x),\qquad
      f(S(y),\vec x)=h\big(y,\,f(y,\vec x),\,\vec x\big).
      \]

\end{enumerate}
\end{definition}

\begin{example}[Addition]
Define \(\mathrm{Add}:\mathbb{N}^2\to\mathbb{N}\) by primitive recursion:
\[
\mathrm{Add}(0,y)=y, \qquad
\mathrm{Add}(S(x),y)=S\big(\mathrm{Add}(x,y)\big).
\]
\end{example}


\begin{definition}[LOOP language]
Let \(\mathrm{Var}=\{x_0,x_1,x_2,\dots\}\).
LOOP programs are generated by the grammar
\[
\begin{aligned}
P ::=~& x_i := 0
\;\mid\; x_i := x_i + 1
\;\mid\; P \,;\, P
\;\mid\; \texttt{LOOP}~x_i~\texttt{DO}~P~\texttt{END},
\end{aligned}
\]
where \(x_i\in\mathrm{Var}\).

\noindent
We assume standard semantics, with a remark that \(\texttt{LOOP}~x_i~\texttt{DO}~P~\texttt{END}\) repeats \(P\) exactly as many times as the value stored in \(x_i\) \emph{at loop entry} (changes to \(x_i\) inside \(P\) do not change the iteration count).
\end{definition}

Interestingly, in~\cite{10.1145/800196.806014} it has been shown that the functions definable by LOOP programs
are precisely the primitive recursive functions.
This simple example actually satisfies our criteria of a `programming language capturing a complexity class', as
the LOOP language captures exactly the primitive recursive functions\footnote{As recognized in \url{https://complexityzoo.net/Complexity_Zoo:P}.}.
Moreover, we can even stratify the primitive recursive functions into a hierarchy like in~\cite{Grzegorczyk1953}.


Historically, the origins of primitive recursion can be traced back to~\cite{Grassmann1861} and~\cite{Dedekind1888},
but the class was probably first considered as the primary object of study in~\cite{Skolem1923-vanHeijenoort}.
For the details of the historical origins, consult~\cite{Adams2011}.


% \section{Implicit Computational Complexity}

% \subsection{What is Implicit Computational Complexity?}
% Implicit computational complexity (ICC) studies how to guarantee resource bounds
% without appealing to external machine models.
% Instead of analysing running time or space after the fact, ICC designs
% languages and recursion schemes whose syntactic constraints ensure that every
% definable function belongs to a chosen complexity class.
% The aim is a principled foundation for programming languages that ``build in''
% complexity guarantees by construction.
% Techniques from proof theory, recursion theory, and linear logic play prominent
% roles in this development.

% ICC can be seen as the proof-theoretic analogue of descriptive complexity.
% While descriptive complexity classifies decision problems through logical
% definability, ICC approaches complexity from within programming languages and
% type systems, enforcing bounds through their typing and recursion disciplines.

% Two prominent ICC traditions illustrate this idea.
% One investigates typed \(\lambda\)-calculi, often inspired by linear logic;
% this thread is discussed in the chapter on linear logic.
% The other starts with basic functions on binary strings (e.g. \( \mathsf{append}_0(x) \),
% \( \mathsf{append}_1(x) \), \( \mathsf{pop}(x) \), \( \mathsf{empty}() \))
% and studies the classes generated by closing under composition and recursion
% subject to carefully crafted restrictions.
% This recursion-theoretic viewpoint is the focus of the remainder of the chapter.

% \subsection{History of the Field}
% The modern study of ICC begins with back-to-back breakthroughs:
% Leivant in 1991~\cite{151625} and Bellantoni-Cook in 1992~\cite{10.1007/BF01201998}
% gave the first implicit characterisations of polynomial-time computable functions.

% Earlier work hinted at the programme.
% Immerman characterised polynomial-time relations in 1987 without explicit
% size bounds~\cite{doi:10.1137/0216051}.
% Compton and LaFlamme~\cite{COMPTON1990241} captured uniform \(\mathsf{NC}^1\),
% and Allen~\cite{ALLEN19911} uniform \(\mathsf{NC}\), though their definitions
% still concealed polynomial bounds and targeted relations instead of functions.
% Other precursors include Gurevich's algebraic view of polynomial-time
% functions~\cite{4568079} and Cobham's seminal 1964 characterisation of
% \(\mathsf{FP}\)~\cite{Cobham1964-COBTIC}, both of which retained explicit
% polynomial constraints.
% For a broader literature survey, see Bloch~\cite{bloch1994function}.

% Since the 1990s numerous classes have been captured implicitly; see, for
% example, Niggl~\cite{NIGGL201047} and Oitavem et al.~\cite{10.1016/j.ic.2015.12.009}
% for overviews of \(\mathsf{FPTIME}\) and \(\mathsf{FNC}\) characterisations.
% Jones connected fragments of Lisp to the decisive classes \(\mathsf{L}\) and
% \(\mathsf{P}\)~\cite{JONES1999151}; Bonfante extended the analysis~\cite{10.1007/11784180_8};
% Kristiansen and Voda~\cite{kristiansenvoda2005} investigated both imperative
% and functional languages whose fragments yield hierarchies containing logspace,
% linear space, polynomial time, and polynomial space.
% Related contributions include work by Kristiansen~\cite{kristiansen2005neat},
% Oitavem~\cite{Oitavem+2010+355+362}, and many others.
% Because this thesis concentrates on \(\mathsf{FPTIME}\) and \(\mathsf{FLOGSPACE}\),
% these classes receive focused attention below.

% Accessible introductions to ICC include the three-part presentation by
% Simone Martini~\cite{martini2006implicit1,martini2006implicit2,martini2006implicit3},
% Simona Ronchi Della Rocca's 2019 talk~\cite{ronchi2019logic},
% and Ugo Dal Lago's short overview~\cite{DalLago2012}.

% \section{Recursion-Theoretic Approach}
% \label{sec:recursion-theory-approach}
% The recursion-theoretic branch of ICC avoids simulating Turing machines.
% Instead, it controls complexity directly by limiting how new functions arise
% from simpler ones.

% Classical recursion theory builds all computable functions from basic ones
% (zero, successor, projections) by closing under composition and recursion.
% Restricting the recursion principles offers refined control: only certain
% recursion patterns are allowed, and the resulting function algebra coincides
% with a target complexity class.

% Primitive recursion and the Grzegorczyk hierarchy already stratify total
% functions by growth rate, yet they do not align neatly with standard classes
% such as \(\mathsf{P}\) or \(\mathsf{L}\).
% The challenge is to refine recursion principles so that they correspond exactly
% to natural complexity classes.

% \subsection{Bellantoni and Cook's Safe Recursion for \texorpdfstring{\(\mathsf{FP}\)}{FP} and \texorpdfstring{\(\mathsf{FL}\)}{FL}}
% Bellantoni and Cook introduced a function algebra \(\mathcal{B}\) whose key
% innovation is the separation of arguments into \emph{normal} inputs (written to
% the left of a semicolon) and \emph{safe} inputs (to the right).
% We write \(f(\vec{x};\vec{a})\), with normal inputs \(\vec{x}\) controlling
% recursion depth and safe inputs \(\vec{a}\) being passed around without
% influencing that depth.
% The computation is performed on non-negative integers; proofs transfer to
% binary strings~\cite{10.1007/BF01201998}.
% For an integer \(x\), let \(|x|\) denote its binary length
% \(\lceil \log_2(x+1)\rceil\); for vectors use component-wise notation.

% \subsubsection{Initial Functions}
% \(\mathcal{B}\) is the smallest class containing the following base functions:
% \begin{enumerate}
%   \item \textbf{Zero}: the nullary function \(0\).
%   \item \textbf{Projections}: for \(n,m \geq 0\) and \(1 \leq j \leq n+m\),
%   \[
%     \pi^{n,m}_j(x_1,\dots,x_n;\,x_{n+1},\dots,x_{n+m}) = x_j.
%   \]
%   \item \textbf{Successors}: appending a bit for \(i \in \{0,1\}\),
%   \( s_i(\,;a) = 2a + i \).
%   \item \textbf{Predecessor}: deleting the last bit,
%   \( p(\,;0)=0 \) and \( p(\,;ai)=a \) for \(i\in\{0,1\}\).
%   \item \textbf{Conditional on the last bit}:
%   \[
%     C(\,;a,b,c) =
%     \begin{cases}
%       b & \text{if } a \bmod 2 = 0,\\
%       c & \text{otherwise.}
%     \end{cases}
%   \]
% \end{enumerate}

% \subsubsection{Closure Principles}
% \(\mathcal{B}\) is closed under:
% \begin{enumerate}
%   \item \textbf{Predicative recursion on notation (PRN)}:
%   given \(g,h_0,h_1 \in \mathcal{B}\), define \(f\) by
%   \[
%   \begin{aligned}
%     f(0,\vec{x};\vec{a})   &= g(\vec{x};\vec{a}),\\
%     f(yi,\vec{x};\vec{a}) &= h_i\!\big(y,\vec{x};\vec{a},f(y,\vec{x};\vec{a})\big),
%   \end{aligned}
%   \]
%   where \(i\in\{0,1\}\) and \(yi\) denotes appending bit \(i\) to \(y\).
%   The recursive value enters only a safe argument position, preventing it from
%   later becoming a normal input.
%   \item \textbf{Safe composition (SC)}:
%   for \(h,\vec{r},\vec{t} \in \mathcal{B}\),
%   \[
%     f(\vec{x};\vec{a}) = h\big(\vec{r}(\vec{x};\,);\ \vec{t}(\vec{x};\vec{a})\big).
%   \]
%   The functions \(\vec{r}\) may depend only on normal inputs, whereas
%   \(\vec{t}\) may depend on both; safe outputs never flow into normal positions.
% \end{enumerate}

% \paragraph{Intuition.}
% Functions in \(\mathcal{B}\) can perform arbitrary polynomial-time computation
% on their normal inputs.
% Safe inputs may increase only by an additive constant, and recursive results
% remain safe, so recursion depth cannot depend on previously computed values.
% This predicative discipline ensures two directions:
% every function definable in \(\mathcal{B}\) is computable in polynomial time,
% and every polynomial-time function can be expressed using only normal arguments.

% \subsection{Characterisation of Polynomial Time}
% The polynomial-time functions are exactly those members of \(\mathcal{B}\) whose
% signature contains only normal inputs.
% Duplicating safe arguments would allow super-polynomial growth, so their role is
% strictly controlled.

% \subsection{Neergaard's \(BC\text{-}\varepsilon\): Definition and Intuition}
% Neergaard's \(BC\text{-}\varepsilon\) algebra follows the Bellantoni--Cook
% separation of normal and safe arguments but adds an \emph{affine} restriction on
% safe data and works directly with binary strings.

% \paragraph{Setup.}
% \begin{itemize}
%   \item Numbers are words over \(\{0,1\}\); the empty word \(\varepsilon\) denotes \(0\).
%   \item Arguments split into normal and safe parts, written
%   \(f(x_1,\dots,x_m : y_1,\dots,y_n)\).
%   \item Safe arguments are affine: each may be used at most once.
%   \item Recursion is permitted only on normal arguments.
% \end{itemize}

% \paragraph{Definition.}
% \(BC\text{-}\varepsilon\) is the least set of functions over binary strings that
% contains the following base functions and is closed under safe affine
% composition and safe affine course-of-value recursion.
% \begin{enumerate}
%   \item \textbf{Base functions}: constant \(0(:)=\varepsilon\); predecessor
%     \(p(:\,\varepsilon)=\varepsilon\) and \(p(:\,yb)=y\); projections
%     \(\pi^{m,n}_j(x_1,\dots,x_m : x_{m+1},\dots,x_{m+n}) = x_j\);
%     successors \(s_0(:\,y)=y0\) and \(s_1(:\,y)=y1\); conditional on the last
%     bit selecting \(y_2\) when the first argument ends in \(1\).
%   \item \textbf{Safe affine composition}:
%     if \(f : N_2^{M,N} \to N_2\),
%     \(g_1,\dots,g_M : N_2^{m,0} \to N_2\), and
%     \(h_1,\dots,h_N\) each consume disjoint subsets of safe inputs, then
%     \[
%       (f \circ \langle g_1,\dots,g_M : h_1,\dots,h_N\rangle)(x : y)
%       = f\big(g_1(x:),\dots,g_M(x:) : h_1(x:\mathbf{y}^{\,1}),\dots\big),
%     \]
%     where the tuple \(y\) is partitioned so that each safe variable appears in
%     at most one block \(\mathbf{y}^{\,j}\).
%   \item \textbf{Safe affine course-of-value recursion}:
%     given \(g : N_2^{m,n}\to N_2\),
%     \(h_0,h_1 : N_2^{m+1,1}\to N_2\),
%     \(d_0,d_1 : N_2^{m+1,0}\to N_2\),
%     define \(f = \mathrm{rec}(g,h_0,\delta_0,h_1,\delta_1)\) by
%     \[
%       f(n,x:y) =
%       \begin{cases}
%         g(x:y), & n=\varepsilon,\\[4pt]
%         h_{b_1}\big(b_k\cdots b_2, x : f(b_k\cdots b_2+\delta, x:y)\big),
%         & n = b_k\cdots b_2 b_1,
%       \end{cases}
%     \]
%     where \(\delta = \bigl|d_{b_1}(b_k\cdots b_2, x:)\bigr|\).
% \end{enumerate}
% Notation follows the original presentation: nullary \(f\) abbreviates
% \(f \circ \langle : \rangle\), and \(\mathrm{rec}(g,h,\delta)\) stands for
% \(\mathrm{rec}(g,h,\delta,h,\delta)\).

% \paragraph{Intuition.}
% Affine use of safe inputs prevents duplication and thus uncontrolled growth,
% while normal inputs control recursion depth.
% Safe affine composition confines normal arguments to depend only on normal data,
% and course-of-value recursion can inspect earlier values via the offset terms,
% yet the recursive result always flows back into a safe position linearly.
% Removing the affine restriction recovers the original Bellantoni--Cook algebra.

% \section{Cobham's Characterisation of \texorpdfstring{\(\mathsf{FP}\)}{FP}}
% \label{sec:cobham-characterisation-stub}
% \textbf{Stub.} Detailed material on Cobham's function algebra for
% \(\mathsf{FP}\) will be added once the supporting definitions are prepared.

% \section{Logspace-Oriented Developments}
% Lind developed early logspace-oriented function algebras in 1973 and
% 1974~\cite{10.1145/1008293.1008295,lind1974logspace}, relying on explicit
% resource bounds.
% Bellantoni and Cook also investigated unary encodings suitable for
% logspace functions with small outputs~\cite{10.1007/BF01201998},
% and Jones explored tail-recursive read-only programs~\cite{JONES1999151}.
% Murawski and Ong introduced \(BC^{-}\) in 2000~\cite{murawski2000can},
% with further refinements in 2004~\cite{MURAWSKI2004197}, and
% Møller-Neergaard proved \(BC\varepsilon^{-} = \mathsf{FL}\) in the same
% period.
% Hofmann summarised logspace languages in 2006~\cite{hofmann2006logspace},
% and Schöpp traced the history of logspace characterisations~\cite{schoepp2006spaceefficiency}.

% The original \(BC\text{-}\varepsilon\) code has been ported from Moscow ML to
% SML/NJ and released under a compatible licence, alongside an unpublished note
% that clarifies the original presentation.\footnote{\url{https://github.com/ruplet/neergaard-logspace-characterization}}
% The accompanying Haskell formalisation reproduces key examples from
% Neergaard's paper; one correction shows that the published definition of
% \(\mathsf{shiftR}\) swaps its arguments.
% The proof of this fact appears in \texttt{thesis-tex/main.tex}.

% \begin{verbatim}
% identity :: Func
% identity = Proj 1 0 1

% oneNormalToZero :: Func
% oneNormalToZero =
%   let g = ZeroFunc in
%   let h = Proj 1 1 2 in
%   let d = identity in
%   Recursion 0 0 g h h d d

% -- Proposition 1. Let m and n be numbers in binary. Right shift shiftR(m : n)
% -- of m by |n| and selection of bit |n| from m are definable in BCe-.

% -- shiftR(n : m) = m >> |n|
% shiftR :: Func
% shiftR =
%   let g = Proj 0 1 1 in
%   let d = oneNormalToZero in
%   -- h(timer : recursive) = tail(recursive)
%   let h = Composition 0 1 1 [1] Tail [] [Proj 1 1 2] in
%   Recursion 0 1 g h h d d
% \end{verbatim}


\printbibliography[heading=bibintoc]
% \begin{thebibliography}{99}
% \addcontentsline{toc}{chapter}{Bibliografia}

% \bibitem[Bea65]{beaman} Juliusz Beaman, \textit{Morbidity of the Jolly
%     function}, Mathematica Absurdica, 117 (1965) 338--9.
% \end{thebibliography}

\end{document}


%%% Local Variables:
%%% mode: latex
%%% TeX-master: t
%%% coding: utf-8
%%% End:
