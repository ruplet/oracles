% Template source: https://www.mimuw.edu.pl/pl/informator-dla-studentow/prace-i-egzaminy-dyplomowe-odbior-dyplomu/

% Niniejszy plik stanowi przykład formatowania pracy magisterskiej na
% Wydziale MIM UW.  Szkielet użytych poleceń można wykorzystywać do
% woli, np. formatujac wlasna prace.
%
% Zawartosc merytoryczna stanowi oryginalnosiagniecie
% naukowosciowe Marcina Wolinskiego.  Wszelkie prawa zastrzeżone.
%
% Copyright (c) 2001 by Marcin Woliński <M.Wolinski@gust.org.pl>
% Poprawki spowodowane zmianami przepisów - Marcin Szczuka, 1.10.2004
% Poprawki spowodowane zmianami przepisow i ujednolicenie
% - Seweryn Karłowicz, 05.05.2006
% Dodanie wielu autorów i tłumaczenia na angielski - Kuba Pochrybniak, 29.11.2016
% drobne poprawki i komentarze - Paweł Goldstein, 24.04.2025

% dodaj opcję [licencjacka] dla pracy licencjackiej
% dodaj opcję [en] dla wersji angielskiej (mogą być obie: [licencjacka,en])
\documentclass[en]  {pracamgr}
%include standalone forest, tikz figures
\usepackage{standalone}

% TODO: CSL doesn't work with Biblatex. Use luatex, citeproc for it: https://github.com/zepinglee/citeproc-lua
% Select citation style (CSL sources live under thesis-tex/).
% https://github.com/citation-style-language/styles/blob/2e3e9964846c9a1ad709e6974307b85f70a37f06/apa.csl
% \PassOptionsToPackage{style=apa}{biblatex} % default APA, see thesis-tex/apa.csl
% https://github.com/citation-style-language/styles/blob/2e3e9964846c9a1ad709e6974307b85f70a37f06/ieee.csl
% \PassOptionsToPackage{style=ieee}{biblatex} % switch to IEEE, see thesis-tex/ieee.csl
% https://github.com/citation-style-language/styles/blob/2e3e9964846c9a1ad709e6974307b85f70a37f06/chicago-author-date-17th-edition.csl
% \PassOptionsToPackage{style=chicago-author-date-17th-edition}{biblatex}

% \usepackage[pl]{babel}
\usepackage[
  backend=biber,
  style=alphabetic,
  bibencoding=utf8 % this is default, but make it explicit
]{biblatex}
% recommended while using biblatex with babel
\usepackage{csquotes}


% Ensure correct handling and display of accented and non-ASCII characters:
% - fontenc[T1]: use 8-bit fonts with full Latin alphabet (proper hyphenation, PDF copy/paste)
% - inputenc[utf8]: tell LaTeX that the source files are saved in UTF-8 encoding
% - lmodern: load Latin Modern fonts (Unicode-aware version of Computer Modern)
\usepackage[T1]{fontenc} 
\usepackage[utf8]{inputenc}
\usepackage{lmodern}

\usepackage{amsmath,amssymb,amsthm}
\usepackage{mathtools}
\usepackage{microtype}
\usepackage{proof}
\usepackage{enumitem}
% https://tex.stackexchange.com/a/119935
\setlist[enumerate]{
  label=(\roman*),
  ref=(\roman*),
  itemsep=2pt,
  parsep=0pt,
  topsep=2pt,
  partopsep=0pt
}

\usepackage{xspace}

% make references clickable, hide boxes around links
\usepackage[hidelinks, pdfpagelabels]{hyperref}
\usepackage{xurl}     % improves breaking points in URLs to avoid overfull

\usepackage{xcolor}
\usepackage{fancyvrb}
\input{listings/pygments-default.tex}
% this is for nice list of listings, bypassing the listings package because
% we compile to TeX anyway, not using pygments from latex
\usepackage{caption}
% Define a "listing" type (non-verbatim) with its own list:
\DeclareCaptionType[fileext=lol]{rawlisting}[Listing][List of Listings]

% https://github.com/leanprover/lean4/blob/20e16f1c75b13057e00b4cc5315aac611df766fe/doc/syntax_highlight_in_latex.md
% \usepackage{fontspec} % this requires lualatex or xelatex
% \setmonofont{FreeMono}
% https://lean-lang.org/documentation/latex-syntax-highlighting/
\usepackage{newunicodechar}
% Logic and set theory
\newunicodechar{→}{\ensuremath{\to}}
\newunicodechar{↔}{\ensuremath{\leftrightarrow}}
\newunicodechar{∀}{\ensuremath{\forall}}
\newunicodechar{∃}{\ensuremath{\exists}}
\newunicodechar{¬}{\ensuremath{\neg}}
\newunicodechar{∧}{\ensuremath{\land}}
\newunicodechar{∨}{\ensuremath{\lor}}
\newunicodechar{⊢}{\ensuremath{\vdash}}
\newunicodechar{⊨}{\ensuremath{\vDash}}

\newunicodechar{∈}{\ensuremath{\in}}
\newunicodechar{∉}{\ensuremath{\notin}}
\newunicodechar{⊆}{\ensuremath{\subseteq}}
\newunicodechar{⊂}{\ensuremath{\subset}}
\newunicodechar{⊇}{\ensuremath{\supseteq}}
\newunicodechar{⊃}{\ensuremath{\supset}}

% Equality / order
\newunicodechar{≠}{\ensuremath{\neq}}
\newunicodechar{≤}{\ensuremath{\le}}
\newunicodechar{≥}{\ensuremath{\ge}}

% Numbers (Lean’s ℕ, ℤ, ℚ, ℝ types)
\newunicodechar{ℕ}{\ensuremath{\mathbb{N}}}
\newunicodechar{ℤ}{\ensuremath{\mathbb{Z}}}
\newunicodechar{ℚ}{\ensuremath{\mathbb{Q}}}
\newunicodechar{ℝ}{\ensuremath{\mathbb{R}}}

% Function arrows and maps
\newunicodechar{⟶}{\ensuremath{\longrightarrow}}
\newunicodechar{⟹}{\ensuremath{\Longrightarrow}}
\newunicodechar{↦}{\ensuremath{\mapsto}}

% Algebraic symbols
\newunicodechar{⊕}{\ensuremath{\oplus}}
\newunicodechar{⊗}{\ensuremath{\otimes}}
\newunicodechar{∘}{\ensuremath{\circ}}

% Top / bottom
\newunicodechar{⊤}{\ensuremath{\top}}
\newunicodechar{⊥}{\ensuremath{\bot}}

% Greek letters commonly used in Lean
\newunicodechar{α}{\ensuremath{\alpha}}
\newunicodechar{β}{\ensuremath{\beta}}
\newunicodechar{γ}{\ensuremath{\gamma}}
\newunicodechar{δ}{\ensuremath{\delta}}
\newunicodechar{ε}{\ensuremath{\varepsilon}} % Lean often uses \varepsilon style
\newunicodechar{φ}{\ensuremath{\varphi}}
\newunicodechar{ψ}{\ensuremath{\psi}}
\newunicodechar{λ}{\ensuremath{\lambda}}

% Angle brackets, often used for tuples / structures
\newunicodechar{⟨}{\ensuremath{\langle}}
\newunicodechar{⟩}{\ensuremath{\rangle}}

% Middle dot for names / multiplication
\newunicodechar{·}{\ensuremath{\cdot}}



% Package todonotes Warning: The length marginparwidth is less than 2cm and will 
% most likely cause issues with the appearance of inserted todonotes. The issue c
% an be solved by adding a line like \setlength {\marginparwidth }{2cm} prior to 
% loading the todonotes package. on input line 228.
\setlength{\marginparwidth}{2cm}
% TODOs in draft
% size=\scriptsize
\usepackage[colorinlistoftodos,textsize=scriptsize]{todonotes}
\usepackage{xcolor}
\setuptodonotes{
  color=orange!20,     % lighter background
  linecolor=orange!50, % border
  bordercolor=orange!50,
  textcolor=black,
  caption={TODO},      % text used in list of todos
  prepend={TODO:} % prefix inside note
}


% https://tex.stackexchange.com/a/404990
% https://mirror.aarnet.edu.au/pub/CTAN/macros/latex/contrib/hyperref/doc/hyperref-doc.html#x1-140004
% Sectioning names (babel-aware)

\addto\extrasenglish{%
  \def\chapterautorefname{Chapter}%
  \def\sectionautorefname{Section}%
  \def\subsectionautorefname{Subsection}%
  \def\subsubsectionautorefname{Subsubsection}%
}

% ?https://tex.stackexchange.com/a/240948 i think this didnt help
% chat suggests using this to fix \autoref displaying an example as a 'Theorem'
\usepackage{amsthm}
\usepackage{aliascnt}

\theoremstyle{plain}
\newtheorem{theorem}{Theorem}[chapter]
\newtheorem{lemma}[theorem]{Lemma}
% \theoremstyle{plain}
% \newtheorem{proposition}[theorem]{Proposition}
% \newtheorem{corollary}[theorem]{Corollary}

\theoremstyle{remark}
\newaliascnt{remark}{theorem}
\newtheorem{remark}[remark]{Remark}
\aliascntresetthe{remark}
\providecommand*{\remarkautorefname}{Remark}

\theoremstyle{definition}
\newtheorem{definition}{Definition}[section]
\providecommand*{\definitionautorefname}{Definition}

\theoremstyle{definition}
\newaliascnt{example}{theorem}
\newtheorem{example}[example]{Example}
\aliascntresetthe{example}
\providecommand*{\exampleautorefname}{Example}

\theoremstyle{definition}
\newtheorem*{definition*}{Definition}

\theoremstyle{definition}
\newtheorem{notation}{Notation}
\providecommand*{\notationautorefname}{Notation}



% Workaround for "duplicate destination name{page.i}" on front matter:
% disable page anchors just for the title pages, then re-enable.
\usepackage{etoolbox}
\makeatletter
\pretocmd{\maketitle}{\hypersetup{pageanchor=false}}{}{}
\apptocmd{\tableofcontents}{\clearpage\hypersetup{pageanchor=true}}{}{}
\makeatother

\addbibresource{chapters/foundations.bib}
\addbibresource{chapters/preliminaries-single-sorted-logic.bib}
\addbibresource{chapters/preliminaries-two-sorted-logic.bib}
\addbibresource{chapters/preliminaries-turing-complexity.bib}
\addbibresource{chapters/preliminaries-circuits.bib}
\addbibresource{chapters/formalized-semantics.bib}
\addbibresource{chapters/descriptive-complexity.bib}
\addbibresource{chapters/reductions.bib}
\addbibresource{chapters/icc.bib}
\addbibresource{chapters/icc-recursion-theory.bib}
\addbibresource{chapters/icc-linear-logic.bib}
\addbibresource{chapters/bounded-arithmetic.bib}
\addbibresource{chapters/appendix-uniformity.bib}


\newcommand{\llbracket}{\mathopen{[\![}}
\newcommand{\rrbracket}{\mathclose{]\!]}}

% \addbibresource{references.bib}


\usepackage{subcaption}
\usepackage{listings}
\lstset{
  basicstyle=\ttfamily,
  columns=fullflexible,
  frame=single,
  breaklines=true,
  literate={ą}{{\k a}}1
  	  {Ą}{{\k A}}1
           {ż}{{\. z}}1
           {Ż}{{\. Z}}1
           {ź}{{\' z}}1
           {Ź}{{\' Z}}1
           {ć}{{\' c}}1
           {Ć}{{\' C}}1
           {ę}{{\k e}}1
           {Ę}{{\k E}}1
           {ó}{{\' o}}1
           {Ó}{{\' O}}1
           {ń}{{\' n}}1
           {Ń}{{\' N}}1
           {ś}{{\' s}}1
           {Ś}{{\' S}}1
           {ł}{{\l}}1
           {Ł}{{\L}}1
}
\usepackage{url}
\usepackage{xr}
\usepackage{tabularx}
\usepackage{booktabs}
\usepackage{graphicx}
\usepackage{forest}
\usepackage{tikz-qtree}
% https://tex.stackexchange.com/a/18927


\usepackage{pdflscape}

\usepackage[titletoc]{appendix}


\autor{Paweł Balawender}{429141}

\title{Practical programming languages capturing complexity classes}
\titlepl{Praktyczne języki programowania wyrażające klasy złożoności}
% \tytulang{An implementation of a difference blabalizer based on the theory of $\sigma$ -- $\rho$ phetors}

%kierunek: 
% - matematyka, informacyka, ...
% - Mathematics, Computer Science, ...
\kierunek{Computer Science}

% informatyka - nie okreslamy zakresu (opcja zakomentowana)
% matematyka - zakres moze pozostac nieokreslony,
% a jesli ma byc okreslony dla pracy mgr,
% to przyjmuje jedna z wartosci:
% {metod matematycznych w finansach}
% {metod matematycznych w ubezpieczeniach}
% {matematyki stosowanej}
% {nauczania matematyki}
% Dla pracy licencjackiej mamy natomiast
% mozliwosc wpisania takiej wartosci zakresu:
% {Jednoczesnych Studiow Ekonomiczno--Matematycznych}

% \zakres{Tu wpisac, jesli trzeba, jedna z opcji podanych wyzej}

% Praca wykonana pod kierunkiem:
% (podać tytuł/stopień imię i nazwisko opiekuna
% Instytut
% ew. Wydział ew. Uczelnia (jeżeli nie MIM UW))
\opiekun{dr hab. Paweł Parys, prof. UW\\
  Institute of Informatics, University of Warsaw\\
  }

% miesiąc i~rok:
\date{July~2025}

%Podać dziedzinę wg klasyfikacji Socrates-Erasmus:
\dziedzina{ 
%11.0 Matematyka, Informatyka:\\ 
%11.1 Matematyka\\ 
%11.2 Statystyka\\ 
11.3 Computer Science\\ 
%11.4 Sztuczna inteligencja\\ 
%11.5 Nauki aktuarialne\\
%11.9 Inne nauki matematyczne i informatyczne
}

% Klasyfikacja tematyczna wedlug AMS (matematyka) lub ACM (informatyka)
% ACM: https://www.acm.org/publications/class-2012
\klasyfikacja{F. Theory of computation\\
  F.3. Logics and meanings of programs\\
  F.3.3. Studies of program constructs}

% Słowa kluczowe:
\keywords{Programming languages, Descriptive complexity, Bounded arithmetic, Lean 4, Rocq, Coq, LOGSPACE, PTIME}

% Tu jest dobre miejsce na Twoje własne makra i~środowiska:
% \newtheorem{defi}{Definicja}[section]
\newcommand{\bcem}{\ensuremath{\text{BC}_{\varepsilon}^-}}

% https://tex.stackexchange.com/a/354690
\DeclareRobustCommand{\bigO}{%
  \text{\usefont{OMS}{cmsy}{m}{n}O}%
}

% We will not use `complexity' CTAN package, because it is wacky
% https://tex.stackexchange.com/q/391604

\newcommand{\compVZero}{%
  \texorpdfstring{%
    \ensuremath{\complexityi{V}{0}}%
  }{V0}%
}

\newcommand{\logicFO}{%
  \texorpdfstring{%
    \complexity{FO_{\mathrm{BIT}}}%
  }{FO_BIT}%
}

\newcommand{\logicFOLFP}{%
  \texorpdfstring{%
    \complexity{FO_{\mathrm{BIT}}[LFP]}%
  }{FO_BIT[LFP]}%
}

\newcommand{\logicFOLFPgraph}{%
  \texorpdfstring{%
    \complexity{FO_{\text{graph}}[LFP]}%
  }{FO(wo BIT)[LFP]}%
}

\newcommand{\logicFOLFPgraphord}{%
  \texorpdfstring{%
    \complexity{FO_{\text{graph}\leqslant}[LFP]}%
  }{FO_BIT[LFP]}%
}

\newcommand{\IDeltaZero}{%
  \texorpdfstring{%
    \ensuremath{\mathrm{I}\Delta_0}%
  }{IDelta0}%
}

\newcommand{\IOPEN}{%
  \texorpdfstring{%
    \ensuremath{\mathrm{IOPEN}}%
  }{IOPEN}%
}

\newcommand{\ISigmaOne}{%
  \texorpdfstring{%
    \ensuremath{\mathrm{I}\Sigma_1}%
  }{ISigma1}%
}

\newcommand{\arithPA}{%
  \texorpdfstring{%
    \ensuremath{\mathrm{PA}}%
  }{PA}%
}


\newcommand{\complexity}[1]{%
  \texorpdfstring{%
    \ensuremath{\mathsf{#1}}%
  }{#1}\xspace%
}
\newcommand{\complexityi}[2]{%
  \texorpdfstring{%
    \ensuremath{\mathsf{#1}^{#2}}%
  }{#1#2}\xspace%
}


\newcommand{\problem}[1]{%
  \texorpdfstring{%
    \ensuremath{\texttt{#1}}%
  }{#1}%
}

\newcommand{\compL}{\complexity{L}}
\newcommand{\compP}{\complexity{P}}
\newcommand{\compFP}{\complexity{FP}}
\newcommand{\compUeAst}{%
  \texorpdfstring{%
    \ensuremath{\complexity{U}_{E^*}}%
  }{UE*}%
}


\newcommand{\ACpoly}[1]{%
  \texorpdfstring{%
    \ensuremath{\complexity{AC}^{#1}_{/poly}}%
  }{AC^{#1}\_/poly}%
}
\newcommand{\NCpoly}[1]{%
  \texorpdfstring{%
    \ensuremath{\complexity{NC}^{#1}_{/poly}}%
  }{NC^{#1}\_/poly}%
}
\newcommand{\TCpoly}[1]{%
  \texorpdfstring{%
    \ensuremath{\complexity{TC}^{#1}_{/poly}}%
  }{TC^{#1}\_/poly}%
}
\newcommand{\ACipoly}{\ACpoly{i}}
\newcommand{\NCipoly}{\NCpoly{i}}
\newcommand{\TCipoly}{\TCpoly{i}}

\newcommand{\FNCpoly}[1]{%
  \texorpdfstring{%
    \ensuremath{\complexity{FNC}^{#1}_{/poly}}%
  }{FNC^{#1}\_/poly}%
}
\newcommand{\FACpoly}[1]{%
  \texorpdfstring{%
    \ensuremath{\complexity{FAC}^{#1}_{/poly}}%
  }{FAC^{#1}\_/poly}%
}
\newcommand{\FTCpoly}[1]{%
  \texorpdfstring{%
    \ensuremath{\complexity{FTC}^{#1}_{/poly}}%
  }{FTC^{#1}\_/poly}%
}
\newcommand{\FACipoly}{\FACpoly{i}}
\newcommand{\FNCipoly}{\FNCpoly{i}}
\newcommand{\FTCipoly}{\FTCpoly{i}}

\newcommand{\AC}[1]{%
  \texorpdfstring{%
    \ensuremath{\complexity{AC}^{#1}}%
  }{AC^{#1}}%
}
\newcommand{\NC}[1]{%
  \texorpdfstring{%
    \ensuremath{\complexity{NC}^{#1}}%
  }{NC^{#1}}%
}
\newcommand{\TC}[1]{%
  \texorpdfstring{%
    \ensuremath{\complexity{TC}^{#1}}%
  }{TC^{#1}}%
}
\newcommand{\ACi}{\AC{i}}
\newcommand{\NCi}{\NC{i}}
\newcommand{\TCi}{\TC{i}}

\newcommand{\FAC}[1]{%
  \texorpdfstring{%
    \ensuremath{\complexity{FAC}^{#1}}%
  }{FAC^{#1}}%
}
\newcommand{\FNC}[1]{%
  \texorpdfstring{%
    \ensuremath{\complexity{FNC}^{#1}}%
  }{NC^{#1}}%
}
\newcommand{\FTC}[1]{%
  \texorpdfstring{%
    \ensuremath{\complexity{FTC}^{#1}}%
  }{FTC^{#1}}%
}
\newcommand{\FACi}{\FAC{i}}
\newcommand{\FNCi}{\FNC{i}}
\newcommand{\FTCi}{\FTC{i}}
\newcommand{\FACZero}{\FAC{0}}




% https://tex.stackexchange.com/a/408965
% \newcommand{\squarequotes}[2][0]{{%
%   {\vphantom{#2}}^{\ulcorner}\kern-\scriptspace{}
%   \mspace{-#1mu}%
%   {{}#2}^{\urcorner}%
% }}
% chat version
\usepackage{xparse}
% usage: \squarequotes[<left mu>][<right mu>]{...}
\NewDocumentCommand{\squarequotes}{O{0} O{0} m}{%
  \vphantom{#3}%
  {}^{\mathopen{\ulcorner}}\mkern-#1mu%
  #3%
  \mkern-#2mu{}^{\mathclose{\urcorner}}%
}


% length / cardinality bars
\usepackage{mathtools}
\DeclarePairedDelimiter{\len}{\lvert}{\rvert}

% base-1 (unary) and base-2 (binary) numeral notation
\newcommand{\unary}[1]{\ensuremath{(#1)_{1}}}
\newcommand{\binary}[1]{\ensuremath{(#1)_{2}}}



% koniec definicji

% \includeonly{chapters/foundations}

\begin{document}
\hypersetup{pageanchor=false}
\maketitle

\begin{abstract}
    In this work, I study which features make sense to add to a programming language
    from the computational complexity perspective. Specifically, 
    I focus on how these features can be used to capture various complexity classes, 
    such as LOGSPACE, PTIME, and PSPACE\@. By analyzing the expressiveness and limitations 
    of these features, I aim to provide insights into the design of programming languages
     that are both practical and theoretically sound.
\end{abstract}

\hypersetup{pageanchor=true}
\tableofcontents
%\listoffigures
%\listoftables
\listofrawlistings{}

\chapter{Introduction}

\todo[inline]{This will be refined. Now it's just essentials}
The purpose of this thesis is to find a convenient way of certifying that a given program
is in some complexity class. We want to do it by defining such a programming language,
that by definition every program written in it has a given complexity.

\begin{enumerate}
    \item The purpose of~\autoref{chap:foundations} (chapter: Models of computation...) is to check in what programming paradigm/computation model,
    it will be the easiest for us to find characterizations. If imperative languages are hard to reason about, maybe consider functional
    languages?
    Also, if computational classes for Turing machines are hard to capture, maybe classes of complexity
    e.g. for lambda terms or for games (not examined here) will be easier to capture?
    \item The purpose of~\autoref{chap:formalized-semantics} is to examine reasoning
    about turing machines straight away, without any characterizations, in Coq. It is way too much work!
    \item In~\autoref{chap:descriptive-complexity} we study languages (logics) for decisional problems
        purely in terms of how complex is the specification of the problem. This gives us
        some characterizations of decisive complexity classes, and we can also
        define functions by bit-graphs and asking "is k-th bit of output f(x) 0 or 1?" repeatedly.
    \item The purpose of~\autoref{chap:reductions} is to find out if we can find language for polytime functions
    by making language for logspace reductions, then adding an oracle for decisional poly-time complete problem.
    We introduce functional complexity classes, which are crucial for our considerations.
    Then, going stronger, if we can have a language for FNP from FP + oracle for NP-complete problem.
    (answer turns out to be NO, as examined in section about TFNP). By the way, \(\complexity{FP}^{\complexity{NP}}\) is
    exactly what we would get this way, and this is a well-studied complexity class.
    And going weaker, if we can have language for FL from L-complete problem and circuit reductions.
    We leave this problem out, and only study circuit reductions/uniformity in appendix.
    We proceed to try find FL characterization right away in Linear Logic (Ugo Dal Lago IntML)
    and in Recursion Theory (Neergaard functional language).
    \item In~\autoref{chap:recursion-theory}, we study simple and cool function algebras
        that capture FL, and FP. Neergaard implemented a programming language for FL.
    \item In~\autoref{chap:linear-types}, we study Ugo Dal Lago's linear-types programming
        language that captures FL and FNL. Briefly! Because introducing linear types is too
        much overhead
    \item In~\autoref{chap:bounded-arithmetic}, we study my contribution!


\end{enumerate}

% - a program documented to perform some computation and return a result should not run into an
%   infinite loop
% - a program implementing an O(n)-time algorithm, and documented to execute in O(n) should not
%   need O(n^2) time to run by an accident in the implementation.
% - thoughtfully written (i.e. "clean") code should convince the user reading it that
%   it does what it is documented to. One way this is typically done is including parts of proof
%   of correctness in the documentation. Another way is declaring a function to have a particular type
%   and ensuring the user can quickly verify the implementation type-checks.




% \section{The full picture}
% States of the process of computation are dots. Computation is arrows between the dots.
% Since computation is sequential (even parallel programs have some intertwine, which is sequential),
% this is a linear graph. What programmer does is describe the arrows.
% The whole graph has bilions of computational steps, which we can't create nor verify.
% Yet, we want to reason about it.
% We want to also be able to construct data, e.g. the input.
% Verification checker has to confirm that if a precondition holds in state S before computation,
% then it also holds for F(S), which is state after computation. F is the semantics
% of some programming language function, so we need to have one functional symbol
% for every possible state function describable.

% The weaker the type system, and the more complicated the function, the less likely
% it is that it will be able to solve the tautology (by automated theorem proving).

% In functional languages, the syntax to describe data and functions is the same:
% lambda terms. Verification has to be supplied by the user as types.

% The interpreter is essentialy a proof that for every state, precisely one consecutive state
% of computation exists.


% # Introduction
% This thesis investigates the problem of certifying computational complexity of standard computer algorithms. The most intuitive solution to it is to formalize a standard computational model (e.g. Turing machines or lambda calculus) and its notion of computational complexity in a proof assistant such as Rocq or Lean. Then, implement the algorithm in such a formalization and prove that it will e.g. execute in polynomial time. This, however, proves to be extremely difficult. The proofs from the area of computational complexity tend to be very "hand-wavy" and making them formal is orders of magnitude more involved than formalizing other areas of mathematics [forster:LIPIcs.CSL.2025.3].
% As of September 2025, no feasible way of doing that on scale has been discovered.

% A promising approach is to first give an “on-paper” proof that the functions in a complexity class are equivalent to the semantics of a specially designed programming language. Then, instead of repeatedly checking complexity proofs for each algorithm, we can just define the algorithms in this language. That way, we only need to manually verify the interpreter once, and afterwards we can let the interpreter automatically check whether any given program is valid.

% This thesis investigates approaches to designing programming languages whose expressiveness is precisely aligned with a target computational complexity class. In such languages, every program would, by construction, operate within the given class, and conversely, any function or problem in that class would be expressible and implementable in the language.

% To illustrate the challenge we are addressing, consider the task of proving that binary multiplication belongs to LOGSPACE. 
% A common way to do it is to first argue that if we show a program that does it using only a finite number of variables, each of which of linear size (e.g. a pointer to the input), then the computation obviously uses only logarithmic amount of memory.
% However, this description is fragile: for instance, simply incrementing a variable in a `for` loop can exceed the LOGSPACE bound. In this work, we investigate more reliable characterizations.

% The ultimate goal of this thesis is to design a practical hierarchy of programming languages that precisely captures resource-bounded computations - for example, languages that express exactly the class of algorithms running in $\mathcal{O}(n^2)$ time or $\mathcal{O}(n)$ space. In the chapters that follow, we explore which restrictions of this kind are feasible, and which are not.

% The structure of this work is as follows:  
% - Chapter 1 reviews open sub-directions that remain unexplored, including descriptive complexity
% - Chapter 2 introduces Implicit Computational Complexity (ICC).  
% - Chapter 3 presents the Curry–Howard approach to ICC, focusing on IntML.  
% - Chapter 4 explores the recursion-theoretic approach, focusing on Neergaard's BC.  
% - Chapter 5 discusses the logical approach, drawing on my work for AITP and at INRIA.  

% Among these, only the logical approach has proven scalable. The others, after extensive investigation, appear unlikely to yield systems that are practical in use.

% \section{Requirements on the product: relation between syntax, semantics, and the interpreter complexity}
% ## Some obvious and obviously impractical approaches
% - The obvious approach: LOOP language.  
% problem: it is PRA. it is not practical.

% - Another obvious approach is: python + polynomial expressing max  
% number of steps. Problem: semantics impossible to reason about.

% - another obvious approach: "natural number coding nth program of a language
% having these properties". problem: syntax might be undecidable! won't write an
% interpreter.

% - another approach: only use Nat type and grzegorczyk hierarchy. problem: ?
\chapter{Preliminaries}\label{chap:preliminaries}
In this chapter we introduce the standard definitions for logic and complexity theory, necessary to
fix the language we will use in the rest of the work. 
We solely focus on first-order logic and don't introduce the similar
definitions of propositional calculus and second-order logic. The semantics is entirely classical
(in the sense of classical or intuitionistic logic). 

In~\autoref{sec:defs-preliminaries-circuits} we exclusively introduce the so-called \emph{non-uniform}
circuit families, to later consider the different notions of uniformity in~\autoref{chap:circuits}.

\begin{remark}[Bibliography]
The definitions in~\autoref{sec:defs-single-sorted} are in the style of~\cites[Section~2B]{CookNguyenDraft}[Section~II.2]{Cook_Nguyen_2010}.
The definitions in~\autoref{sec:defs-two-sorted} are in the style of~\cites[Section~4B]{CookNguyenDraft}[Section~IV.2]{Cook_Nguyen_2010}.
We will mostly need them for~\autoref{chap:bounded-arithmetic}, but also for~\autoref{chap:descriptive-complexity} --- 
the results discussed in the latter are mostly from~\cite{Immerman1999-IMMDC},
where different style of definitions is used.
However,~\cite{Immerman1999-IMMDC} doesn't introduce two-sorted logic.
The single-sorted definitions differences are mostly negligible and the consistent treatment of
single- and two-sorted logic, is important for~\autoref{chap:bounded-arithmetic}.

Much effort has been put into
considering different versions of the definitions in~\autoref{sec:decisional-complexity-classes}
to keep them consistent
with the definitions of functional complexity classes studied later in~\autoref{sec:functional-complexity-classes},
e.g.\ the class \complexity{FNP} defined in~\autoref{def:complexity-fnp}.
These are much less standard, and often have no clear consensus in the literature.

The definitions of decisional circuit classes are from~\cite{10.5555/520668}.
\end{remark}


\section{Single-sorted first-order logic}\label{sec:defs-single-sorted}

\begin{definition}[First-order vocabulary and syntax]
A \emph{first-order vocabulary} (or \emph{language}) $\mathcal{L}$ consists of:
\begin{enumerate}
  \item For each $n \ge 0$, a (possibly empty) set of $n$-ary \emph{function symbols}.
        We use $f,g,h,\dots$ as meta-variables for function symbols.
        A $0$-ary function symbol is called a \emph{constant symbol};
  \item For each $n \ge 0$, a set of $n$-ary \emph{predicate symbols}, which is
        nonempty for at least one $n$.
        We use $P,Q,R,\dots$ as meta-variables for predicate symbols.
\end{enumerate}

In addition, the following logical symbols are available to build first-order
terms and formulas:
\begin{enumerate}
  \item An infinite set of \emph{variables}. We use $x,y,z,\dots$ and sometimes
        $a,b,c,\dots$ as meta-variables for variables;
  \item The connectives $\lnot$, $\land$, $\lor$ (not, and, or) and
        logical constants $\bot$, $\top$ (False, True);\todo{consistency: semicolon in enumerate}
  \item The quantifiers $\forall$, $\exists$ (for all, there exists);
  \item Parentheses $(\, ,\,)$.
\end{enumerate}
\end{definition}

% Expressive power of different possibilities
% \item $\text{FO}[+, *] = \text{FO}[\mathrm{BIT}]: Section~1.2.1 Immerman $.
% \item $\text{FO}[+]$ is less expressive than $\text{FO}[<, *] = \text{FO}[<, /] = \text{FO}[<, \mathrm{COPRIME}]$~\cite{10.1002/malq.200310041}.


\begin{definition}[$\mathcal{L}$-terms]
Let $\mathcal{L}$ be a first-order vocabulary.
The set of \emph{$\mathcal{L}$-terms} is defined inductively as follows:
\begin{enumerate}
  \item Every variable is an $\mathcal{L}$-term;
  \item If $f$ is an $n$-ary function symbol of $\mathcal{L}$ and
        $t_1,\dots,t_n$ are $\mathcal{L}$-terms, then
        \[
          f(t_1,\dots,t_n)
        \]
        is an $\mathcal{L}$-term.
\end{enumerate}
\end{definition}

\begin{definition}[$\mathcal{L}$-formulas]
Let $\mathcal{L}$ be a first-order vocabulary.
The set of \emph{first-order formulas in $\mathcal{L}$} (or
\emph{$\mathcal{L}$-formulas}) is defined
inductively as follows:
\begin{enumerate}
  \item The logical constants $\bot$ and $\top$ are atomic formulas;
  \item If $P$ is an $n$-ary predicate symbol in $\mathcal{L}$ and
        $t_1,\dots,t_n$ are $\mathcal{L}$-terms, then
        \[
          P(t_1,\dots,t_n)
        \]
        is an \emph{atomic} $\mathcal{L}$-formula;
  \item If $A$ and $B$ are $\mathcal{L}$-formulas, then
        $\lnot A$, $(A \land B)$, and $(A \lor B)$ are $\mathcal{L}$-formulas;
  \item If $A$ is an $\mathcal{L}$-formula and $x$ is a variable, then
        $\forall x\,A$ and $\exists x\,A$ are $\mathcal{L}$-formulas.
\end{enumerate}
For example,
\[
  (\lnot \forall x\,P x \,\lor\, \exists x\,\lnot P x)
  \quad\text{and}\quad
  (\forall x\,\lnot P x y \,\land\, \lnot \forall z\,P f y z)
\]
are $\mathcal{L}$-formulas (for suitable choices of $P$ and $f$ in $\mathcal{L}$).
\end{definition}

\begin{definition}[The language of arithmetic]\label{def:LA}
The \emph{language of arithmetic} is
\[
  L_{A} = [\,0,\,1,\,+,\,\cdot \;;\; =,\,\le\,],
\]
where \(0\) and \(1\) are constant symbols, \(+\) and \(\cdot\) are binary
function symbols, and \(=\) and \(\le\) are binary predicate symbols.
We will write these symbols in infix form.
\end{definition}

\begin{definition}[Free and bound variables]\label{def:free-bound}
Let $A$ be a formula and $x$ a variable.
An occurrence of $x$ in $A$ is \emph{bound} if it lies within a subformula
of $A$ of the form $\forall x\,B$ or $\exists x\,B$.
Any other occurrence of $x$ in $A$ is called \emph{free}.
\end{definition}

\begin{definition}[Closed terms, closed formulas, sentences]\label{def:closed-sentence}
A formula is \emph{closed} if it contains no free occurrence of any variable.
A term is \emph{closed} if it contains no variables at all.
A closed formula is also called a \emph{sentence}.
\end{definition}

\begin{definition}[$\mathcal{L}$-structure]\label{def:L-structure}
Let $\mathcal{L}$ be a first-order vocabulary.
An \emph{$\mathcal{L}$-structure} $\mathcal{M}$ consists of:
\begin{enumerate}
  \item A nonempty set $M$, called the \emph{universe}
        (Variables are intended to range over $M$);
  \item For each $n$-ary function symbol $f$ in $\mathcal{L}$, an associated function
        \(
          f^{\mathcal{M}} : M^n \to M
        \);
  \item For each $n$-ary predicate symbol $P$ in $\mathcal{L}$, an associated relation
        \(
          P^{\mathcal{M}} \subseteq M^n
        \).
\end{enumerate}
\end{definition}

\begin{remark}
    Note that to ``syntactical'' relations, we assign ``real'' relations defined on
    the underlying elements of the structure. We will want to treat some of these
    relations specially, e.g.\ to make sure that the ``\(=\)'' relation is
    interpreted as the actual equality, or that a designated ``\(\text{PLUS}(x, y, z)\)''
    relation holds only if the underlying objects are actual natural numbers,
    for which we have $x + y = z$. \todo{add autoref to where we talk about that}
% Thus the predicate symbol $=$ receives special treatment: it must always be
% interpreted as actual equality on the universe. We can also consider
% logics where we don't take the $=$ symbol as granted. For our purposes, however,
% the things we will be able to say about the class of models of a given formula
% will be more interesting if we already assume that whenever $p = q$ holds
% in the structure, then the underlying objects of the universe are also equal
% (in the meta-mathematical sense).
\end{remark}


\begin{definition}[Object Assignment]
Let $\mathcal{M}$ be a structure with universe $M$.  
An \emph{object assignment} \(\sigma\) for $\mathcal{M}$ is a mapping
from variables to the universe $M$.
\end{definition}

\begin{notation}
Let $x$ be a variable and $m \in M$.  
We write $\sigma(m/x)$ for the
assignment that is the same as $\sigma$ except that it maps $x$ to $m$.
\end{notation}

\begin{definition}[Basic Semantic Definition]
Let $\mathcal{L}$ be a first-order vocabulary, let $\mathcal{M}$ be an $\mathcal{L}$-structure
with universe $M$, and let $\sigma$ be an object assignment for $\mathcal{M}$.

\paragraph{Interpretation of terms.}
Each $\mathcal{L}$-term $t$ is assigned an element $t^{\mathcal{M}}[\sigma] \in M$,
defined by structural induction on $t$:
\begin{enumerate}
  \item For each variable $x$,
  \(
    x^{\mathcal{M}}[\sigma] = \sigma(x).
  \)
  \item 
  \(
    (f t_1 \dots t_n)^{\mathcal{M}}[\sigma]
      = f^{\mathcal{M}}\bigl(t_1^{\mathcal{M}}[\sigma],\dots,t_n^{\mathcal{M}}[\sigma]\bigr).
  \)
\end{enumerate}

\paragraph{Satisfaction of formulas.}
For an $\mathcal{L}$-formula $A$, the relation
\[
  \mathcal{M} \models A[\sigma]
\]
(read: ``$\mathcal{M}$ satisfies $A$ under $\sigma$'') is defined by structural
induction on $A$:
\begin{enumerate}
  \item $\mathcal{M} \models \top$ and $\mathcal{M} \not\models \bot$.
  \item For an atomic formula $P t_1 \dots t_n$ (with $P$ an $n$-ary
        predicate symbol),
  \[
    \mathcal{M} \models (P t_1 \dots t_n)[\sigma]
    \;\;\text{iff}\;\;
    \bigl\langle t_1^{\mathcal{M}}[\sigma],\dots,t_n^{\mathcal{M}}[\sigma]\bigr\rangle
    \in P^{\mathcal{M}}.
  \]
  \item If $\mathcal{L}$ contains $=$, then for terms $s,t$,
  \[
    \mathcal{M} \models (s = t)[\sigma]
    \;\;\text{iff}\;\;
    s^{\mathcal{M}}[\sigma] = t^{\mathcal{M}}[\sigma].
  \]
  \item $\mathcal{M} \models \neg A[\sigma]$ iff $\mathcal{M} \not\models A[\sigma]$.
  \item $\mathcal{M} \models (A \lor B)[\sigma]$ iff
        $\mathcal{M} \models A[\sigma]$ or $\mathcal{M} \models B[\sigma]$.
  \item $\mathcal{M} \models (A \land B)[\sigma]$ iff
        $\mathcal{M} \models A[\sigma]$ and $\mathcal{M} \models B[\sigma]$.
  \item $\mathcal{M} \models (\forall x\,A)[\sigma]$ iff
        $\mathcal{M} \models A[\sigma(m/x)]$ for all $m \in M$.
  \item $\mathcal{M} \models (\exists x\,A)[\sigma]$ iff
        $\mathcal{M} \models A[\sigma(m/x)]$ for some $m \in M$.
\end{enumerate}

\noindent
If $t$ is a closed term, then $t^{\mathcal{M}}[\sigma]$
is independent of $\sigma$, and we simply write $t^{\mathcal{M}}$.
Similarly, if $A$ is a sentence, we often write
$\mathcal{M} \models A$ instead of $\mathcal{M} \models A[\sigma]$, since
the choice of $\sigma$ does not matter.
\end{definition}

\section{Two-sorted first-order logic}\label{sec:defs-two-sorted}
% Sec. 2.2: Akapit 1 sugeruje, jakbyśmy nic nie mówili o logice 2-sortowej, a zaraz po nim miało nastąpić coś
% innego. Przeformułuj, aby był jaśniejszy (w stylu "pomijamy rzeczy podobne, a wypiszemy tylko różnice").

% Ponadto rozdziały 2.1 i 2.2 mają zupełnie inny charakter. W 2.1 jest logika nad dowolną sygnaturą, a w 2.2
% konkretnie liczby i napisy. Pasuje coś powiedzieć na początku 2.2 - że choć logikę 2-sortową można ogólnie
% definiować dla dowolnych sortów, to my piszemy tylko o tych konkretnych, liczby i napisy.

% I też właśnie trzeba konkretnie napisać, że sorty to liczby i napisy, przed Def. 2.2.1 - bo teraz w środku
% tej definicji nagle się pojawia coś o napisach i liczbach i nie wiadomo o co chodzi.



Two-sorted first-order logic extends the single-sorted setting in a routine way, so we only record
the additions that we use later.  A systematic presentation can be found
in~\cites[Section~4B]{CookNguyenDraft}[Section~IV.2]{Cook_Nguyen_2010}.
In principle one could work with arbitrary pairs of sorts, but in this thesis we instantiate the
framework to the familiar number sort (ranging over $\mathbb{N}$) and string sort (ranging over
finite binary strings).  The goal of this section is therefore to emphasise what changes when we
move from the definitions of \autoref{sec:defs-single-sorted} to this concrete two-sorted
setting.

\begin{definition}[Two-sorted first-order vocabularies]\label{def:two-sorted-vocabulary}
A \emph{two-sorted first-order vocabulary} (often abbreviated simply as a
two-sorted language) \(\mathcal{L}\) consists of collections of function and predicate
symbols, much like an ordinary single-sorted vocabulary, but now the
symbols may accept arguments of either of the two sorts.  Moreover, the
function symbols come in two varieties:
\begin{enumerate}
  \item \emph{number-valued} function symbols, whose outputs lie in the
        number sort; and
  \item \emph{string-valued} function symbols, whose outputs lie in the
        string sort.
\end{enumerate}

For any pair \(n,m \in \mathbb{N}\), the vocabulary contains:
\begin{enumerate}
  \item a set of \((n,m)\)-ary number-function symbols,
  \item a set of \((n,m)\)-ary string-function symbols, and
  \item a set of \((n,m)\)-ary predicate symbols.
\end{enumerate}
A \((0,0)\)-ary function symbol is simply a constant symbol, which may be
either a constant of the number sort or a constant of the string sort.

We use \(f,g,h,\dots\) as metavariables for number-valued function symbols,
\(F,G,H,\dots\) for string-function symbols, and \(P,Q,R,\dots\) for predicate
symbols.
\end{definition}

% TODO: TUTAJ DOPISAC REMARK ZE TO BEDZIE JAKO AKSJOMAT!
% Okolice Def. 2.2.2: Implicite zakładamy związek między |X| oraz t\in X, więc nie dowolna struktura tylko
% coś w stylu że nie zachodzi t\in X dla liczb >= |X|.

\begin{definition}[The language \(\mathcal{L}^{2}_{A}\)]\label{def:L2A}
As an example, consider the following two-sorted extension of the
arithmetical language \(\mathcal{L}_{A}\)~(\autoref{def:LA}):
\[
  \mathcal{L}^{2}_{A} \;=\; [\,0,\ 1,\ +,\ \cdot,\ |\cdot| \;;\ =_{1},\ =_{2},\ \le,\ \in\,].
\]

Here the symbols \(0,1,+,\cdot,=_{1},\le\) are
symbols of \(\mathcal{L}_{A}\) (with \(=_{1}\) corresponding to the usual equality of
numbers).  
The symbol \(|X|\) is a number-valued function symbol giving the
length of a string \(X\).  
The binary predicate \(\in\) relates a number and a string and is used to
express membership: intuitively, \(i \in X\) means that the \(i\)-th bit of
the string \(X\) is \(1\).  
The symbol \(=_{2}\) denotes equality between objects of the second sort.

For convenience, when \(t\) is a number term, we abbreviate
\[
  X(t) \;\coloneqq\; t \in X.
\]
Thus \(X(i)\) plays the role of the \(i\)-th bit of the binary string \(X\).

In \(\mathcal{L}^{2}_{A}\), the symbols \(+\) and \(\cdot\) each have arity \((2,0)\); the
length function \(|\cdot|\) has arity \((0,1)\); and the predicate
\(\in\) has arity \((1,1)\).
\end{definition}



\begin{notation}[Bounded formulas]\label{def:bounded-formulas}
Let \(\mathcal{L}\) be a two-sorted vocabulary.  
If \(x\) is a number variable and \(X\) a string variable that do not occur
in the \(\mathcal{L}\)-number term \(t\), we use the following abbreviations:
\begin{align*}
  \exists x \le t\ldotp \varphi
    &\;\;\text{stands for}\;\;
      \exists x\ldotp (x \le t \;\wedge\; \varphi), \\[2pt]
  \forall x \le t\ldotp \varphi
    &\;\;\text{stands for}\;\;
      \forall x\ldotp (x \le t \;\to\; \varphi), \\[2pt]
  \exists X \le t\ldotp \varphi
    &\;\;\text{stands for}\;\;
      \exists X\ldotp (\,|X| \le t \;\wedge\; \varphi), \\[2pt]
  \forall X \le t\ldotp \varphi
    &\;\;\text{stands for}\;\;
      \forall X\ldotp (\,|X| \le t \;\to\; \varphi).
\end{align*}

A quantifier appearing in one of these forms is called \emph{bounded},
and a \emph{bounded formula} is a formula in which every quantifier is
bounded.
\end{notation}


\paragraph{Notation.}
The expression
\(\exists \vec{x} \le \vec{t}\ldotp \varphi\)
abbreviates a block of bounded number quantifiers
\(\exists x_1 \le t_1\ldotp \cdots \exists x_k \le t_k\ldotp \varphi\)
for some \(k\), where no variable \(x_i\) occurs in any term \(t_j\)
(even when \(i < j\)).  
The same convention applies to \(\forall \vec{x} \le \vec{t}\), \(\exists \vec{X} \le \vec{t}\), and
\(\forall \vec{X} \le \vec{t}\).


\begin{definition}[The \texorpdfstring{$\Sigma^1_{1}(\mathcal{L})$, $\Sigma^B_{i}(\mathcal{L})$,
  and $\Pi^B_{i}(\mathcal{L})$}{Σ¹₁(L), Σᵢᴮ(L), Πᵢᴮ(L)} formulas]\label{def:SigmaB-PiB-hierarchy}
Let \(\mathcal{L} \supseteq \mathcal{L}^{2}_{A}\) be a two-sorted vocabulary.  
% Druga potencjalna modyfikacja to dopuszczenie kwantyfikatorów po liczbach między tymi po napisach.
% Obecnie w ogóle nie może być kwantyfikatora po napisie wewnątrz takiego po liczbie. To istotne?
% Warto to napisać wprost jeszcze raz pod spodem.
\begin{enumerate}
  \item The class \(\Sigma^{B}_{0}(\mathcal{L}) = \Pi^{B}_{0}(\mathcal{L})\) consists of all
        \(\mathcal{L}\)-formulas whose only quantifiers are \emph{bounded number quantifiers}
        (string variables may occur free).

  \item For \(i \ge 0\), the class \(\Sigma^{B}_{i+1}(\mathcal{L})\) (resp.\ \(\Pi^{B}_{i+1}(\mathcal{L})\))
        consists of formulas of the form
        \[
          \exists \vec{X} \le \vec{t}\ldotp \varphi(\vec{X})
          \quad\text{(resp.\ }\forall \vec{X} \le \vec{t}\ldotp \varphi(\vec{X})\text{)},
        \]
        where:
        \begin{enumerate}
            \item \(\vec{X}\) is a vector of string variables,
            % TODO: Def. 2.2.3 (ii) - "not involving variables from X" -
            % nasuwa się pytanie co by było, gdybyśmy dopuścili? Nic by się nie zmieniło.
            % Więc warto gdzieś tam dopisać, że to tylko założenie dla porządku, które nic nie zmienia.
            \item \(\vec{t}\) is a vector of \(\mathcal{L}^{2}_{A}\)-terms not involving variables from \(\vec{X}\),
            \item \(\varphi\) is a \(\Pi^{B}_{i}(\mathcal{L})\) formula  
                  (resp.\ a \(\Sigma^{B}_{i}(\mathcal{L})\) formula).
        \end{enumerate}

  \item A \(\Sigma^{1}_{1}(\mathcal{L})\) formula is a formula of the form
        \(
          \exists \vec{X}\ldotp \varphi,
        \)
        where \(\vec{X}\) is a vector of zero or more string variables and
        \(\varphi\) is a \(\Sigma^{B}_{0}(\mathcal{L})\) formula.
\end{enumerate}

We usually write \(\Sigma^{B}_{i}\) for \(\Sigma^{B}_{i}(\mathcal{L}^{2}_{A})\) and
\(\Pi^{B}_{i}\) for \(\Pi^{B}_{i}(\mathcal{L}^{2}_{A})\).
\end{definition}


\begin{remark}
The formalism described above coincides with the usual weak monadic second-order logic (WMSO) on
words.  We phrase it as a two-sorted first-order system only to match the presentation
in~\cite{CookNguyenDraft,Cook_Nguyen_2010}, but in practice we freely identify it with WMSO and use
that terminology when convenient, following the convention in~\cite{COOK2003193}.
% Ostatnie zdanie rozdziału, o MSO: w MSO na słowach skończonych też kwantyfikujemy po zbiorach skończonych.
% Więc oprócz tego, że zbiory są skończone warto napisać, że "pomimo iż liczby są z nieskończonego zbioru
% liczb naturalnych" (w sumie też nigdzie nie jest napisane, że to liczby naturalne, a nie całkowite,
% wymierne, czy jeszcze jakieś inne).
% W tym kontekście używa się często nazwy WMSO (weak MSO) - czy ona nie byłaby odpowiednia?
\end{remark}

\section{Functions computable by Turing machines}\label{sec:preliminaries-turing-functions}
We introduce the notion of computing a general $\{0,1\}^{\ast} \to \{0,1\}^{\ast}$ function
early, as this is the primary interest of this thesis. Most of the literature in computational complexity
focuses solely
on computing Boolean functions which we introduce in~\autoref{sec:preliminaries-turing}.
To properly discuss this imbalance, we postpone introducing the \emph{functional}
complexity classes until~\autoref{chap:reductions}.


\begin{definition}[{\cites[Definition~1.3]{10.5555/1540612}[Definition~1.4]{DRAFT10.5555/1540612}}]
Let~\(f : \{0,1\}^{\ast} \to \{0,1\}^{\ast}\) and \(T : \mathbb{N} \to \mathbb{N}\) be functions, and let
\(M\) be a Turing machine.  We say that \(M\) \emph{computes} \(f\) if, for every input
        \(x \in \{0,1\}^{\ast}\), when \(M\) is started in its initial configuration
        on input \(x\), it eventually halts with the string \(f(x)\) written on its
        output tape.
\end{definition}


\section{Decisional Turing machine complexity classes}\label{sec:preliminaries-turing}

In this section we introduce standard complexity classes such as 
\(\complexity{L}\), \(\complexity{P}\) and \(\complexity{NP}\). It is important to note that these 
classes only contain \emph{decision} problems, i.e.\ only require
computing a function \(f : \{0,1\}^\ast \to \{0,1\}\). Complexity classes for general
functions will appear in.

\begin{notation}
    We say that a machine \emph{decides} a language \(L \subseteq \{0, 1\}^\ast\) iff it computes
    the function \(f_L: \{0, 1\}^\ast \rightarrow \{0, 1\}\), where \(f_L(x) = 1 \iff x \in L\).
\end{notation}

\begin{definition}[Time complexity~{\cites[Definition~1.19]{DRAFT10.5555/1540612}[Definition~1.12]{10.5555/1540612}}]\label{def:turing-dtime}
\hfill\\
Let~\(T~:~\mathbb{N} \to \mathbb{N}\) be a function.
We define \(\complexity{DTIME}(T(n))\) to be the class of all Boolean
functions that are computable by some deterministic
Turing machine running in at most \(c_1 \cdot T(n) + c_2\) steps on every input of
length \(n\), for some constants \(c_1, c_2 > 0\).
\end{definition}

\begin{definition}[Space complexity~{\cites[Definition~4.1]{DRAFT10.5555/1540612}[Definition~4.1]{10.5555/1540612}}]\label{def:turing-dspace}
\hfill\\
Let~\(S~:~\mathbb{N} \to \mathbb{N}\) be a function and let
\(L \subseteq \{0,1\}^{\ast}\) be a language.  
% \begin{enumerate}
%   \item 
  We say that \(L \in \complexity{SPACE}(S(n))\) iff there exist constants
        \(c_1, c_2 > 0\) and a deterministic Turing machine \(M\) deciding \(L\) such that,
        on every input of length \(n\), the machine \(M\) visits at most
        \(c_1 \cdot S(n) + c_2\) distinct cells on its read-write work tapes
        (the input tape is read-only and does not count toward the space
        bound).

%   \item Likewise, \(L \in \complexity{NSPACE}(S(n))\) if there exist a constant
%         \(c > 0\) and a nondeterministic Turing machine \(M\) deciding \(L\) such
%         that, for every input of length \(n\) and for every computation branch,
%         \(M\) uses no more than \(c \cdot S(n)\) nonblank work-tape cells.
% \end{enumerate}
\end{definition}


\begin{definition}[Logarithmic space~{\cites[Definition~4.5]{10.5555/1540612}[Definition~4.5]{DRAFT10.5555/1540612}}]\label{def:complexity-l}
\[
    \complexity{L} = \complexity{SPACE}(\log n).
\]
\end{definition}


\begin{definition}[Polynomial time~{\cites[Definition~1.13]{10.5555/1540612}[Definition~1.20]{DRAFT10.5555/1540612}}]\label{def:complexity-p}
\[
    \complexity{P} = \bigcup_{c \ge 1} \complexity{DTIME}(n^c).
\]
\end{definition}



\begin{definition}[The class \texorpdfstring{\complexity{NP}}{NP}{~\cites[Definition~2.1]{10.5555/1540612}[Definition~2.1]{DRAFT10.5555/1540612}}]\label{def:complexity-np}
\hfill\\
A~language~\(L \subseteq \{0,1\}^{\ast}\) belongs to \(\complexity{NP}\) if there exist
\begin{enumerate}
  \item a polynomial \(p : \mathbb{N} \to \mathbb{N}\) (bounding the length of a certificate); and
  \item a deterministic polynomial-time Turing machine \(M\)
        (called a \emph{verifier} for \(L\)),
\end{enumerate}
such that for every input string \(x \in \{0,1\}^{\ast}\),
\[
  x \in L
  \;\Longleftrightarrow\;
  \exists\,u \in \{0,1\}^{\,p(\len{x})} \ \text{with}\ M(x,u) = 1.
\]

Whenever \(x \in L\) and a string \(u \in \{0,1\}^{p(\len{x})}\) satisfies \(M(x,u)=1\),
the string \(u\) is called a \emph{certificate} (or \emph{witness}) for \(x\)
with respect to the language \(L\) and the verifier~\(M\).
\end{definition}

\section{Classes of binary circuits}\label{sec:defs-preliminaries-circuits}


\begin{definition}
\todo[inline]{Define what is a circuit perhaps?}
\end{definition}

\begin{definition}[\NCipoly, \ACipoly, \TCipoly]\label{def:ac-nc-tc-poly}

Fix \(i \ge 0\).  
A language \(L \subseteq \{0,1\}^\ast\) belongs to one of the following
circuit classes if there exists a family of circuits
\(\{C_n\}_{n\in\mathbb{N}}\) such that \(C_n(x)=1\) iff \(x\in L\) and:

\begin{enumerate}
    \item\label{itm:circ-size} every circuit \(C_n\) has polynomially many gates w.r.t. \(n\);
    \item\label{itm:circ-depth} every circuit \(C_n\) has depth \(\bigO((\log n)^i)\);
    \item\label{itm:circ-nc} \(L \in \NCpoly{i}\) (Nick's class)  
          if each \(C_n\) uses only fan-in~2 \(\wedge\)-, fan-in~2 \(\vee\)-gates
          and fan-in~1 \(\neg\)-gates;

  \item\label{itm:circ-ac} \(L \in \ACpoly{i}\)  (Alternating circuits)
        if each \(C_n\) uses  
        unbounded fan-in \(\wedge\)-, unbounded fan-in \(\vee\)-gates and
        fan-in~1 \(\neg\)-gates;

  \item \label{itm:circ-tc}\(L \in \TCpoly{i}\)  (Threshold circuits)
        if each \(C_n\) uses unbounded fan-in \(\wedge\)-, unbounded fan-in \(\vee\)-gates,
        fan-in~1 \(\neg\)-gates and unbounded fan-in \emph{majority} gates
        (i.e.\ a gate that outputs 1 iff at least half of its inputs are one).
\end{enumerate}

Note that the condition \(C_n(x)=1\) iff \(x\in L\) requires the circuits
to have precisely one output node. We lift this requirement in~\autoref{def:aci-faci}.
% The full hierarchies are:
% \[
%   \complexity{AC}_{/poly}
%     = \bigcup_{i\ge 0} \ACpoly{i}, 
%   \qquad
%   \complexity{NC}_{/poly}
%     = \bigcup_{i\ge 0} \NCpoly{i}, 
%   \qquad
%   \complexity{TC}_{/poly}
%     = \bigcup_{i\ge 0} \TCpoly{i}.
% \]
\end{definition}


\begin{definition}[\FNCipoly, \FACipoly, \FTCipoly]\label{def:faci-poly}
A function \(F : \{0, 1\}^\ast \rightarrow \{0, 1\}^\ast\) belongs to
\FNCipoly (resp. \FACipoly, \FTCipoly) iff there exists a family of circuits
with multiple output nodes
\(
  \langle C_n \rangle_{n \in \mathbb{N}}
\)
such that whenever the input bits of \(C_n\) encode \(X \in \{0, 1\}^n\), the output bits
encode \(F(X)\); \(C_n\)
satisfies the size and depth
conditions~\ref{itm:circ-size},~\ref{itm:circ-depth} of~\autoref{def:ac-nc-tc-poly};
and additionally \(C_n\) satisfies the class condition~\ref{itm:circ-nc}
(resp.~\ref{itm:circ-ac},~\ref{itm:circ-tc}) of~\ref{def:ac-nc-tc-poly}.
\end{definition}

\begin{definition}[\compL-uniform circuit families]\label{def:logspace-uniformity}
      We say that a family of circuits \(\langle C_n \rangle_{n \in \mathbb{N}}\)
      is \compL-uniform if there exists a Turing machine operating in space \(\bigO(\log n)\),
      computing the function \(1^n \to C_n\) for some representation of \(C_n\).
\end{definition}

\begin{definition}[\NCi, \ACi, \TCi, \FNCi, \FACi, \FTCi]\label{def:aci-faci}
      A function \(F : \{0, 1\}^\ast \rightarrow \{0, 1\}^\ast\) belongs to
\FNCi (resp. \FACi, \FTCi) iff there exists a \compL-uniform family of circuits
satisfying the conditions from~\autoref{def:faci-poly}.
\end{definition}


% \subsection{Cicuit complexity classes}
% The complexity classes in this subsection are \emph{circuit complexity} classes,
% which means that the computation is done by \emph{boolean circuits}, which we will
% now introduce. Note that this is a different computational model to Turing machines, finite
% automata or lambda calculi, and thus comes with its own notion of complexity. The definitions
% in this subsection are based on~\cite[Definition 5.17]{Immerman1999-IMMDC}.



% \begin{definition}[Words accepted by a circuit]
% We say that a circuit \emph{accepts} a given binary word iff the value of its root is 1 with values of leaves
% set according to the input.
% \end{definition}

% \begin{definition}[Language decided by circuit family]
% We say that a family $\langle C_n \rangle_{n \in \mathbb{N}}$
% of boolean circuits decides a language $L$ iff for every $w \in \{0, 1\}^*$, $C_{|w|}$ accepts
% $w$ iff $w \in L$.
% \end{definition}

% \begin{definition}[\complexity{NC} circuits (Nick's class)]
% A boolean circuit is \emph{\complexity{NC}} iff the gates are only binary AND and OR gates.
% \end{definition}

% \begin{definition}[\complexity{AC} circuits (Alternating circuits)]
% A boolean circuit is \emph{\complexity{AC}} iff the gates are only unlimited fan-in AND and OR gates.
% It is a theorem that we can also allow unary NOT gates at leaves only.
% \end{definition}

% \begin{definition}[\complexity{TC} circuits (Threshold circuits)]
% A boolean circuit is \emph{\complexity{AC}} iff the gates are only unlimited fan-in Threshold(?) gates.
% It is a theorem that we can also allow unlimited fan-in AND and OR gates(?).
% \end{definition}

% \begin{definition}[\complexity{NC[t(n)]_{/poly}}, \complexity{AC[t(n)]_{/poly}}, \complexity{TC[t(n)]_{/poly}}]
% We define $\complexity{NC[t(n)]_{/poly}}$ to be the class of $\complexity{NC}$ circuits that:
% \begin{enumerate}
%    \item have polynomially-many internal nodes w.r.t. $n$, the number of leaves.
%    \item have depth $\bigO(t(n))$.
% \end{enumerate}
% Let $\complexityi{NC_{/poly}}{i} = \complexity{NC[(\log n)^i]_{/poly}}$. We define
% $\complexity{AC[t(n)]_{/poly}}, \complexity{TC[t(n)]_{/poly}}, \complexityi{AC_{/poly}}{i}, \complexityi{TC_{/poly}}{i}$ analogously.

% In particular, $\complexityi{NC_{/poly}}{0}$, $\complexityi{AC_{/poly}}{0}$, $\complexityi{TC_{/poly}}{0}$ are families of circuits with
% uniformly constant depth.

% For now we have not talked about how do we generate the consecutive circuits $C_n$.
% Later in~\autoref{sec:uniformity} we will restrict the family of (some standard) descriptions of the circuits
% to have to be \emph{computable efficiently} and then we will be able to drop the $_{/poly}$ suffix in our complexity classes.
% \end{definition}


% \section{\texorpdfstring{$\complexityi{NC}{i}_{/poly}$}{NC\string^ipoly}}

% \subsection{\complexityi{NC}{0}}
% For example, each output of an \complexityi{NC}{0} computable function can depend on only finitely many
% inputs. Thus, \complexityi{NC}{0} can't even compute an AND of all its inputs (in contrast, the unbounded
% fan-in AND is an \complexityi{AC}{0} function).
% \paragraph{TODO: Notes on $\complexityi{NC}{0}$.}
% TODO: Provide an introductory overview of the key references on $\complexityi{NC}{0}$ listed below.
% \begin{enumerate}
% \item TODO: Summarize the construction of one-way permutations in $\complexityi{NC}{0}$~\cite{10.1016/0020-01908790053-6}.
% \item TODO: Explain why all sets complete under $\complexityi{AC}{0}$ reductions are already complete under $\complexityi{NC}{0}$ reductions~\cite{10.1145/258533.258671,AGRAWAL1998127}.
% \item TODO: Describe Immerman's page~81 discussion of addition in $\complexityi{NC}{0}$ and MAJORITY in $\complexityi{NC}{1}$~\cite{Immerman1999-IMMDC}.
% \item TODO: Detail why addition and subtraction of binary numbers lies in $\complexityi{AC}{0}$~\cite{27676}.
% \end{enumerate}


% Given $x, y, z$: binary representations of natural numbers,
% it is decidable in uniform \complexityi{AC}{0} if $x + y = z$, but it is not decidable 
% in \complexityi{AC}{0} if $x * y = z$.

% \subsection{\complexityi{AC}{0}}
% \begin{enumerate}
% \item TODO: Summarize why addition is in $\complexityi{AC}{0}$~\cite{BussLectureNotes}.
% \item TODO: Explain the equivalence $\text{FO}[+, *] = \complexity{DLOGTIME}$-uniform $\complexityi{AC}{0}$ (see \url{https://complexityzoo.net/Complexity_Zoo:A#ac0}).
% \item TODO: Discuss the characterization of \complexity{DLOGTIME}-uniform $\complexityi{AC}{0}$ presented in~\cite{hella2023regularrepresentationsuniformtc0}.
% \item TODO: Review the definitions of $\complexityi{AC}{0}$, $\complexityi{TC}{0}$, FATC0, and FTC0 provided in~\cite{612309}.
% \end{enumerate}



% TODO: Check whether $\complexityi{TC}{i}$ is contained in $\complexityi{NC}{i + 1}$.
% \subsection{\complexityi{TC}{0}}
% \begin{enumerate}
% \item TODO: Confirm that multiplication is in $\complexityi{TC}{0}$~\cite{BussLectureNotes}, and locate the supporting details in~\cite{doi:10.1137/0213028}.
% \end{enumerate}

\chapter{Models of computation, programming paradigms and complexity measures}\label{chap:foundations}
One of the first decisions a programming language designer has to make is choosing
the programming paradigm convenient for writing the programs of interest.
In the practice of programming, imperative languages
have no competition when a user needs to reason about the computational complexity of
programs.
The structure of imperative programs closely mirrors how the computation
is executed on modern CPUs and GPUs.
In turn, it is inherently unintuitive to reason about the
complexity of programs written e.g.\ in Haskell or Prolog.

Yet if we want to understand what classes of functions can be characterized syntactically,
we have to temporarily step away from the imperative mindset. As argued in~\cite{10.1007/978-3-642-27660-6_3},
the very notion of an ``algorithm'' is still evolving, so we shouldn't limit our considerations
to a single paradigm. This chapter tests whether
alternative models of computation could be better suited
to form the basis of languages that capture popular complexity classes.

We can foreshadow that the answer is (perhaps surprisingly) positive: even though the complexity classes are defined on Turing machines,
the characterizations studied in literature are rarely imperative. 

\section{Finite automata and transducers}
Finite-state transducers compute string-to-string functions and have simple descriptions.
In~\cite{bojańczyk2018polyregularfunctions},
four characterizations are given for the class of \emph{polyregular} functions --- a~class of string-to-string
functions computed by a particular kind of transducer. The definitions described there readily
constitute the basis of a programming language. Another programming
language for transducers is studied in~\cite{DBLP:conf/fsmnlp/Schmid05}. A particular
class of string-to-string functions defined using logic, \emph{MSO transductions},
is characterized to be precisely the class of functions computable by two-way deterministic finite transducers (2DFT)
in~\cite{engelfriet1999msodefinablestringtransductions}. All of these results
provide  basis for new programming languages and are therefore more than relevant to our work.

Because transducer classes remain poorly understood, it is usually unclear whether an arbitrary problem 
belongs to the class recognized by a given flavor of transducer.
Writing programs in such a programming language would therefore often be equivalent
to doing new research on transducers, so for now we focus
on more well-established classes of functions. A~good overview of existing research on
transducers can be found in~\cite{muscholl_et_al:LIPIcs.STACS.2019.2}.

The existing research on finite-state automata has not been directly useful for our work ---
perhaps because the expressiveness of this model is inherently limited to Boolean-valued functions.

\section{Turing machines}
This model of computation underpins the imperative programming style.
There is a variety of flavors of Turing machines, and the details of a specific definition will most
of the time not affect our considerations in this work. When not explicitly stating otherwise,
when describing a computational process we will implicitly assume the realization of it on some kind
of a Turing machine.

\begin{definition}[Time complexity]\label{def:turing-dtime}
Let \(T : \mathbb{N} \to \mathbb{N}\) be some function.  A language \(L\) belongs to \(\complexity{DTIME}(T(n))\) iff there exists a deterministic Turing machine that satisfies:
\begin{enumerate}[label= (\roman*), ref= (\roman*)]
    \item\label{itm:dtime} for some constant \(c > 0\) and every input \(w \in \{0,1\}^{n}\) it terminates within \(c \cdot T(n)\) steps;
    \item for every \(w \in L\) the machine outputs \(1\);
    \item for every \(w \notin L\) the machine outputs \(0\).
\end{enumerate}
We will say that a Turing machine belongs to \(\complexity{DTIME}(T(n))\) iff it satisfies~\ref{itm:dtime}.
\end{definition}

\begin{definition}[Space complexity]\label{def:turing-dspace}
Let \(T : \mathbb{N} \to \mathbb{N}\) be some function.  A language \(L\) belongs to \(\complexity{DSPACE}(T(n))\) iff there exists a deterministic Turing machine that satisfies:
\begin{enumerate}[label= (\roman*), ref= (\roman*)]
    \item\label{itm:dspace} for some constant \(c > 0\) and every input \(w \in \{0,1\}^{n}\) it halts while using at most \(c \cdot T(n)\) work-tape cells;
    \item for every \(w \in L\) the machine outputs \(1\);
    \item for every \(w \notin L\) the machine outputs \(0\).
\end{enumerate}
We will say that a Turing machine belongs to \(\complexity{DSPACE}(T(n))\) iff it satisfies~\ref{itm:dspace}.
\end{definition}

The most popular complexity classes are directly based on the above notions.
Yet surprisingly few elegant, high-level
imperative languages are known to characterize well-studied complexity classes beyond the trivial ones.
This scarcity is precisely why much of the work surveyed in later chapters relies on non-imperative paradigms.

\subsection{Random-access Turing machines}
Random-access Turing machines are Turing machines with a special ``pointer'' tape,
of length logarithmic to the size of input, and a special state such that when the binary
number on the pointer tape is \(n\), \(n\)-th digit of the input is written to the work tape.

Reasoning about computation in complexity classes such as \(\complexity{L}\) or \(\complexity{P}\) is the same
for traditional Turing machines and random-access Turing machines. The choice of model starts to matter for
notions of complexity such as \(\complexity{DTIME}(n)\) or \(\complexity{DTIME}(n^2)\) (\emph{fine-grained} complexity classes).
Due to insufficient existing research on implicit characterizations of
fine-grained complexity classes, we will not consider them besides a brief discussion
in~\autoref{subsec:fine-grained-reductions}. Computation on transducers is a promising way to obtain robust characterizations of these
classes --- e.g.~\emph{polyregular} functions are computable in linear time by a kind of a transducer~\cite{bojańczyk2018polyregularfunctions}.
This is not a \emph{characterization}, however, because it is unlikely that every \(\complexity{DTIME}(n)\) function is definable in this language.
The notion of \(\complexity{DLOGTIME} = \complexity{DTIME}(\log n)\)-uniformity explored in~\autoref{subsec:dlogtime-uniformity} is also
only defined for random-access Turing machines.

\section{Circuits}
Circuits as a model of computation are the theoretical foundation of parallel programming.
The theoretical implications turn out to not be widely applicable in practice, however.
In this work, we will mostly use circuits to reason about very weak complexity.

As we will see in~\autoref{sec:uniformity}, we will almost always want to 
define a circuit family by a function (of a low complexity) \(n \rightarrow C_n\), computing the description
of the circuit for a given input size \(n\). If we obtain a language for
such functions, we will be able to use it to \todo{compute with complexity bound,
i.e.\ that if the resulting circuit is correct, it will be logspace-uniform ac0}compute circuit families. We will study
a logical characterization of a variant of \(\complexity{AC}^0\) in~\autoref{sec:vac0}.

A very rich \todo{there is nothing new for me
in this paper; i have also a very good literature overview, but scattered around this PDF}
overview of different characterizations of circuit complexity classes is
in~\cite{antonelli2025characterizingsmallcircuitclasses}.

\section{Discrete differential equations}
An original point of view on computation is to describe functions
as solutions to discrete differential equations. For example, in~\cite{bournez_et_al:LIPIcs.MFCS.2019.23},
\(\complexity{FP}\)~(\autoref{def:fp}) and \(\complexity{FNP}\)~(\autoref{def:fnp}) are characterized.
Characterizations of various circuit classes from
\(\complexity{FAC}^0\) to \(\complexity{FAC}^1\)~(\autoref{def:faci}) 
are described in~\cite{antonelli2025characterizingsmallcircuitclasses}.

\section{Logic programming and descriptive complexity}
Logic provides very deep complexity-theoretic connections,
primarily through descriptive complexity theory, which we explore in more detail in~\autoref{chap:descriptive-complexity}.

\section{Untyped recursion}\label{sec:untyped-lambda}
Another classical paradigm is that of general recursive functions, or equivalently the untyped
lambda calculus. Because these systems are Turing complete, the interesting question
is how to constrain recursion so that the resulting language captures a specific class.
We treat these ideas in~\autoref{chap:recursion-theory}.

\section{Typed lambda calculus}
Typed lambda calculus underpins functional programming. In this section we focus on typed variants,
unlike in~\autoref{sec:untyped-lambda}.

Lambda calculus does
not line up cleanly with traditional Turing machine complexity measures.  For instance,
for some representation of strings \(\{0, 1\}^\ast\),~\cite[Theorem~3.4]{10.5555/788018.788832} identifies
the functions \(\{0, 1\}^\ast \rightarrow \{0, 1\}\) definable in simply typed
lambda calculus (STLC), with regular languages. But with a different encoding of inputs,~\cite{HILLEBRAND1996117} relates STLC to the whole \(\complexity{ELEMENTARY}\) class. 
Moreover,~\cite{zakrzewski2007definablefunctionssimplytyped} states that
with a different ``standard'' encoding, STLC instead characterizes extended polynomials, and further shows
that if we slightly modify the encoding, the class is yet different.
For more discussion, see also~\cite{27863}.
Consequently, it is not obvious how to reason about complexity theory in the language of lambda calculus.

Nevertheless, typed lambda calculi have been utilized
very successfully to syntactically characterize complexity classes.
Recall that one of the reasons linear logic is studied is the potential
to reason about resource creation and utilization.
Concepts from linear logic have been implemented in the theory of type systems
to transfer the resource interpretation.
This will be discussed further in~\autoref{chap:linear-types}.

\section{Set theory as inspiration for model of computation}
An interesting connection appears when we think of traditional notions of ``complexity'' of sets in set theory
from the point of view of computational complexity. 
If we treat taking complements, intersections, countable unions as operations in a programming language,
perhaps we could design a programming language for constructing sets. By itself such a language would probably
not be the most interesting one. However, thinking about mathematical reasoning
in terms of a computational process is a very powerful technique. It has been deeply explored
under the name of Curry-Howard or proofs-as-programs correspondence.\footnote{A good introduction to the
immensely deep topic of proofs-as-programs is~\cite{10.5555/1197021}.} For the computational content
of set theory in particular, an interesting brief discussion is presented in~\cite{Tao2010ComputationalPerspective}.

For our purposes, an interesting connection appears when we look at descriptive set theory.
The most basic classes of sets distinguished there are open and closed sets. Slightly higher,
an \(F_\sigma\) set is a countable union of closed sets; a \(G_\delta\) set is a countable intersection
of open sets. A~few levels higher up the \emph{Boldface hierarchy}, which in a way quantify 
the complexity of sets, Borel sets are considered:

\begin{definition}[Borel sets]
Let \(X\) be a topological space. The class of Borel sets of \(X\), \(\mathcal{B}(X)\) is the smallest class of sets
containing every open set of \(X\) and closed under~\ref{itm:borel-comp} and~\ref{itm:borel-union}:
\begin{enumerate}[label= (\roman*), ref= (\roman*)]
    \item\label{itm:borel-comp} if \(A\) is Borel, then its complement \(X \setminus A\) is Borel;
    \item\label{itm:borel-union} if \(A_n\) is Borel for each \(n \in \mathbb{N}\), then the countable union \(\bigcup_{n \in \mathbb{N}} A_n\) is Borel.
\end{enumerate}
\end{definition}

As this is a standard least fixed point definition, it is strikingly similar to
definitions of classes of recursive functions that we will consider later, e.g.~\autoref{def:primitive-recursive}.
We can also take a computational point of view on theorems about the determinacy of Gale-Stewart games (which we shall not introduce here).
It is widely known that Gale-Stewart games are determined when the underlying set is open or closed.
Allowing the underlying set to be more and more complicated, we quickly 
reach the limits of provability in Zermelo-Fraenkel set theory: determinacy for Borel sets is a
difficult theorem, and determinacy for analytic and projective sets is
independent of ZF, yet provable assuming as axiom the existence of an appropriately large cardinal.

A good question to ask is if by carefully curating the axioms, we could
obtain a mathematical theory such that theorems corresponding
to computation in our desired complexity class are provable, and
the theorems that wouldn't be ``implementable'' are not.

We will circle back
to this intuition while considering the \(\texttt{PIGEON}\) computational problem in~\ref{subsubsec:tfnp} and
the unprovability of the related pigeonhole principle in the weak theories studied in~\ref{subsec:vac0-php}.
Especially the last fact about unprovability is interesting for us in this section,
as it resembles e.g.\ independence of continuum hypothesis from ZFC, but in a strictly computational setting
of not being able to perform enough computation in a low complexity class.
% This is also a hint that going this route, a collection of powerful tools used in logic to prove
% independence, will become available to us as tools for proving a problem not being solvable
% in a given complexity class.

While this line of thought is very far from ``practical programming languages'', these considerations
inspired\footnote{This line of study grew out of presenting the topic
at the JAiO master's seminar at the University of Warsaw ---
slides are available online at~\cite{Balawender2025BorelDeterminacySeminar}.}
the approach that we study in~\autoref{chap:bounded-arithmetic}.


% \section{Models of hypercomputation}
% Notes (for future work):
% \begin{itemize}
% \item Sequential Time. An algorithm determines a sequence of computational
% states for each valid input. Specifically, the time is discrete. what if time is not a successor structure? i.e. we can travel back in Time
% \item Malament-Hogarth spacetime.
% \end{itemize}

\chapter{Formalized semantics}\label{chap:formalized-semantics}
\todo[inline]{In this short chapter, I describe why a language: C++ + mathematical proofs of
being in complexity, is infeasible. This is because reasonign about Turing machines
in Coq is a nightmare and I cite resources for it. On the other hand, C++ + just a polynomial
cutting off the computation after step n is a bad idea, as we can't reason abot the 
semantics at all}
\chapter{Descriptive Complexity}\label{chap:descriptive-complexity}
In the rest of this thesis, we usually measure complexity of algorithms:
how much time and memory will a given algorithm need to run to solve a problem of size \(n\)?
In this chapter we will instead focus on the complexity of defining the problem itself.

The central problem of Descriptive Complexity is to characterize a
complexity class by the \emph{power of logic required to define its problems}.
This is in contrast to \emph{Implicit Complexity Theory}, discussed later in~\autoref{chap:icc},
which seeks the weakest system sufficient to \emph{implement algorithms}
of the class; and in contrast to \emph{Bounded Arithmetic}, which studies
the weakest \emph{theory} required to define a function \emph{and prove its correctness}.
This perspective has led to elegant logical characterizations of many traditional
complexity classes, as we will discuss in~\autoref{sec:descriptive-results}.

Before going into details, we will discuss an example to recall what it means for
a structure to model a formula~(\autoref{def:L-structure}):

\begin{remark}
In Descriptive Complexity, there is no notion of a proof.
The problem of deciding if a given sentence is provable is independent of this chapter's considerations,
and will be studied by us only in~\autoref{chap:bounded-arithmetic}.
\end{remark}

\begin{example}\label{exm:logic-simple-model}
Consider a vocabulary \(\mathcal{L}\) consisting of unary relations \(\mathrm{Zero}(x), \mathrm{One}(x)\) and binary relations
\(\mathrm{=}, \mathrm{\leqslant}\). If we think of positions in a binary string as elements of the universe,
we can define e.g. for a string \(01011\) an \(\mathcal{L}\)-structure \(\mathcal{M}\) with:
\begin{enumerate}
\item universe \(\{1, 2, 3, 4, 5\}\);
\item \(\mathrm{Zero} := \{1, 3\}; \mathrm{One} := \{2, 4, 5\}; \mathrm{=} := \{\langle1, 1\rangle, \langle2, 2 \rangle, \dots, \langle5, 5 \rangle\}; \mathrm{\leqslant} := \{\dots\}\).
\end{enumerate}

Now, we can reason about the original binary string using logic.
The formula  \(\exists x \ldotp \mathrm{One}(x)\) is true in \(\mathcal{M}\). However, the formula
\(\forall x \ldotp \mathrm{Zero}(x)\) is not.
It is insightful to analyze in general what kind of logical formulas are true in \(\mathcal{M}\).
\end{example}


We can use the language of logic to describe computational queries about the underlying structure.
This is the intuition behind the languages designed for querying databases. To formally connect
logic and computation, consider the following

\begin{definition}[Language of binary strings]
  Consider a vocabulary \(\tau_{\text{string}}\) to contain only the unary relations \(\mathrm{Zero}(x), \mathrm{One}(x)\) and
  the binary relations \(=, \leqslant\). Intuitively, \(x \leqslant y\) means that memory cell $x$ comes before
  memory cell $y$. As we can represent most of the real-world structures of interest as some binary
  string on a computer, this simple vocabulary already allows us to ask interesting questions.
\end{definition}

\begin{definition}[Language generated by a formula]\label{def:language-decided-by-formula}
  For a binary string \(w \in \{0, 1\}^n\) define its corresponding structure \(\mathcal{M}_w\)
  over vocabulary \(\tau_{\text{string}}\)
  to have the universe \(M := \{1, \dots, n\}\) and the standard semantics of relations from \(\tau_{\text{string}}\).
  
  We define the language \(L_\varphi\) to be \emph{generated} by a first-order sentence
  $\varphi$ over $\tau_{\text{string}}$
  iff
  \[\mathcal{M}_w \models \varphi \;\Longleftrightarrow\; w \in L_\varphi.\]
\end{definition}

Now let's shift our focus to a more standard definition.

\begin{definition}[First order logic on ordered structures]\label{def:fo-order}
  Define the vocabulary $\tau_\leqslant$ to contain:
  \begin{enumerate}
  \item the constants $0, 1, \mathrm{max}$;
  \item the binary relations $=, \leqslant$.
  \end{enumerate}

  \complexity{FO_\leqslant} is the class of sentences of first-order logic in the language \(\tau_\leqslant\).

  From now on, whenever talking about the semantics of any such sentence, we will require
  the elements of the (finite) universe to be interpreted as actual natural numbers $0, 1, \dots$;
  the equality and order to be interpreted as the actual equality and order on the elements of the model;
  $0, 1, \mathrm{max}$ to be interpreted as the minimum, second, and maximum elements under $\leqslant$.

  \begin{remark}[Bibliography]
    In literature, \(\complexity{FO_\leqslant}\) is called \complexity{FO (wo~BIT)}; see~\cite[Ordering~Proviso~1.14]{Immerman1999-IMMDC}.
  \end{remark}
\end{definition}


\begin{definition}[First order logic with arithmetical predicates]\label{def:fo-arith}
  \logicFO{} is the class of sentences of first-order logic over $\tau_\leqslant$ extended with
  the binary relation $\mathrm{BIT}$.
  Semantically, we require \(\mathrm{BIT}(x, y)\) to hold iff bit $y$ in the binary representation of $x$ is 1.

  \begin{remark}
    By default in the literature, the arithmetic predicates and order are included.
  Usually, arithmetic predicates $\mathrm{PLUS}(x, y, z)$, $\mathrm{TIMES}(x, y, z)$, $\mathrm{SUCC}(x, y)$, denoting
  that $x + y = z, x * y = z, x + 1 = y$ respectively, are also added. We use the fact that $\mathrm{PLUS}, \mathrm{TIMES}$ are
  first-order definable from $\mathrm{BIT}$~\cite[Theorem~1.17]{Immerman1999-IMMDC} and that $\mathrm{SUCC}$ is
  first-order definable from $\leqslant$~\cite[Section~1.2]{Immerman1999-IMMDC}.
  \end{remark}
  % \cite[Proposition~9.16]{Immerman1999-IMMDC}: BIT is definable in FO(wo BIT)(DTC), so also (TC) and (LFP).
  % SOLUTION: JUST DON'T CARE ABOUT THIS! JUST ADD BIT NORMALLY TO LOGIC WITH ZERO, ONE.
\end{definition}



% \section{First-order expressible decisional problems}
% The definition below is from~.

% \begin{definition}[{%
%   \cite[Definition~4.24]{Immerman1999-IMMDC}}%
%   ~The complexity class \complexity{FO[t (n)]}%
% ]
% Let $\mathcal{L}$ be a vocabulary and let $S$ be a class of $\mathcal{L}$-structures. 
% We say that $S$ is of complexity $\complexity{FO[t(n)]}$ iff there exist

% \todo{consistency: bar c vs c1, c2,\dots,ck.}
% \begin{enumerate}
%   \item quantifier-free $\mathcal{L}$-sentences $M_i$ ($0 \le i \le k$);
%   \item a quantifier block
%   \[
%      QB \;=\; (Q_1 x_1 \ldotp M_1)\,\dots\,(Q_k x_k \ldotp M_k)
%   \]
% \end{enumerate}

% such that for all $\mathcal{L}$-structures $\mathcal{A}$,
% \[
%   \mathcal{A} \in S
%   \quad\Longleftrightarrow\quad
%   \mathcal{A} \models [QB]^{\,t(|A|)} M_0.
% \]

% \begin{remark}
% In the original definition, free variables in $M_i$ are allowed which need to be treated. We ignore them here.
% \end{remark}

% In particular, for \(t(n) = \bigO(1)\), then \(\complexity{FO[ t(n)]} = \complexity{FO}\) are formulas
% with the same single block of quantifiers for all input sizes. This case will be
% the most important for~\autoref{thm:}.
% \end{definition}








\section{Results}\label{sec:descriptive-results}
The field of descriptive complexity has a long history and we should not introduce all the results
in this work, as our primary purpose is to examine if, in the first place, they are relevant to
our problem. We will only describe the logical characterizations of the classes that are the
most interesting for us: \compP{} and uniform \complexityi{AC}{0} which will underpin our~\autoref{chap:bounded-arithmetic}.
A~concise overview of the classical results is presented in~\cite[Section~15.1]{Immerman1999-IMMDC},
where characterizations of \(\complexity{SPACE}(n^k)\), \complexity{L},
\complexity{NL}, \complexity{P}, \complexity{NP} and \complexity{PSPACE} are described.
Please also see the insightful~\cite{doi:10.1137/0216051} that introduces
model-theoretical characterizations of numerous decisional complexity classes. 


One intensely studied logic is \logicFOLFP{}, defined inductively.
\begin{samepage}
\begin{definition}[{\cite[Definition~4.5]{Immerman1999-IMMDC}~Least fixed-point logic}]
\logicFOLFP{} is a class of logical sentences such that:
\begin{enumerate}
  \item if \(\varphi \in \logicFO{}\) then \(\varphi \in \logicFOLFP{}\);
  \item if \(\varphi_R(x_1, \dots, x_k) \in \logicFOLFP{}\), where \(R\) is a \(k\)-ary relation
    and \(R\) only occurs positively in $\varphi_R(x_1, \dots, x_k)$ (i.e.\ every occurrence of $R$ is preceded
    by an even number of negations), then \(\complexity{LFP}_{R(x_1, \dots, x_k)} \varphi\) may be used
    as a new $k$-ary relation symbol denoting the least fixed-point of $\varphi$.
\end{enumerate}
\end{definition}
\end{samepage}

\begin{theorem}[$\logicFOLFP{} = \compP$]\label{thm:fo-lfp-eq-ptime}
  The class of languages generated by
  sentences from \logicFOLFP{} is precisely \(\complexity{P}\).

  \begin{remark}
    We can think of the $\complexity{LFP}$ operator as allowing us to write recursive formulas.
  \end{remark}

  \begin{remark}
    This result has been proved independently by Immerman~\cite{IMMERMAN198686}
    and Vardi~\cite{10.1145/800070.802186}. For a more uniform treatment, see~\cite[Theorem~4.10]{Immerman1999-IMMDC}.
  \end{remark}
\end{theorem}

\begin{theorem}[{%
  \cite[Corollary~5.32]{Immerman1999-IMMDC}~\logicFO{}
  = \complexity{FO}-uniform \complexityi{AC}{0}%
}]\label{thm:fo-eq-ac0}
Languages \(L \subseteq \{0, 1\}^\ast\) decidable by uniform \complexityi{AC}{0}
  circuits are precisely the languages generated by \logicFO{} formulas,
  in the sense of~\autoref{def:language-decided-by-formula}.
  % \todo[inline]{Tutaj sie poplątałem,
  % bo zdefiniowałem moje FO jako komórki pamięci + Zero(x), One(x) - a u Immermana elementy dziedziny
  % to liczby naturalne, i jego predykaty to są normalne operacje arytmetyczne na liczbach. Nie wiem czy wyniki się przekładają...}

\begin{remark}[Bibliography]
  The theorem is originally stated in terms of \complexity{FO_{\mathrm{BIT}}[t (n)]} defined in~\cite[Definition~4.24]{Immerman1999-IMMDC}.
  We limit to \(t(n) = \bigO(1)\), i.e.\ in our case $\complexity{FO_{\mathrm{BIT}}[t (n)]} = \logicFO{}$.
  The uniformity condition is also different to the one we fixed in~\autoref{def:logspace-uniformity}:
  their circuits are so-called \complexity{FO}-uniform, which is a much stronger condition.
  For details about the different notions of uniformity, refer to~\autoref{chap:uniformity}.
  % The~\cite[Corollary~5.32]{Immerman1999-IMMDC} is implied from~\cite[Theorem~5.22]{Immerman1999-IMMDC} and earlier
  % from~\cite[Theorem~5.2]{Immerman1999-IMMDC}. It is proved that
  % \complexity{FO} is in \complexityi{AC}{0} in~\cite[Lemma~5.4]{Immerman1999-IMMDC}.
  % The proof that \complexityi{AC}{0} is in \complexity{FO} is in two steps,~\cite[Lemma~5.3]{Immerman1999-IMMDC}
  % and then~\cite[Lemma~4.25]{Immerman1999-IMMDC}.
\end{remark}
\end{theorem}


% As an example, Grädel's Theorem in descriptive complexity states that
% \(\complexity{NL}\) is the class of finite models of the second-order Krom formulas
% \cite{GRADEL199235}.
% Here, a \emph{Krom formula} is a formula in conjunctive normal form (CNF)
% where each clause contains at most
% two literals. The satisfiability problem for Krom formulas, Krom-SAT, is
% complete for \(\complexity{coNL}\) (or equivalently \(\complexity{NL}\), by the Immerman-Szelepcsényi
% Theorem).

% % second-order horn formula: https://en.wikipedia.org/wiki/Second-order_propositional_logic
% Similarly, Grädel~\cite{GRADEL199235} shows that second-order Horn formulas
% express precisely the properties decidable in polynomial time.
% A \emph{Horn clause} is a clause with at most one positive literal and any number
% of negative literals, and a Horn formula is a conjunction of such clauses.
% Similar results exist for most commonly studied complexity classes:
% \(\complexity{NP}\) (Fagin's theorem), \(\complexity{coNP}\), \(\complexity{PH}\), \(\complexity{PSPACE}\), and \(\complexity{EXPTIME}\).

% For an accessible introduction, see the classical and self-contained reference
% by Immerman~\cite{Immerman1999-IMMDC}.

% Fagin's theorem says that existential second-order logic characterizes
% \(\complexity{NP}\).
% It is worth noting that this theorem does \emph{not} assume order on the input ---
% intuitively, \(\complexity{NP}\) is strong enough to ``guess'' the order relation.




\section{The Quest for a Logic Capturing \complexity{PTIME}}\label{subsec:unordered-structures}
Many of the problems in \compP{} are graph problems, i.e.\ they ask to decide a property $\varphi(G)$
for some abstract graph $G$. It is natural to represent a graph as a logical structure:
the nodes of the graph correspond to the elements of the universe and we add to the vocabulary a special relation
$E(x, y)$ denoting there is an edge from $x$ to $y$. This is reflected by the following

\begin{definition}[Logic on graphs]\label{def:fo-unordered}
  Define the vocabulary \(\tau_{\text{graph}}\) to contain only the binary relations \(x = y, E(x, y)\). 
  The class \complexity{FO_{\text{graph}}} contains precisely the sentences of first-order logic over
  the vocabulary \(\tau_{\text{graph}}\).

  From now on, when talking about the semantics of
  sentences from $\complexity{FO}_{\text{graph}}$, we will assume the interpretation of elements of the universe
  as nodes of the graph $G$, and the interpretations of $x = y$, $E(x, y)$ to agree with
  the underlying node equality and edge relation of $G$. For example, we will assume that $E(x, y)$ holds 
  if and only if
  there is an edge between nodes of $G$ corresponding to the universe elements $x, y$.

  \begin{remark}[Bibliography]
    Typically, the vocabulary of logic on graphs also contains unary symbols $a(x), b(x), \dots$ denoting that
    the color of node $x$ is $a$ or $b$ etc.~\cites[Definition~12.2]{Immerman1999-IMMDC}[Theorem~1.36]{Immerman1999-IMMDC}.
    As we will not consider
    coloring of nodes in this thesis,
    we don't need to introduce that.
    In the literature, \(\complexity{FO}_{\text{graph}}\) is typically called $\complexity{FO (wo~\leqslant)}$,
    modulo the addition of the edge relation.
    For more details, see~\cite[Ordering~Proviso~1.14]{Immerman1999-IMMDC}
    and also discussion under~\cite[Question~12.1]{Immerman1999-IMMDC}.
  \end{remark}
\end{definition}

Defining graph problems
rarely requires us to impose any numbering on the nodes of the graph. However,
to talk about deciding a property of a graph on a Turing machine, we need to
encode the graph as a binary string. This imposes some artificial ordering on the vertices
of the graph. Perhaps a more suitable definition of logic of graphs would thus be

\begin{definition}[Logic on graphs with order]\label{def:fo-graph-ordered}
  We denote by \complexity{FO_{\text{graph}\leqslant}} the class of first-order sentences over
  $\tau_{\text{graph}}$ extended with the constants $0, 1, 
  \mathrm{max}$, the binary relations $\mathrm{BIT}, \leqslant$ and the operator $\complexity{LFP}$,
  interpreted as earlier.
\end{definition}

Now, let's consider adding the least fixed-point operator to logic on graphs.
We will denote by \logicFOLFPgraph{} the logic obtained by extending \complexity{FO_{\text{graph}}}
with the $\complexity{LFP}$ operator.
For the \complexity{FO_{\text{graph}\leqslant}} with $\complexity{LFP}$, we need to notice two things.
First, we refer to~\cite[Proposition~9.16]{Immerman1999-IMMDC} stating that
the relation $\mathrm{BIT}$ is first-order definable with ordering and $\complexity{LFP}$.
Second, we notice that the addition of the binary edge relation doesn't change the expressive power here.
Indeed, with ordering we can encode the edge relation as part of input.
Thus, we will treat this logic as equally strong to $\logicFO{}$ introduced earlier.
In particular, we will assume without
transferring the proof of~\autoref{thm:fo-lfp-eq-ptime} that $\complexity{FO_{\text{graph}\leqslant}} = \compP$.
Thus, we obtain two similarly defined theories: \logicFOLFPgraph{}, not having access to the order relation,
and \logicFOLFPgraphord{}, only operating on ordered structures.

For graph problems we typically want the Turing machine $M$ to return the same
answer regardless of the order we pass the vertices in. However, when looking at the code
of a particular Turing machine, it's usually difficult to tell if it returns the same answer
for all permutations of the input graph. The implicit assumption of being always given \emph{some} ordering
of input graph nodes, is inherent in computation on Turing machines. We may now wonder:
does this assumption limit us in the generality of programs we write? If someone forced us
to not rely on this ordering in our programs, and write programs in a \emph{permutation-invariant}
way, would anything change in our expressive power? This turns out to be a very deep and difficult question.

We will say that a graph problem $P$
is in the complexity class \complexity{inv\text{-}P} iff there is a polynomial-time Turing machine $M$
such that for every graph $G$ and every permutation $\pi$ of vertices of $G$,
\[M(\mathrm{enc}(\pi(G))) \leftrightarrow P(G).\]
That is: the problem $P(G)$ has such a decider $M$ that it returns the correct answer regardless
of how we label the input nodes. The class \complexity{inv\text{-}P} is also called \compP \emph{on unordered structures}.

As motivated earlier, \logicFOLFPgraphord{} is strong
enough to express any property from \compP. However, it turns out that when we take the order out,
\(\logicFOLFPgraph{}\) \textbf{cannot}
express every problem from~\(\complexity{inv\text{-}P}\). Equivalently: that \logicFOLFPgraph{}
cannot capture \compP~\cite{Cai1992}. This means that there are graph problems that
have a robust, order-invariant decider $M$, yet can't be expressed by a logical formula having access
to the $\complexity{LFP}$ operator, but not having access to the arithmetical predicates and ordering.
The existence of a logic that characterizes \complexity{inv\text{-}P} (or \(\complexity{P}\) on
unordered structures) remains a major open problem in computer science as of 2025.
Good overviews of this problem are~\cite{dawar2012syntactic} and~\cite[Chapter 12, The Role of Ordering]{Immerman1999-IMMDC}.

An important result that treats unordered structures is Fagin's theorem~\cite{Fagin1974} that states
that the class of languages generated by sentences of existential second-order logic is precisely \complexity{NP}.
Intuitively, ordering of the domain is not necessary to assume for Fagin's theorem because
in \complexity{NP}, we can \emph{guess} it.

% In a paper from 2019, Jean-Yves Moyen and Jakob Grue Simonsen discuss the problem of providing syntax for PTIME ,
% concluding that in the light of their generalization of Rice's theorem \cite{10.1007/978-3-030-22996-2_19}.
% (problem w reprezentacji:
% mozemy miec jezyk programowania dla P, ale odkladamy nierozstrzygalnos do problemu sprawdzenia czy mozna
% przetlumaczyc dana maszyne turinga na ten jezyk)








\section{Defining functional problems in logic}
\subsection{First-order queries (\complexity{FO}-reductions)}
Despite logic being only able to naturally define decisive problems,
some approaches have been used to reason logically about functions.
Most importantly, to study completeness of problems in low complexity classes
such as \compL, \complexity{FO}-reductions are used. They have a rather
complicated definition, which we don't display here and for the details
refer to~\cite[Definition~1.26]{IMMERMAN198686}. In the same work,
even weaker notions of reducibility are studied: first-order projections (fops) and
quantifier-free projections (qfps) are
defined in~\cite[Definition~11.7]{Immerman1999-IMMDC}.
An interesting property of the complexity classes we are studying in this work is
that their complete problems are already complete under surprisingly weak reductions.
In~\cite[Proposition~11.10]{Immerman1999-IMMDC} it is proved that
\problem{SAT} is \complexity{NP}-complete via fops and even via qfps.
% There is a problem \(\texttt{S}\) that is \complexity{NP}-complete via first-order
% reductions, but not via fops\cite[Proposition~11.14]{Immerman1999-IMMDC}.
% The following interesting property of first-order projections, which says that
% there is only one complete problem via first-order projections for each ``nice''
% complexity class.
% Let \(\texttt{C}\) be one of the complexity classes \complexityi{NC}{1}, \complexity{L}, \complexity{NL},
% \complexity{P}, \complexity{NP}, \complexity{PSPACE}.
% Let \(\texttt{A}\) and \(\texttt{B}\) be problems complete for \complexity{C} via fops. Then there is a first-order
% definable isomorvarphism between \(\texttt{A}\) and \(\texttt{B}\).

\subsection{Bit-graph definitions}\label{subsec:bit-graph-definitions}
An easy way to define a general function from Boolean functions is to
use the Boolean functions to decide if ``$i$-th bit of the output $f(x)$ is $0$ or $1$''.
For the definition to be precise, two more technicalities are needed (the output's length itself
must be polynomial and easy to compute), which we will
discuss while defining \compL-reductions in~\autoref{def:logspace-reductions}.
For now, we will just say that this style of definitions is very important for
some of the results we will study in~\autoref{chap:bounded-arithmetic}, as 
the ``semantic'' definition of definability of functions (\cites[Definition~V.4.12]{Cook_Nguyen_2010}[Definition~5.37]{CookNguyenDraft})
is precisely that.


\subsection{\complexity{FO} and \complexity{MSO} transductions}
Some works use the notion of \complexity{FO}-transductions, e.g.\ in~\cite[Section~2]{nesetril2021structuralpropertiesfirstordertransduction},
where they are also defined. They are,
however, completely different from anything we study in this work and are defined as compositions of
\emph{copying}, \emph{coloring} and \emph{simple interpretations}, which we will not discuss.
A similar, and much more important notion is of \complexity{MSO} transductions defined
e.g.\ in~\cite[Section~2]{COURCELLE199453}
% https://www.mimuw.edu.pl/~bojan/posts/who-to-cite-mso-transductions











\section{Descriptive Complexity and programming languages}\label{sec:descriptive-for-programming}
From the point of view of programming languages, the meaning of
e.g.\ the result discussed in~\autoref{subsec:unordered-structures} is as follows: given a logical formula
\(\varphi \in \logicFOLFP{}\), we \emph{know that there exists some} Turing machine
of complexity \complexity{P} that will check for us if \(w \models \varphi\) for any input \(w\).
There is a huge leap of faith, however --- just that the machine \emph{exists} and runs fast,
we can't conclude that we will ever be able to actually use it. We have to inspect
the proof of such a theorem and tell if we can compute the description itself of such a machine,
and how fast we can do it. An illustrative example is infeasibility of using the below theorem
to design a programming language for finite automata:

\begin{theorem}[{\cite[Theorem~1.36]{Immerman1999-IMMDC}}~Büchi-Elgot-Trakhtenbrot theorem]
  For the alphabet \(\Sigma := \{0, 1\}\),
  the set of boolean queries expressible in second-order monadic logic (which we skip the definition of)
  over the vocabulary \(\tau_\leqslant\) consists exactly of the regular languages. In other words, \(\complexity{MSO} = \complexity{REG}\).
  \begin{remark}[Bibliography]
    The theorem was originally proved by Büchi in~\cite{Bchi1960WeakSA}, Elgot in~\cite{8cb177c1-4df9-3823-9c81-47f75da88fff}
    and Trakhtenbrot in~\cite{Trakhtenbrot1962}.
    The statement from~\cite[Theorem~1.36]{Immerman1999-IMMDC} is slightly more accessible.
    It uses slightly different wording to ours. For the details of the definitions, refer to~\cite[Proviso~1.14]{Immerman1999-IMMDC} and~\cite[Proviso~1.15]{Immerman1999-IMMDC}.
  \end{remark}
\end{theorem}


However, not always is this problem that difficult. Results from descriptive complexity have been crucial
for the field of database theory, which we don't discuss here. The approach of using logic for programming
has also coined the paradigm of logic programming.

\subsection{Logic programming}
The most famous example of a logic programming language is Prolog. Prolog doesn't have a reputation
of a language well-suited for computational complexity analysis. However, it relates very well
with the notions of deciding computational problems we discussed in this chapter. Please see~\autoref{lst:prolog-example}
for a demonstration how we can transfer our considerations from~\autoref{exm:logic-simple-model} to practical computation.

\begin{rawlisting}
\begin{Verbatim}[commandchars=\\\{\}]
\PY{c+c1}{\PYZpc{} Run at: https://swish.swi\PYZhy{}prolog.org/}
\PY{n+nf}{succ}\PY{p}{(}\PY{l+s+sAtom}{bit1}\PY{p}{,} \PY{l+s+sAtom}{bit2}\PY{p}{)}\PY{p}{.}
\PY{n+nf}{succ}\PY{p}{(}\PY{l+s+sAtom}{bit2}\PY{p}{,} \PY{l+s+sAtom}{bit3}\PY{p}{)}\PY{p}{.}
\PY{n+nf}{succ}\PY{p}{(}\PY{l+s+sAtom}{bit3}\PY{p}{,} \PY{l+s+sAtom}{bit4}\PY{p}{)}\PY{p}{.}
\PY{n+nf}{succ}\PY{p}{(}\PY{l+s+sAtom}{bit4}\PY{p}{,} \PY{l+s+sAtom}{bit5}\PY{p}{)}\PY{p}{.}

\PY{n+nf}{zero}\PY{p}{(}\PY{l+s+sAtom}{bit1}\PY{p}{)}\PY{p}{.}
\PY{n+nf}{zero}\PY{p}{(}\PY{l+s+sAtom}{bit3}\PY{p}{)}\PY{p}{.}
\PY{n+nf}{one}\PY{p}{(}\PY{l+s+sAtom}{bit2}\PY{p}{)}\PY{p}{.}
\PY{n+nf}{one}\PY{p}{(}\PY{l+s+sAtom}{bit4}\PY{p}{)}\PY{p}{.}
\PY{n+nf}{one}\PY{p}{(}\PY{l+s+sAtom}{bit5}\PY{p}{)}\PY{p}{.}

\PY{n+nf}{exists\PYZus{}1}  \PY{o}{:\PYZhy{}} \PY{n+nf}{one}\PY{p}{(}\PY{n+nv}{X}\PY{p}{)}\PY{p}{.} 							  \PY{c+c1}{\PYZpc{} true}
\PY{n+nf}{exists\PYZus{}00} \PY{o}{:\PYZhy{}} \PY{n+nf}{succ}\PY{p}{(}\PY{n+nv}{X}\PY{p}{,} \PY{n+nv}{Y}\PY{p}{)}\PY{p}{,} \PY{n+nf}{zero}\PY{p}{(}\PY{n+nv}{X}\PY{p}{)}\PY{p}{,} \PY{n+nf}{zero}\PY{p}{(}\PY{n+nv}{Y}\PY{p}{)}\PY{p}{.}        \PY{c+c1}{\PYZpc{} false}
\PY{n+nf}{no\PYZus{}00}     \PY{o}{:\PYZhy{}} \PY{l+s+sAtom}{\PYZbs{}+} \PY{p}{(} \PY{n+nf}{succ}\PY{p}{(}\PY{n+nv}{X}\PY{p}{,} \PY{n+nv}{Y}\PY{p}{)}\PY{p}{,} \PY{n+nf}{zero}\PY{p}{(}\PY{n+nv}{X}\PY{p}{)}\PY{p}{,} \PY{n+nf}{zero}\PY{p}{(}\PY{n+nv}{Y}\PY{p}{)} \PY{p}{)}\PY{p}{.} \PY{c+c1}{\PYZpc{} true}
\PY{n+nf}{no\PYZus{}11}     \PY{o}{:\PYZhy{}} \PY{l+s+sAtom}{\PYZbs{}+} \PY{p}{(} \PY{n+nf}{succ}\PY{p}{(}\PY{n+nv}{X}\PY{p}{,} \PY{n+nv}{Y}\PY{p}{)}\PY{p}{,} \PY{n+nf}{one}\PY{p}{(}\PY{n+nv}{X}\PY{p}{)}\PY{p}{,} \PY{n+nf}{one}\PY{p}{(}\PY{n+nv}{Y}\PY{p}{)} \PY{p}{)}\PY{p}{.}   \PY{c+c1}{\PYZpc{} false}
\end{Verbatim}

\caption{Prolog example}\label{lst:prolog-example}
\end{rawlisting}


\subsection{Datalog}
Datalog is a programming language that uses the paradigm of logic programming, similarly to Prolog.
Unlike Prolog, however, it \emph{captures} the complexity class of \compP~\cite[Theorem~4.4]{10.1145/502807.502810} (on ordered, finite structures).
That is, it is a language such that any query definable in it will be checkable in \compP{} for a given model.
We have not investigated the subtleties of Datalog's implementation and whether the complexity doesn't blow up
somewhere.



\subsection{Logic as the type system}
Could the results from descriptive complexity be used to design
a specification mechanism (such as a type system\footnote{Language doesn't have to be typed to foster
partial proofs of correctness; see e.g.~\cite{10.1145/319301.319317},~\cite{paulson2000settheoryverificationii}.})
for a programming language?
While such a language would be very interesting and possibly have highly desired
properties, our ``typechecker'' in the naive case would essentially be
a proof checker for first order logic. This would be a very difficult system to
design properly. We were not successful in curating such a small set of basic
programming instructions that,
given a pre- and a post-condition in \complexity{FO} and an operation from that set,
checking for the validity of such a triple would be an easy problem.

Nonetheless, this line of inquiry led to a broader investigation of
type systems that enforce resource bounds --- particularly those inspired
by linear logic and implicit computational complexity.
The results of this exploration are presented in~\autoref{sec:linear-types}.

% todo: coin the term Oracle Oriented Programming
\chapter{Reductions}\label{chap:reductions}
Plenty of theorems of the form ``problem $P$ is \emph{complete} for class $C$ under
reductions in class $R$'' have been described in the literature. In this chapter
we analyze when are they useful to capture complexity classes syntactically,
and when they don't suffice.

\begin{definition}[Circuit Value Problem (\problem{CVP})]
  Given a representation of a Boolean circuit on input, compute its output.
  \todo[inline]{more formality}
\end{definition}

\begin{theorem}[%
  {\cite{10.1145/990518.990519}}%
]\label{thm:cvp-pcomplete}
  The circuit value problem is complete for \compP{} under \compL{} reductions.
  \begin{remark}
    We define \compL-reductions formally in~\autoref{def:logspace-reductions}.
  \end{remark}
\end{theorem}

Knowing~\autoref{thm:cvp-pcomplete}, we could design a language for \compP{} in such a way:
first, design a language for \compL{} which is (perhaps) simpler; then, as the compiler provider,
include a standard library function \(P\), which
the \compL-functions from the language could call like an oracle to get solution for \(P\).
In this chapter we try to understand
if such a language would truly have the full power of polynomial-time functions.
The core of this chapter is~\autoref{thm:fl-is-l-red}, stating that this holds at least for the class
of logspace-computable functions and a particular notion of weak reductions. But the heart
of this intuition is not developed until~\autoref{thm:fc-is-fac0-closure}, where we get
such characterizations for most of the complexity classes that we consider in this thesis.

\section{Decisional and functional complexity classes}\label{sec:decisional-and-functional-complexity-classes}
We can't answer the question of expressive power of \(\problem{CVP} + \compL \text{-reductions}\)
by comparing it with any decisional class.
Focusing solely on the complexity of Boolean functions
as we did for now e.g.\ in~\autoref{sec:preliminaries-turing} is 
not sufficient to reason about general functions with output.
In this chapter we will introduce the still standard, but less talked about, 
\emph{functional} complexity classes,
that study general functions \(f : \{0,1\}^\ast \to \{0,1\}^\ast\). In complexity theory these are
usually implicitly used to define \emph{reductions}. The results
about these classes  transfer much better to our interests than results about the decisional complexity
classes. As we discuss  in~\autoref{chap:recursion-theory}, programming-language-like characterizations
of decisional complexity classes are far more abundant and predate the characterizations of
functional complexity classes. It is the latter, however, that is viable for our purposes of
designing a programming language.

A thorough overview of complexity classes of functions is described in~\cite{SELMAN1994357}.
A more thorough discussion of decision vs search is in~\cite{10.5555/889581}.

\paragraph{Exponential-length output}
The difference between \compP{} and \compFP{} is obvious when we look at the below example:
\begin{example}
    Running time of any Turing machine computing the function \(x \rightarrow 2^x\) for input and output in binary,
    is exponential. At the same time, given an input \(x, y\), checking if \(y = 2^x\) is easily in polynomial time.
\end{example}
This problem is, however, usually artificially mitigated by requiring the length \(|f(x)|\)
of the function's output to be polynomially bounded as in~\autoref{def:logspace-reductions}.


\paragraph{Self-reducibility}
A typical, and ubiquitous in the literature way of defining function problems is to require
an external proof that \(|f(x)|\) is polynomially bounded,
then repeatedly decide if \(i\)-th bit of \(f(x)\) is 0 or 1. For example, let's look at the below

\begin{definition}[{\cites[Definition~4.16]{10.5555/1540612}[Definition~4.14]{DRAFT10.5555/1540612}}~\compL-reductions]\label{def:logspace-reductions}
A function \(f : \{0,1\}^{\ast} \to \{0,1\}^{\ast}\) is said to be
\emph{logspace computable} if
\begin{enumerate}
  \item \(f\) is \emph{polynomially bounded}, meaning that there exists a
        constant \(c > 0\) such that
        \[
          |f(x)| \le |x|^{\,c}
          \qquad\text{for all } x \in \{0,1\}^{\ast},
        \]
        and
  \item the following two languages lie in \(\mathsf{L}\):
        \[
          L_{f}
            = \{\,\langle x,i\rangle \mid f(x)_{i} = 1\,\},
          \qquad
          L'_{f}
            = \{\,\langle x,i\rangle \mid i \le |f(x)|\,\}.
        \]
\end{enumerate}

In other words, a deterministic \(\bigO(\log |x|)\)-space machine can, given
\((x,i)\), determine whether \(i\) is within the length of \(f(x)\) and, if so,
whether the \(i\)-th bit of \(f(x)\) is \(1\).

A language \(B\) is \emph{logspace reducible} to a language \(C\), if there exists
a function
\(f : \{0,1\}^{\ast} \to \{0,1\}^{\ast}\) that is logspace
computable and such that \(x \in B\) iff \(f(x) \in C\) for every
\(x \in \{0,1\}^{\ast}\).
\end{definition}


Famously, we can also apply this trick to solve the problem \problem{FSAT}, the corresponding functional
problem to the \problem{SAT} of finding a specific
satisfying assignment with polynomially many calls to the decision procedure \problem{SAT}. 
In general, many functional problems are solvable in polynomial time with polynomially
many calls to their corresponding decisional problems. We say that
problems with that property are \emph{self-reducible}.

However, some search problems are unlikely to be self-reducible.
A good example is the problem of integer factorization, which
is still, as of November 2025, conjectured to not be in \compP even despite
the recent breakthrough in which \problem{PRIME} was proved to be in \compP.
A particularly important class of such problems is considered in~\ref{subsubsec:tfnp}.
But first, let's define the classes \complexity{FP} and \complexity{FNP}.

\section{Functional complexity classes}\label{sec:functional-complexity-classes}

\todo{perhaps define poly-time computable functoins in preliminaries, and here just link to that}
\begin{definition}[{\cites[Section~17.2]{10.5555/1540612}[Section~9.1]{DRAFT10.5555/1540612}}~The class \complexity{FP} (Version 1)]\label{def:fp-ver1}
  \(\mathsf{FP}\)~consists of all functions
\[
  f : \{0,1\}^{\ast} \to \{0,1\}^{\ast}
\]
that are computable by a deterministic polynomial-time Turing machine.
In contrast to decision problems (which output a single bit), functions
in \(\mathsf{FP}\) may produce outputs of arbitrary polynomial length.

\begin{remark}
  This definition is used e.g.\ in~\cite{COOK19852} besides the work cited above.
  It is rather equivalent to~\autoref{def:fp} below, also ubiquitous in the literature.
\end{remark}
\end{definition}

\todo[inline]{I have a definition of also poly-time reductions e.g. for NP in tex, but seems like i'm not using it anywhere}
\begin{definition}[{\cite[Section 28.10]{Rich2007Automata}}~\complexity{FP} (Version 2)]\label{def:fp}

A binary relation \(P(x, y)\) is in \(\complexity{FP}\) iff there exists a polynomial-time Turing machine
that, given an arbitrary input \(x\):
\begin{enumerate}[label=(\roman*)]
    \item outputs some \(y\) such that \(P(x, y)\) if any exists;
    \item signals that no such \(y\) exists otherwise.
\end{enumerate}

\begin{remark}
  This version can make it more obvious to compare \(\complexity{FP}\) with \(\complexity{FNP}\) (defined later) ---
  but only assuming that \(\complexity{FNP}\) is defined using nondeterministic Turing machines,
  which is not true in our case; we will use the \emph{verifier}-style definition.
\end{remark}


% transducers of polynomial growth are studied in~\cite{10.1145/3531130.3533326}, where also
% are given pebble, functinal, impearitve and logical models of computation.
\end{definition}


\begin{remark}[\complexity{P} vs \complexity{FP}]
These two classes are often identified due to similar properties.
The notion of completeness for both of them, despite being differently defined, practically is practically the same
to the robustness of \(\complexity{P}\) computations under being repeated for every bit of the output.
Indeed, even in Stephen Cook's 1982 ACM Turing Award lecture~\cite[Section 6]{10.1145/358141.358144},
it is not clearly distinguished between
\(\complexity{P}\)-completeness and \(\complexity{FP}\)-completeness: the 3 proofs cited in this lecture
as proofs of \(\complexity{FP}\)-completeness of some functions \(f(x)\) only themselves prove the
\(\complexity{P}\)-completeness of problems of the form ``decide if \(i\)-th bit of the result \(f(x)\) is zero''.

The two classes, however, are not the same.
In~\cite[Theorem 4.1]{KRENTEL1988490}, it is proved that
if 
\(\complexity{FP}^{\complexity{SAT}}[\bigO(\log{n})] = \complexity{FP}^{\complexity{SAT}}[n^{\bigO(1)}]\)
then also \(\complexity{P}=\complexity{NP}\).
In turn, as noted in~\cite[discussion after Theorem 8]{doi:10.1142/9789812794499_0029}, the corresponding result for
\(\complexity{P}^\complexity{NP}\) versus \(\complexity{P}^\complexity{NP}[\bigO(\log{n})]\) is not known,
and indeed fails relative to some oracles.

For a good discussion specifically on \complexity{FP}-completeness,
which is relatively hard to find, there is an argument that finding the lexicographically
first maximal clique in an undirected graph
is \complexityi{NC}{i}-complete for \complexity{FP} in~\cite[Proposition~6.1]{COOK19852}.
\end{remark}

% POLYTIME-REDUCIBILITY
% We will say that a function is \emph{polynomial-time computable} if it is
% computable by a Turing machine running in polynomial time, not using the
% decider trick for \emph{logspace computability} in~\autoref{def:logspace-reductions}.
% \begin{definition}[\compP-reductions~{\cites[Definition~2.7]{10.5555/1540612}[Definition~2.7]{DRAFT10.5555/1540612}}]\label{def:p-reductions}
% Let \(A,B \subseteq \{0,1\}^{\ast}\) be languages.  
% We say that \(A\) is \emph{polynomial-time Karp (many-one) reducible} to \(B\),
% if there exists a polynomial-time
% computable function
% \[
%   f : \{0,1\}^{\ast} \to \{0,1\}^{\ast}
% \]
% such that for every input \(x \in \{0,1\}^{\ast}\),\todo{consistency: \(\iff\) vs \(\Longleftrightarrow\)}
% \[
%   x \in A
%   \;\Longleftrightarrow\;
%   f(x) \in B.
% \]
% In this case, the function \(f\) is called a \emph{polynomial-time reduction}
% from \(A\) to \(B\).
% \end{definition}









\subsection{\complexity{FNP}}
The definition of \complexity{FNP} is tricky to get right.
A very good discussion of the awkwardness of the definitions is present in~\cite{37813}.
For extensive discussion on the different definitions, see~\cite{37812},~\cite{71617}.
In Papadimitriou's book, it's defined in yet another way, as a class function problems for \complexity{NP},
not in terms of a specific computational model.

\begin{definition}[The class \texorpdfstring{\complexity{FNP}}{FNP}]\label{def:complexity-fnp}
A binary relation \(P(x, y)\) is in \(\complexity{FNP}\) if there exist:
\begin{enumerate}
  \item a polynomial \(p : \mathbb{N} \to \mathbb{N}\) such that if for a given \(x\) exists a solution \(y\) such
  that \(P(x, y)\), then there also exist a ``short'' solution \(y'\) such that
  \(P(x, y')\) and \(|y'| \leqslant p(|x|)\);
  \item a deterministic polynomial-time Turing machine \(M\)
        (a \emph{verifier}), 
        such that for every input pair \((x, y)\),
        \[
            P(x, y)
            \;\Longleftrightarrow\;
            M(x, y) = 1.
            \]
\end{enumerate}

\begin{remark}
  This definition is in style of~\cite[28.10~and~Theorem~28.9]{Rich2007Automata}, where also the
  other, nondeterministic Turing machines-based definition is listed.

  The other definition might come off as more intuitive:
  that a relation $P$ is in \complexity{FNP} iff there is a nondeterministic polynomial-time
  algorithm that, given an arbitrary input $x$,
  can find some $y$ such that $P(x, y)$ or signal that it doesn't exist~\cite{bournez_et_al:LIPIcs.MFCS.2019.23}.
  However, as such nondeterministic Turing machines don't seem to be physically realisable, we don't want to
  introduce that computational model in this work.
\end{remark}

\end{definition}




\subsection{\complexity{NP} vs \complexity{FNP} and the total search problems}\label{subsubsec:tfnp}
\begin{definition}[\complexity{TFNP}]\label{def:tfnp}
A binary relation \(P(x, y)\) is in \(\complexity{TFNP}\) (total \(\complexity{FNP}\)) iff it is
in \(\complexity{FNP}\) and for every \(x\) exists at least one \(y\) such that \(P(x, y)\).
\end{definition}

An interesting example of a problem in \(\complexity{TFNP}\) is \problem{PIGEON} defined below,
for which we mathematically know that the answer exists, but finding it is not trivial.

\begin{definition}[\problem{PIGEON}]\label{def:pigeon}
Given a binary string encoding a Boolean circuit \(C:\{0,1\}^n\!\to\!\{0,1\}^n\), return either
an input \(x\) such that \(C(x) = 0^n\), or two distinct inputs \(x \neq y\) such that \(C(x) = C(y)\).
\end{definition}

\begin{remark}\label{remark:link-pigeonhole}
    This class will be of our interest in~\autoref{subsec:vac0-php}, where we will discuss mathematical theories
    so weak that the pigeonhole principle is not their theorem. The intuition behind it is that
    the computational content of these theories is not strong enough to perform an exhaustive
    linear search of the whole domain.
\end{remark}

The class \complexity{PPP}, a subclass of \complexity{TFNP} problems for which the solution is
guaranteed to exist by the pigeonhole principle, is conjectured to not be equal to \complexity{FP}.
If \(\complexity{PPP} = \complexity{FP}\), then one-way permutations do not exist~\cite[Proposition~3]{PAPADIMITRIOU1994498},
which would have tremendous implications for cryptography.

The class \complexity{TFNP} is discussed in yet more detail in~\cite[Section 1.1]{10.5555/1104410}.







% CIRCUIT-REDUCIBILITY
% uniform FAC0 does NOT admit a nice circuit characterization. FAC0/poly
% is standard (see LogicalFoundations Definition V.2.3) circuits with output.
% But the notion of uniformity doesn't generalize to circuits with outputs!
% Our only hope is in ``polynomial number of AC0-decidable outputs''!

% FAC0 vs AC0-reduction: Definition IX.1.1: AC0-reductions are Turing reductions
% and are circuits with oracle gates for some problem L!
% Section IX.2.1: if C is relations ac0-reducible to F, then FC is FAC0 closure of F.


% \section{\texorpdfstring{$\complexityi{AC}{0}$-reduction}{AC\string^0-reduction}}
% \label{sec:ac0red}
% Definition IX.1.1 CN10. We say that a string function F
% (resp. \  a number function f) is $\complexityi{AC}{0}$-reducible to $L$ if there is a sequence
% of string functions $F_1, \dots, F_n (n \geqslant 0)$ such that
% $F_i$ is $\Sigma^B_0$-definable from $L \cup \{F_1, \dots , F_{i-1}\}$, for $i = 1, \dots, n$
% and F (resp. \ f) is $\Sigma^B_0$-definable from $L \cup \{F_1, \dots , F_{i-1}\}$. A relation R is
% $\complexityi{AC}{0}$-reducible to $L$ if there is a sequence $F_1, \dots, F_n$ as above, and R is
% represented by a $\Sigma^B_0(L \cup \{F_1, \dots, F_n\})$ formula.

% In Chapter~2 of~\cite{edbd4873718c414f90d22dadf0dba2b1} there is an extensive discussion about
% the different subtleties of defining $\complexityi{AC}{0}$ functions and numerous different characterizations
% of Dlogtime-uniform $\complexityi{AC}{0}$-computable functions.

\begin{definition}
  \todo[inline]{I really don't want to introduce circuit reductions... skipping it.}
\end{definition}


\subsection{Language for FL}

\begin{theorem}[{\cite[Proposition~4.1]{COOK19852}}]\label{thm:fl-is-l-red}
  We obtain precisely the class \complexity{FL} from the closure of \compL under \complexityi{NC}{1} circuit reductions,
  symbolically: \(\complexity{FL} = \complexity{L}^\ast\)
  
  \begin{remark}
    Originally, this theorem is proved with \complexityi{NC}{1}-reducibility meaning
    reducibility by \compUeAst-uniform \complexityi{NC}{1} circuits. We don't introduce these
    notions in this work (except for a brief discussion of \compUeAst-uniformity in~\autoref{sec:uniformity-ueast})
  \end{remark}
\end{theorem}

An overview of problems complete for \compL is present in~\cite{COOK1987385}.

\subsection{Language for \complexity{FP}}
We can derive an analogous result to~\autoref{thm:fl-is-l-red} for the class \complexity{FP}.
We are discussing it here, postponing the considerations to~\autoref{thm:fc-is-fac0-closure}.

% # Circuit Value Problem
% - For a given single-tape, polynomial-time Turing machine `M` and input `x`, in [@Kozen2006],
% there is an explicit construction of a boolean circuit over (0, 1, `and`, `or`, `not`)
% (with fan-in 2 for `and`, `or` and 1 for `not`), with one output node, such that its value
% is 1 if and only if machine `M` accepts input `x`. The construction is in LOGSPACE.
% So CVP is P-complete w.r.t. LOGSPACE-reductions.
% - This is a good example of a LOGSPACE-reduction, being a good benchmark for the LF programming
% language and for the circuit description language
% - The problem is that we can't generate tests for it; we have no database of Turing machines descriptions
The notion of \compP-completeness is defined formally e.g.\ in~\cites[Definition~6.25]{DRAFT10.5555/1540612}[Definition~6.28]{10.5555/1540612}.
A very detailed description of one problem complete for \compP under \compL-reductions is in~\cite{Kozen2006}.

\subsection{Not-a-Language for \complexity{FNP}}
There is a relatively agreed-upon notion of reductions between \complexity{FNP} problems:

\begin{definition}[{~\cites{Goebel2011NashComplexity}{Goldberg2021SearchTotal}}~Polynomial-time reductions for \complexity{FNP}]
    Let \(\texttt{HardProblem}\), \(\texttt{NewProblem}\) be search problems in \(\complexity{FNP}\).
    We say that \(\texttt{HardProblem}\) (many-one) reduces to \(\texttt{NewProblem}\) if there exist
    \(f, g\) in \complexity{FP} such that:
    \[\texttt{NewProblem}(f(x), y) \implies \texttt{HardProblem}(x, g(y))\]

    For a given input \(x\) of \(\texttt{HardProblem}\), we can run \(\texttt{NewProblem}(f(x))\)
    to obtain some result \(y\), such that \(g(y)\) is the result of \(\texttt{HardProblem}(x)\).
\end{definition}

There is also plenty of \complexity{NP}-complete problems described in the literature.
However, it is very unclear if we will get \complexity{FNP} this way. The class 
\(\complexity{FP}^{\complexity{NP}}\) is well-studied and nothing suggests it to be equal to \complexity{FNP}.


\subsection{Semantic and syntactic complexity classes}
Some of the complexity classes remain notoriously difficult to be characterized implicitly,
by e.g.\ showing a complete problem and reductions for it.
However, as it turns out, not all complexity classes studied have known
complete problems.
The classes for which a complete problem exists are called ``syntactic'' complexity classes,
as opposed to ``semantic'', e.g.~in~\cite{DBLP:conf/innovations/GoldbergP18}, the authors define a new complexity class
\(\complexity{PTFNP \subseteq \complexity{TFNP}}\), for which
they prove the existence of a complete problem, and then call this class ``syntactic''.

An interesting discussion of this problem, centered around the class \complexity{inv-P}
we discussed in~\autoref{remark:unordered-structures}, is present in~\cite{dawar2012syntactic}.
For the class \complexity{BPP}, some discussion is in~\cite{35236}.
Despite that, a ``less implicit'' characterization of BPP was studied in~\cite{lago2012higherordercharacterizationprobabilisticpolynomial}.

Interestingly, PP has been characterized implicitly by Ugo Dal Lago:~\cite{dallago_et_al:LIPIcs.MFCS.2021.35}.



\begin{remark}[Bibliography]
For probably the first published recognition of the widespread inconsistency of decisional vs functional
complexity classes in the literature, with examples of inconsistent places see~\cite[Page~131]{10.5555/114872}
(and our bibliographical~\autoref{remark:bibliography-david-s-johnson}).

\(\complexity{TFNP}\) was first introduced in~\cite{MEGIDDO1991317}.
\end{remark}







\section{Oracle-oriented programming}
If you use reductions + a single oracle for a difficult problem, you can get
a powerful programming language. This would be a new, nice paradigm that I aimed to
realize.

Due to possible quadratic blowup of output size (even for ac0 circuits i think? but maybe not if fo-uniform?),
it is unsatisfactory for us to have a single complete problem solving e.g. sat in worst-time \(2^n\).
Because then it gets \(2^{n^2}\) etc., which is bad. Transductions (or fine-grained time complexity classes
such as dlintime) are a potential nice class for this.

\subsection{Oracle Turing machines and the technique of forcing}\label{subsec:oracle-forcing}
\todo[inline]{optional. Baker-Gill-Solovay proof uses forcing. }
% Some discussion in~\cite{14091}.



% ### Proving unprovability: Kripke semantics
% - even though searching for a countermodel in Kripke semantics is completely infeasible computationally, we have a good tool for the job!
% - i tested and it works, find countermodels and proofs of intuitionistic formulas. code: https://github.com/ferram/jtabwb_provers/tree/master

% ### Proving unprovability: forcing
% - Extending Type Theory with Forcing (INRIA, 2012)
% > Implementation of forcing in Coq as a program transformation and show a proof of the negation of CH  
% > https://hal.science/hal-00685150/document

% - A beginner’s guide to forcing
% > https://arxiv.org/pdf/0712.1320

% - Forcing for dummies blogpost
% > https://timothychow.net/mathstuff/forcingdum.txt

% - Baker-Gill-Solovay theorem proof
% > Forcing as a method to prove that something can or cannot be done using an oracle  
% > https://math.stackexchange.com/questions/2616541/simple-applications-of-forcing-in-recursion-theory  
% > https://en.wikipedia.org/wiki/Forcing_(computability)

% forcing zeby badac jezyki programowania: to jak oracles w computational complexity!
% https://cstheory.stackexchange.com/a/14093
% see here for oracle A such that NEXP^A = P^{NP^{A}}
% what it means in logic when you have P^A,B vs P^A^B?
% https://link.springer.com/article/10.1007/s00037-001-8190-2




\subsection{Fine-grained reductions}\label{subsec:fine-grained-reductions}
% We will not realistically capture $\text{TIME}(\bigO(n))$ or anything of this kind,
% as the field of fine-grained complexity is relatively modern and little or none interesting
% characterizations of these classes have been found as of writing this work.
% \begin{enumerate}
% \item TODO: Review Neil D. Jones's ``Constant Time Factors Do Matter'' for its discussion of NLIN-complete problems (\url{https://dl.acm.org/doi/pdf/10.1145/167088.167244}).
% \item TODO: Summarize the insights from Gurevich and Shelah's ``Nearly Linear Time'' concerning the definition of $\complexity{DTIME}(n)$ and nearly-linear-time-complete problems under QL reductions (\url{https://link.springer.com/content/pdf/10.1007/3-540-51237-3_10.pdf}).
% \end{enumerate}
\todo[inline]{just mention QL (quasilinear-time functions), NLT (robust complexity class for 
    \( \complexity{DTIME} (n (\log{n})^{\bigO(1)}) \) 
    on RAM machines~\cite{10.1007/3-540-51237-3_10})
}




% \item TODO: Example programming language characterizing \complexity{L}: finite number of variables each bounded by $n$.
% \item TODO: Explore the alternative characterization using a finite number of input pointers, relating it to multi-head two-way automata~\cite{423885},~\cite{10.1007/BF00289513}.



% \subsection{Related works on specifically the \complexity{FL} class}
% Early function algebras for \complexity{FL} appeared in~\cite{10.1145/1008293.1008295} and~\cite{lind1974logspace},
% but these were explicit characterizations.
% In~\cite{10.1007/BF01201998} it was shown how to readily use their concept
% to characterize functions from \complexity{FL} with ``small output'', but this characterization
% relied on using unary representation of natural numbers on input, which is more of a 
% hack than a true characterization of this class.
% In~\cite{murawski2000can}, with further refinements in~\cite{MURAWSKI2004197},
% \(BC^{-}\) was introduced, an algebra that was contained in \complexity{FL},
% but was not known (and unlikely) to be \complexity{FL}-complete.
% In~\cite{Neergaard04} this was improved to the result that \(BC_\varepsilon^{-} = \complexity{FL}\),
% with a short discussion that using course-of-value affine recursion instead of predicative affine recursion
% seem to be the reason behind \(BC_\varepsilon^{-}\) being FL-complete, and \(BC^{-}\) being probably not.

% In~\cite{4276584}, Stratified Bounded Affine Logic is introduced to capture \complexity{FL} computation.
% In~\cite{10.1007/978-3-662-46678-0_27}, an interesting approach using coinduction is utilized to capture \complexity{FL}.

% In~\cite{hofmann2006logspace} a good overview of languages for \complexity{FL} is presented,
% and in~\cite{schoepp2006spaceefficiency}, the history of \complexity{FL} characterizations is traced.

\chapter{Implicit Computational Complexity}\label{chap:icc}
Implicit computational complexity (ICC) studies how to guarantee resource bounds
without appealing to external machine models.
Instead of analysing running time or space after the fact, ICC designs
languages and recursion schemes whose syntactic constraints ensure that every
definable function belongs to a chosen complexity class.
The aim is a principled foundation for programming languages that ``build in''
complexity guarantees by construction.

Our goals in this chapter are twofold.
For one, we want to introduce
the most promising ICC approaches we have found towards obtaining a language for
\complexity{FL} or for \complexity{FP}. Indeed, the characterizations we will study
resulted in the creation of two actual programming languages. They are purely of academic
interest, but there are close to none alternative languages for complexity classes
that reached the stage of being actually implemented.
In the rest of this chapter we explain why the approach we studied most ---
of using techniques from ICC --- did not lead to a practical programming language for our purposes,
even though it initially seems very appealing.
However, more importantly, we want to share \emph{negative} evidence of
usefullness of the characterizations studied in ICC towards our ultimate goal
of not just creating a programming language, but using it to \emph{certify}
the complexity of known algorithms.

We will begin by sharing our experience with ICC,
then proceed to discuss the characterizations we have studied, most importantly:
in~\autoref{def:bc-eps} we examine a programming language based on the first
function algebra (untyped) for \complexity{FL} due to Neergaard; in~\autoref{sec:intml} we study
Dal Lago and Schöpp's programming language \texttt{IntML} with the type system 
guaranteeing the capturing of \complexity{FL} and \complexity{FNL} (nondeterministic version of \complexity{FL},
which we don't introduce formally).

\begin{remark}
  Recall that, given a description of a Turing machine, the problem of deciding whether it halts
  is undecidable. Hence, deciding whether an arbitrary computer program belongs to the complexity
  class \compP{} is also undecidable. This is only a problem, however, if we take \emph{arbitrary}
  programs as input. If we limit the scope to programs written in a special, limited programming
  language, we can easily design the language so that it does not admit constructs such as
  \texttt{while}-loops or general recursion, and therefore every program in it necessarily terminates.
  Moreover, membership in the syntax of such a language can be easy to decide. In this way, the
  undecidability problem is not resolved, but ``shifted'' to the difficulty of programming: given a
  \texttt{C} program implementing an algorithm, it becomes undecidable whether there exists a
  corresponding program in our restricted language.
\end{remark}


That was the approach that took us the most time to study and did not result in obtaining
``practical programming languages''. We want to place ourselves in the program of ``we need a better way of
communicating negative results (failed research)'', and signal why, despite the apparent
usefulness of the results for our purpose, our goal failed. The reason is the difference between
\emph{intenstional} and \emph{extensional} expressive power of characterizations of complexity classes.
With type systems capturing low complexity classes, it becomes very apparent that these function
classes are only \emph{extensionally} equivalent to their corresponding complexity classes.
For an example of how easily we can lose \emph{intensional} expressive power in a language with
linear types, please look at~\autoref{lst:haskell-linear-example}.
They shift the problem of undecidability of deciding if a given Turing machine works in some complexity,
to the problem if the algorithm implemented by it is transferrable to the particular programming language.

\subsection{Extensional and intensional equality of sets of functions}\label{subsec:intensional}

The field appear very fragmented and the approaches don't scale:~\cite{DalLagoMartini2006MuTutorial}.

\begin{remark}
    We don't introduce tiered recursion, ramified recursion and Leivant's works in general here:~\cite{LeivantRamyaa11}.
    See also~\cite{3ffa7833-e2d2-3419-abb5-7f266190ba48} for discussion on tiering as recursion technique.
\end{remark}



Accessible introductions to ICC include the three-part presentation~\cite{martini2006implicit1,martini2006implicit2,martini2006implicit3},
the talk~\cite{ronchi2019logic},
and a short overview~\cite{DalLago2012}.


\begin{remark}
    We focused on the characterizations of \complexity{FL} and \complexity{FP}.
\end{remark}

% \subsection{Related works on specifically the \complexity{FL} class}
% Early function algebras for \complexity{FL} appeared in~\cite{10.1145/1008293.1008295} and~\cite{lind1974logspace},
% but these were explicit characterizations.

% In~\cite{10.1007/978-3-662-46678-0_27}, an interesting approach using coinduction is utilized to capture \complexity{FL}.

% In~\cite{hofmann2006logspace} a good overview of languages for \complexity{FL} is presented,
% and in~\cite{schoepp2006spaceefficiency}, the history of \complexity{FL} characterizations is traced.

Description of logspace, ptime (decisive):~\cite{Jones99}; logspace, linspace:~\cite{Kristiansen05}.~\cite{Bonfante06}
For a broad literature survey, see~\cite{bloch1994function}.




\subsection{Characterizations not easily adjustable for a programming language}
Before the seminal works that founded the field of Implicit Complexity, many characterizations
of complexity classes had been known already. All of them suffered at least one of the two problems:
either it only characterized a class of relations in a given complexity (as opposed to functions),
or the characterization wasn't purely syntactic. We will refer to the latter of being ``explicit''
instead of ``implicit''.

\subsection{Characterizations of classes of relations}
Characterizations of classes of relations, such as \complexity{P} (as opposed to \complexity{FP}),
are not of interest to us because they don't generalize at all to a programming language
allowing to write functions with output. Nevertheless, we investigated the concepts used
there and describe some of them briefly in this subsection.

Polynomial-time relations have been characterized without explicit
size bounds in~\cite{doi:10.1137/0216051}.
In~\cite{COMPTON1990241}, uniform \(\complexityi{NC}{1}\) was characterized,
and in~\cite{ALLEN19911} uniform \(\complexity{NC}\), though their definitions
still concealed polynomial bounds and targeted relations instead of functions.

In more modern works, decisive complexity classes have been successfully characterized in~\cite{JONES1999151} by a fragment of Lisp in~\(\complexity{L}\) and
\(\complexity{P}\). The same concept has been extended to account for
for nondeterminism in~\cite{10.1007/11784180_8}.
The authors of~\cite{kristiansenvoda2005} investigated both imperative
and functional programming languages whose fragments yield hierarchies containing \emph{decisional} \complexity{L},
\complexity{LINSPACE}, \complexity{P}, and \complexity{PSPACE}.
Related contributions include~\cite{kristiansen2005neat} and~\cite{Oitavem+2010+355+362}.



The modern study of ICC begins with two breakthroughs:~\cite{151625}~and~\cite{10.1007/BF01201998}
gave the first implicit characterisations of polynomial-time computable functions.

However, the idea of Bellantoni and Cook seemed to best align with being the foundation
of a practical programming language. Hence, we decided to solely focus on it and its successors.

Since then (since BC, leivant), numerous classes have been captured implicitly; see, for
example,~\cite{NIGGL201047}~and~\cite{10.1016/j.ic.2015.12.009}
for overviews of \(\complexity{FP}\) and \(\complexity{FNC}\) characterisations.

See~\cite{10.1007/s00153-022-00828-4} for implicit characterizations of counting classes such as $\complexity{\mathbin{\#}P}$ (not introduced here).

\section{Recursion-theoretic approach}\label{sec:recursion-theory}
In this section we focus on techniques from recursion theory that were successfully
utilized in Implicit Computational Complexity.

\subsection{Origins of recursion theory}
While not the primary focus of this work, the field of recursion theory developed concepts
that later became foundational for ICC\@. An important formal system studied there is \emph{primitive recursion}.


\begin{definition}[Primitive recursive functions]\label{def:primitive-recursive}
      \(\complexity{PR}\) is the smallest class of functions containing~\ref{itm:pra-const}--\ref{itm:pra-proj} and closed under~\ref{itm:pra-comp} and~\ref{itm:pra-rec}:
\begin{enumerate}
\item\label{itm:pra-const}\textbf{(constants)} for every \(n\in\mathbb{N}\) and \(k\ge 0\), the \(k\)-ary constant function
      \(c_{n}^{(k)}(\vec x)=n\);
\item\label{itm:pra-succ} \textbf{(successor)} \(S(x)=x+1\);
\item\label{itm:pra-proj} \textbf{(projections)} for \(k\ge 1\) and \(1\le i\le k\),
      \(\pi_i^{(k)}(x_1,\dots,x_k)=x_i\);
\item\label{itm:pra-comp} \textbf{(composition)} if \(h:\mathbb{N}^m\to\mathbb{N}\) and
      \(g_1,\dots,g_m:\mathbb{N}^k\to\mathbb{N}\) are in \(\complexity{PR}\), then
      \(f(\vec x)=h\big(g_1(\vec x),\dots,g_m(\vec x)\big)\) is in \(\complexity{PR}\);
\item\label{itm:pra-rec} \textbf{(primitive recursion)} if \(g:\mathbb{N}^k\to\mathbb{N}\) and
      \(h:\mathbb{N}^{k+2}\to\mathbb{N}\) are in \(\complexity{PR}\), then the unique
      \(f:\mathbb{N}^{k+1}\to\mathbb{N}\) is in \(\complexity{PR}\) with:
      \[
      f(0,\vec x)=g(\vec x),\qquad
      f(S(y),\vec x)=h\big(y,\,f(y,\vec x),\,\vec x\big).
      \]
\end{enumerate}
\end{definition}

\begin{example}[Addition]
Define \(\mathrm{Add}:\mathbb{N}^2\to\mathbb{N}\) by primitive recursion:
\[
\mathrm{Add}(0,x)=x, \qquad
\mathrm{Add}(S(y),x)=S\big(\mathrm{Add}(y,x)\big).
\]
\end{example}

\begin{definition}[LOOP language]
Let \(\mathrm{Var}=\{x_0,x_1,x_2,\dots\}\).
LOOP programs are generated by the grammar
\[
\begin{aligned}
P ::=~& x_i := 0
\;\mid\; x_i := x_i + 1
\;\mid\; P \,;\, P
\;\mid\; \texttt{LOOP}~x_i~\texttt{DO}~P~\texttt{END},
\end{aligned}
\]
where \(x_i\in\mathrm{Var}\).

\noindent
We assume standard semantics, with a remark that \(\texttt{LOOP}~x_i~\texttt{DO}~P~\texttt{END}\) repeats \(P\) exactly as many times as the value stored in \(x_i\) \emph{at loop entry} (changes to \(x_i\) inside \(P\) do not change the iteration count).
\end{definition}

\begin{theorem}[{\cite{10.1145/800196.806014}}~\complexity{LOOP} captures precisely \complexity{PR}]\label{thm:loop-captures-pr}
      % TODO: make statement precise, perhaps based on https://www.cs.cornell.edu/courses/cs6110/2012sp/MeyerAndRitchie-67.pdf
      The functions computable by \complexity{LOOP} programs are precisely the primitive recursive functions
\end{theorem}

This simple connection actually satisfies our criteria of a ``programming language capturing a complexity class'', as
the \(\complexity{LOOP}\) language captures exactly \(\complexity{PR}\)\footnote{As recognized at \url{https://complexityzoo.net/Complexity_Zoo:P}.}, the class of primitive recursive functions.
Moreover, we can even stratify the primitive recursive functions into a hierarchy like in~\cite{Grzegorczyk1953}.

\begin{remark}[Bibliography]
      Historically, the origins of primitive recursion can be traced back to~\cite{Grassmann1861} and~\cite{Dedekind1888},
      but the class was probably first considered as the primary object of study in~\cite{Skolem1923-vanHeijenoort}.
      For the details of the historical origins, consult~\cite{Adams2011}.
\end{remark}


\subsection{Explicit characterizations}\label{subsec:explicit}
Some characterizations discussed in the literature contain so-called \emph{explicit}
conditions --- e.g.\ they explicitly require a function to not grow faster than polynomially.
Such conditions cannot be checked syntactically and must be enforced
by an additional proof outside of the algebra. We call these characterizations
explicit, as opposed to implicit. Despite not being convenient to use for our purpose, these concepts
have been very important for the field and we will discuss one example in this subsection.
For more examples of explicit characterizations, please see~\cite{4568079} for an algebra for polynomial-time
functions;~\cite{COMPTON1990241} for uniform \complexityi{NC}{1};~\cite{ALLEN19911} for uniform \complexity{NC}.

For a good overview of implicit versus explicit characterizations, refer to~\cite{aubert:hal-01111737}.

A famous example of an explicit characterization is Cobham's algebra
for polynomial-time functions, using the recursion scheme defined below.
We will use the notation \(S_0(y)=2y\) and \(S_1(y)=2y+1\) for appending a binary digit to \(y\).
We will denote by $\len{x}$ the length of binary representation of $x$; in particular, $\len{0} = 1, \len{1}=1, \len{2}=2$.

\begin{definition}[{\cite[Definition~VIII.2.14~(Page~174)]{Odifreddi1999CRT2}}]\label{def:bounded-binary-primitive-recursion}
A function \(f\) is defined from functions \(g\), \(h_0\), \(h_1\), and \(s\) by \emph{bounded primitive recursion on binary notation}
if, for every \(\vec x\) and \(y\in\mathbb{N}\),
\begin{align}
  f(\vec x, 0)      &= g(\vec x);\\
  f(\vec x, S_0(y)) &=
    \begin{cases}
      0 & \text{if } y = 0,\\
      h_0\big(\vec x, y, f(\vec x, y)\big) & \text{otherwise;}
    \end{cases}\\
  f(\vec x, S_1(y)) &= h_1\big(\vec x, y, f(\vec x, y)\big);\\
  f(\vec x, y)      &\le s(\vec x, y)\label{eq:explicit-cobham}.
\end{align}
\end{definition}

\begin{remark}\label{remark:cobham-recursion}
  The recursion parameter $y$ is written in binary, so the definition of $f(\vec x,y)$
  unfolds only $\mathcal{O}(\len{y})$ many steps.
  Moreover, the
  side condition $f(\vec x,y)\le s(\vec x,y)$ implies that every value of $f(\vec x,y)$
  is at most $s(\vec x,y)$. Hence, if the defining functions
  $g,h_0,h_1$ and the bounding function $s$ are polynomially bounded (in the lengths
  of their arguments in binary), then $f$ will also be polynomially bounded.
  
  In this definition, the
  bound~\eqref{eq:explicit-cobham} is \emph{explicit}: when one writes a function in this
  style, there is no obvious way to check mechanically that $f(\vec x,y)\le s(\vec x,y)$
  holds, other than by supplying a separate mathematical proof.
\end{remark}


\begin{definition}[Cobham's algebra for \complexity{FP}]\label{def:cobham}
The class \complexity{Cob} is the smallest class of functions containing~\ref{itm:cobham-zero}--\ref{itm:cobham-smash} and closed
under composition and bounded primitive recursion on binary notation:
\begin{enumerate}
\item\label{itm:cobham-zero} for every $k \ge 0$, the $k$-ary constant function $0(\vec{x}) = 0$;
\item the binary successor functions \(S_0(x)=2x\) and \(S_1(x)=2x+1\);
\item for \(k\ge 1\) and \(1\le i\le k\), projections
                  \(\pi_i^{(k)}(x_1,\dots,x_k)=x_i\);

\item\label{itm:cobham-smash} the weak exponential \((x,y)\mapsto x^{\len{y}}\) denoted $x\mathbin{\#}y$; sometimes called \emph{the smash function}.
\end{enumerate}
\end{definition}

\begin{remark}
This algebra was originally defined for decimal digits.
Note that all of our initial functions are polynomially bounded in the lengths
of their arguments. For the smash function we have, for $x,y \ge 1$,
\[
  x < 2^{\len{x}},
  \qquad
  x^{\len{y}} < 2^{\len{x}\cdot\len{y}},
  \qquad
  \len{x^{\len{y}}} \le \len{x}\cdot\len{y} + 1,
\]
hence
\[
  \len{x \mathbin{\#} y}
  = \len{x^{\len{y}}}
  \le \len{x}\cdot\len{y} + 1.
\]
Composition preserves polynomial boundedness, and bounded recursion on binary
notation also preserves it (recall~\autoref{remark:cobham-recursion}),
so every function in $\complexity{Cob}$ is polynomially bounded.
\end{remark}

\begin{theorem}[{\cite[Proposition~VIII.2.15~(Page~175)]{Odifreddi1999CRT2}}]\label{thm:cobham-is-fp}
      The class \complexity{Cob} contains precisely the functions from \complexity{FP}.
\end{theorem}

\begin{remark}
      Cobham's characterization underlies the arithmetical theory \complexity{PV},
      which is also designed to capture
      polynomial time reasoning in the style discussed in~\autoref{chap:bounded-arithmetic}.
      We will not introduce \complexity{PV} in this work, but we will mention it once again in~\autoref{sec:bounded-arith-imppv}
      to discuss another concept for a programming language. Please see~\cite[End~of~section~6]{10.1007/BF01201998} for
      a brief discussion.
\end{remark}

\begin{remark}[Bibliography]
This algebra from~\autoref{def:cobham} was originally published in~\cite{Cobham1964-COBTIC} and never proved by the author to
actually capture \complexity{FP}.
In~\cite[Remark~15.3.22]{Tourlakis2022Computability} there is a discussion of several proofs of this
theorem from the literature, each based on different, and in general non-equivalent, definitions.
The proof we referenced in the header of~\autoref{thm:cobham-is-fp} is due to Odifreddi.
\end{remark}



\subsection{Characterization of \complexity{FP} with safe-recursion}

Bellantoni and Cook introduced a function algebra \(\complexity{BC}\) whose key
innovation is the separation of arguments into \emph{normal} inputs (controlling
recursion depth) and \emph{safe} inputs (being passed around without
influencing that depth).
We write \(f(\vec{x};\vec{a})\), with normal inputs \(\vec{x}\) to the left of
the semicolon and safe inputs \(\vec{a}\) to the right.

\begin{definition}[Bellantoni and Cook's algebra for \complexity{FP}]\label{def:bellantoni-cook}
We will use the notation $a0$ for the binary representation of $2*a$. Similarily, we will use $a1$ for $2*a + 1$,
and the more general $ai$ when $i$ is known from the context to be equal to either $0$ or $1$.

      The class \(\complexity{BC}\) is the smallest class of functions on non-negative
integers that contains~\ref{itm:bc-constant}--\ref{itm:bc-cond} and is closed under~\ref{itm:bc-rec} and~\ref{itm:bc-comp}:
\begin{enumerate}
  \item\label{itm:bc-constant}\textbf{(constant)} \(0(;)=0\);
  \item\textbf{(projection)} for \(m,n\ge 0\) and \(1\le j\le m+n\),
        \[
          \pi_j(x_1,\dots,x_m;\,a_1,\dots,a_n)=
          \begin{cases}
            x_j     & \text{if } j\le m,\\
            a_{j-m} & \text{otherwise.}
          \end{cases};
        \]
  \item \textbf{(successors)} \(s_i(;a)=2a+i\) for \(i\in\{0,1\}\);
  \item \textbf{(predecessor)} \(p(;0)=0\) and \(p(;ai)=a\);
  \item\label{itm:bc-cond}\textbf{(conditional)}\[
        C(;a,b,c)=
        \begin{cases}
          b & \text{if } a\bmod 2 = 0,\\
          c & \text{otherwise};
        \end{cases}
        \]
  \item\label{itm:bc-rec}\textbf{(predicative recursion on notation (\emph{safe} recursion))}\
        if \(g,h_0,h_1\in\complexity{BC}\), then also \(f \in \complexity{BC}\) with
        \begin{align*}
        f(0,\vec{x};\vec{a}) &= g(\vec{x};\vec{a}),\\
        f(yi,\vec{x};\vec{a}) &= h_i\bigl(y,\vec{x};\vec{a},f(y,\vec{x};\vec{a})\bigr)\qquad \text{for } i \in \{0, 1\}
        \end{align*}
  \item\label{itm:bc-comp}\textbf{(safe composition)}\
        if \(h,\vec{r},\vec{t}\in\complexity{BC}\) with each component of
        \(\vec{r}\) taking only normal arguments and each component of
        \(\vec{t}\) taking both normal and safe arguments, then also \(f \in \complexity{BC}\) with
        \[
          f(\vec{x};\vec{a}) =
          h\bigl(\vec{r}(\vec{x};\,);\ \vec{t}(\vec{x};\vec{a})\bigr).
        \]
\end{enumerate}
\end{definition}

\begin{theorem}[{\cite[Theorem~3.3,4.2]{10.1007/BF01201998}}]\label{thm:icc-belantoni-cook-fp}
      Let $f(\vec{x})$ be a function in \complexity{FP}. Then $f(\vec{x};)\in\complexity{BC}$.
      Let $f(\vec{x};\vec{y})$ be a function in \complexity{BC}. Then $f(\vec{x}, \vec{y}) \in \complexity{FP}$.
\end{theorem}


\begin{remark}[Intuition]
  The key idea of safe recursion is that safe arguments can never flow back into normal positions.
  In safe composition, the normal arguments of $h$ are obtained only from
  functions that themselves take only normal inputs. In the recursion scheme, the
  recursive value $f(y,\vec{x};\vec{a})$ is passed to $h_i$ as a safe argument.
  As a result, the
  depth of recursion can depend only on the normal inputs, and not on values computed
  during the recursion.
\end{remark}

\begin{remark}\label{remark:icc-bc-algebra-bitstrings}
The theorem is formulated in terms of computation on non-negative integers, but the proof transfers to
the case of general binary strings (e.g.\ being able to start with a zero)~\cite{10.1007/BF01201998}.
\end{remark}

\begin{remark}[Formalization of polynomial time functions]\label{remark:heraud-nowak}
Interestingly, in~\cite{10.1007/978-3-642-22863-6_11} the authors claim formalizing a~proof that
the Bellantoni-Cook algebra formulated on binary strings (recall~\autoref{remark:icc-bc-algebra-bitstrings})
captures precisely \complexity{FP}.


This sparked our interest, as it could suggest that the authors have formalized the notion of an \complexity{FP}
function in a proof assistant. In particular, one could hope for a compiler translating \complexity{BC} programs
into Turing machines running in \complexity{FP}. This is not the case, however, as
they have only formalized the proof that the class \complexity{BC} is the same as the class
\complexity{Cob}.\footnote{Link to the source code: \url{https://github.com/davidnowak/bellantonicook}.}
\end{remark}


\begin{remark}
In~\cite{10.1007/BF01201998} it is also shown how to readily use their safe recursive algebra
to characterize functions from \complexity{FL} with ``small output'', but this characterization
relied on using unary representation of natural numbers on input, which is more of a 
hack than a true characterization of this class.
\end{remark}


\subsection{Characterization of \complexity{FL} with affine safe recursion}
Møller-Neergaard refined the safe recursion discipline to capture \complexity{FL}
by strengthening the treatment of safe data: each safe value may be \emph{used} at most once.
An analogy from quantum computing is that after we ``measure'' a value by, for example, testing
one of its bits in a conditional, it disappears (goes out of scope). As in the
Bellantoni-Cook setting, arguments are split into normal and safe ones, and recursion is
permitted only on the normal arguments. In addition, the composition and recursion schemes are
designed so that safe arguments are never duplicated, and recursion has a course-of-value
flavour: successive recursive calls are allowed to ``jump back'' along the recursion chain and
do not need to visit every intermediate value. These restrictions together yield a function
algebra that characterizes \complexity{FL}. As the precise definition is quite technical and
will not be used later, we refer the interested reader to~\autoref{def:bc-eps}
and~\autoref{thm:neergaard-theorem} in~\autoref{chap:appendix-neergaard}.

This algebra was very important for us, as its description in~\cite{10.1007/978-3-540-30477-7_21}
explicitly names it as a programming language. It seemed very promising indeed to be implemented
on a computer as a simple, useful programming language.

However, in practice we found it very difficult to program in this algebra.
Neergaard's system can define every function in \complexity{FL}, but not every
\emph{algorithmic technique} commonly used to implement \complexity{FL} Turing machines.
In other words, it matches \complexity{FL} extensionally, but its intensional
expressive power (cf.~\autoref{subsec:intensional}) is quite limited. For instance,
even a function that always returns $0$ requires a non-trivial use of the recursion
scheme, simply to introduce normal arguments.

It seemed unlikely that we could add support for structures such as pairs and lists to this programming language.
In an unpublished technical report by the author, accessible at~\cite{Neergaard2004BCeps},
it was discussed that the type of \emph{pairs} of numbers seems not to be implementable in this
language. The property of the whole data structure \emph{disappearing}
after we check one bit of it seems to render implementing typical data structures very difficult.


\paragraph{Neergaard's original code from 2004}
The author's original code of the interpreter referenced in the paper is not publicly accessible.
We have managed to obtain the original Moscow ML code from 2004 from the author on a permissive license,
port it to a modern version of SML/NJ and release it with the author's
permission.\footnote{The code is available at:~\url{https://github.com/ruplet/neergaard-logspace-characterization}.}

\begin{remark}[Bibliography]
      The technical report~\cite{Neergaard2004BCeps} has never been published and seems to be
      a preliminary version of the author's publication from the same year. Despite that,
      it contains crucial insight on the limits of this characterization.
      There are papers closely related to Neergaard's publication and necessary
      to also be studied while exploring Neergaard's work. Please also
      see~\cite{NeergaardMairson2003HowLight}. There is also~\cite{MurawskiOng2000SafeRecursion},
      but it seems inaccessible online. Probably the contents of that work is similar to~\cite{MURAWSKI2004197}
\end{remark}

\begin{remark}[Origins of this thesis]
For insight into the timeline of this thesis, it's worth to note that we have first discovered
this paper ourselves around February 2023.
\end{remark}
\section{Linear types}\label{sec:linear-types}
The type systems of most popular programming languages are fairly weak:
they do not provide direct support
for verifying high-level correctness or complexity properties.
Functional programming languages with strong type systems exist, allowing us to
verify quite useful correctness properties --- one example is the language Haskell.
Going yet
stronger, Lean and Rocq are at the same time programming languages and proof assistants, i.e.\
their type systems allow us to prove even very abstract mathematical properties of the functions they define.
However, none of the mainstream languages allows us to enforce useful computational complexity properties
of programs.

One of the properties that is notoriously difficult to enforce
is preventing data from being copied\footnote{We can delete the copy constructor in C++, but this is of course bypassable.}
or discarded without being used.\footnote{A linter can detect unused variables, but this does not enforce a guarantee.}
We will call variables that are \emph{non-copyable} \emph{affine}, and reserve the term \emph{linear} for variables
that must be used \emph{exactly once}, i.e.\ they may be neither duplicated nor discarded.
The affine requirement can be enforced by making the variable go out of scope just after it was used.
Affine variables are important in logspace computations, where copying large enough data
is simply not implementable. Linear variables, on the other hand, are reminiscent of
the requirement that every class construction must be matched by a corresponding destruction operation
on every execution path.
We will see an example of affinity enforced by linear types in Haskell in~\autoref{lst:haskell-linear-example}.

These ideas have been studied for decades in proof theory.
There, a proof system may or may not allow the so-called \emph{structural} rules:
weakening, contraction and exchange.
Proof systems that lack one of these rules are studied in the field of linear logic,
introduced in~\cite{GIRARD19871} with a clear intent to be used in computer science.
The semantics of linear logic has a natural interpretation in terms of \emph{resources}:
a proposition may be used exactly once, or at most once, rather than freely duplicated and discarded.
The concepts from linear logic have been carried over almost directly to \emph{linear type systems},
which, in (typically functional) programming languages, can control the ability to clone and discard data.

As it turns out, this level of control is also enough to limit the computational complexity of definable functions.
The class \complexity{FL} has been captured by a variant of affine logic in~\cite{4276584};
it was also studied in~\cite{10.1007/11874683_40},~\cite{DaLagoSchopp10} and~\cite{mazza:LIPIcs.CSL.2015.24}.
The class \complexity{FP} was characterized in~\cite{Leivant93}.
Most importantly for us, the programming language \texttt{IntML}
was introduced in~\cite{DALLAGO2016150}, which we will discuss in~\autoref{sec:intml}.

\begin{remark}
The formal bridge between linear logics and linear type systems is the
Curry-Howard correspondence,
also known as ``propositions as types'' or ``proofs as programs''.
The connections between logic, type systems and complexity theory are already well explored in the literature,
e.g.\ \cite{BenedettiPhd}.
We will not repeat these definitions here.
A good introduction to the Curry-Howard correspondence is the book by S\o{}rensen and Urzyczyn~\cite{10.5555/1197021}.
\end{remark}

\begin{remark}
Controlling the computational complexity of programs through the complexity of their specification
was already discussed in~\autoref{chap:descriptive-complexity}, where the control went through model theory.
Here, the restrictions are entirely in the world of proof theory, which is more directly connected to
computation than model theory.
\end{remark}

\begin{remark}[Support for linear types in mainstream programming languages]
Some mainstream programming languages offer some support for linear  types.
Haskell (GHC~$\ge 9.0.1$) supports a limited version of linear types.
In Rust, affine reasoning can be expressed through the ownership and borrowing mechanism.
Going less mainstream, Idris~2, F$^\star$ and Q$^\star$ (a quantum programming language)
also support different variants of linear types.
\end{remark}

\begin{rawlisting}
\begin{Verbatim}[commandchars=\\\{\}]
\PY{c+cm}{\PYZob{}\PYZhy{}}\PY{c+cm}{\PYZsh{} LANGUAGE LinearTypes \PYZsh{}}\PY{c+cm}{\PYZhy{}\PYZcb{}}\PY{+w}{ }\PY{c+c1}{\PYZhy{}\PYZhy{} compilation: `ghc Linear.hs`, ghc \PYZgt{}= 9.0.1}
\PY{k+kr}{module}\PY{+w}{ }\PY{n+nn}{Linear}\PY{+w}{ }\PY{k+kr}{where}
\PY{k+kr}{import}\PY{+w}{ }\PY{n+nn}{Prelude}

\PY{c+c1}{\PYZhy{}\PYZhy{} Define own bitstring type for ints, as operations from Prelude}
\PY{c+c1}{\PYZhy{}\PYZhy{} on Int are *not* linear and will not typecheck.}
\PY{k+kr}{data}\PY{+w}{ }\PY{k+kt}{Bit}\PY{+w}{  }\PY{o+ow}{=}\PY{+w}{ }\PY{k+kt}{Zero}\PY{+w}{ }\PY{o}{|}\PY{+w}{ }\PY{k+kt}{One}
\PY{k+kr}{data}\PY{+w}{ }\PY{k+kt}{Bits}\PY{+w}{ }\PY{o+ow}{=}\PY{+w}{ }\PY{k+kt}{Nil}\PY{+w}{ }\PY{o}{|}\PY{+w}{ }\PY{k+kt}{B0}\PY{+w}{ }\PY{k+kt}{Bits}\PY{+w}{ }\PY{o}{|}\PY{+w}{ }\PY{k+kt}{B1}\PY{+w}{ }\PY{k+kt}{Bits}\PY{+w}{  }\PY{c+c1}{\PYZhy{}\PYZhy{} Prepend 0 or 1 as the LSB.}
\PY{n+nf}{const\PYZus{}5}\PY{+w}{ }\PY{o+ow}{::}\PY{+w}{ }\PY{k+kt}{Bits}\PY{+w}{ }\PY{o+ow}{=}\PY{+w}{ }\PY{k+kt}{B1}\PY{+w}{ }\PY{p}{(}\PY{k+kt}{B0}\PY{+w}{ }\PY{p}{(}\PY{k+kt}{B1}\PY{+w}{ }\PY{k+kt}{Nil}\PY{p}{)}\PY{p}{)}
\PY{n+nf}{const\PYZus{}6}\PY{+w}{ }\PY{o+ow}{::}\PY{+w}{ }\PY{k+kt}{Bits}\PY{+w}{ }\PY{o+ow}{=}\PY{+w}{ }\PY{k+kt}{B0}\PY{+w}{ }\PY{p}{(}\PY{k+kt}{B1}\PY{+w}{ }\PY{p}{(}\PY{k+kt}{B1}\PY{+w}{ }\PY{k+kt}{Nil}\PY{p}{)}\PY{p}{)}

\PY{n+nf}{burn}\PY{+w}{ }\PY{o+ow}{::}\PY{+w}{ }\PY{k+kt}{Bits}\PY{+w}{ }\PY{o}{\PYZpc{}}\PY{l+m+mi}{1}\PY{o+ow}{\PYZhy{}\PYZgt{}}\PY{+w}{ }\PY{n+nb}{()}
\PY{n+nf}{burn}\PY{+w}{ }\PY{k+kt}{Nil}\PY{+w}{      }\PY{o+ow}{=}\PY{+w}{ }\PY{n+nb}{()}
\PY{n+nf}{burn}\PY{+w}{ }\PY{p}{(}\PY{k+kt}{B0}\PY{+w}{ }\PY{n}{xs}\PY{p}{)}\PY{+w}{  }\PY{o+ow}{=}\PY{+w}{ }\PY{n}{burn}\PY{+w}{ }\PY{n}{xs}
\PY{n+nf}{burn}\PY{+w}{ }\PY{p}{(}\PY{k+kt}{B1}\PY{+w}{ }\PY{n}{xs}\PY{p}{)}\PY{+w}{  }\PY{o+ow}{=}\PY{+w}{ }\PY{n}{burn}\PY{+w}{ }\PY{n}{xs}

\PY{n+nf}{evenBits}\PY{+w}{ }\PY{o+ow}{::}\PY{+w}{ }\PY{k+kt}{Bits}\PY{+w}{ }\PY{o}{\PYZpc{}}\PY{l+m+mi}{1}\PY{o+ow}{\PYZhy{}\PYZgt{}}\PY{+w}{ }\PY{k+kt}{Prelude}\PY{o}{.}\PY{k+kt}{Bool}
\PY{n+nf}{evenBits}\PY{+w}{ }\PY{k+kt}{Nil}\PY{+w}{      }\PY{o+ow}{=}\PY{+w}{ }\PY{k+kt}{Prelude}\PY{o}{.}\PY{k+kt}{True}
\PY{n+nf}{evenBits}\PY{+w}{ }\PY{p}{(}\PY{k+kt}{B0}\PY{+w}{ }\PY{n}{xs}\PY{p}{)}\PY{+w}{  }\PY{o+ow}{=}\PY{+w}{ }\PY{k+kr}{case}\PY{+w}{ }\PY{n}{burn}\PY{+w}{ }\PY{n}{xs}\PY{+w}{ }\PY{k+kr}{of}\PY{+w}{ }\PY{n+nb}{()}\PY{+w}{ }\PY{o+ow}{\PYZhy{}\PYZgt{}}\PY{+w}{ }\PY{k+kt}{Prelude}\PY{o}{.}\PY{k+kt}{True}
\PY{n+nf}{evenBits}\PY{+w}{ }\PY{p}{(}\PY{k+kt}{B1}\PY{+w}{ }\PY{n}{xs}\PY{p}{)}\PY{+w}{  }\PY{o+ow}{=}\PY{+w}{ }\PY{k+kr}{case}\PY{+w}{ }\PY{n}{burn}\PY{+w}{ }\PY{n}{xs}\PY{+w}{ }\PY{k+kr}{of}\PY{+w}{ }\PY{n+nb}{()}\PY{+w}{ }\PY{o+ow}{\PYZhy{}\PYZgt{}}\PY{+w}{ }\PY{k+kt}{Prelude}\PY{o}{.}\PY{k+kt}{False}

\PY{n+nf}{half}\PY{+w}{ }\PY{o+ow}{::}\PY{+w}{ }\PY{k+kt}{Bits}\PY{+w}{ }\PY{o}{\PYZpc{}}\PY{l+m+mi}{1}\PY{o+ow}{\PYZhy{}\PYZgt{}}\PY{+w}{ }\PY{k+kt}{Bits}
\PY{n+nf}{half}\PY{+w}{ }\PY{k+kt}{Nil}\PY{+w}{      }\PY{o+ow}{=}\PY{+w}{ }\PY{k+kt}{Nil}
\PY{n+nf}{half}\PY{+w}{ }\PY{p}{(}\PY{k+kt}{B0}\PY{+w}{ }\PY{n}{xs}\PY{p}{)}\PY{+w}{  }\PY{o+ow}{=}\PY{+w}{ }\PY{n}{xs}
\PY{n+nf}{half}\PY{+w}{ }\PY{p}{(}\PY{k+kt}{B1}\PY{+w}{ }\PY{n}{xs}\PY{p}{)}\PY{+w}{  }\PY{o+ow}{=}\PY{+w}{ }\PY{n}{xs}

\PY{n+nf}{plus2x1}\PY{+w}{ }\PY{o+ow}{::}\PY{+w}{ }\PY{k+kt}{Bits}\PY{+w}{ }\PY{o}{\PYZpc{}}\PY{l+m+mi}{1}\PY{o+ow}{\PYZhy{}\PYZgt{}}\PY{+w}{ }\PY{k+kt}{Bits}
\PY{n+nf}{plus2x1}\PY{+w}{ }\PY{n}{x}\PY{+w}{ }\PY{o+ow}{=}\PY{+w}{ }\PY{k+kt}{B1}\PY{+w}{ }\PY{n}{x}

\PY{n+nf}{branchConst}\PY{+w}{ }\PY{o+ow}{::}\PY{+w}{ }\PY{k+kt}{Bits}\PY{+w}{ }\PY{o}{\PYZpc{}}\PY{l+m+mi}{1}\PY{o+ow}{\PYZhy{}\PYZgt{}}\PY{+w}{ }\PY{k+kt}{Bits}
\PY{n+nf}{branchConst}\PY{+w}{ }\PY{n}{x}\PY{+w}{ }\PY{o+ow}{=}
\PY{+w}{  }\PY{k+kr}{if}\PY{+w}{ }\PY{n}{evenBits}\PY{+w}{ }\PY{n}{x}
\PY{+w}{    }\PY{k+kr}{then}\PY{+w}{ }\PY{n}{half}\PY{+w}{ }\PY{n}{const\PYZus{}5}
\PY{+w}{    }\PY{k+kr}{else}\PY{+w}{ }\PY{n}{plus2x1}\PY{+w}{ }\PY{n}{const\PYZus{}6}

\PY{c+c1}{\PYZhy{}\PYZhy{} collatzBad2 :: Bits \PYZpc{}1\PYZhy{}\PYZgt{} Bits}
\PY{c+c1}{\PYZhy{}\PYZhy{} collatzBad2 x =}
\PY{c+c1}{\PYZhy{}\PYZhy{}   if evenBits x}
\PY{c+c1}{\PYZhy{}\PYZhy{}     then half x       \PYZhy{}\PYZhy{} ERROR: x already consumed by \PYZsq{}evenBits x\PYZsq{}}
\PY{c+c1}{\PYZhy{}\PYZhy{}     else plus2x1 x    \PYZhy{}\PYZhy{} ERROR: x already consumed by \PYZsq{}evenBits x\PYZsq{}}
\end{Verbatim}

\caption{Example of linear types in Haskell}\label{lst:haskell-linear-example}
\end{rawlisting}
\begin{remark}\label{remark:haskell-linear}
  In~\autoref{lst:haskell-linear-example}, the type system prevents us from using a linear argument
  more than once. The function \texttt{collatzBad2} fails to type-check precisely because the
  argument \texttt{x} would be consumed by \texttt{evenBits x} and then used again in the branches.
This effect makes most of the standard algorithms not transferrable to this formalism.  
\end{remark}

\subsection{IntML}\label{sec:intml}
In 2013, Dal Lago and Sch\"opp introduced \texttt{IntML}, a functional language with a linear type
system that characterizes \complexity{FL}~\cite{DALLAGO2016150}.
An implementation of \texttt{IntML} is available on
GitHub.\footnote{\url{https://github.com/uelis/IntML}. Following private communication with the authors, 
a permissive license was added to the repository, as it was not included originally.}
To the best of our knowledge, it remains the only language within the linear-logic branch of ICC that has both
a working implementation and some potential for (academic) practical use.

From the point of view of this thesis, however, linear-logic-based approaches --- including \texttt{IntML} ---
run into the \emph{same} issue of intensional vs.\ extensional expressive power discussed in~\autoref{subsec:intensional}.
These systems characterize classes such as \complexity{FL} and \complexity{FP} \emph{extensionally}.
The characterizations capture the right functions, but not the usual \emph{algorithmic techniques}
used to implement them. \texttt{IntML} looks like a very good starting point for
a practical programming language. However, it would still be very hard to use it
to certify the complexity of standard algorithms.

In this thesis, we will not pursue this line further as a practical basis for certifying
the complexity.

\chapter{Bounded arithmetic}\label{chap:bounded-arithmetic}

In mathematics we typically assume some (pretty strong) foundational axioms we rely on
to prove theorems. If we choose set theory as the foundation (as we usually do), a debatable
concept is whether we should use the axiom of choice or not. More popularly in computer science,
we often want to be explicit about using Kőnig's
lemma\footnote{note that Kőnig's lemma is a form of countable choice from finite sets.}
and Ramsey's theorem.
If we think of the concept
we introduced in~\autoref{chap:reductions} (i.e.\ writing a program in a language in which calls
of the computation-heavy oracle are explicit), we would often ask ourselves the same question ---
can we write the program without relying on that function, i.e.\ write the program in a lower complexity?
It turns out the similarity is not a coincidence,
and to explore this connection further we need to study the theories of \emph{bounded arithmetic}.

We will start by looking at interesting theorems about one such theory, just after
introducing the necessary definitions.

\section{Single-sorted logic and \IDeltaZero}

\begin{definition}[{\cite[Definition~III.1.1]{Cook_Nguyen_2010}}]
A \emph{theory} over a vocabulary $\mathcal{L}$ is a set $\mathcal{T}$ of $\mathcal{L}$-formulas that is closed
under logical consequence and under universal closure.

Note that we have not defined ``logical consequence'', which refers to a particular \emph{proof system}
(also named \emph{proof calculus}). We will not define proof systems for logic in this work.
For those interested, we will just mention that
all the results discussed in this chapter assume standard Gentzen-style proof calculus for classical logic,
$\mathrm{LK}$~\cite[Section~II.2.3]{Cook_Nguyen_2010} for single-sorted logic
and $\mathrm{LK}^2$~\cite[Section~IV.4]{Cook_Nguyen_2010} for two-sorted logic.
\end{definition}

\begin{definition}[{\cite[Definition~II.2.3]{Cook_Nguyen_2010}}]
The vocabulary of arithmetic is
\[
\mathcal{L}_A = \langle 0,1,+,\cdot\ ;\ =,\le \rangle .
\]
Here $0,1$ are constant symbols; $+$ and $\cdot$ are binary function symbols; and
$=$ and $\le$ are binary predicate symbols. We will implicitly assume the
standard interpretation of these symbols as the appropriate functions on natural numbers
whenever talking about the semantics of $\mathcal{L}_A$-formulas.
\end{definition}

\begin{definition}[{\cite[Figure~1]{Cook_Nguyen_2010}} Axioms 1-BASIC of Peano arithmetic]
    \[
\begin{array}{@{}l l@{}}
\mathrm{B1.}\; x + 1 \neq 0
&
\mathrm{B5.}\; x \cdot 0 = 0
\\[2pt]
\mathrm{B2.}\; x + 1 = y + 1 \rightarrow x = y
&
\mathrm{B6.}\; x \cdot (y + 1) = (x \cdot y) + x
\\[2pt]
\mathrm{B3.}\; x + 0 = x
&
\mathrm{B7.}\; (x \le y \land y \le x) \rightarrow x = y
\\[2pt]
\mathrm{B4.}\; x + (y + 1) = (x + y) + 1
&
\mathrm{B8.}\; x \le x + y
\\[6pt]
\multicolumn{2}{@{}l@{}}{\text{C.}\; 0 + 1 = 1}
\end{array}
\]
\end{definition}

\begin{definition}[{\cite[Definition~III.1.4]{Cook_Nguyen_2010}} Induction Scheme]
Let $\Phi$ be a set of formulas. The \emph{$\Phi$-IND axioms} are all formulas of the form
\begin{equation}\label{eq:Phi-IND}
\bigl(\varphi(0)\ \land\ \forall x\,(\varphi(x)\rightarrow \varphi(x+1))\bigr)
\ \rightarrow\ \forall z\,\varphi(z),
\end{equation}
where $\varphi$ ranges over formulas in $\Phi$.  Note that $\varphi(x)$ may have free
variables other than~$x$.
\end{definition}

\begin{definition}[{\cite[Definition~III.1.5]{Cook_Nguyen_2010}} Peano Arithmetic]
The theory $\arithPA$ has as axioms $\mathrm{B1},\ldots,\mathrm{B8}$, together with the $\Phi$-IND axioms,
where $\Phi$ is the set of all $\mathcal{L}_A$-formulas.

Peano Arithmetic is a powerful theory capable of formalizing the major theorems of
number theory. We define subsystems of $\arithPA$ by restricting the induction axioms
to certain sets of formulas. 
\end{definition}

\begin{definition}[{\cite[Definition~III.1.7]{Cook_Nguyen_2010}} \IOPEN, \IDeltaZero, \ISigmaOne]
Let $\mathrm{OPEN}$ be the set of \emph{open} (i.e.\ quantifier-free) formulas, let $\Delta_{0}$
be the set of \emph{bounded} formulas, and let $\Sigma_{1}$ be the set of formulas
of the form $\exists \vec{x}\,\varphi$, where $\varphi$ is bounded and $\vec{x}$ is a
(possibly empty) tuple of variables.  

The theories $\IOPEN$, $\IDeltaZero$, and $\ISigmaOne$ are the subsystems of $\arithPA$
obtained by restricting the induction scheme so that $\Phi$ is $\mathrm{OPEN}$, $\Delta_{0}$,
and $\Sigma_{1}$, respectively.
\end{definition}

\begin{lemma}[{\cite[Example~III.1.8]{Cook_Nguyen_2010}}]
The following formulas (and their universal closures) are theorems of $\IOPEN$:
\[
\begin{array}{@{}l l@{}}
\mathrm{O1.}\; (x+y)+z = x+(y+z)
& \text{(Associativity of $+$)} \\[2pt]
\mathrm{O2.}\; x+y = y+x
& \text{(Commutativity of $+$)} \\[2pt]
\mathrm{O3.}\; x\cdot (y+z) = (x\cdot y)+(x\cdot z)
& \text{(Distributive law)} \\[2pt]
\mathrm{O4.}\; (x\cdot y)\cdot z = x\cdot (y\cdot z)
& \text{(Associativity of $\cdot$)} \\[2pt]
\mathrm{O5.}\; x\cdot y = y\cdot x
& \text{(Commutativity of $\cdot$)} \\[2pt]
\mathrm{O6.}\; x+z = y+z \rightarrow x=y
& \text{(Cancellation for $+$)} \\[2pt]
\mathrm{O7.}\; 0 \le x
& \\[2pt]
\mathrm{O8.}\; x \le 0 \rightarrow x=0
& \\[2pt]
\mathrm{O9.}\; x \le x
& \\[2pt]
\mathrm{O10.}\; x \neq x+1
&
\end{array}
\]
\end{lemma}

\begin{lemma}[{\cite[Example~III.1.9]{Cook_Nguyen_2010}}]
The following formulas (and their universal closures) are theorems of $\IDeltaZero$:
\[
\begin{array}{@{}l l@{}}
\mathrm{D1.}\; x\neq 0 \rightarrow \exists y\le x\,(x=y+1)
& \text{(Predecessor)} \\[2pt]
\mathrm{D2.}\; \exists z\,\bigl(x+z=y \,\lor\, y+z=x\bigr)
& \\[2pt]
\mathrm{D3.}\; x\le y \leftrightarrow \exists z\,(x+z=y)
& \\[2pt]
\mathrm{D4.}\; (x\le y \land y\le z) \rightarrow x\le z
& \text{(Transitivity)} \\[2pt]
\mathrm{D5.}\; x\le y \lor y\le x
& \text{(Total order)} \\[2pt]
\mathrm{D6.}\; x\le y \leftrightarrow x+z \le y+z
& \\[2pt]
\mathrm{D7.}\; x\le y \rightarrow x\cdot z \le y\cdot z
& \\[2pt]
\mathrm{D8.}\; x\le y+1 \leftrightarrow (x\le y \lor x=y+1)
& \text{(Discreteness 1)} \\[2pt]
\mathrm{D9.}\; x<y \leftrightarrow x+1\le y
& \text{(Discreteness 2)} \\[2pt]
\mathrm{D10.}\; x\cdot z = y\cdot z \land z\neq 0 \rightarrow x=y
& \text{(Cancellation for $\cdot$)}
\end{array}
\]
\end{lemma}

Using the above lemmas as building blocks, we can prove quite a few nontrivial theorems.
We will now introduce the core notion of arithmetic --- what does it mean to \emph{define} a function
\emph{in a theory}.


\begin{definition}[{\cite[Definition~III.3.2]{Cook_Nguyen_2010}} Predicates and Functions definable in a Theory]
    Let $\mathcal{T}$ be a theory with vocabulary $\mathcal{L}$, and let $\Phi$ be a set of $\mathcal{L}$-formulas.

\begin{enumerate}
    \item
    A predicate symbol $P(x)\notin \mathcal{L}$ is \emph{$\Phi$-definable in $\mathcal{T}$} if there exists
    an $\mathcal{L}$-formula $\varphi(x)\in\Phi$ such that
    \begin{equation}\label{eq:defining-predicate}
    P(x)\;\leftrightarrow\;\varphi(x).
    \end{equation}

    \item
    A function symbol $f(x)\notin \mathcal{L}$ is \emph{$\Phi$-definable in $\mathcal{T}$} if there exists
    a formula $\varphi(x,y)\in\Phi$ such that
    \begin{equation}\label{eq:unique-existence}
    \mathcal{T} \vdash \forall x\,\exists! y\,\varphi(x,y),
    \end{equation}
    and moreover
    \begin{equation}\label{eq:defining-function}
    y=f(x)\;\leftrightarrow\;\varphi(x,y).
    \end{equation}
\end{enumerate}

    We call \eqref{eq:defining-predicate} a \emph{defining axiom} for $P(x)$ and
    \eqref{eq:defining-function} a \emph{defining axiom} for $f(x)$.
    A symbol is \emph{definable in $\mathcal{T}$} if it is $\Phi$-definable in $\mathcal{T}$ for some $\Phi$.
\end{definition}

\begin{definition}
    We will say that a function is \emph{provably total} in $\mathcal{T}$ iff it
    is $\Sigma_1$-definable in $\mathcal{T}$.
\end{definition}


In~\cite[Section~III.3]{Cook_Nguyen_2010} it is argued that: functions $\lfloor x/y \rfloor,\;\big \lfloor \sqrt{x} \big \rfloor,
\;\max(0, x - y),\;x\!\!\mod y$ are definable in \IDeltaZero; relation $x \mid y$ is definable in \IDeltaZero{},
and, interestingly, the relation $\;\exp(x, y)$ where $\exp(x, y)$ iff $y = 2^x$, is also definable in \IDeltaZero{}.
We don't introduce the specific logical formula defining the relation $\exp(x, y)$, as it is complicated and discussed
in~\cite[Section~III.3]{Cook_Nguyen_2010}.
For a different point of view to these problems, e.g.\ in~\cite{Jumelet1995} it is shown
that Euler's $\varphi$ function is provably total in \IDeltaZero{}.
However, the limits of expressive power of \IDeltaZero{} are low.

\begin{theorem}[{\cite[Section~III.2]{Cook_Nguyen_2010}}]
\[
\IDeltaZero \nvdash \forall x\,\exists y\,\exp(x,y).
\]
Note that $\arithPA$ easily proves $\forall  x\,\exists y\,\exp(x,y)$.
\end{theorem}

It is interesting to study the theory $\IDeltaZero + \exp$ of $\IDeltaZero$ axioms with an additional axiom
stating that the exponential function is definable. As it turns out, this theory enables us to
reason about syntactic constructs such as coding of sets and sequences or context-free
grammar parsing~\cite[Chapter~V,~Section~3]{HajekPudlak1993Metamathematics}
\footnote{note that they use the name $\mathrm{I}\Sigma_0 + \Omega_1$ instead of $\IDeltaZero + \exp$ which is the same.}.

% In~\cite{Buchholz1987ProvablyCF} is is shown that a function to not be provably total in Peano
% arithmetic requires it to be growing too fast. An intuition behind it for the sake of our
% thesis is that functions that are difficult to \emph{prove correct}, but grow slowly (in particular,
% solve decisional problems and only output a boolean value), must have graphs that are not
% easily definable by a logical sentence.

It turns out that a function is $\Sigma_1$-definable in \IDeltaZero{} iff it is in \complexity{FLTH}, functional
version of linear-time hierarchy~\cite[Theorem~III.4.8]{Cook_Nguyen_2010}; for the definition of \complexity{LTH},
refer to~\cite[Section~III.4.1]{Cook_Nguyen_2010} --- as this complexity class is far from what we call ``feasible''
in this work, we don't introduce the details here. Instead, we will now introduce a theory with a good
computational complexity characterization.

\section{Two-sorted logic and \compVZero}\label{sec:theory-v0}

\begin{definition}[Axioms of 2-BASIC]
\[
\begin{array}{@{}l l@{}}
\mathrm{B1.}\; x + 1 \neq 0
&
\mathrm{B7.}\; (x \le y \land y \le x) \rightarrow x = y
\\[2pt]
\mathrm{B2.}\; x + 1 = y + 1 \rightarrow x = y
&
\mathrm{B8.}\; x \le x + y
\\[2pt]
\mathrm{B3.}\; x + 0 = x
&
\mathrm{B9.}\; 0 \le x
\\[2pt]
\mathrm{B4.}\; x + (y + 1) = (x + y) + 1
&
\mathrm{B10.}\; x \le y \lor y \le x
\\[2pt]
\mathrm{B5.}\; x \cdot 0 = 0
&
\mathrm{B11.}\; x \le y \leftrightarrow x < y + 1
\\[2pt]
\mathrm{B6.}\; x \cdot (y + 1) = (x \cdot y) + x
&
\mathrm{B12.}\; x \neq 0 \rightarrow \exists y \le x\, (y + 1 = x)
\\[6pt]
\text{L1.}\; X(y) \rightarrow y < \len{X}
&
\text{L2.}\; y + 1 = \len{X} \rightarrow X(y)
\\[6pt]
\multicolumn{2}{@{}l@{}}{
\text{SE.}\;
\bigl(\len{X} = \len{Y} \land \forall i < \len{X}\, (X(i) \leftrightarrow Y(i))\bigr)
\rightarrow X = Y
}
\end{array}
\]

\end{definition}


\begin{definition}[Comprehension Axiom]\label{def:V.1.2}
Let $\Phi$ be a set of formulas. The \emph{comprehension axiom scheme for $\Phi$},
denoted $\Phi\text{-}\mathrm{COMP}$, consists of all formulas of the form
\begin{equation}\label{eq:Phi-COMP}
\exists X \le y\;\forall z<y\;\bigl(X(z)\leftrightarrow \varphi(z)\bigr),
\end{equation}
where $\varphi(z)\in\Phi$ and $X$ does not occur free in $\varphi(z)$.
In \eqref{eq:Phi-COMP}, the formula $\varphi(z)$ may have free variables of both
sorts in addition to~$z$.  We are mainly interested in the cases where
$\Phi$ is one of the classes $\Sigma^{B}_{i}$.
\end{definition}

% \begin{notation}\label{not:V.1.2}
% Since \eqref{eq:Phi-COMP} asserts the existence of a finite set $X$ of numbers,
% we will sometimes use standard set-theoretic notation to describe~$X$:
% \begin{equation}\label{eq:set-notation}
% X=\{\,z : z<y \land \varphi(z)\,\}.
% \end{equation}
% \end{notation}

\begin{definition}[$V^{i}$]\label{def:arith-vi}
For $i\ge 0$, the theory $V^{i}$ has vocabulary $\mathcal{L}^{2}_{A}$ and is axiomatized by
2-BASIC together with $\Sigma^{B}_{i}\text{-}\mathrm{COMP}$.

Note that there are no explicit induction axioms for $\mathrm{V}^{i}$.
\end{definition}

\begin{theorem}[{\cite[Corollary~V.1.8]{Cook_Nguyen_2010}}]
    Induction is provable in $\mathrm{V}^{i}$. Induction for $\Delta_0$ formulas
    is a theorem of \compVZero.

    Note that this implies that any theorem $\varphi$ provable in \IDeltaZero{} is also
    provable in \compVZero.
\end{theorem}

\begin{theorem}[{\cite[Theorem~V.1.9]{Cook_Nguyen_2010}}]
    For every formula $\varphi$ in the vocabulary $\mathcal{L}_A$ of single-sorted arithmetic,
    if $\compVZero \vdash \varphi$, then also $\IDeltaZero \vdash \varphi$.
    In other words, \compVZero{} is a \emph{conservative extension} of \IDeltaZero.
\end{theorem}

\begin{remark}[{\cite[Section~IV.3]{Cook_Nguyen_2010}} Two-sorted complexity classes]
    When operating in two-sorted logic, we need to redefine what does it mean for a relation to be in a complexity
    class. We will think of numerical arguments $x_i$ of a relation $R(\vec{x}, \vec{X})$ to be passed to
    the deciding Turing machine in unary representation. The string arguments $X_i$ representing finite sets
    of numbers are passed as follows. For a string argument $S$ define $S(i) = 1$ when $i \in S$, 0 otherwise.
    Then the representation $\squarequotes[4]{S}$ of $S$, when the largest member of $S$ is $n$, is defined as
    the following concatenation of bits:
    \[\squarequotes[4]{S} = S(n)S(n - 1) \dots S(1)S(0)\]
    If $S$ is empty then $\squarequotes[4]{S}$ is the empty string.
    Note that $\len{\squarequotes[4]{S}}$ is the same as our
    interpretation of $S$ inside of the theory: $\len{S} = \max(S) + 1$ or $0$ if $S$ empty.
    
    We will write $\unary{x}$ to denote unary representation of $x$, i.e.\ $1^{\len{x}}$
    and $\binary{x}$ to denote binary representation.
    The ultimate input to the Turing machine deciding if $R(\vec{x}, \vec{X})$
    for $\len{\vec{x}}= n, \len{\vec{X}} = N, \len{X_i}=N_i$ is:

    \[
    \unary{n}\;0 \quad \unary{x_1}\;0\;\unary{x_2}\;0\;\dots\;0\;\unary{x_n}\;0\quad\unary{N}
    \; 0 \quad
    \unary{N_1}\;0\;\squarequotes[5][7]{X_1}\;0\;\dots\;0\;\unary{N_N}\;0\;\squarequotes[5][7]{X_N}
    \]


    Note that a purely numerical relation $R(x)$ is in two-sorted polynomial time iff it is computed
    in time $2^{\bigO(n)}$ for $n = \len{\binary{x}}$. The notion of polynomial-time complexity for
    relations with only string arguments $R(\vec{X})$ coincides with our standard intuition.
\end{remark}

\begin{definition}[{\cite[Definition~V.2.1]{Cook_Nguyen_2010}}]
A number function $f$ or string function $F$ is
(\emph{$p$-bounded}) iff there exists a polynomial $p(x,y)$ such that, for all
inputs $x,Y$,
\[
f(x,Y)\ \le\ p\bigl(x,\lvert Y\rvert\bigr)
\qquad\text{or}\qquad
\lvert F(x,Y)\rvert\ \le\ p\bigl(x,\lvert Y\rvert\bigr),
\]
respectively.
\end{definition}


\begin{definition}[{\cite[Definition~V.2.3]{Cook_Nguyen_2010}} Two-sorted functional complexity classes]
Let $\complexity{C}$ be a two-sorted complexity class of relations.  
The corresponding \emph{function class} $\complexity{FC}$ consists of:
\begin{enumerate}
\item all $p$-bounded number functions whose graphs belong to $\complexity{C}$, and
\item all $p$-bounded string functions whose bit graphs belong to $\complexity{C}$.
\end{enumerate}

Note that the classes \complexityi{FAC}{0}, \complexity{FP}, \complexity{FL} are defined in a different way
to what we have used earlier. However, the difference will not matter in this work.
\end{definition}


We don't repeat the definitions of definability in a theory
for the two-sorted case~\cite[Definition~V.4.1]{Cook_Nguyen_2010}.
Recall the definition of $\Sigma_0^B$ formulas~(\autoref{def:SigmaB-PiB-hierarchy}).

\begin{theorem}[{\cite[Corollary~V.5.3]{Cook_Nguyen_2010}}]
    A function is in \FACZero{} iff it is $\Sigma_0^B$-definable in \compVZero.
\end{theorem}

\begin{definition}
    The theory \complexity{VC} for a complexity class $\complexity{C}$ has vocabulary $\mathcal{L}^2_A$
    and is axiomatized
    by the axioms of \compVZero and one additional axiom depending on the choice of the class $\complexity{C}$.
    The additional axiom can be thought of adding an oracle for a $C$-complete problem to \compVZero.
    We skip the (lengthy) technicalities of~\cite[Definition~IX.2.1]{Cook_Nguyen_2010}.
\end{definition}

The below theorem is the central result of our interest in this thesis.

\begin{theorem}[{\cite[Theorem~IX.2.3]{Cook_Nguyen_2010}}]
    A function is provably total in \complexity{VC} iff it is in \complexity{FC}.
\end{theorem}

By adding a single axiom to the theory of \compVZero, we can obtain arithmetical hierarchies
in which the functions that we can define and prove correct are precisely the functions
from a given complexity class $\complexity{C} = \complexityi{FTC}{0}, \complexityi{FNC}{1}, \complexity{FL}, \complexity{FP}$.

This way, we obtain theories with very nice properties. They foster certification of complexity
of an algorithm (if the proof of
correctness itself is feasible, see~\autoref{subsec:complexity-alg-proof}).
At the same time, they enable us to prove theorems about the correctness of functions defined.
In~\cite{buss2025logspaceconstructiveprooflsl}, the authors formalize the breakthrough
result \(\complexity{L}=\complexity{SL}\) of~\cite{10.1145/1391289.1391291} inside of the weak
theory of bounded arithmetic~\complexity{VL}.
The complexity of computational content of proofs of the Discrete Jordan Curve Theorem is
examined in~\cite{10.1145/2071368.2071377}.
Expander construction in \complexityi{VNC}{1} was conducted in~\cite{BUSS2020102796}.

Another elegant property of these theories is that the proof of a problem
not being solvable in a given complexity is exactly a proof of independence
of the axiom (corresponding to the problem) from the theory (corresponding to the complexity class).

\begin{theorem}[{\cites[Corollary~7.21]{CookNguyenDraft}[Corollary~VII.2.4]{Cook_Nguyen_2010}}~Independence of \problem{PHP} from \complexityi{VAC}{0}]\label{subsec:vac0-php}
    \[\complexityi{VAC}{0} \nvdash \problem{PHP}\]
\end{theorem}

\subsection{Complexity of algorithm vs complexity of proof}\label{subsec:complexity-alg-proof}
Even when an algorithm is simple, it seems to not always be trivial to ``feasibly'' prove
that it computes the correct result. In our setting, this results in knowing that
a problem can be solved in a complexity class $\complexity{C}$, but not knowing if the corresponding
function can be defined in the theory $\complexity{VC}$ (i.e.\ proved total and correct).
See \cites[Section~IX.7.3]{Cook_Nguyen_2010}[Section~9G.3]{CookNguyenDraft} for
an open problem whether the breakthrough result that binary integer division
is in \complexity{DLOGTIME}-uniform \complexityi{TC}{0}~\cite{HESSE2002695},
means that it can also be proved in the corresponding \complexityi{VTC}{0} theory.
Note that this problem apparently has been solved (affirmatively) in~\cite{Jerabek2022}.


% \begin{definition}[V.4.12] (semantic)
% A string function is said to be \(\Sigma^B_0\)\textit{-definable from a collection} \(L\) of
% two-sorted functions and relations if it is \(p\)-bounded and its bit graph is represented by
% a \(\Sigma^B_0(L)\) formula.  
% Similarly, a number function is \(\Sigma^B_0\)\textit{-definable from} \(L\) if it is \(p\)-bounded
% and its graph is represented by a \(\Sigma^B_0(L)\) formula.
% \medskip
% This \emph{semantic} notion of \(\Sigma^B_0\)-definability should not be confused with \(\Sigma^B_0\)-definability \emph{in a theory} (Definition~V.4.1), which involves provability.  
% The next result connects these two notions.
% \end{definition}

% there are functions whose graphs are in \complexityi{AC}{0} (representable by sigma0b formulas),
% but which do not belong to \complexityi{FAC}{0} (section: proof of witnessing theorem for v0)


\begin{remark}[Bibliography]
The field of bounded arithmetic was initiated by Samuel Buss in his PhD thesis:~\cite{Buss1986}, in which
the theories $\mathrm{S}^1_2$ were introduced to capture reasoning about the polynomial-time hierarchy \complexity{PH}
(not introduced in our thesis).
The first theory designed to capture polynomial time reasoning was the
equational theory $\mathrm{PV}$ (as in: polynomially-verifiable [proofs])
theory from~\cite{10.1145/800116.803756}.
The two-sorted logic language for capturing complexity classes has been introduced by Zambella in~\cite{00d3b11b-ff1c-386f-a929-6943478c4a28}.
Despite the theories being designed to reason about computation, they are theories of classical logic,
which might come off as worrying given our considerations from \todo[inline]{sec: chapter icc, section intuit. logic}.
Intuitionistic counterparts such as $\mathrm{IS}^1_2$ for $\mathrm{S}^1_2$ and $\mathrm{IPV}$ for $\mathrm{PV}$
have also been studied. However, much less is known about their expressive power.
For the relation of $\mathrm{IS}^1_2$ and $\mathrm{S}^1_2$, please see~\cite{10.1007/3-540-16486-3_91}. In
particular,~\cite[Conjecture~3]{10.1007/3-540-16486-3_91} asks: if $\mathrm{IS}^1_2 \vdash \exists y \ldotp \phi(y, c)$,
then is it true that there is a function $f$, provably correct in $\mathrm{S}^1_2$, such that $f$ \emph{computes}
the Gödel encoding of that $\mathrm{IS}^1_2$ proof? In~\cite[Corollary~8.19]{COOK1993103}, that conjecture
is answered affirmatively. The intuitionistic version $\mathrm{IPV}$ of the theory $\mathrm{PV}$ is discussed
in some detail in~\cite{COOK1993103}.

For a good introduction to \emph{bounded reverse mathematics},
with a very thorough overview of arithmetical theories corresponding to complexity classes below \compFP,
refer to~\cite{Ngu08}.
\end{remark}




\section{Programming language}
\todo[inline]{In progress}
Now, we want to formalize these arithmetical theories so that a computer can check our programs and ensure we didn't
go out of a given complexity at any point.
With such a programming language (and a formalization of arithmetic in general),
we will be able to readily transfer a huge amount of results from paper to computer.

We have two goals for formalization:
\begin{enumerate}
    \item for logicians to believe us the formalization is sound
    \item to be able to extract code with certified complexity from proofs
\end{enumerate}

There is very little work available on the formalization of arithmetic.
A formalization of consistency of Peano arithmetic in Coq was presented in~\cite{O_Connor_2005}.
A formalization of the so-called \emph{Hydra battles} related to unprovability in Peano arithmetic
was shown in~\cite{casteran:hal-03404668}. There is an impressive ongoing project of formalization
of bounded arithmetic in the model-theoretical style in the Lean
community.\footnote{\url{https://github.com/FormalizedFormalLogic/Foundation}} Their approach doesn't
align with our goal of certifying complexity, as their focus is on other arithmetical theories, which differ
significantly from what we need. Somehow related, some work on intuitionistic logic in Lean has also been done
in~\cite{Trufa__2024} even though Lean is not a natural environment for intuitionistic thinking, as it
assumes classical axioms very deeply in its standard libraries, unlike Rocq which is constructive by heart.


An idea for a programming language based on bounded arithmetic was discussed
in~\cite{Li2025FeasibleMathematics}. The language they discuss is $\mathrm{IMP}~(\complexity{PV})$,
based on the equational theory $\complexity{PV}$ which is different from (and less interesting than)
the theories we have discussed. There, the authors show how to design an imperative programming language
with Hoare logic as the verification mechanism (a.k.a. a type system). Note that for their concept
to be implementable in practice, a \emph{formalization} of $\complexity{PV}$ is necessary.

\todo[inline]{Say how i use curry-howard correspondence to extract code from proofs as introduced in chapter linear logic}


% design:
% deep embedding of proof theory vs going with model theory: rocq internship
% proof relevance/irrelevance vs "extension of a theory" / actually defining new functions in theory
% for formalizing V0: go with single-sorted interpretation, or modify Mathlib
% for formalizing anything: separate Ex and All vs set Ex phi = !All x !phi

% This was presented at AITP2025. And was the topic of
% my visit of Yannick Forster at INRIA. My source code is here:~\url{https://github.com/ruplet/formalization-of-bounded-arithmetic}.
% The presentations PDFs are also there, and reviews of my abstract from aitp.

% deep vs shallow embeddings:
% https://people.cs.nott.ac.uk/psztxa/publ/tt-in-tt.pdf
% https://research-information.bris.ac.uk/ws/portalfiles/portal/330955816/LIPIcs_ITP_2022_28.pdf
% https://drops.dagstuhl.de/storage/00lipics/lipics-vol237-itp2022/LIPIcs.ITP.2022.28/LIPIcs.ITP.2022.28.pdf
% https://cstheory.stackexchange.com/questions/1370/shallow-versus-deep-embeddings
% My work on formalizing bounded arithmetic is here: [https://github.com/ruplet/formalization-of-bounded-arithmetic](https://github.com/ruplet/formalization-of-bounded-arithmetic) - in this repo there is also my presentation from AITP, the abstract and the reviews it received.
% This is also the subject of my visit at INRIA, beginning 8th September 2025.





\begin{rawlisting}
\begin{Verbatim}[commandchars=\\\{\}]
\PY{c+c1}{\PYZhy{}\PYZhy{} D1. x ≠ 0 → ∃ y ≤ x, x = y + 1  (Predecessor)}
\PY{c+c1}{\PYZhy{}\PYZhy{} proof: induction on x}
\PY{k+kn}{theorem}\PY{+w}{ }\PY{n}{pred\PYZus{}exists}\PY{+w}{ }\PY{o}{:}
\PY{+w}{  }\PY{n+nb+bp}{∀}\PY{+w}{ }\PY{o}{\PYZob{}}\PY{n}{x}\PY{+w}{ }\PY{o}{:}\PY{+w}{ }\PY{n}{M}\PY{o}{\PYZcb{}}\PY{o}{,}\PY{+w}{ }\PY{n}{x}\PY{+w}{ }\PY{n+nb+bp}{≠}\PY{+w}{ }\PY{l+m+mi}{0}\PY{+w}{ }\PY{n+nb+bp}{→}\PY{+w}{ }\PY{n+nb+bp}{∃}\PY{+w}{ }\PY{n}{y}\PY{+w}{ }\PY{n+nb+bp}{≤}\PY{+w}{ }\PY{n}{x}\PY{o}{,}\PY{+w}{ }\PY{n}{x}\PY{+w}{ }\PY{n+nb+bp}{=}\PY{+w}{ }\PY{n}{y}\PY{+w}{ }\PY{n+nb+bp}{+}\PY{+w}{ }\PY{l+m+mi}{1}\PY{+w}{ }\PY{o}{:=}
\PY{k}{by}
\PY{+w}{  }\PY{k}{let}\PY{+w}{ }\PY{n}{ind1}\PY{+w}{ }\PY{o}{:}\PY{+w}{ }\PY{n}{peano}\PY{n+nb+bp}{.}\PY{n}{Formula}\PY{+w}{ }\PY{o}{(}\PY{n}{Vars2}\PY{+w}{ }\PY{n+nb+bp}{.}\PY{n}{y}\PY{+w}{ }\PY{n+nb+bp}{.}\PY{n}{x}\PY{o}{)}\PY{+w}{ }\PY{o}{:=}\PY{+w}{ }\PY{n}{x}\PY{+w}{ }\PY{n+nb+bp}{=}\PY{n+nb+bp}{\PYZsq{}}\PY{+w}{ }\PY{o}{(}\PY{n}{y}\PY{+w}{ }\PY{n+nb+bp}{+}\PY{+w}{ }\PY{l+m+mi}{1}\PY{o}{)}
\PY{+w}{  }\PY{k}{let}\PY{+w}{ }\PY{n}{ind2}\PY{+w}{ }\PY{o}{:}\PY{+w}{ }\PY{n}{peano}\PY{n+nb+bp}{.}\PY{n}{Formula}\PY{+w}{ }\PY{o}{(}\PY{n}{Vars1}\PY{+w}{ }\PY{n+nb+bp}{.}\PY{n}{x}\PY{o}{)}\PY{+w}{ }\PY{o}{:=}
\PY{+w}{    }\PY{o}{(}\PY{n}{Formula}\PY{n+nb+bp}{.}\PY{n}{iBdEx\PYZsq{}}\PY{+w}{ }\PY{n}{x}\PY{+w}{ }\PY{o}{(}\PY{n}{display2}\PY{+w}{ }\PY{n+nb+bp}{.}\PY{n}{y}\PY{+w}{ }\PY{n}{ind1}\PY{o}{)}\PY{n+nb+bp}{.}\PY{n}{flip}\PY{o}{)}
\PY{+w}{  }\PY{k}{let}\PY{+w}{ }\PY{n}{ind}\PY{+w}{ }\PY{o}{:=}\PY{+w}{ }\PY{n}{idelta0}\PY{n+nb+bp}{.}\PY{n}{delta0\PYZus{}induction}\PY{+w}{ }\PY{n+nb+bp}{\PYZdl{}}\PY{+w}{ }\PY{n}{display1}\PY{+w}{ }\PY{n+nb+bp}{\PYZdl{}}\PY{+w}{ }\PY{o}{(}\PY{n}{x}\PY{+w}{ }\PY{n+nb+bp}{≠}\PY{n+nb+bp}{\PYZsq{}}\PY{+w}{ }\PY{l+m+mi}{0}\PY{o}{)}\PY{+w}{ }\PY{n+nb+bp}{⟹}\PY{+w}{ }\PY{n}{ind2}

\PY{+w}{  }\PY{n}{unfold}\PY{+w}{ }\PY{n}{ind2}\PY{+w}{ }\PY{n}{ind1}\PY{+w}{ }\PY{k}{at}\PY{+w}{ }\PY{n}{ind}

\PY{+w}{  }\PY{n}{specialize}\PY{+w}{ }\PY{n}{ind}\PY{+w}{ }\PY{o}{(}\PY{k}{by}
\PY{+w}{    }\PY{n}{rw}\PY{+w}{ }\PY{o}{[}\PY{n}{IsDelta0}\PY{n+nb+bp}{.}\PY{n}{display1}\PY{o}{]}
\PY{+w}{    }\PY{c+c1}{\PYZhy{}\PYZhy{} TODO: this lemma can\PYZsq{}t be in @[delta0\PYZus{}simps],}
\PY{+w}{    }\PY{c+c1}{\PYZhy{}\PYZhy{} as it creates a goal \PYZsq{}φ.IsOpen\PYZsq{} \PYZhy{} which might be not true!}
\PY{+w}{    }\PY{n}{rw}\PY{+w}{ }\PY{o}{[}\PY{n}{IsDelta0}\PY{n+nb+bp}{.}\PY{n}{of\PYZus{}open}\PY{n+nb+bp}{.}\PY{n}{imp}\PY{o}{]}
\PY{+w}{    }\PY{n+nb+bp}{·}\PY{+w}{ }\PY{n}{constructor}
\PY{+w}{      }\PY{n+nb+bp}{·}\PY{+w}{ }\PY{n}{unfold}\PY{+w}{ }\PY{n}{Term}\PY{n+nb+bp}{.}\PY{n}{neq}
\PY{+w}{        }\PY{n}{rw}\PY{+w}{ }\PY{o}{[}\PY{n}{IsDelta0}\PY{n+nb+bp}{.}\PY{n}{of\PYZus{}open}\PY{n+nb+bp}{.}\PY{n}{not}\PY{o}{]}
\PY{+w}{        }\PY{n}{constructor}\PY{n+nb+bp}{;}\PY{+w}{ }\PY{n}{constructor}\PY{n+nb+bp}{;}\PY{+w}{ }\PY{n}{constructor}
\PY{+w}{        }\PY{n}{constructor}\PY{n+nb+bp}{;}\PY{+w}{ }\PY{n}{constructor}
\PY{+w}{      }\PY{n+nb+bp}{·}\PY{+w}{ }\PY{n}{constructor}
\PY{+w}{    }\PY{n+nb+bp}{·}\PY{+w}{ }\PY{n}{unfold}\PY{+w}{ }\PY{n}{Term}\PY{n+nb+bp}{.}\PY{n}{neq}
\PY{+w}{      }\PY{n}{rw}\PY{+w}{ }\PY{o}{[}\PY{n}{IsOpen}\PY{n+nb+bp}{.}\PY{n}{not}\PY{o}{]}
\PY{+w}{      }\PY{n}{constructor}\PY{n+nb+bp}{;}\PY{+w}{ }\PY{n}{constructor}
\PY{+w}{  }\PY{o}{)}
\PY{+w}{  }\PY{n}{simp\PYZus{}induction}\PY{+w}{ }\PY{k}{at}\PY{+w}{ }\PY{n}{ind}

\PY{+w}{  }\PY{n}{apply}\PY{+w}{ }\PY{n}{ind}\PY{+w}{ }\PY{n+nb+bp}{?}\PY{n}{base}\PY{+w}{ }\PY{n+nb+bp}{?}\PY{n}{step}\PY{+w}{ }\PY{n+nb+bp}{\PYZlt{}}\PY{n+nb+bp}{;}\PY{n+nb+bp}{\PYZgt{}}\PY{+w}{ }\PY{n}{clear}\PY{+w}{ }\PY{n}{ind}\PY{+w}{ }\PY{n}{ind1}\PY{+w}{ }\PY{n}{ind2}
\PY{+w}{  }\PY{n+nb+bp}{·}\PY{+w}{ }\PY{n}{simp}\PY{+w}{ }\PY{n}{only}\PY{+w}{ }\PY{o}{[}\PY{n}{IsEmpty}\PY{n+nb+bp}{.}\PY{n}{forall\PYZus{}iff}\PY{o}{]}
\PY{+w}{  }\PY{n+nb+bp}{·}\PY{+w}{ }\PY{n}{intro}\PY{+w}{ }\PY{n}{a}\PY{+w}{ }\PY{n}{hind}\PY{+w}{ }\PY{n}{h}
\PY{+w}{    }\PY{n}{exists}\PY{+w}{ }\PY{n}{a}
\PY{+w}{    }\PY{n}{constructor}
\PY{+w}{    }\PY{n+nb+bp}{·}\PY{+w}{ }\PY{n}{exact}\PY{+w}{ }\PY{n}{B8}\PY{+w}{ }\PY{n}{a}\PY{+w}{ }\PY{l+m+mi}{1}
\PY{+w}{    }\PY{n+nb+bp}{·}\PY{+w}{ }\PY{n}{rfl}
\end{Verbatim}

\caption{Lean example}\label{lst:lean-example}
\end{rawlisting}




\appendix
\chapter{Uniformity}\label{chap:uniformity}
In this chapter, we focus on the descriptions of uniformity conditions used for
families of circuits.
There is a very thorough overview of uniformity conditions for
complexity classes below \complexityi{NC}{1} in~\cite{MIXBARRINGTON1990274}.

\section{\complexity{FO}-uniformity}\label{sec:uniformity-fo}
The definition of first-order uniformity is rather technical and is based on the notion of
first-order queries, introduced in~\cite[Definition~5.16]{Immerman1999-IMMDC}.
One of the results that we discussed which uses this notion is~\autoref{thm:fo-eq-ac0}.

% immerman:
% Definition 5.16 (Uniform) Let C be a sequence of circuits as in Equation (5.14).
% Let l' E {l'eo l'thc} be the vocabulary of circuits or threshold circuits. Let / :
% STRUC[l's] ~ STRUC[l'] be a query such that for all n E N, /(0") = Cn. That
% is, on input a string of n zero's the query produces circuit n. If / E FO, then C is
% 80 5. Parallelism
% afirst-order uniform sequence of circuits. Similarly, if I E L, then Cis logspace
% uniform. If I E P, then C is polynomial-time uniform, and so on. 


\section{\compUeAst-uniformity}\label{sec:uniformity-ueast}
The notions of \(U_E\) and \(U_{E^\ast}\)-uniformity (sometimes also called $E^\ast$-uniformity) were studied in early work on circuit uniformity.
These notions are defined in terms of the \emph{direct connection language}
and the \emph{extended connection language} of a circuit family; see~\cites[Definitions~2.24 and~2.43]{10.5555/520668}.

Since then, $U_E$- and $U_{E^\ast}$-uniformity have largely been displaced by \complexity{DLOGTIME}-uniformity.
Interesting arguments about why \complexity{DLOGTIME} uniformity is the most reasonable to consider
are presented in a breakthrough paper proving that binary integer division is in 
(\complexity{DLOGTIME}-uniform) \TC{0}~\cite{HESSE2002695}.

\section{\complexity{DLOGTIME}-uniformity}\label{sec:uniform-nc1}
For very low complexity classes, the most commonly used notion of
uniformity is \complexity{DLOGTIME}-uniformity, based on random-access
Turing machines. Similarly as in~\autoref{sec:uniformity-ueast},
this notion is also based on direct and extended connection language of a circuit family.
A~circuit family is \complexity{DLOGTIME}-uniform when we can decide its
direct connection language on a random-access Turing machine in logarithmic time.
For the details, we refer to:~\cite[Section~6]{MIXBARRINGTON1990274}.

\complexity{DLOGTIME}-uniformity has some very elegant properties.
The strength of such Turing machines is well-studied.
It's mentioned that \(\complexity{AC}^0_0 \subseteq \complexity{DLOGTIME} \subseteq \complexity{AC}^0_2\) in~\cite[Page~141]{10.5555/114872},
where $\complexity{AC}^0_k$ denotes $\complexity{AC}^0$ circuits of depth $k$.
It has also been shown that the class \complexity{DLOGTIME}-uniform \complexityi{NC}{1} is equal to \complexity{ALOGTIME} (alternating
logarithmic time, which we don't introduce here)
and also equal to \complexityi{NC}{1}-uniform \complexityi{NC}{1}:~\cite[Lemma~6.2]{MIXBARRINGTON1990274}.

% Similarly to using this class of Turing machines to decide the connection languages,
% functions can also be studied. This is the basis of the notion of \complexity{DLOGTIME}-reductions.
% A very detailed example of a concrete \complexity{DLOGTIME}-reduction is presented in~\cite{694595},
% showing that the tree isomorphism problem for string-represented trees is $\complexityi{NC}{1}$-complete.


% barrington:
%We define a log-time Turing machine to have a read-only input tape of length n,
% a constant number of read-write work tapes of total length O(log n), and a read-
% write input address tape of length log n. On a given time step the machine has
% access to the bit of the input tape denoted by the contents of the address tape (or
% to the fact that there is no such bit, if the address tape holds too large a number).
% We will assume (without loss of generality) that the machine always takes the same
% amount of time on inputs of a given length (this is because some of the work tape
% can always used as a clock). The following lemma summarizes some useful
% capabilities of such a machine.

% Important: tm can decide formula langugae: {<c, i, y>: |y|=n and ith char of nth formula is c}
% very important:

\begin{remark}[Bibliography]\label{remark:bibliography-david-s-johnson}
We refer to a source that is difficult to access (e.g.\ the above~\cite[Page~141]{10.5555/114872}): this is Chapter 2 ``A Catalog of Complexity Classes'' by
David S. Johnson~\cite{10.5555/114872.114874},
which appeared in January 1991 in ``Handbook of theoretical computer science (vol. A): algorithms and complexity'',
edited by Jan van Leeuwen~\cite{10.5555/114872}.
\end{remark}


% \section{p-projections}
% In~\cite{10.1145/3149.3158}, the notion of p-projections are introduced. In~\cite{10.1007/BFb0028550},
% a complete problem for \AC{0} of depth $k$ is discussed: complete for uniform~\AC{0} under 


% REMOVE THIS? THIS IS FOR P-PROJECTIONS FOR AC0k
% \begin{definition}[Projection of Boolean functions]\label{def:boolean-proj}
% Let $f:\{0,1\}^n\!\to\!\{0,1\}$ and $g:\{0,1\}^m\!\to\!\{0,1\}$.
% We say that $f$ is a \emph{projection} of $g$ if there is a mapping
% $\varsigma : \{y_1, \dots, y_m\} \rightarrow \{0, 1, x_1, \dots, x_n, \neg x_1, \dots, \neg x_n\}$
% such that
% \[
% f(x_1, \dots, x_n) = g\big(\varsigma(y_1),\dots,\varsigma(y_m)\big).
% \]
% \end{definition}
% \begin{definition}[p\mbox{-}projection between families]\label{def:boolean-p-proj}
% Let $\mathcal{P}=(P_i)_{i\in\mathbb{N}}$ and $\mathcal{Q}=(Q_j)_{j\in\mathbb{N}}$ be families
% of Boolean functions (each $P_i,Q_j:\{0,1\}^{*}\!\to\!\{0,1\}$, arity arbitrary).
% We say that $\mathcal{P}$ is a \emph{p\mbox{-}projection} of $\mathcal{Q}$, if 
% there exists a polynomial $t:\mathbb{N}\to\mathbb{N}$ such that
% for every $i\in\mathbb{N}$ there is some $j\le t(i)$ with $P_i$ a projection of $Q_j$.
% \end{definition}

% \begin{remark}[Why the polynomial bound]
% Unrestricted projection lets a simple $P_i$ be realized only by projecting
% some $Q_j$ at an arbitrarily large index $j$, which makes the comparison vacuous.
% The polynomial bound $j\le t(i)$ enforces an \emph{efficient} correspondence of indices,
% yielding a robust, reduction-like notion. The relation $\preceq_p$ is a preorder;
% modulo $\equiv_p$ it induces a partial order on equivalence classes of families.
% \end{remark}

% \section{Bibliographical remark}
% The notions of~\autoref{def:boolean-proj} and~\autoref{def:boolean-p-proj} are from~.
% \begin{definition}[\texorpdfstring{\(\complexity{AC}^0_k\)}{AC0 depth k}]\label{def:complexity-ac0-k}
%     This is \complexityi{AC}{0} of depth $k$.
%     Complete problem:\cite{10.1007/BFb0028550}
% \end{definition}

% \begin{proposition}
%     for \(k \geqslant 3\), \(\texttt{MAZE}_k\) is complete for non-uniform $\Pi_k$ under p-projections
%     and complete for uniform $\Pi_k$ under \complexity{DLOGTIME}-uniform projections~\cite{10.1007/BFb0028550}.
% \end{proposition}
\chapter{Definition of Neergaard's safe affine recursion}\label{chap:appendix-neergaard}

We continue to write functions as $f(\vec{x};\vec{y})$, with normal inputs
$\vec{x}$ and safe inputs $\vec{y}$. For $\delta \in \mathbb{N}$, put
$\texttt{shift}(y,\delta) = \lfloor y / 2^\delta \rfloor$, i.e.\ the number
obtained from $y$ by dropping its $\delta$ least significant bits.
Let $\mathbb{N}_2$ denote the natural numbers in binary notation.
  For $m,n \ge 0$, we write $\mathbb{N}_2^{m,n}$ for the set of functions
  with $m$ normal and $n$ safe arguments. All the arguments and the return value are in $\mathbb{N}_2$.

\begin{definition}[{\cite{10.1007/978-3-540-30477-7_21}}~Neergaard's $\complexity{BC}_\varepsilon^{-}$ algebra]\label{def:bc-eps}
  The class $\complexity{BC}_\varepsilon^{-}$ is the smallest class of functions over
  $\mathbb{N}_2$ that contains the initial functions~\ref{itm:bc-eps-zero}--\ref{itm:bc-eps-cond}
  and is closed under
  \ref{itm:neergaard-composition} and~\ref{itm:neergaard-recursion} below:
  \begin{enumerate}
    \item\label{itm:bc-eps-zero}\textbf{(zero)}
      \(
        0(;) \;=\; \varepsilon;
      \)
      \item\label{itm:bc-eps-pred}\textbf{(predecessor)}
      \(
        p(;\varepsilon) \;=\; \varepsilon,\quad
        p(; yb) \;=\; y \quad\text{for } b \in \{0,1\};
      \)
      \item\label{itm:bc-eps-proj}\textbf{(projections)}
      for $m,n \ge 0$ and $1 \le j \le m+n$,
      \(
        \pi^{m,n}_j(x_1,\dots,x_m ; x_{m+1},\dots,x_{m+n}) \;=\; x_j;
      \)
      \item\label{itm:bc-eps-succ}\textbf{(successors)}
      \(
        s_0(; y) \;=\; y0,\quad
        s_1(; y) \;=\; y1;
      \)
      \item\label{itm:bc-eps-cond}\textbf{(conditional)}
      \[
        c(; y_1,y_2,y_3) \;=\;
        \begin{cases}
          y_2 & \text{if $y_1$ ends in $1$ (i.e.\ $y_1 = z1$ for some $z$)},\\
          y_3 & \text{otherwise;}
        \end{cases}
      \]
    \item\label{itm:neergaard-composition}
      \textbf{(safe affine composition)}\
      let $f\in \complexity{BC}_\varepsilon^{-}$ have arity $(M,N)$, let
      $g_1,\dots,g_M \in \complexity{BC}_\varepsilon^{-}$ with
      $g_i$~of~arity~$(m,0)$, and let
      $h_1,\dots,h_N \in \complexity{BC}_\varepsilon^{-}$ with
      $h_i$ of arity $(m, n_i)$.
      Put $n := n_1 + \dots + n_N$.
      The \emph{safe affine composition}
      $f \circ \langle g_1,\dots,g_M ; h_1,\dots,h_N \rangle$
      is the function $F$ of arity $(m, n)$ defined by
      \[
        F(\vec{x};\vec{y}) =
        f\Bigl(g_1(\vec{x};\,),\dots,g_M(\vec{x};\,)\ ;\ 
              h_1(\vec{x};\vec{y}_1),\dots,h_N(\vec{x};\vec{y}_N)\Bigr),
      \]
      where each $\vec{y}_j$ is a (possibly empty) subtuple of the safe inputs
      $\vec{y}$ and every safe variable occurs in \emph{at most} one of the
      tuples $\vec{y}_1,\dots,\vec{y}_N$. In particular, no safe input may be
      duplicated, but some safe inputs may be unused.

    \item\label{itm:neergaard-recursion}
      \textbf{(safe affine course-of-value recursion)}\
      let $g,h_0,h_1,d_0,d_1 \in \complexity{BC}_\varepsilon^{-}$ be such that
      \[
        g : \mathbb{N}_2^{m,n} \to \mathbb{N}_2,\quad
        h_0,h_1 : \mathbb{N}_2^{m+1,1},\quad
        d_0,d_1 : \mathbb{N}_2^{m+1,0}.
      \]
      \emph{Safe affine course-of-value recursion} of them is the function
      $f : \mathbb{N}_2^{m+1,n}$ given by
      \[
      \begin{aligned}
        f(0,\vec{x};\vec{y}) &= g(\vec{x};\vec{y}),\\
        f(tb,\vec{x};\vec{y}) &=
          h_b \Bigl(t,\vec{x};\,f(\texttt{shift}(t,\delta_b),\vec{x};\vec{y})\Bigr)
      \end{aligned}
      \]
      for $b \in \{0,1\}$, where
      $\delta_b = \len{d_b(t,\vec{x};\,)}$ depends only on the normal
      arguments $(y,\vec{x})$. Here the recursive value
      $f(\texttt{shift}(t,\delta_b),\vec{x};\vec{y})$ is supplied to $h_b$ as
      its unique safe argument and can therefore be used at most once.
  \end{enumerate}
\end{definition}

\begin{theorem}[{\cite[Theorem~1]{10.1007/978-3-540-30477-7_21}}]\label{thm:neergaard-theorem}
    For any function definable in $\complexity{BC}_\varepsilon^{-}$ there is a Turing machine evaluating
    the function in \complexity{FL}. The Turing machine can be constructed from the function expression
    in logarithmic space in the size of the $\complexity{BC}_\varepsilon^{-}$-expression.
\end{theorem}



\begin{remark}[Intuition]
The recursive result is passed back in a safe
  position and can be used only once.
  Once a safe value is
  ``measured'' by the conditional operator,
    it must be recomputed; we \emph{cannot} duplicate it.
Dropping the affinity constraint collapses back to the original
\complexity{BC} algebra for \(\complexity{FP}\).
\end{remark}


\begin{remark}[Bibliography]
      In~\cite{murawski2000can}, with further refinements in~\cite{MURAWSKI2004197},
      \(\complexity{BC}^{-}\) (note the lack of $\varepsilon$ subscript) was introduced, an algebra that was contained in \complexity{FL},
      but was not known (and unlikely) to be \complexity{FL}-complete.
      In~\cite{10.1007/978-3-540-30477-7_21} this was improved to the result that
      \(\complexity{BC}_\varepsilon^{-} = \complexity{FL}\),
      with a short discussion that using course-of-value affine recursion instead of predicative affine recursion
      seems to be the reason why \(\complexity{BC}_\varepsilon^{-}\) is \complexity{FL}-complete,
      whereas \(\complexity{BC}^{-}\) is probably not.
\end{remark}

Writing programs in \(\complexity{BC}_\varepsilon^{-}\) is not reminiscent of any mainstream programming language.
The types being linear make most of the standard programming techniques not usable.
As part of our work, we have implemented an interpreter for the algebra in Haskell and tried to reproduce
some of the propositions from~\cite{10.1007/978-3-540-30477-7_21}. We present one listing, just to give
a hint of how the corresponding function look like; this is a Haskell representation of a function from the
algebra, not a programming language directly for the algebra. We present the full source code as
attachment to this thesis.\todo{How to add attachment to thesis?}

  % \begin{rawlisting}
  % \begin{Verbatim}[commandchars=\\\{\},numbers=left,firstnumber=1,stepnumber=1]
\PY{n+nf}{constZero}\PY{+w}{ }\PY{o+ow}{::}\PY{+w}{ }\PY{k+kt}{Int}\PY{+w}{ }\PY{o+ow}{\PYZhy{}\PYZgt{}}\PY{+w}{ }\PY{k+kt}{Int}\PY{+w}{ }\PY{o+ow}{\PYZhy{}\PYZgt{}}\PY{+w}{ }\PY{k+kt}{Func}
\PY{n+nf}{constZero}\PY{+w}{ }\PY{l+m+mi}{0}\PY{+w}{ }\PY{l+m+mi}{0}\PY{+w}{ }\PY{o+ow}{=}\PY{+w}{ }\PY{k+kt}{ZeroFunc}
\PY{n+nf}{constZero}\PY{+w}{ }\PY{l+m+mi}{0}\PY{+w}{ }\PY{n}{nSafe}\PY{+w}{ }\PY{o+ow}{=}
\PY{+w}{  }\PY{k+kt}{Composition}\PY{+w}{ }\PY{l+m+mi}{0}\PY{+w}{ }\PY{p}{(}\PY{n}{nSafe}\PY{+w}{ }\PY{o}{+}\PY{+w}{ }\PY{l+m+mi}{1}\PY{p}{)}\PY{+w}{ }\PY{l+m+mi}{0}\PY{+w}{ }\PY{p}{[}\PY{n}{nSafe}\PY{p}{,}\PY{+w}{ }\PY{l+m+mi}{0}\PY{p}{]}\PY{+w}{ }\PY{p}{(}\PY{k+kt}{Proj}\PY{+w}{ }\PY{l+m+mi}{0}\PY{+w}{ }\PY{l+m+mi}{2}\PY{+w}{ }\PY{l+m+mi}{2}\PY{p}{)}\PY{+w}{ }\PY{k+kt}{[]}\PY{+w}{ }\PY{p}{[}\PY{k+kt}{Proj}\PY{+w}{ }\PY{l+m+mi}{0}\PY{+w}{ }\PY{n}{nSafe}\PY{+w}{ }\PY{l+m+mi}{1}\PY{p}{,}\PY{+w}{ }\PY{k+kt}{ZeroFunc}\PY{p}{]}
\PY{n+nf}{constZero}\PY{+w}{ }\PY{n}{nNormal}\PY{+w}{ }\PY{n}{nSafe}\PY{+w}{ }\PY{o+ow}{=}
\PY{+w}{  }\PY{k+kr}{let}\PY{+w}{ }\PY{n}{g}\PY{+w}{ }\PY{o+ow}{=}\PY{+w}{ }\PY{n}{constZero}\PY{+w}{ }\PY{p}{(}\PY{n}{nNormal}\PY{+w}{ }\PY{o}{\PYZhy{}}\PY{+w}{ }\PY{l+m+mi}{1}\PY{p}{)}\PY{+w}{ }\PY{n}{nSafe}\PY{+w}{ }\PY{k+kr}{in}
\PY{+w}{  }\PY{k+kr}{let}\PY{+w}{ }\PY{n}{h}\PY{+w}{ }\PY{o+ow}{=}\PY{+w}{ }\PY{k+kt}{Proj}\PY{+w}{ }\PY{n}{nNormal}\PY{+w}{ }\PY{l+m+mi}{1}\PY{+w}{ }\PY{p}{(}\PY{n}{nNormal}\PY{+w}{ }\PY{o}{+}\PY{+w}{ }\PY{l+m+mi}{1}\PY{p}{)}\PY{+w}{ }\PY{k+kr}{in}
\PY{+w}{  }\PY{k+kr}{let}\PY{+w}{ }\PY{n}{d}\PY{+w}{ }\PY{o+ow}{=}\PY{+w}{ }\PY{k+kt}{Proj}\PY{+w}{ }\PY{n}{nNormal}\PY{+w}{ }\PY{l+m+mi}{0}\PY{+w}{ }\PY{l+m+mi}{1}\PY{+w}{ }\PY{k+kr}{in}
\PY{+w}{  }\PY{k+kt}{Recursion}\PY{+w}{ }\PY{p}{(}\PY{n}{nNormal}\PY{+w}{ }\PY{o}{\PYZhy{}}\PY{+w}{ }\PY{l+m+mi}{1}\PY{p}{)}\PY{+w}{ }\PY{n}{nSafe}\PY{+w}{ }\PY{n}{g}\PY{+w}{ }\PY{n}{h}\PY{+w}{ }\PY{n}{h}\PY{+w}{ }\PY{n}{d}\PY{+w}{ }\PY{n}{d}
\end{Verbatim}

  % \caption{Constant zero function of arity $(M, N)$ }\label{lst:neergaard-const-zero}
  % \end{rawlisting}
  % \begin{remark}
  %   Notice that the function depicted in~\autoref{lst:neergaard-const-zero} only returns 0,
  %   but nevertheless we need to use recursion to introduce normal arguments.
  % \end{remark}

\begin{rawlisting}
\begin{Verbatim}[commandchars=\\\{\},numbers=left,firstnumber=1,stepnumber=1]
\PY{c+c1}{\PYZhy{}\PYZhy{} Proposition 1. Let m and n by numbers in binary. Right shift shiftR(m : n) of}
\PY{c+c1}{\PYZhy{}\PYZhy{} m by |n| and selection of bit |n| from m are definable in BCeps\PYZhy{}.}

\PY{c+c1}{\PYZhy{}\PYZhy{} shiftR(n : m) = m \PYZgt{}\PYZgt{} |n|}
\PY{n+nf}{shiftR}\PY{+w}{ }\PY{o+ow}{::}\PY{+w}{ }\PY{k+kt}{Func}
\PY{n+nf}{shiftR}\PY{+w}{ }\PY{o+ow}{=}
\PY{+w}{  }\PY{k+kr}{let}\PY{+w}{ }\PY{n}{g}\PY{+w}{ }\PY{o+ow}{=}\PY{+w}{ }\PY{k+kt}{Proj}\PY{+w}{ }\PY{l+m+mi}{0}\PY{+w}{ }\PY{l+m+mi}{1}\PY{+w}{ }\PY{l+m+mi}{1}\PY{+w}{ }\PY{k+kr}{in}
\PY{+w}{  }\PY{k+kr}{let}\PY{+w}{ }\PY{n}{d}\PY{+w}{ }\PY{o+ow}{=}\PY{+w}{ }\PY{n}{oneNormalToZero}\PY{+w}{ }\PY{k+kr}{in}
\PY{+w}{  }\PY{c+c1}{\PYZhy{}\PYZhy{} h(timer : recursive) = tail(recursive)}
\PY{+w}{  }\PY{k+kr}{let}\PY{+w}{ }\PY{n}{h}\PY{+w}{ }\PY{o+ow}{=}\PY{+w}{ }\PY{k+kt}{Composition}\PY{+w}{ }\PY{l+m+mi}{0}\PY{+w}{ }\PY{l+m+mi}{1}\PY{+w}{ }\PY{l+m+mi}{1}\PY{+w}{ }\PY{p}{[}\PY{l+m+mi}{1}\PY{p}{]}\PY{+w}{ }\PY{k+kt}{Tail}\PY{+w}{ }\PY{k+kt}{[]}\PY{+w}{ }\PY{p}{[}\PY{k+kt}{Proj}\PY{+w}{ }\PY{l+m+mi}{1}\PY{+w}{ }\PY{l+m+mi}{1}\PY{+w}{ }\PY{l+m+mi}{2}\PY{p}{]}\PY{+w}{ }\PY{k+kr}{in}
\PY{+w}{  }\PY{k+kt}{Recursion}\PY{+w}{ }\PY{l+m+mi}{0}\PY{+w}{ }\PY{l+m+mi}{1}\PY{+w}{ }\PY{n}{g}\PY{+w}{ }\PY{n}{h}\PY{+w}{ }\PY{n}{h}\PY{+w}{ }\PY{n}{d}\PY{+w}{ }\PY{n}{d}
\end{Verbatim}

\caption{Function \texttt{shiftR} from the original paper.}\label{lst:neergaard-shiftr}
\end{rawlisting}
\begin{remark}
  The example in~\autoref{lst:neergaard-shiftr} shows the implementation of a function that
  shifts bit to the right (dropping least-significant bits). The implementation is standard,
  but it's worth to note that in the original paper, there is an error --- the function \texttt{shiftR} originally
  has flipped safe and normal arguments, making it impossible to implement. We outline our proof by contradiction for this fact.
  Consider a function $\mathtt{cond'(; x, y, z)}$ returning $y$ if $x$ is empty, $z$ otherwise.
  We can prove directly that this function is not implementable in \(\complexity{BC}^{-}\) due to
  the algebra not being able to differentiate between an empty string and a string beginning with a sufficiently
  long prefix of zeros.
  Assume the original \texttt{shiftR} is implementable. Then we can implement $\mathtt{cond'}$ using
  \texttt{shiftR}, which is a contradiction.
\end{remark}

\printbibliography[heading=bibintoc]

\end{document}


%%% Local Variables:
%%% mode: latex
%%% TeX-master: t
%%% coding: utf-8
%%% End:
